\documentclass[12pt]{article}
\usepackage{verbatim}
\usepackage[dvips]{epsfig}
\usepackage{color}
\usepackage{url}
\usepackage[colorlinks=true]{hyperref}

\begin{document}

\section*{GENESIS: Documentation}

{\bf Related Documentation:} \\
\href{../genesis-system/genesis-system.tex}{\bf GENESIS\,System}
% start: userdocs-tag-replace-items related-do-nothing
% end: userdocs-tag-replace-items related-do-nothing

\section*{Developer Installation Fedora 12 and Higher}

\subsection*{Introduction}

The documentation below gives specifics for installing GENESIS 3 on Fedora
Linux versions 12 and higher. It also contains some more general
information that will later be included in the general 
\href {../developer-installation/developer-installation.html}{\bf
Developer Installation} documentation.

The documentation 
\href {../installation-developer/installation-developer.html}{\bf GENESIS
Developer Installation} gives an overview and prerequisites for the
construction of a GENESIS 3 (G-3) development environment. Note that this
involves a lot more steps than installing a User Installation, because of
the software tools required for G-3 development.  If you do not plan to
check in changes to the software, it is not necessary to create a monotone
passphrase.

The document 
\href {../genesis-dependencies/genesis-dependencies.html}{\bf GENESIS Software
Dependencies} describes the system software tools required for a developer
installation of G-3 on any Unix/Linux platform.  In general, you will want
to install a full set of software development tools such as the gcc
compiler and Perl.  Python is required for use of the new 
\href {../sspy/sspy.html}{\bf SSPy} (Simple Scheduler in Python) scheduler.

\section*{Installation Steps}

Installing GENESIS on Fedora 12 requires executing the following major steps. {\bf Note}: Installation requires administrator privileges.
\begin{itemize}
   \item[] Prepare and upgrade the system software.
   \item[] Download and install the {\it DeveloperPackage}.
   \item[] Install the software packages.
   \item[] Check if the installation was successful. 
\end{itemize}


\subsection*{Note on configuring sudo}

When you install the G-3 developers release, it will create a
``neurospaces\_project" directory for development work in the home directory
of the user doing the installation.  This will hold subdirectories for each
of the G-3 software components.  Normally, you will want this user to be
yourself, rather than root.  In order to make it possible to install files
in system directories, the install process makes use of sudo.  Thus, you
will need to install sudo and add yourself to the {\it /etc/sudoers file}.
Ideally, you should use a tool called {\it visudo} to edit this file,
in order to add your username.  However, the easiest way is as root to do (for
example for user 'joeuser'):

\begin{verbatim}
        echo 'joeuser ALL=(ALL) ALL' >> /etc/sudoers
\end{verbatim}

During the install process, you will be prompted for your password.
However, when directing the output of an install or a
``neurospaces\_check" to a file, this can become a problem. The automated
tests performed by a check take a long time, and it is easy to
miss a password request and have it time out.  If you are on a secure
single-user system, you may wish to avoid the password check and set
it with (for example):

\begin{verbatim}
        echo 'joeuser ALL=(ALL) NOPASSWD:ALL' >> /etc/sudoers
\end{verbatim}

\subsection*{Prepare and upgrade the system software}

\begin{itemize}
   \item[] Check for the existence of any needed system tools described in
the documents referenced above.  Install any that are not found.  This may be
done on the command line with "{\tt rpm -q <packagename$>$}".
To install a package you can use the System -$>$ Administration -$>$ Add/Remove
Software menu or the command line "{\tt yum install <package$>$}" as root user.

      \begin{itemize}
         \item perl-CPAN (only required to run tests, see below)
         \item perl-YAML
         \item perl-Inline
         \item perl-Parse-RecDescent
         \item perl-Expect (and its dependency perl-IO)
         \item perl-ExtUtils-Embed
         \item perl-XML-Simple
         \item perl-libwww-perl (for LWP::Simple)
         \item readline
         \item readline-devel
         \item perl-Term-ReadLine-Gnu
         \item tetex-tex4ht (this will also install tex/latex if not present)
         \item swig
         \item \href{http://monotone.ca/}{\bf monotone-0.44} or higher and its dependency--the crypto libraries. You only need to install the client package (not the server package) 
      \end{itemize}
   \item[] For G-Tube, GUI development, and SSPy these additional packages and their
dependencies must be installed:

    \begin{itemize}
         \item PyYAML (required by the SSPy component)
         \item wxPython and wxPython-devel
         \item perl-Gtk2
         \item perl-SDL
         \item perl-GraphViz
         \item python-devel
         \item ncurses-devel
         \item perl-devel
    \end{itemize}

\end{itemize}
 
\subsection*{Download and install the {\it DeveloperPackage}}

\begin{enumerate}
   \item You can download the {\it DeveloperPackage} from the \href{http://repo-genesis3.cbi.utsa.edu/src/}{\bf GENESIS package archive}. It is called {\it developer-release-label.tar.gz}, where {\it release-label} is the current release identifier.
   \item For the most up to date version of the {\it DeveloperPackage}, you can download the latest snapshot from the development repository \href{http://repo-genesis3.cbi.utsa.edu/src/}{here.}
   \item Change to the directory where you downloaded the file.
   \item Unpack the archive by typing ``{\tt tar xfz developer-release-label.tar.gz}''.
   \item Change to the directory with the content of the archive by typing ``{\tt cd developer-release-label}.
   \item Configure by typing ``{\tt ./configure}''.
   \item Compile by typing ``{\tt make}''.
   \item Install by typing ``{\tt sudo make install}''  (requires administrator privileges). 
\end{enumerate}

\subsection*{Install the dependencies for running tests}

Check for the packages needed for running tests.  Most of these will
have been installed  by the developer package. For those that have not,
issue these commands as root:

      \begin{itemize}
         \item {\tt perl -MCPAN -e 'install Test::More}'
         \item {\tt perl -MCPAN -e 'install Clone}'
         \item {\tt perl -MCPAN -e 'install Data::Utilities}'
         \item {\tt perl -MCPAN -e 'install File::Find::Rule}'
         \item {\tt perl -MCPAN -e 'install Digest::SHA}' 
      \end{itemize}
      {\bf Note:} The running of tests is optional, but strongly advised.

To search for a Perl module that might have been previously installed
through CPAN or other than via an rpm, you can use "locate" if it was
installed before the (usually nightly) refresh of the database.	 For
example, to look for the module "Digest::SHA", type "locate Digest/SHA".
This may show it to be in /usr/lib64/perl5/Digest/SHA.pm on a 64 bit
version of linux, or return nothing if hasn't been installed.

\subsection*{Install software packages}

\begin{enumerate}
   \item Use the installer script to create the correct directory layout by typing ``{\tt neurospaces\_create\_directories}''.
   \item Pull the archives of the source code by typing ``{\tt neurospaces\_pull}''.
   \item Update the source code in the working directories by typing ``{\tt neurospaces\_update}''.
   \item Generate {\it make} files by typing ``{\tt neurospaces\_configure}''.
   \item Compile and install the software by typing ``{\tt neurospaces\_install}''.
\end{enumerate}

\subsection*{Checking for installation errors}

A first attempt at a developer installation of G-3 may fail because
one of the required software dependencies was not installed.  It is
therefore a good idea to capture the output and all error messages
resulting from the ``neurospaces\_install" command into a log file.
For example when using the bash shell:

\begin{verbatim}
    neurospaces_install > install.log 2>&1
\end{verbatim}

or for csh or tcsh:

\begin{verbatim}
    neurospaces_install > & ! install.log
\end{verbatim}

You may observe the progress in another terminal window in your home
directory with:

\begin{verbatim}
    tail -f install.log
\end{verbatim}

This is particularly useful if you expect to be prompted for a password
by sudo.  Once the installation has completed, look for errors in
the log file with ``{\tt grep Error install.log}".  A closer examination
may show a missing dependency.

\subsection*{Enabling G-Tube}

The GUI interface for G-3, g-tube is still under development, so its
installation is disabled in the developer package by default.
To perform an install with g-tube, give the commands:

\begin{verbatim}
        neurospaces_create_directories --enable g-tube
        neurospaces_init --regex g-tube --enable g-tube
        neurospaces_pull --enable g-tube
        neurospaces_update --enable g-tube
        neurospaces_configure
        neurospaces_install --enable g-tube
\end{verbatim}


\subsection*{Check if installation was successful}

This step is optional.
\begin{itemize}
   \item[] Run tests of all the packages and save tester output to a file  by typing
   ``{\tt neurospaces\_check >/tmp/check.out 2>\&1}'' when using bash, or
      ``{\tt neurospaces\_check >  \& !  tmp/check.out}" when using csh or tcsh as
      your shell.

   \item[] Check the output by typing ``{\tt less /tmp/check.out}''.
   Importantly, search for lines containing the string {\tt error\_count}.

\end{itemize}

\subsection*{Updating the installation}

While G-3 is under rapid development, it is important to frequently
upgrade your installation with the ``neurospaces\_upgrade" command.
As with ``neurospaces\_install", this is performed in your home directory,
with the output and error messages directed to a log file.

If Error messages are found in the upgrade log file, they may indicate a
new dependency that needs to be installed.  When upgrading an older
version, sometimes the installation of a particular software component
will fail, but repeating the ``neurospaces\_upgrade" will result in
a successful installation.  If there have been changes in the
DeveloperPackage that prevent a successful upgrade, you may update
this package first, by using the command:

\begin{verbatim}
    neurospaces_upgrade --regex developer
\end{verbatim}

and then perform a normal neurospaces\_upgrade.
 
\end{document}
