\documentclass[12pt]{article}
\usepackage[dvips]{epsfig}
\usepackage{color}
%e.g.  \textcolor{red,green,blue}{text}
\usepackage{url}
\usepackage[colorlinks=true]{hyperref}
\usepackage{scrtime}

\begin{document}

\section*{GENESIS: Documentation}

{\bf Related Documentation:}
% start: userdocs-tag-replace-items related-do-nothing
% end: userdocs-tag-replace-items related-do-nothing

\section*{Welcome to the GENESIS Documentation System}

As GENESIS 3 is developed, look for documentation and technical information at \\

\href{../documentation-overview/documentation-overview.tex}{\textcolor{red}{\bf GENESIS\,Documentation\,System:\,Overview\,and\,Contents}} \\

\noindent as well as at: \\

\href{http://www.neurospaces.org}{\bf Neurospaces}. \\

For your convenience, following the GENESIS Quick Start (below). \\

Links are also provided to the documentation for previous versions of GENESIS (Version 2).

\subsection*{GENESIS 3}

\subsubsection*{Quick Start}

\begin{itemize}

   \item \href{../genesis-intro/genesis-intro.tex} {\bf GENESIS Introduction}

   \item \href{../genesis-overview/genesis-overview.tex}{\bf Overview\,of\,GENESIS}

   \item {\bf GENESIS Installation Guides} (Currently, we recommend the Developer Installation for all users)

      \begin{itemize}
         \item \href{../installation-debian/installation-debian.tex}{\bf Brief\,User\,Installation\,Guide}
         \item \href{../developer-installation/developer-installation.tex}{\bf Summary\,of\,Developer\,Installation\,Documentation}
      \end{itemize}

   \item \href{../tutorial-genesis/tutorial-genesis.tex} {\bf Introductory
          Tutorials on creating and running simulations}

   \item \href{../background-material/background-material.tex}
          {\bf Recommended Background Reading}

    \item  {\bf Relevant technical documentation for advanced users}
      \begin{itemize}
         \item \href{../technical-guide-1/technical-guide-1.tex}
                {\bf Technical User Guide 1}
         \item \href{../units-conversion/units-conversion.tex}
                {\bf Conversion From Physiological to SI Units in GENESIS 3}
      \end{itemize}

    \item \href{../genesis-extend-functionality/genesis-extend-functionality.tex}
          {\bf Extending GENESIS Functionality}
\end{itemize}

\subsection*{GENESIS 2}

\begin{itemize}
 
 \item \href{http://www.genesis-sim.org/GENESIS/bog/bog.html}{\bf The Book of GENESIS}\\
 This is the PDF version of {\it The Book of GENESIS}. The ``BoG'' (Bower and Beeman, 1998) is the second edition of a practical introduction to realistic neural modeling through the use of the GENESIS simulator. Chapters 4, 5, 7, and 9 cover the basics of realistic neural modeling.  Either \href{http://www.genesis-sim.org/GENESIS/iBoG/iBoGpdf/index.html}{\bf view} the PDF files or \href{http://www.genesis-sim.org/GENESIS/iBoG/index.html}{\bf download} PDF or Postscript files.

\item \href{http://www.genesis-sim.org/GENESIS/UGTD.html}{\bf The Ultimate GENESIS Tutorial Distribution}\\
This is the latest version of a complete self-paced course on biologically realistic modeling, based on an extended version of the GENESIS Neural Modeling Tutorials package. It contains more tutorials, the Book of GENESIS, GENESIS 2.3 source and binary distributions with installation instructions, and additional GENESIS network simulation scripts, examples, and exercises. Download this if you want everything in one package.

\item \href{http://www.genesis-sim.org/GENESIS/GNMT.html}{\bf The GENESIS Neural Modeling Tutorials}\\
These tutorials are an evolving package of HTML tutorials intended to teach the process of constructing biologically realistic neural models with the GENESIS simulator. This is the best entry point to access a large collection of documentation and tutorials about neural modeling in general, and GENESIS in particular. More information and downloads for the GENESIS Neural Modeling Tutorials

 \item \href{http://www.genesis-sim.org/GENESIS/Hyperdoc/Manual.html}{\bf Reference Manual}\\
This is a hypertext Reference Manual for GENESIS commands and simulation object types (classes). Although it will not teach you neural modeling, it is the indispensible reference for creating GENESIS 2 simulation scripts. \href{http://www.genesis-sim.org/genesis-ftp/}{\bf Download} the gzipped manual.

\item \href{http://www.genesis-sim.org/GENESIS/Tutorials_summary.html}{\bf Neurobiological Tutorials}\\
The GENESIS distribution includes a number of demonstration simulations in the form of ``user-friendly'' tutorials. Several of these have been adapted from recent research simulations, and may be modified and used to create one's own simulations. \href{http://www.genesis-sim.org/GENESIS/bog/bog.html}{\it The Book of GENESIS} uses these simulations to teach concepts in computational neuroscience, as well as the use of GENESIS. These are some \href{http://www.genesis-sim.org/GENESIS/illtuts/illtuts.html}{\bf examples and screen shots} of the tutorial simulations provided with GENESIS.

\item {\bf User-contributed Documentation and Tutorials}\\
These are found at the top of this page (under the Education tab on the \href{http://www.genesis-sim.org}{\bf GENESIS home page}), along with short mini-tutorials, If you would like to contribute a short ``HOWTO'' document, commented example simulation or demo, or educational material suitable for course use, please upload it under Education.

\item \href{../g2-gap-junction/g2-gap-junction.tex}{\bf Create gap junction object and gap junctions in GENESIS 2}

\end{itemize}

{\bf Build date: \today, \thistime}

\end{document}
