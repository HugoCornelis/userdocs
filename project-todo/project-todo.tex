\documentclass[12pt]{article}
\usepackage[dvips]{epsfig}
\usepackage{color}
%e.g.  \textcolor{red,green,blue}{text}
\usepackage{url}
\usepackage[colorlinks=true]{hyperref}

\setlength{\textwidth}{15.7cm}
\setlength{\textheight}{22cm}

\setlength{\oddsidemargin}{0pt}

\setlength{\topmargin}{0cm}
\setlength{\headheight}{1cm}
\setlength{\headsep}{0.5cm}
\setlength{\topskip}{0cm}

\begin{document}

\section*{GENESIS: Documentation}

\section{The TODO and the DONE List}

This document contains a list of functions and descriptions that need
to be implemented or incorporated in the software or software
infrastructure.  The level of detail of the descriptions can vary
between items.  Some item is described, and sometimes a checklist is
given that can be checked by inserting the word 'DONE' into this
document.

Overtime this document can grow because people add new functions that
await implementation, and it can be consolidated by moving items that
have been completed from the TODO list to the DONE list.  Some
functions are related to maintenance and they never disappear (such as
Packaging and Distribution).

The growing DONE list gives a tangible measure of project progress.


\section{TODO List}

\subsection{Packaging and Distribution}

Packaging involves the creation of tarballs with source code to the
creation of binary packages for different OSs.

For packaging and distribution of the alpha release of GENESIS 3.0,
each software component must contain the following makefile targets:

\begin{itemize}
\item 'dist': for creation of the tarball with source code.
\item 'pkg-deb': for creation of a binary archive for debian based
  Linux distributions (including Ubuntu).
\item 'pkg-rpm': for creation of a binary archive for rpm based Linux
  distributions (such as Redhat and Fedora).
\end{itemize}


\subsection{Benchmarking}

Test the performance of the Purkinje cell in each of the following
environments:

\begin{itemize}
\item GENESIS 2 (old GENESIS 2 enviroment).
\item NS-SLI (GENESIS 3 - GENESIS 2 backward compatibility).
\item SSP (GENESIS 3 batch simulation environment).
\item gshell (GENESIS 3 interactive simulation environment).
\end{itemize}


\subsection{Report Document for Testers}

Construct a report document with instructions for alpha and beta
testers.  The instructions consist of installation instructions and a
questionnaire about basic application functionality.


\subsection{Release Notes}

** Document that describes the release
*** Compile a list of G3 capabilities
**** compile them from each of the components
*** Make statements about
**** backward compatibility
***** clarify the level of backward compatibility 
***** emphasize on commitment of backward compatibility
***** explain policies for feature requests
***** rallpacks
***** does the purkinje cell and its various stimulation protocols
***** please try your simulations and report problems
**** NeuroML support
**** PyNN support
***** placeholder for an interface to PyNN through the pynn\_load command
*** Installation
**** Compile a list of dependencies
**** Create an integrated installation tar ball or so.
**** add deb and rpm packages
http://www.artificialworlds.net/blog/2007/02/22/creating-deb-and-rpm-packages/
**** Release plan: alpha, beta, full release
**** synchronize userdocs and wiki installation docs
*** mailing lists, discussion boards and other genesis related systems
**** what are they used for?
**** what do we want them to be used for? why?
**** Add hg repository
**** Add bulletin boards and mailing lists
**** Add NS related systems
***** www.neurospaces.org
***** blog
*** alpha: Mid-October
*** beta1: Mid-January
*** beta2: Mid-March
*** full: Mid-July


\subsection{Level 4 Documentation: Technical Specifications}

For each software component a set of (html) documents must be
generated that specify the capabilities of the component.  For Unix
shell scripts, technical specifications can be generated from the
usage messages generated by a '--help' argument.

\subsection{Level 3-7 Documentation: Creation Guide}

An outline of how Level 3-7 documentation is created and included in
the appropriate locations is required.


\subsection{Document Dependencies}

Perl dependencies:
\begin{itemize}
\item YAML
\item Clone
\item Data::Utilities
\end{itemize}


\subsection{Progress Reports to the Supplement}


\subsection{Add Network Modeling Support}

Because this is an important function, it requires substantial
development on one software component, and also impacts other software
components.

\begin{itemize}
\item DES
  \begin{itemize}
  \item instantiation of DES queuer stand-alone (DONE)
  \item instantiation of DES distributor stand-alone (DONE)
  \item tests of DES stand-alone (DONE)
  \end{itemize}
\item model-container
  \begin{itemize}
  \item instantiation of projections / connections by model-container (DONE)
  \item inspection of projections / connections by model-container API (DONE)
  \end{itemize}
\item SSP
  \begin{itemize}
  \item instantiation of DES queuer from ssp (partial)
  \item instantiation of DES distributor from ssp (partial)
  \item translation of model-container connections to DES connections (partial)
  \item tests for ssp
  \end{itemize}
\item gshell
  \begin{itemize}
  \item add integration commands to gshell
  \item tests for gshell
  \end{itemize}
\item backward compatibility
  \begin{itemize}
  \item integration of planar / volume connect command
  \item tests for backward compatibility
  \end{itemize}
\end{itemize}


\section{DONE List}

\subsection{Monotone Clone Operation Integration with the Installer Scripts}
Integrate 'mtn clone' into the installer scripts (DONE).

\end{document}

%%% Local Variables: 
%%% mode: latex
%%% TeX-master: t
%%% End: 
