\documentclass[12pt]{article}
\usepackage[dvips]{epsfig}
\usepackage{color}
%e.g.  \textcolor{red,green,blue}{text}
\usepackage{url}
\usepackage[colorlinks=true]{hyperref}
\usepackage{scrtime}
 
\begin{document}

{\bf Related Documentation:}
% start: userdocs-tag-replace-items related-todo
% end: userdocs-tag-replace-items related-todo

\section*{GENESIS: Documentation}

\section{The TODO and the DONE List}

This document contains a list of functions and descriptions that need
to be implemented or incorporated in the software or software
infrastructure.  The level of detail of the descriptions can vary
between items.  After an item is described, it can be followed by a
checklist with information of the level of implementation.  For
example, subitems in the checklist can be marked with the word `DONE'.

This document can grow when people add new functions that await
implementation, and it can be consolidated by moving items that have
been completed from the TODO list to the top of the DONE list.  Items
are never removed from the DONE list.
%  Some functions
%are related to maintenance and they never disappear (such as Packaging
%and Distribution).  They can be marked with 'WORKING'.

This TODO list was built on \today, \thistime.


\section{TODO List}


\subsection{testing reconfiguration}

Mando, I have no idea what is going on, but it seems autoconf related
and related to uninstallation of STARTUP\_FILES.  You?

I don't know how this would influence the consistency of the tester
output, for instance if it leaves foreign files that contaminate the
filesystem.  I also don't know how long this has been happening on my
machine because this type of error is not reported by
neurospaces\_cron when not run from a crontab.


\$ grep -v 't/died' /tmp/neurospaces\_cron.stdout | grep -v 't/dont\_overwrite\_die\_handler' | grep -v 't/subtest/die' | grep -v 't/Tester/tbt\_03die' | grep -B 15 die
make[2]: Entering directory
`/home/cornelis/neurospaces\_project/ns-sli/source/snapshots/0/src'
 ( cd '/usr/local/bin' \&\& rm -f ns-sli )
rm -rf /usr/local/ns-sli/
 ( cd '/usr/local/ns-sli/' \&\& rm -f startup/grey startup/xmacros.g startup/xtools.g startup/vclamp.g startup/redhot startup/rainbow2 startup/escapelist.g startup/schedule.g startup/rainbow startup/defaults.g~ startup/nxsimrc-ns-sli startup/minsimrc-ns-sli~ startup/Xdefaults.example startup/simrc-ns-sli.example startup/hot startup/defaults.g startup/Usermake startup/minsimrc-ns-sli startup/schedule.g~ startup/simrc~ startup/simrc-ns-sli~ startup/xstartup.g startup/simrc-ns-sli startup/nxsimrc~ startup/configrc startup/Libmake startup/Makefile startup/minschedule.g startup/minsimrc~ )
/bin/bash: line 4: cd: /usr/local/ns-sli/: No such file or directory
make[2]: *** [uninstall-nobase\_dist\_ns\_slistartupDATA] Error 1
make[2]: Leaving directory
`/home/cornelis/neurospaces\_project/ns-sli/source/snapshots/0/src'
make[1]: *** [uninstall-recursive] Error 1
make[1]: Leaving directory
`/home/cornelis/neurospaces\_project/ns-sli/source/snapshots/0/src'
make: *** [uninstall-recursive] Error 1
/usr/local/bin/neurospaces\_build: *** Error: package ns-sli: 512 at
/usr/local/bin/neurospaces\_build line 4269.
 at /usr/local/bin/neurospaces\_build line 4269
       main::operation\_execute('ARRAY(0x99b9668)', 'HASH(0x9a3f360)',
'ARRAY(0x9a94e90)') called at /usr/local/bin/neurospaces\_build line
781
       main::all\_operations\_execute() called at
/usr/local/bin/neurospaces\_build line 546
       main::main() called at /usr/local/bin/neurospaces\_build line 4344
*** die: /usr/local/bin/neurospaces\_build: *** Error: package ns-sli:
512 at /usr/local/bin/neurospaces\_build line 4269.


\subsection{testing reconfiguration}

neurospaces\_clean has a --distclean option, but I don't think it has
ever been tested.  We should incorporate it into neurospaces\_cron and
neurospaces\_upgrade.


\subsection{userdocs-sync error messages}

The userdocs-sync command in the g-tube source directory gives error
messages about the descriptor.yml files part of the g-tube source
code.  This is a bug because userdocs-sync is supposed to only run in
the userdocs source directory.


\subsection{ns-sli warning registry}

The ns-sli does more soundness checks on a model and the simulation
run-time and gives many more warnings than G-2.  Typical examples
include:

\begin{itemize}
\item When running
  tests/scripts/test-traub91-v0/traub91\_asym\_simple3.g:
\begin{verbatim}
*** Warning: the setmethod command is deprecated in G-3
\end{verbatim}
\item When running tests/scripts/PurkM9\_model/CURRENT9.g:
\begin{verbatim}
Warning: ParameterResolveToPidinStack() does not perform any operations.
\end{verbatim}
\item When running
  tests/scripts/test-traub91-v0/traub91\_asym\_simple3.g:
\begin{verbatim}
*** Warning: zero initial value for the concentration of the following:
/cell/apical_18/Ca_conc
\end{verbatim}
\item When running
  tests/scripts/test-traub91-v0/traub91\_asym\_simple3.g:
  \begin{verbatim}
Warning: Tabchannel parameter xmin has already been set to 0.000000 for K_C
Warning: Tabchannel parameter xmax has already been set to 1000.000000 for K_C
  \end{verbatim}
\item During execution of tests/scripts/rallpack3-simple.g (1):
\begin{verbatim}
warning: cell found during solver_mathcomponent_typer()
warning:     /axon/solve
warning: cell found during solver_mathcomponent_processor()
warning:     /axon/solve
warning: cell found during solver_mathcomponent_finalizer()
warning:     /axon/solve
\end{verbatim}
\item During execution of tests/scripts/rallpack3-simple.g (2):
\begin{verbatim}
Warning: Heccer object /axon exists, resetting it instead.
\end{verbatim}
\item During readcell tests/scripts/test-RSnet2/RScell.p in
  tests/scripts/test-RSnet2/RScell-G3-nohole.g:
\begin{verbatim}
*** Warning: during readcell: channel /library/Na_pyr_dp has no Gbar / gmax defined
*** Warning: during readcell: channel /library/Kdr_pyr_dp has no Gbar / gmax defined
*** Warning: during readcell: channel /library/KM_pyr_dp has no Gbar / gmax defined
\end{verbatim}
\end{itemize}

The implementation of a warning registry that registers warnings
without displaying them avoids annoying a user.  A new SLI command
'show\_warnings' shows all the registered warnings.


\subsection{website images}

The documentation system uses ht4tex which is the successor of
latex2html.

It looks like a set of configuration files of dvipng determine the
quality of the generated images.  Each linux distro puts these files
in different locations.

So far it is not clear how exactly to modifiy the configuration to
improve the quality of the images, neither to the choice of
distribution we would like to support.

Proposal to support at least one user-oriented operating system for
generating higher quality images, and I still think we should do this.
Then we can understand how well this works.

The genesis website runs on CentOS, not user-oriented.  The best
candidate is maybe ubuntu.


\subsection{gshell output\_resolution and output\_interval}

The problem comes when I want to set an output interval (i.e. a time
step for a G2 clock) that is not related to the integration time step.
Presently one can give a gshell command, such as:

\begin{verbatim}
genesis > output\_resolution 5
\end{verbatim}

to output every fifth integration step, but the resolution has to be
an integer multiplier of simulation time step. I had a case where I
wanted to compare to a G2 result where the simulation clock was 20
microsec and the output clock was 50 microsec, and it was not
posssible.  Furthermore, it is only possible to set the output
resolution if the output generator is running in 'steps' output\_mode,
which means that the file output will list step number instead of
simulation time.  Perhaps that should be an option, but one usually
cares about the simulation time, not the step number, nor the relation
between the output interval and the simulation time step. In fact, if
we at some point incorporate variable time step solver methods, it
wouldn't be possible to relate the "output resolution" to the time
step.

So, my suggestion is to replace 'output\_resolution' with
'output\_interval', and that this should be an option within g-tube
also.

Another slightly related issue that came up in the benchmarking is
that the rallpack axon model produces two output files, one for each
end of the axon.  Alternatively, it could have used asc\_file to
produce a single file with three columns for time, axon0, axonx
values.  But neither option is presently possible with G3.  (I have
not yet implemented the G3plot capability to plot multiple columns
yet, but it is on the list.)  These options should be part of the
gshell/g-tube output model.


\subsection{tester cron job mail reporting}

The email reporting facility of the tester needs to be refactored and
cleaned up.  It currently is highly sensitive to small
misconfiguration errors, see also the entry of 29th of August in
system-administration-log.

It is likely that the best way forward is to refactor things across
the different scripts by creating a couple of perl packages.


\subsection{tester framework and compilation errors}

The script neurospaces\_build does not detect compilation errors.
Maybe because the option keep-going is set, needs further
investigation.  The priority of this problem is very low.  Compilation
errors are assumed to be detected on the machine of checkin rather
than the automated tester.


\subsection{when adding a new software component}

\begin{itemize}
\item Figure out the workflow from adding the experiment component.
\item Default monotone branch pattern: The neurospaces\_build script
  should set a default branch pattern during the execution of the
  procedure to add a new software component.
\item It's also possible to have the public keys in a text file so
  that when a ne repo is initialized, those public keys are
  automatically put in
\end{itemize}

See also the document that
describes~\href{../genesis-addto-component-developerpackage/genesis-addto-component-developerpackage.tex}{how
  to add a new software component}.


\subsection{ns-sli installation}

The ns-sli module install files under /usr/local/ns-sli rather than
/usr/local/neurospaces/ns-sli.  This installation needs to be ligned
up with the installation of other modules such as gshell.


\subsection{morphology\_summarize}

To characterize a cell morphology it would be good to extend the
morphology\_summarize command to include the average (+/- SD, SE ?)
number of branch point from the soma to each tip and the number of
primary, secondary, tertiary . . . etc, branch points.


\subsection{GShell -- Endogeneous Activation}

Define GShell tests for endogeneous activation.


\subsection{building documentation}

Correct build of the documentation depends on a working G-3
installation.  It looks as if userdocs\_cron should be integrated into
neurospaces\_cron.  Both the build of the documenation and the
execution of the tests should be configurable from the
neurospaces\_cron options.  The userdocs\_cron script should call the
neurospaces\_docs script.

See also chat between Mando and me date 23th of August 2010.


\subsection{ns-sli: follow-up after the October developers meeting 1}

Add monotone version information identifiers to the gshell command
'list components'.


\subsection{ns-sli: follow-up after the October developers meeting 1}

Implement the following SLI commands and objects:

\begin{itemize}
\item scaletabchan: also try to implement a tagging mechanism that
  allows to browse the 'history of a tabchan' (Hugo).
\item spikegen: likely requires a trivial mapping to the spikegen
  token in the model-container (DONE).
\item delta compression of ascii files: requires the glue of the
  Experiment::Output object to create a filesystem fifo with a dummy
  name, bind it to a perl oneliner and pipe it to bzip2 to write
  compressed output (Mando).
\item randomspike: likely needs network support to be finished first
  (Hugo).
\item pulsegen: likely needs network support to be finished first
  (Mando).
\item asc\_file: has several common fields currently not supported by
  the backward-compatibility module: float\_format, append, initialize,
  leave\_open (Mando).
\end{itemize}

\subsection{ns-sli: follow-up after the October developers meeting 2}

Others that need implementation:

\begin{itemize}
\item Table object functions and modes, some are spike related, some
  are input and output related.
\item Analysis functions related to table objects.
\item Interspike and peristim objects (low priority).
\item Createmap needs proper testing.
\item Volumeconnect needs to be linked to the volumeconnect algorithm
  in the model-container.
\item Script\_out: requires a hook in ssp.
\item Disk\_in and disk\_out equivalents.  Likely the delta-compressor
  and the annotations provided by DAVIS are already sufficient.
\end{itemize}

\subsection{Model container and readline library}

The model-container does not use the readline library for its
querymachine.  This function was present in the past and should be
restored because it helps in querying models via the querymachine
interface.

There is logic to detect readline in the filesystem present in the
configure script, but somehow this part of the configuration does not
propagate through to the compilation phase of the model-container.

{
  \vspace{5mm}
  \centering
  \begin{tabular}{|c|c|}
    \hline
    Notes
    & \\
    \hline
    Time Estimate
    & 1 days \\
    \hline
    Assigned to
    & Mando \\
    \hline
  \end{tabular}
}


\subsection{File permission errors}

Certain tests output to a hardcoded filename, such as /tmp/output.
When one user runs the tests this file is given permissions for their
user id. If a second user tries to run the same test, it writes to the
same filename, the test will fail and the test will bail out with an
error (as was displayed by the regression tester machine).  This
bottlenecks the number of users that can reliably install a developer
installation on the same machine.


{
  \vspace{5mm}
  \centering
  \begin{tabular}{|c|c|}
    \hline
    Notes
    &  \\
    \hline
    Time Estimate
    & ??? \\
    \hline
    Assigned to
    & ??? \\
    \hline
  \end{tabular}
}


\subsection{Mercurial}

The source code of most GENESIS 3 software components are maintained
using the Monotone version control system.  The installer scripts
integrate with Monotone to provide for automated building and testing.

The {\bf G-Tube} is maintained using the Mercurial version control system.
We must continue and finish the integration of the Mercurial version
control system and the {\it DeveloperScript neurospaces\_build}.


\subsection{Gating Variables with Coupled Equations}

There is currently no way to solve coupled equations of gating
variables.

dm/dt = A * m + B * n
dn/dt = C * m + D * n

In GENESIS 2, this is possible using forward integration methods only.


\subsection{Populate the Model Library}

Build up a library of cells that can be used with G2 or G3.

\begin{itemize}
\item Purkinje cell model edsjb1994 (present in the model container).
\item Purkinje cell model Sergio (hopefully by email).
\item Purkinje cell model Allan (may be included in dash).
\item Rapp's passive model (must be reconstructed from the paper).
\item Different species purkinje cell morphologies (gp, rat, turtle,
  finch (present in project-browser).
\item Rallpacks 1, 2, 3 (present in NS-SLI).
\item Traub91 (present in NS-SLI).
\end{itemize}

\subsection{Level 4 Documentation: Technical Specifications}

For each software component a set of (html) documents must be
generated that specify the capabilities of the component.  For Unix
shell scripts, technical specifications can be generated from the
usage messages generated by a ``{\tt --help}'' argument.

\begin{itemize}
\item Developer package:
  \begin{itemize}
  \item neurospaces\_build
  \item neurospaces\_cron
  \item release-expand
  \item neurospaces\_harness
  \item tests\_2\_html
  \item td-labels
  \item td-majors
  \end{itemize}
\item model-container:
  \begin{itemize}
  \item neurospacesparse
  \item morphology2ndf
  \end{itemize}
\item experiment:
\item heccer:
  \begin{itemize}
  \item heccer
  \end{itemize}
\item ssp:
  \begin{itemize}
  \item ssp
  \item ssp\_directory
  \end{itemize}
\item studio:
  \begin{itemize}
  \item neurospaces
  \end{itemize}
\item ns-sli:
  \begin{itemize}
  \item ns-sli
  \item convert
  \end{itemize}
\item exchange:
  \begin{itemize}
  \item neurospaces\_exchange
  \end{itemize}
\item gshell:
  \begin{itemize}
  \item genesis-g3
  \end{itemize}
\item userdocs:
  \begin{itemize}
  \item userdocs-build
  \item userdocs-check
  \item userdocs-create
  \item userdocs-rename
  \item userdocs-snippet
  \item userdocs-sync
  \item userdocs-tag-filter
  \item userdocs-tag-replace-items
  \item userdocs-version
  \end{itemize}
\end{itemize}


\subsection{Level 3-7 Documentation: Creation Guide}

Documentation from Levels 3--7 are "automatically" generated. What is
required is a description for developers of how to create these levels
of documentation and how to add them to the appropriate places where
they can be accessed by the GENESIS Documentation System.  These are
level 2 documents.

\begin{itemize}
\item Level 3: explain the html-upload target, and how the html-pages
  are copied to a directory visible to apache.
\item Level 4: Each software package creates its own technical
  specifications, that are copied to a directory visible to apache.
\item Level 5: Only contains links to external websites.  As an
  example the document effic84.
\item Level 6: Explains the proper format for inserting doxygen comments,
 doxygen configuration, and doxygen output formats.
\item Level 7: Inline source code documentation, and Xrefactory configuration. 
\end{itemize}

\subsection{Documentation Browsing}

Tag all the documentation with keywords to define flows through the
documentation and ease browsing and reading of related documents.

Should be provided via a browser-based framework to give
`professional' look to documentation. See Documentation Framing
(following).


\subsection{Progress Reports to the Supplement}

Every three months a report about progress on the work of the
administrative supplement must be submitted to NIH.

This can be done by picking items from the DONE list related to the
work proposed in the administrative supplement.

First move conceptualized items from the supplement to this TODO list.


\section{DONE List}

\subsection{gshell help components}

Diagnosis of problem and documented procedure for troubleshooting the Inline::Python module is found in \href{../genesis-dependencies/genesis-dependencies.tex}{GENESIS dependencies} document, in the {\it Note about Inline::Python} section. (Working)

Fails on Mac OSX after reinstallation of GENESIS developer package for Python module:
\begin{verbatim}
Other components:
  python:
    description: interface to python scripting
    module: GENESIS3::Python
    status: |
      Error. You have specified 'Python' as an Inline programming language.
      
      I currently only know about the following languages:
          C, Foo, foo
      
      If you have installed a support module for this language, try deleting the
      config file from the following Inline DIRECTORY, and run again:
      
          /Users/adcmobile/.genesis3/gshell/InlineCode
      
       at /usr/local/glue/swig/perl/GENESIS3/Python.pm line 25
       at /usr/local/bin/genesis-g3 line 52
      	main::__ANON__('Error. You have specified \'Python\' as an Inline programming...') called at /System/Library/Perl/5.8.8/Carp.pm line 269
      	Carp::croak('Error. You have specified \'Python\' as an Inline programming...') called at blib/lib/Inline.pm (autosplit into blib/lib/auto/Inline/check_config_file.al) line 711
      	Inline::check_config_file('Inline=HASH(0xa3043c)') called at /Library/Perl/5.8.8/Inline.pm line 244
      	Inline::glue('Inline=HASH(0xa3043c)') called at /Library/Perl/5.8.8/Inline.pm line 146
      	Inline::import('Inline', 'Python', 'import sys\x{a}sys.path.append(\'/usr/local/glue/swig/python\')\x{a}i...') called at /usr/local/glue/swig/perl/GENESIS3/Python.pm line 25
      	GENESIS3::Python::BEGIN() called at /usr/local/glue/swig/perl/GENESIS3/Python.pm line 52
      	eval {...} called at /usr/local/glue/swig/perl/GENESIS3/Python.pm line 52
      	require GENESIS3/Python.pm called at (eval 41) line 3
      	eval 'require GENESIS3::Python
      ;' called at /usr/local/glue/swig/perl/GENESIS3.pm line 3611
      	GENESIS3::profile_environment() called at /usr/local/glue/swig/perl/GENESIS3.pm line 3635
      	require GENESIS3.pm called at /usr/local/bin/genesis-g3 line 256
      	main::main() called at /usr/local/bin/genesis-g3 line 354
      BEGIN failed--compilation aborted at /usr/local/glue/swig/perl/GENESIS3/Python.pm line 52.
       at /usr/local/bin/genesis-g3 line 52
      	main::__ANON__('Error. You have specified \'Python\' as an Inline programming...') called at /usr/local/glue/swig/perl/GENESIS3/Python.pm line 52
      	require GENESIS3/Python.pm called at (eval 41) line 3
      	eval 'require GENESIS3::Python
      ;' called at /usr/local/glue/swig/perl/GENESIS3.pm line 3611
      	GENESIS3::profile_environment() called at /usr/local/glue/swig/perl/GENESIS3.pm line 3635
      	require GENESIS3.pm called at /usr/local/bin/genesis-g3 line 256
      	main::main() called at /usr/local/bin/genesis-g3 line 354
      Compilation failed in require at (eval 41) line 3.
       at /usr/local/bin/genesis-g3 line 52
      	main::__ANON__('Error. You have specified \'Python\' as an Inline programming...') called at (eval 41) line 3
      	eval 'require GENESIS3::Python
      ;' called at /usr/local/glue/swig/perl/GENESIS3.pm line 3611
      	GENESIS3::profile_environment() called at /usr/local/glue/swig/perl/GENESIS3.pm line 3635
      	require GENESIS3.pm called at /usr/local/bin/genesis-g3 line 256
      	main::main() called at /usr/local/bin/genesis-g3 line 354
    type:
      description: scriptable user interface
      layer: 2
\end{verbatim}


\subsection{Incorrect path to NDF model file corrupts G-Shell}
\begin{verbatim}
dhcp-129-111-247-96:0 adcmobile$ genesis-g3
Welcome to the GENESIS 3 shell
genesis > ndf_load cells/edsjb1994_partitioned.ndf
Could not find file (number 1, 1), path name (cells/edsjb1994_partitioned.ndf)
Set one of the environment variables NEUROSPACES_NMC_USER_MODELS,
NEUROSPACES_NMC_PROJECT_MODELS, NEUROSPACES_NMC_SYSTEM_MODELS or NEUROSPACES_NMC_MODELS
to point to a library where the required model is located,
or use the -m switch to configure where neurospaces looks for models.

genesis > ndf_load cells/purkinje/edsjb1994_partitioned.ndf
genesis-g3: Parse of cells/purkinje/edsjb1994_partitioned.ndf failed with 1 (cumulative) error.
genesis > list_elements
---
- /Purkinje
genesis > delete /Purkinje
genesis > list_elements
---

genesis > ndf_load cells/purkinje/edsjb1994_partitioned.ndf
genesis-g3: Parse of cells/purkinje/edsjb1994_partitioned.ndf failed with 1 (cumulative) error.
genesis > exit
dhcp-129-111-247-96:0 adcmobile$ genesis-g3
Welcome to the GENESIS 3 shell

genesis > ndf_load cells/purkinje/edsjb1994_partitioned.ndf
genesis >
\end{verbatim}


\subsection{gshell list\_elements}

The gshell command 'list\_elements' sometimes loses part of the
output.  In the following snippet parts of the 'km' and 'kh' channels
have disappeared:


\begin{verbatim}
> genesis-g3 --execute "ndf_load cells/purkinje/edsjb1994.ndf"
Welcome to the GENESIS 3 shell
genesis > list_elements /Purkinje/segments/soma/**
---
- /Purkinje/segments/soma/km/km
- /Purkinje/segments/soma/kdr/kdr_steadystate
- /Purkinje/segments/soma/kdr/kdr_tau
- /Purkinje/segments/soma/ka/ka_gate_activation
- /Purkinje/segments/soma/ka/ka_gate_inactivation
- /Purkinje/segments/soma/kh/kh
- /Purkinje/segments/soma/nap/nap
- /Purkinje/segments/soma/naf/naf_gate_activation
- /Purkinje/segments/soma/naf/naf_gate_inactivation
- /Purkinje/segments/soma/cat/cat_gate_activation
- /Purkinje/segments/soma/cat/cat_gate_inactivation
\end{verbatim}

This is obviously wrong behavior.  The querymachine command 'expand'
can be used to compare with the correct output.

\begin{verbatim}
genesis > querymachine expand /Purkinje/segments/soma/**
\end{verbatim}


\subsection{documentation synchronization}

When the internet is down, the userdocs-sync script continues its
execution after a synchronization failure.


\subsection{userdocs-check}

Integrate userdocs-check with userdocs-sync.


\subsection{Split I/O Objects from the Heccer package}

The Heccer package currently contains the Heccer software component
and various Input and Output Simulation objects.  It is better to have
an I/O package separate from the Heccer package.


\subsection{Perl Inline code directory}

Create a consolidated directory for inline perl code so that Perl does not 
constantly create "\_Inline" directories in every working directory
that ssp and the gshell are called in. 

\subsection{Descriptor file tags recognized by {\it neurospaces\_cron} (2)}

The following tag should be recognized by the documentation build
system (cf. draft, published):
\begin{itemize}
    \item[]{\bf - local} Private (local) documentation publication.
\end{itemize}

The {\bf local} tag is currently implemented by invoking the following
command from a UNIX shell:

\begin{verbatim}
  $ userdocs-build --tags local
\end{verbatim}

This command builds a local website of all the documents that have the
tag 'local'.

\begin{verbatim}
  $ userdocs-build --tags local --tags published
\end{verbatim}

This command builds a local website with all the published and the
local documents.

\subsection{Implement more consistent G-Shell command names}

Assigned to Hugo (G-Shell) and Mando (G-Tube).

\begin{verbatim}
   runtime_parameter_add <name>
   runtime_parameter_show <name>
   runtime_parameter_delete <name>

   input_add <name>
   input_show <name>
   input_delete <name>

   inputclass_add <name>
   inputclass_show <name>
   inputclass_delete <name>
\end{verbatim}

Also doing similar commands for inputclass\_templates.


\subsection{Bibliographies}

Publication documentation does not currently support bibtex references.
e.g.

\href{http://www.genesis-sim.org/userdocs/pub-purkinje-deschutter1-conductance1-ka1/pub-purkinje-deschutter1-conductance1-ka1.html}{\bf Ka channel}.

References appear as {\bf [?]} in text and the {\bf References} section is missing.

This probably requires the following sequence to be implemented in {\it userdocs-build}:
\begin{verbatim}
   latex bibtex latex latex
\end{verbatim}


\subsection{bibliographies}

Documentation that use citations are currently not supported by the
documentation system.  Such documents are built differently on
different machines.

Seems at present nobody knows what the original problem was, but this
description being in the done list means that the problem has been
solved.


\subsection{Fix trash cleanup in automated tester.} (DONE)

The automated tester (neurospaces\_cron) currently has a problem with
not deleting all of the files it generates in /tmp. Over time the
``files'' grow and consume large amounts of disk space. The trash file
removal must be fixed so that it keeps from contaminating the file
system further. It must also be improved by making the trash list an
configurable input file that can be passed as an argument to the
neurospaces\_cron script.

{
  \vspace{5mm}
  \centering
  \begin{tabular}{|c|c|}
    \hline
    Notes
    & Won't take wildcards for file deleting. \\
    \hline
    Time Estimate
    & 1 day \\
    \hline
    Assigned to
    & Mando \\
    \hline
  \end{tabular}
}


\subsection{ns-sli .simrc}

The backward-compatibility module expects the .simrc file in the users
home directory.  The default place of this file should be a place in
{\it /usr/local}. (DONE)

This file may also be renamed to {\it .ns-sli}.

{
  \vspace{5mm}
  \centering
  \begin{tabular}{|c|c|}
    \hline
    Notes
    & Required for alpha1 Release 3.0 \\
    \hline
    Time Estimate
    & 1 day \\
    \hline
    Assigned to
    & Hugo \\
    \hline
  \end{tabular}
}


\subsection{Descriptor file tags recognized by {\it neurospaces\_cron} (1)}

Extend the number of file formats for documentation that are
recognized by the documentation build system when a document is placed
in a document folder, as for example, currently happens for PDF
formated documents. The following tags should be recognized for the
alpha release.
\begin{itemize}
   \item[]{\bf - html} Hypertext markup language. (DONE)
    \item[]{\bf - png} Portable network graphics. (DONE)
    \item[]{\bf - ps} Postscript. (DONE)
\end{itemize}
This means that tags given above, should also be placed in the
descriptor.yml file and recognized.

\subsection{Unify the Methods of Handling of Inputs to a Model}

There are several fundamentally different ways to specify inputs to a
simulated model from the {\bf G-Shell}.  For example, a voltage clamp
protocol is specified by creating input objects (pclamp object).  A
current injection and endogeneous stimulation is specified by setting
parameters.


\subsection{GUI: Other}
\begin{itemize}
\item Finish yaml loader to load the clickable graphic complete with
  roll overs dynamically (DONE).
\item Have the GUI use a `project' file which will track various
  aspects of a users project. (DONE)
\item Get {\it compileall} script to work and make python byte code for
  various distros (DONE).
\item Have GUI track Workflow state and guide user accordingly. (DONE)
\end{itemize}

{
  \vspace{5mm}
  \centering
  \begin{tabular}{|c|c|}
    \hline
    Notes
    & Required for alpha1 Release 3.0 \\
    \hline
    Time Estimate
    & 1 week \\
    \hline
    Assigned to
    & Mando \\
    \hline
  \end{tabular}
}


\subsection{GUI: Output}
\begin{itemize}
\item Plot: Get plot to take parameters from the simulation to fill in
  the legend, X and Y axis as well as the title of a graph. - (DONE)
\end{itemize}

{
  \vspace{5mm}
  \centering
  \begin{tabular}{|c|c|}
    \hline
    Notes
    & Required for alpha1 Release 3.0 \\
    \hline
    Time Estimate
    & 1 - 2 weeks \\
    \hline
    Assigned to
    & Mando \\
    \hline
  \end{tabular}
}


\subsection{GUI: Run Simulation (1)}
\begin{itemize}
\item CHECK: Appears to be working. (DONE)
\item RUN: Appears to be working. (DONE)      
\item RESET: Get reset functionality to work. (DONE)
\end{itemize}


\subsection{GUI: Design Experiment}
\begin{itemize}
\item Choose Inputs: create a menu of selectable inputs from the
  loaded model. (DONE).
\item Choose Outputs: create a menu of selectable outputs from the
  loaded model (DONE).
\end{itemize}

{
  \vspace{5mm}
  \centering
  \begin{tabular}{|c|c|}
    \hline
    Notes
    & Required for alpha1 Release 3.0 \\
    \hline
    Time Estimate
    & 1 - 2 days for integration \\
    \hline
    Assigned to
    & Mando \\
    \hline
  \end{tabular}
}


\subsection{Find out about Dependencies}

Write a document that briefly describes the dependencies.  These
dependencies are known:
\begin{itemize}
\item Perl dependencies:
  \begin{itemize}
  \item YAML
  \item Clone
  \item Data::Utilities
  \end{itemize}
\item GUI dependencies:
  \begin{itemize}
  \item wxWidgets
  \item pyYAML
  \end{itemize}
\end{itemize}

Compile other dependencies from the NS wiki.

{
  \vspace{5mm}
  \centering
  \begin{tabular}{|c|c|}
    \hline
    Notes
    & Required for alpha1 Release 3.0 \\
    \hline
    Time Estimate
    & 1 day \\
    \hline
    Assigned to
    & Mando and Hugo \\
    \hline
  \end{tabular}
}


\subsection{G-Shell enhancements/bugs on Mac}

\begin{enumerate}
   \item 
   \begin{verbatim}
      genesis > help ndf_save
      ---
      *** Error: no help for topic ndf_save yet
      genesis > ndf_save mypc.ndf
      Bus error
      dhcp-129-111-247-96:0 adcmobile$ genesis-g3
      cat: neurospaces/config.h: No such file or directory
      Welcome to the GENESIS 3 shell
      genesis > ndf_load cells/purkinje/edsjb1994.ndf
      genesis > ndf_save
      Bus error
      dhcp-129-111-247-96:0 adcmobile$
   \end{verbatim}
   
   \item After a {\tt ce} command there should be no requirement for the absolute path argument for commands such as {\it set\_model\_parameter} and {\it show\_parameter}, etc.
\begin{verbatim}
genesis > ndf_load cells/purkinje/edsjb1994.ndf
genesis > pwe
/
genesis > show_parameter /Purkinje/segments/b3s45[10] RM
value = 3
genesis > ce /Purkinje/segments/b3s45[10]  
genesis > show_parameter RM
symbol not found      
genesis > show_parameter /Purkinje/segments/b3s45[10] RM
value = 3
\end{verbatim} 

     The use of the command {\it show\_parameter} above is invalid.
     Use of the current element as the default element for the GENESIS
     2 {\it setfield} command interferes with future extensibility of
     this command and will not be implemented.

     This can be easily worked around using the '.' to refer to the
     current working element.  The following snippet illustrates this
     use:
\begin{verbatim}
genesis > ndf_load cells/purkinje/edsjb1994.ndf
genesis > ce /Purkinje/segments/
genesis > pwe
/Purkinje/segments
genesis > ce b3s45[10]
genesis > pwe
/Purkinje/segments/b3s45[10]
genesis > show_parameter . RM
value = 3
genesis > ce ../
genesis > pwe
/Purkinje/segments
genesis > show_parameter b3s45[10] RM
value = 3
\end{verbatim}
\end{enumerate}

\subsection{Allow use of token names for model components}

This requires component names to be stringified in the NDF file
format. E.g. cell, neuron, channel, etc.

{
  \vspace{5mm}
  \centering
  \begin{tabular}{|c|c|}
    \hline
    Notes
    & \\
    \hline
    Time Estimate
    & 2 days \\
    \hline
    Assigned to
    & Hugo \\
    \hline
  \end{tabular}
}

\subsection{Current Clamp Protocol}

Make a current clamp protocol available to the gshell.  Based on the
voltage clamp protocol, add an inputclass, set the clamp value and
connect it to a variable.


\subsection{Documentation Summary Tag}

Add support for a summary descriptor attribute that will append
some text to end of a link in the documentation conents page. (DONE)


\subsection{Use neurospaces\_build to serve the source code}

Right now custom scripts are used to serve the GENESIS 3 source code
to the internet.  This leads to duplication of configuration data such
as filenames.  The installer script neurospaces\_build has dedicated
function to serve the source code without configuration data
duplication.  Rather than using custom scripts, the neurospaces\_build
script should be used to serve the source code to the outside world. (DONE)

{
  \vspace{5mm}
  \centering
  \begin{tabular}{|c|c|}
    \hline
    Notes
    & Required for alpha1 Release 3.0 \\
    \hline
    Time Estimate
    & 1 day \\
    \hline
    Assigned to
    & Mando \\
    \hline
  \end{tabular}
}


\subsection{gshell studio}

Disable the studio in the gshell, it should only be loaded on explicit
request (DONE).

{
  \vspace{5mm}
  \centering
  \begin{tabular}{|c|c|}
    \hline
    Notes
    & Required for alpha1 Release 3.0 \\
    \hline
    Time Estimate
    & 4 hours \\
    \hline
    Assigned to
    & Hugo \\
    \hline
  \end{tabular}
}


\subsection{make dist for ns-sli}

Building a tarball of the backward compatibility modules is broken.
This needs to be fixed before a debian package can be built. (DONE)

{
  \vspace{5mm}
  \centering
  \begin{tabular}{|c|c|}
    \hline
    Notes
    & Required for alpha1 Release 3.0 \\
    \hline
    Time Estimate
    & 1 day \\
    \hline
    Assigned to
    & Mando \\
    \hline
  \end{tabular}
}


\subsection{Repository Server}
\label{sec:repository-server}
Monotone sometimes hangs on the repository server.  Upgrade the
repository server (DONE).

{
  \vspace{5mm}
  \centering
  \begin{tabular}{|c|c|}
    \hline
    Time Estimate
    & 1 day \\
    \hline
    Assigned to
    & Mando \\
    \hline
  \end{tabular}
}


\subsection{Automated Regression Tester}

After an update of the model-container, monotone reports that there
are files missing from the local model-container workspace.  This
inhibits further updates such that the tester starts testing old
software.  An example monotone checkin that produces this problem is
the model-container checkin of version
d4cbeb603334d009ba2797976db15f53db5046fe.  This version contains a
white-space only change but corrupts the model-container workspace on
the tester machine.

{
  \vspace{5mm}
  \centering
  \begin{tabular}{|c|c|}
    \hline
    Notes
    & \\
    \hline
    Time Estimate
    & 2 days \\
    \hline
    Assigned to
    & Mando \\
    \hline
  \end{tabular}
}


\subsection{ns-sli showfield}

Implement ns-sli showfield. (DONE)

{
  \vspace{5mm}
  \centering
  \begin{tabular}{|c|c|}
    \hline
    Notes
    & Required for alpha1 Release 3.0 \\
    \hline
    Time Estimate
    & 1 days \\
    \hline
    Assigned to
    & Mando \\
    \hline
  \end{tabular}
}


\subsection{G-Shell}
\begin{itemize}
\item Implement the {\it ssp\_save} and {\it ssp\_load} commands.
\item Complete the link with python.
\end{itemize}

{
  \vspace{5mm}
  \centering
  \begin{tabular}{|c|c|}
    \hline
    Notes
    & Required for alpha1 Release 3.0 \\
    \hline
    Time Estimate
    & 2 days \\
    \hline
    Assigned to
    & Hugo \\
    \hline
  \end{tabular}
}


\subsection{{\it configure.ac} and {\it neurospaces.h}}

The configure script of {\bf Heccer} and {\bf NS-SLI} does not find a valid
{\it neurospaces.h} header even when one is installed in a standard
location.

{
  \vspace{5mm}
  \centering
  \begin{tabular}{|c|c|}
    \hline
    Notes
    & \\
    \hline
    Time Estimate
    & 0.5 days \\
    \hline
    Assigned to
    & Hugo \\
    \hline
  \end{tabular}
}


\subsection{asc\_file object}

The asc\_file object test in ns-sli needs to be completed.

{
  \vspace{5mm}
  \centering
  \begin{tabular}{|c|c|}
    \hline
    Notes
    & \\
    \hline
    Time Estimate
    & 1 day \\
    \hline
    Assigned to
    & Hugo \\
    \hline
  \end{tabular}
}


\subsection{Installation documentation}

Consolidate and correct the installation documentation for software
developers.

{
  \vspace{5mm}
  \centering
  \begin{tabular}{|c|c|}
    \hline
    Notes
    & Required for alpha1 Release 3.0 \\
    \hline
    Time Estimate
    & 3 days, includes installation and uninstall of the software on a laptop \\
    \hline
    Assigned to
    & Allan \\
    \hline
  \end{tabular}
}


\subsection{Repository Upgrade}

Upgrade the source code repository (also test machine) to the latest
stable Debian version.  This was a duplicate
of~\ref{sec:repository-server}.


\subsection{Monotone Clone Operation Integration with the Installer Scripts}
Integrate `mtn clone' into the installer scripts (DONE).

\subsection{Convert Developer Documentation to Doxygen}

\begin{itemize}
\item Finish {\it mcad2doxy} convertor (DONE).
\item Convert and Check Output (DONE).
\item Configure and Makefile Integration (DONE).
\end{itemize}

\subsection{Backwards Compatibility for Single Neurons}

\begin{itemize}
\item Ca pools (DONE).
\item {\it ascii\_out} (DONE).
\item {\it readcell} (DONE).
\item {\it hsolve} (DONE).
\end{itemize}


\subsection{Webcheck Output}

Correct the errors reported by webcheck. (DONE)

{
  \vspace{5mm}
  \centering
  \begin{tabular}{|c|c|}
    \hline
    Notes
    & Required for alpha1 Release 3.0 \\
    \hline
    Time Estimate
    & \\
    \hline
    Assigned to
    & Allan \\
    \hline
  \end{tabular}
}

\subsection{Webcheck}

Go through the Webcheck output and identify critical and easy to fix
problems. (DONE)

{
  \vspace{5mm}
  \centering
  \begin{tabular}{|c|c|}
    \hline
    Notes
    & Required for alpha1 Release 3.0 \\
    \hline
    Time Estimate
    & 1 day \\
    \hline
    Assigned to
    & Allan \\
    \hline
  \end{tabular}
}


\end{document}


