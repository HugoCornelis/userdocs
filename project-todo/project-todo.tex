\documentclass[12pt]{article}
\usepackage[dvips]{epsfig}
\usepackage{color}
%e.g.  \textcolor{red,green,blue}{text}
\usepackage{url}
\usepackage[colorlinks=true]{hyperref}
\usepackage{scrtime}
 
\begin{document}

{\bf Related Documentation:}
% start: userdocs-tag-replace-items related-do-nothing
% end: userdocs-tag-replace-items related-do-nothing

\section*{GENESIS: Documentation}

\section{The TODO and the DONE List}

This document contains a list of functions and descriptions that need
to be implemented or incorporated in the software or software
infrastructure.  The level of detail of the descriptions can vary
between items.  After an item is described, it can be followed by a
checklist with information of the level of implementation.  For
example, subitems in the checklist can be marked with the word `DONE'.

This document can grow when people add new functions that await
implementation, and it can be consolidated by moving items that have
been completed from the TODO list to the DONE list.
%  Some functions
%are related to maintenance and they never disappear (such as Packaging
%and Distribution).  They can be marked with 'WORKING'.

This TODO list was built on \today, \thistime.


\section{TODO List}

\subsection{Automated Regression Tester}

After an update of the model-container, monotone reports that there
are files missing from the local model-container workspace.  This
inhibits further updates such that the tester starts testing old
software.  An example monotone checkin that produces this problem is
the model-container checkin of version
d4cbeb603334d009ba2797976db15f53db5046fe.  This version contains a
white-space only change but corrupts the model-container workspace on
the tester machine.

{
  \vspace{5mm}
  \centering
  \begin{tabular}{|c|c|}
    \hline
    Notes
    & \\
    \hline
    Time Estimate
    & 2 days \\
    \hline
    Assigned to
    & Mando \\
    \hline
  \end{tabular}
}


\subsection{ns-sli showfield}

Implement ns-sli showfield.

{
  \vspace{5mm}
  \centering
  \begin{tabular}{|c|c|}
    \hline
    Notes
    & Required for alpha1 Release 3.0 \\
    \hline
    Time Estimate
    & 1 days \\
    \hline
    Assigned to
    & Mando \\
    \hline
  \end{tabular}
}


\subsection{symmetric compartments}

Implement symmetric compartments.  We will have to decide how to do
this: core implementation, or converting each symmetric compartment to
two asymmetric compartments.  This will also likely involve core
design decisions, because the model-container does not distinguish
between symmetric and asymmetric compartments (only the solvers do).

{
  \vspace{5mm}
  \centering
  \begin{tabular}{|c|c|}
    \hline
    Notes
    & Required for alpha1 Release 3.0 \\
    \hline
    Time Estimate
    & \\
    \hline
    Assigned to
    & Hugo \& Mando \\
    \hline
  \end{tabular}
}


\subsection{SSP Optimizer}

The SSP optimizer use the perl Inline module to inline C code with
perl code.  The Inline module creates a directory hierarchy inside the
current directory to do its work.  The Inline module must be
configured to do this in a standard directory, where all users share
the result.

The installation of SSP must run SSP to do the compilation of the
optimizer, and install it afterwards.  This is to avoid race
conditions between simultaneously running ssp processes.

{
  \vspace{5mm}
  \centering
  \begin{tabular}{|c|c|}
    \hline
    Notes
    & Required for alpha1 Release 3.0 \\
    \hline
    Time Estimate
    & 2 days \\
    \hline
    Assigned to
    & Hugo \\
    \hline
  \end{tabular}
}


\subsection{Daily Releases and Release Labels}

The installer scripts can update the release labels in the source code
automatically, and check them in into the source code repositories.
The checkin should be done on a branch 'dist' to prevent contamination
of the default '0' branch with versions that reconfigure the software
with new release labels.

The checkin should be made visible from the repository server (ie. the
tester should push its changes from its repository to the server
repository).

{
  \vspace{5mm}
  \centering
  \begin{tabular}{|c|c|}
    \hline
    Notes
    & \\
    \hline
    Time Estimate
    & 3 days \\
    \hline
    Assigned to
    & Hugo \\
    \hline
  \end{tabular}
}


\subsection{ns-sli .simrc}

The backward-compatibility module expects the .simrc file in the users
home directory.  The default place of this file should be a place in
{\it /usr/local}.

This file may also be renamed to {\it .ns-sli}.

{
  \vspace{5mm}
  \centering
  \begin{tabular}{|c|c|}
    \hline
    Notes
    & Required for alpha1 Release 3.0 \\
    \hline
    Time Estimate
    & 1 day \\
    \hline
    Assigned to
    & Hugo \\
    \hline
  \end{tabular}
}


\subsection{gshell python}

Document the python dependencies of the gshell.

{
  \vspace{5mm}
  \centering
  \begin{tabular}{|c|c|}
    \hline
    Notes
    & Required for alpha1 Release 3.0 \\
    \hline
    Time Estimate
    & 1 day \\
    \hline
    Assigned to
    & \\
    \hline
  \end{tabular}
}


\subsection{gshell studio}

Disable the studio in the gshell, it should only be loaded on explicit
request (DONE).

{
  \vspace{5mm}
  \centering
  \begin{tabular}{|c|c|}
    \hline
    Notes
    & Required for alpha1 Release 3.0 \\
    \hline
    Time Estimate
    & 4 hours \\
    \hline
    Assigned to
    & Hugo \\
    \hline
  \end{tabular}
}


\subsection{make dist for ns-sli}

Building a tarball of the backward compatibility modules is broken.
This needs to be fixed before a debian package can be built.

{
  \vspace{5mm}
  \centering
  \begin{tabular}{|c|c|}
    \hline
    Notes
    & Required for alpha1 Release 3.0 \\
    \hline
    Time Estimate
    & 1 day \\
    \hline
    Assigned to
    & Mando \\
    \hline
  \end{tabular}
}


\subsection{Repository Server}
\label{sec:repository-server}
Monotone sometimes hangs on the repository server.  Upgrade the
repository server (DONE).

{
  \vspace{5mm}
  \centering
  \begin{tabular}{|c|c|}
    \hline
    Time Estimate
    & 1 day \\
    \hline
    Assigned to
    & Mando \\
    \hline
  \end{tabular}
}


\subsection{Webcheck}

Go through the Webcheck output and identify critical and easy to fix
problems.

{
  \vspace{5mm}
  \centering
  \begin{tabular}{|c|c|}
    \hline
    Notes
    & Required for alpha1 Release 3.0 \\
    \hline
    Time Estimate
    & 1 day \\
    \hline
    Assigned to
    & Allan \\
    \hline
  \end{tabular}
}


\subsection{Use neurospaces\_build to serve the source code}

Right now custom scripts are used to serve the GENESIS 3 source code
to the internet.  This leads to duplication of configuration data such
as filenames.  The installer script neurospaces\_build has dedicated
function to serve the source code without configuration data
duplication.  Rather than using custom scripts, the neurospaces\_build
script should be used to serve the source code to the outside world.

{
  \vspace{5mm}
  \centering
  \begin{tabular}{|c|c|}
    \hline
    Notes
    & Required for alpha1 Release 3.0 \\
    \hline
    Time Estimate
    & 1 day \\
    \hline
    Assigned to
    & Mando \\
    \hline
  \end{tabular}
}





\subsection{Fix trash cleanup in automated tester.}

The automated tester (neurospaces\_cron) currently has a problem with 
not deleting all of the files it generates in /tmp. Over time the 
``files'' grow and consume large amounts of disk space. The trash file 
removal must be fixed so that it keeps from contaminating the file system
further. It must also be improved by making the trash list an configurable 
input file that can be passed as an argument to the neurospaces\_cron script.

{
  \vspace{5mm}
  \centering
  \begin{tabular}{|c|c|}
    \hline
    Notes
    & Won't take wildcards for file deleting. \\
    \hline
    Time Estimate
    & 1 day \\
    \hline
    Assigned to
    & Mando \\
    \hline
  \end{tabular}
}




\subsection{File permission errors}

Since GENESIS 3 is to be installed on machines for multiple users
a serious bug must be addressed. Certain programs output to a hardcoded
filename, such as /tmp/output. When one user starts up a simulation which 
writes to such a file, it is given permissions for thier user id. If a 
second user tries to start up a similar simulation, which writes to the 
same filename, the simulation will fail and GENESIS will bail out with an
error (as was displayed by the tester). This bottlenecks the number of 
users that can be using GENESIS 3 at the same time on the same machine.


{
  \vspace{5mm}
  \centering
  \begin{tabular}{|c|c|}
    \hline
    Notes
    &  \\
    \hline
    Time Estimate
    & ??? \\
    \hline
    Assigned to
    & ??? \\
    \hline
  \end{tabular}
}



\subsection{Packaging and Distribution (for the alpha1 Release)}

Packaging involves the creation of tarballs with source code to the
creation of binary packages for different OSs.

For packaging and distribution of the alpha release of GENESIS\,3.0,
each software component must implement the following makefile targets:

\begin{itemize}
\item {\it dist}: for creation of the tarball with source code.
  Create an installer script that builds the tarball of each package
  (neurospaces\_dist) (DONE).
\item {\it pkg-deb}: for creation of a binary archive for debian based
  Linux distributions (including Ubuntu). - (IN PROGRESS)
\item {\it pkg-rpm}: for creation of a binary archive for rpm based Linux
  distributions (such as Redhat and Fedora). - (IN PROGRESS)
\item MAC .dmg file. - (IN PROGRESS)
\end{itemize}

An additional package must be created that has all other software
components of GENESIS\,3 as dependencies.  An installer script must
be created that builds all the OS specific archives and puts them in a
common directory (neurospaces\_dist\_archives).

Communicate with the previous debian maintainers about this.  They are
Sam Hocevar (Debian packages) sam+deb at zoy.org and Matt Zimmerman
mdz at csh.rit.edu

{
  \vspace{5mm}
  \centering
  \begin{tabular}{|c|c|}
    \hline
    Notes
    & Required for alpha1 Release 3.0 \\
    \hline
    Time Estimate
    & 4 days \\
    \hline
    Assigned to
    & Mando \\
    \hline
  \end{tabular}
}



\subsection{QueryHandlerPrintSpikeRecieverCount() on MAC OSX}

There is an issue with executing {\bf ./neurospacesparse -v 1 -q
  legacy/networks/supernetwork.ndf} on mac osx. The system stalls on a
call to the function QueryHandlerPrintSpikeRecieverCount() with no CPU
activity.

{
  \vspace{5mm}
  \centering
  \begin{tabular}{|c|c|}
    \hline
    Notes
    & Required for alpha1 Release 3.0 \\
    \hline
    Time Estimate
    & 1 day for a detailed diagnosis \\
    \hline
    Assigned to
    & Mando \\
    \hline
  \end{tabular}
}



\subsection{Benchmarking}

Test the performance of the Purkinje cell in each of the following
environments:

\begin{itemize}
\item GENESIS\,2 (old GENESIS\,2 enviroment).
\item NS-SLI (GENESIS\,3--GENESIS\,2 backward compatibility).
\item SSP (GENESIS\,3 batch simulation environment).
\item G-Shell (GENESIS\,3 interactive simulation environment).
\end{itemize}

{
  \vspace{5mm}
  \centering
  \begin{tabular}{|c|c|}
    \hline
    Notes
    & Required for alpha1 Release 3.0 \\
    \hline
    Time Estimate
    & 1 day \\
    \hline
    Assigned to
    & Hugo \\
    \hline
  \end{tabular}
}


\subsection{Report Document for Testers}

Construct a report document with instructions for alpha and beta
testers.  The instructions consist of installation instructions and a
questionnaire about basic application functionality.

Questionnaire must be put on the website.


{
  \vspace{5mm}
  \centering
  \begin{tabular}{|c|c|}
    \hline
    Notes
    & Required for alpha1 Release 3.0 \\
    \hline
    Time Estimate
    & 3 days \\
    \hline
    Assigned to
    &  Allan and Dave? \\
    \hline
  \end{tabular}
}


\subsection{Webcheck Output}

Correct the errors reported by webcheck.

{
  \vspace{5mm}
  \centering
  \begin{tabular}{|c|c|}
    \hline
    Notes
    & Required for alpha1 Release 3.0 \\
    \hline
    Time Estimate
    & \\
    \hline
    Assigned to
    & Allan and Dave? \\
    \hline
  \end{tabular}
}


\subsection{Demonstration Scripts}

The following scripts have been selected for use as demonstration
scripts:

\begin{itemize}
\item Purkinje Cell Scripts from Erik De Schutter: some of the scripts
  are used for regression tests (and work correctly).  Other Purkinje
  cell scripts still need validation.
\item Rallpacks: for benchmarking it would be useful to implement
  scripts for the rallpack standard.  To show the relationships
  between the different GENESIS\,3 interfaces, it is useful to
  implement these scripts in the different interfaces currently
  available (Perl, Python, NS-SLI, NDF, Heccer, G-Shell).

  For visibility and good integration with the rest of the system, it
  would be best to integrate these scripts with the regression tester,
  such that they are validated and documented on a daily basis.
  \begin{itemize}
  \item Rallpack 1: Passive unbranched cable.
  \item Rallpack 2: Branched cable.
  \item Rallpack 3: Squid axon/Active cable.
  \item Integrate with Tests.
  \end{itemize}
\end{itemize}

{
  \vspace{5mm}
  \centering
  \begin{tabular}{|c|c|}
    \hline
    Notes
    & Required for alpha1 Release 3.0 \\
    \hline
    Time Estimate
    & 2 days per scripting language \\
    \hline
    Assigned to
    & Allan and Dave? \\
    \hline
  \end{tabular}
}


\subsection{Release Notes}

\begin{itemize}
\item See also \href{../release-notes/release-notes.tex}{release-notes.tex}
\item Document that describes the release.
  \begin{itemize}
  \item Compile a list of G-3 capabilities, compile them from each of the components.
  \item Statements about:
    \begin{itemize}
    \item Backward compatibility.
      \begin{itemize}
      \item Clarify the level of backward compatibility. 
      \item Emphasize on commitment of backward compatibility.
      \item Support of conversions from G2 to G3 for the next two years.
      \item explain policies for feature requests.
      \item Rallpacks.
      \item Does the purkinje cell and its various stimulation protocols.
      \item please try your simulations and report problems
      \end{itemize}
    \item NeuroML support.
    \item PyNN support, placeholder for an interface to PyNN through the
      {\it pynn\_load} command
    \end{itemize}
  \item Installation.
    \begin{itemize}
    \item Compile a list of dependencies.
    \item Create an integrated installation tar ball or so.
    \item add deb and rpm packages.
      % (http://www.artificialworlds.net/blog/2007/02/22/creating-deb-and-rpm-packages/).
    \item Release plan: alpha, beta, full release.
    \item Synchronize userdocs and wiki installation docs.
    \end{itemize}
  \item Mailing lists, discussion boards and other genesis related systems.
    \begin{itemize}
    \item What are they used for?
    \item What do we want them to be used for? Why?
    \item Add {\ hg} repository.
    \item Add bulletin boards and mailing lists.
    \item Add NS related systems.
      \begin{itemize}
      \item www.neurospaces.org
      \item Blog.
      \end{itemize}
    \end{itemize}
  \item alpha: Mid-October.
  \item beta1: Mid-January.
  \item beta2: Mid-March.
  \item full: Mid-July.
  \end{itemize}
\end{itemize}

{
  \vspace{5mm}
  \centering
  \begin{tabular}{|c|c|}
    \hline
    Notes
    & Required for alpha1 Release 3.0 \\
    \hline
    Time Estimate
    & 2 days \\
    \hline
    Assigned to
    & Allan and Dave \\
    \hline
  \end{tabular}
}


\subsection{Find out about Dependencies}

Write a document that briefly describes the dependencies.  These
dependencies are known:
\begin{itemize}
\item Perl dependencies:
  \begin{itemize}
  \item YAML
  \item Clone
  \item Data::Utilities
  \end{itemize}
\item GUI dependencies:
  \begin{itemize}
  \item wxWidgets
  \item pyYAML
  \end{itemize}
\end{itemize}

Compile other dependencies from the NS wiki.

{
  \vspace{5mm}
  \centering
  \begin{tabular}{|c|c|}
    \hline
    Notes
    & Required for alpha1 Release 3.0 \\
    \hline
    Time Estimate
    & 1 day \\
    \hline
    Assigned to
    & Mando and Hugo \\
    \hline
  \end{tabular}
}


\subsection{Add Network Modeling Support}

Because this is an important function, it requires substantial
development on one software component, and also impacts other software
components.

\begin{itemize}
\item DES
  \begin{itemize}
  \item instantiation of DES queuer stand-alone (DONE)
  \item instantiation of DES distributor stand-alone (DONE)
  \item tests of DES stand-alone (DONE)
  \end{itemize}
\item model-container
  \begin{itemize}
  \item instantiation of projections/connections by model-container (DONE)
  \item inspection of projections/connections by model-container API (DONE)
  \end{itemize}
\item SSP
  \begin{itemize}
  \item instantiation of DES queuer from SSP (partial)
  \item instantiation of DES distributor from SSP (partial)
  \item translation of model-container connections to DES connections (partial)
  \item tests for SSP
  \end{itemize}
\item G-Shell
  \begin{itemize}
  \item add integration commands to G-Shell
  \item tests for G-Shell
  \end{itemize}
\item backward compatibility
  \begin{itemize}
  \item integration of planar/volume connect command
  \item tests for backward compatibility
  \end{itemize}
\end{itemize}


{
  \vspace{5mm}
  \centering
  \begin{tabular}{|c|c|}
    \hline
    Notes
    & Is this required for alpha1 Release 3.0 ?? \\
    \hline
    Time Estimate
    & 2 weeks \\
    \hline
    Assigned to
    & Hugo \\
    \hline
  \end{tabular}
}


\subsection{GUI: Construct Model}

\begin{itemize}
\item Import from External Library:
  \begin{itemize}
  \item Displaying NDF library is working. (DONE)
  \item Get the GUI read from external libraries on the net.
  \item Add support for different file types like SWC.
  \end{itemize}
\item Load Tutorial: Display a list of clickable tutorial links and
  have the GUI open them in a web browser of html widget.
\item Load Project: Get GUI to load a project based on a yaml file
  which includes all relevant input and result files.
\item Create Model: Create interface to allow a user to create a
  simple model via text and/or graphics.
\item Explore Model: Create a graphics widget which displays the
  morphology when clicking 'explore model'.
\item Save Model: Get save model to work properly (save a project file
  as well as all included files).
\end{itemize}

{
  \vspace{5mm}
  \centering
  \begin{tabular}{|c|c|}
    \hline
    Notes
    & Required for alpha1 Release 3.0 \\
    \hline
    Time Estimate
    & 2 - 3 weeks \\
    \hline
    Assigned to
    & Mando \\
    \hline
  \end{tabular}
}


\subsection{GUI: Design Experiment}
\begin{itemize}
\item Choose Inputs: create a menu of selectable inputs from the loaded model. (DONE)
\item Choose Outputs: create a menu of selectable outputs from the loaded model.
\end{itemize}

{
  \vspace{5mm}
  \centering
  \begin{tabular}{|c|c|}
    \hline
    Notes
    & Required for alpha1 Release 3.0 \\
    \hline
    Time Estimate
    & 1 - 2 days for integration \\
    \hline
    Assigned to
    & Mando \\
    \hline
  \end{tabular}
}


\subsection{GUI: Run Simulation}
\begin{itemize}
\item CHECK: Appears to be working. (DONE)
\item RUN: Appears to be working. (DONE)      
\item STOP: Get 'stop' simulation to properly work (not sure how this
  will work since the G-Shell is stop and wait).
\item RESET: Get reset functionality to work. (DONE)
\item SAVE: Save model state. 
\item LOAD: Get GUI to load a previous save state.
\item Runtime Option: create a menu for adding runtime options.
\end{itemize}

{
  \vspace{5mm}
  \centering
  \begin{tabular}{|c|c|}
    \hline
    Notes
    & Required for alpha1 Release 3.0 \\
    \hline
    Time Estimate
    & 2 - 3 days \\
    \hline
    Assigned to
    & Mando \\
    \hline
  \end{tabular}
}


\subsection{GUI: Output}
\begin{itemize}
\item Plot: Get plot to take parameters from the simulation to fill in
  the legend, X and Y axis as well as the title of a graph. - (DONE)
\item Studio: Remove Studio from output menu? Same functionality will
  be in 'Explore Model.'
\item Matlab: Create export menu/functionality for matlab.
\item xmgrace: Create export menu/functionality for xmgrace.
\item Mathematica: Create export menu/functionality for Mathematica.
\end{itemize}

{
  \vspace{5mm}
  \centering
  \begin{tabular}{|c|c|}
    \hline
    Notes
    & Required for alpha1 Release 3.0 \\
    \hline
    Time Estimate
    & 1 - 2 weeks \\
    \hline
    Assigned to
    & Mando \\
    \hline
  \end{tabular}
}


\subsection{GUI: Iterators}
Add menus/functionality to the iterator menu.

Put a couple of iterator scripts of the purkinje cell comparison and
the mutual information projects in a directory.  The Iterators menu
lists all the scripts found in this directory, clicking on one of them
runs the appropriate script.  The script generates SSP configuration
files that can be loaded into the GUI.

{
  \vspace{5mm}
  \centering
  \begin{tabular}{|c|c|}
    \hline
    Notes
    & Required for alpha1 Release 3.0 \\
    \hline
    Time Estimate
    & 1 week \\
    \hline
    Assigned to
    & Mando \\
    \hline
  \end{tabular}
}


\subsection{GUI: Other}
\begin{itemize}
\item Finish yaml loader to load the clickable graphic complete with
  roll overs dynamically.
\item Have the GUI use a `project' file which will track various
  aspects of a users project. (DONE)
\item Get {\it compileall} script to work and make python byte code for
  various distros.
\item Have GUI track Workflow state and guide user accordingly. (DONE)
\item Screenshot functionality for plot and explore model.
\end{itemize}

{
  \vspace{5mm}
  \centering
  \begin{tabular}{|c|c|}
    \hline
    Notes
    & Required for alpha1 Release 3.0 \\
    \hline
    Time Estimate
    & 1 week \\
    \hline
    Assigned to
    & Mando \\
    \hline
  \end{tabular}
}


\subsection{G-Shell}
\begin{itemize}
\item Implement the {\it ssp\_save} and {\it ssp\_load} commands.
\item Complete the link with python.
\end{itemize}

{
  \vspace{5mm}
  \centering
  \begin{tabular}{|c|c|}
    \hline
    Notes
    & Required for alpha1 Release 3.0 \\
    \hline
    Time Estimate
    & 2 days \\
    \hline
    Assigned to
    & Hugo \\
    \hline
  \end{tabular}
}


\subsection{model-container}
Complete the implementation of the {\it ndf\_save} command.  Add tests for
saving a full model, saving a library of models and saving a
modularized part of a model.

{
  \vspace{5mm}
  \centering
  \begin{tabular}{|c|c|}
    \hline
    Notes
    & Required for alpha1 Release 3.0 \\
    \hline
    Time Estimate
    & 3 days \\
    \hline
    Assigned to
    & Hugo \\
    \hline
  \end{tabular}
}


\subsection{Mercurial}

The source code of most GENESIS 3 software components are maintained
using the Monotone version control system.  The installer scripts
integrate with Monotone to provide for automated building and testing.

The G-Tube is maintained using the Mercurial version control system.
We must continue and finish the integration of the Mercurial version
control system and the installer script {\it neurospaces\_build}.


\subsection{Gating Variables with Coupled Equations}

There is currently no way to solve coupled equations of gating
variables.

dm/dt = A * m + B * n
dn/dt = C * m + D * n

In GENESIS 2, this is possible using forward integration methods only.


\subsection{Populate the Model Library}

Build up a library of cells that can be used with G2 or G3.


\subsection{Level 4 Documentation: Technical Specifications}

For each software component a set of (html) documents must be
generated that specify the capabilities of the component.  For Unix
shell scripts, technical specifications can be generated from the
usage messages generated by a `--help' argument.


\subsection{Level 3-7 Documentation: Creation Guide}

An outline of how Level 3-7 documentation is created and included in
the appropriate locations is required.  A level 2 document.


\subsection{Documentation Browsing}

Tag all the documentation with keywords to define flows through the
documentation and ease browsing and reading of related documents.

Should be provided via a browser-based framework to give
`professional' look to documentation. See Documentation Framing
(following).


\subsection{Documentation Framing}

Look and feel of documentation website.

{
  \vspace{5mm}
  \centering
  \begin{tabular}{|c|c|}
    \hline
    Notes
    & Required for beta1 Release 3.0 \\
    \hline
    Time Estimate
    & 3 days \\
    \hline
    Assigned to
    & Mando and Allan \\
    \hline
  \end{tabular}
}



\subsection{Progress Reports to the Supplement}

Every three months a report about progress on the work of the
administrative supplement must be submitted to NIH.

This can be done by picking items from the DONE list related to the
work proposed in the administrative supplement.

First move conceptualized items from the supplement to this TODO list.


\subsection{Unify the Methods of Handling of Inputs to a Model}

There are several fundamentally different ways to specify inputs to a
simulated model from the G-Shell.  For example, a voltage clamp
protocol is specified by creating input objects (pclamp object).  A
current injection and endogeneous stimulation is specified by setting
parameters.


\subsection{Announce GENESIS\,3 alpha1}

Compose an announcement of the GENESIS\,3 alpha1 release to the
connectionist, compneuro and genesis-users email lists.  Signed with
Allan Coop, Hugo Cornelis, Mando Rodriguez, Dave Beeman, James Bower.

{
  \vspace{5mm}
  \centering
  \begin{tabular}{|c|c|}
    \hline
    Notes
    & Required for alpha1 Release 3.0 \\
    \hline
    Time Estimate
    & \\
    \hline
    Assigned to
    & Hugo \\
    \hline
  \end{tabular}
}


\section{DONE List}

\subsection{Installation documentation}

Consolidate and correct the installation documentation for software
developers.

{
  \vspace{5mm}
  \centering
  \begin{tabular}{|c|c|}
    \hline
    Notes
    & Required for alpha1 Release 3.0 \\
    \hline
    Time Estimate
    & 3 days, includes installation and uninstall of the software on a laptop \\
    \hline
    Assigned to
    & Allan \\
    \hline
  \end{tabular}
}


\subsection{Repository Upgrade}

Upgrade the source code reopsitory (also test machine) to the latest
stable Debian version.  This was a duplicate
of~\ref{sec:repository-server}.


\subsection{Monotone Clone Operation Integration with the Installer Scripts}
Integrate `mtn clone' into the installer scripts (DONE).

\subsection{Convert Developer Documentation to Doxygen}

\begin{itemize}
\item Finish {\it mcad2doxy} convertor (DONE).
\item Convert and Check Output (DONE).
\item Configure and Makefile Integration (DONE).
\end{itemize}

\subsection{Backwards Compatibility for Single Neurons}

\begin{itemize}
\item Ca pools (DONE).
\item {\it ascii\_out} (DONE).
\item {\it readcell} (DONE).
\item {\it hsolve} (DONE).
\end{itemize}


\end{document}
 