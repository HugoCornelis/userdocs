\documentclass[12pt]{article}
\usepackage{verbatim}
\usepackage[dvips]{epsfig}
\usepackage{color}
\usepackage{url}
\usepackage[colorlinks=true]{hyperref}

\begin{document}

\section*{GENESIS: Documentation}

{\bf Related Documentation:}
% start: userdocs-tag-replace-items related-do-nothing
% end: userdocs-tag-replace-items related-do-nothing

\section*{De Schutter: Purkinje Cell Model}

\subsection*{NaF SPIKES}

In intracellular recordings in brain slice preparations
some Purkinje cells are silent; others fire spontaneously.
Our model is quiet without stimulation, with a stable
resting potential of -68\,mV. For intrasomatic current injections
below the spiking threshold, no somatic spikes
could be generated regardless of the length of the current
pulses (simulation results not shown). Low-amplitude
current injections in Purkinje cell somata above this threshold
caused firing of fast somatic spikes, with the relatively
high minimum spike frequency (30--40\,Hz) characteristic
of these cells (Figs.\,\href{../pub-purkinje-deschutter1-fig-3/pub-purkinje-deschutter1-fig-3.tex}{\bf 3{\it A}}, \href{../pub-purkinje-deschutter1-fig-4/pub-purkinje-deschutter1-fig-4.tex}{\bf 4{\it A}\,and\,{\it B}}, and \href{../pub-purkinje-deschutter1-fig-6/pub-purkinje-deschutter1-fig-6.tex}{\bf 6{\it A}}). Somatic firing
frequency increased relatively rapidly with current amplitude
(range 22--250\,Hz).

In in vitro intracellular recordings there is often a delay
between the onset of the current injection and the occurrence
of the first somatic spike, which decreases with increasing
amplitude of current. This variable delay was also
present in the simulations (Figs. \href{../pub-purkinje-deschutter1-fig-3/pub-purkinje-deschutter1-fig-3.tex}{\bf 3}, \href{../pub-purkinje-deschutter1-fig-4/pub-purkinje-deschutter1-fig-4.tex}{\bf 4}, \href{../pub-purkinje-deschutter1-fig-5/pub-purkinje-deschutter1-fig-5.tex}{\bf 5}, and \href{../pub-purkinje-deschutter1-fig-6/pub-purkinje-deschutter1-fig-6.tex}{\bf 6{\it B}}). Note also
that in \href{../pub-purkinje-deschutter1-fig-4/pub-purkinje-deschutter1-fig-4.tex}{\bf Fig.\,4{\it D}} we reproduce another phenomenon observed
in slice. When the injected current amplitude became
too high, somatic spiking saturated at a depolarized
level of about -30\,mV. This saturation could be reversed
either spontaneously (at this current amplitude) or by shutting
off the current.

\subsection*{SOMATIC CURRENTS AND THE NaF SPIKE}

In the soma (\href{../pub-purkinje-deschutter1-fig-11/pub-purkinje-deschutter1-fig-11.tex}{\bf Fig.\,11{\it A}}), the main active currents were the fast and persistent
Na$^+$ channels and the delayed rectifier. NaF currents and
Kdr played their classic role in generating the fast action
potentials\,\cite{hodgkin52:_quantitative_description}. Note also that the
amplitude of these currents was quite large compared with
currents in the dendritic compartments; this was caused by
the large size of the soma and the large gin the model. The
other somatic currents did not play a large role in the firing
patterns shown here. KA and CaT currents were transiently
activated when the cell was depolarized from resting membrane
or hyperpolarized potentials, but inactivated to a
large degree during long current steps (data not shown).

\bibliographystyle{plain}
\bibliography{../tex/bib/g3-refs.bib}

\end{document}
