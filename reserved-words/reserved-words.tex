\documentclass[12pt]{article}
\usepackage[dvips]{epsfig}
\usepackage{color}
%e.g.  \textcolor{red,green,blue}{text}
\usepackage{url}
\usepackage[colorlinks=true]{hyperref}

\begin{document}

\section*{GENESIS: Documentation}

{\bf Related Documentation:}
% start: userdocs-tag-replace-items related-do-nothing
% end: userdocs-tag-replace-items related-do-nothing

\section*{Reserved Words}

\begin{itemize}

\item {\bf GENESIS ``Shell'':} An interactive shell provides a GENESIS ``session''. A GENESIS session is evoked by typing ``{\tt genesis-g3}'' at a UNIX prompt. The GENESIS shell provides a unified interface and allows for convenient interaction with other software components of a given GENESIS environment.  Two introductory tutorials (\href{../tutorial1/tutorial1.tex}{Tutorial 1} and \href{../tutorial2/tutorial2.tex}{Tutorial 2}) focus on the use of GENESIS from such an interactive shell.

\item {\bf GENESIS ``Component'':}  A stand-alone software module that generates specific simulator functionality by providing commands to the GENESIS shell. Component functionality is evoked by GENESIS on an as-needed basis.

To generate a list of GENESIS components type ``{\tt list components}'' in the GENESIS shell. To get help for a component type ``{\tt help <component name>}''. The {\it help} option gives an example of how to use a component and a one sentence description of what the component does. GENESIS software components include, the model container which loads and provides internal storage for neuronal models and can be used to perform model analysis, an equation solver (Heccer), and a simple scheduler in Perl (SSP) which coordinates and runs the GENESIS software components required to perform a simulation. We note that specific components recognize specific commands and tokens (see below).

\item {\bf GENESIS ``Command'':} Utility functions that allow operations with or on a model. Most GENESIS commands are provided by the specific components supporting a simulation. To generate a list of GENESIS commands type ``{\tt list commands}'' in the GENESIS shell. To get help for a GENESIS command type ``{\tt help <command name>}''. The {\it help} option gives an example of how to use a command and a one sentence description of what the command does.

\item {\bf GENESIS ``Token'':}  Lexical tokens are used to construct the different parts of a model. There are four sets of tokens:  section, physical, function, and structure. To generate a list of tokens type ``{\tt list tokens <set name>}''. To get help for a GENESIS token, type ``{\tt help <token name>}''. The {\it help} option gives an example of how to use a token and a one sentence description of what the token does. To learn more about the different categories of tokens, see \href{../shell-tokens/shell-tokens.tex}{Shell Tokens}.

\item{\bf GENESIS ``Element'':} This is the building block used to create simulations under GENESIS. They are created with tokens. Once an element has been created, it can be assigned a variety of properties.

\end{itemize}

\end{document}
