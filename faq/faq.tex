\documentclass[12pt]{article}
\usepackage{verbatim}
\usepackage[dvips]{epsfig}
\usepackage{color}
\usepackage{url}
\usepackage[colorlinks=true]{hyperref}

\begin{document}

\section*{GENESIS: Documentation}

{\bf Related Documentation:}
% start: userdocs-tag-replace-items related-do-nothing
% end: userdocs-tag-replace-items related-do-nothing

\section*{Frequently Asked Questions}

This page provides answers to frequently asked questions (FAQ). The answers to particular questions contain links to more detailed documentation. Questions and their answers are added to the lists as they are received by either the genesis-sim-users mailing list or the GENESIS Development Team.

The categories of FAQ are as listed below. They can be extended as needed or requested.

\section*{USER FAQ}

\begin{enumerate}

\item {\bf Why should I subscribe to the GENESIS mailing list?} \\
Although anyone can email a question or reply to the GENESIS mailing list, if you are not registered, you will only receive a reply if the person answering your question remembers to ``reply to all'', something that cannot be guaranteed.

\item {\bf How do I join the genesis-sim-users mailing list?} \\
Instructions can be found at:

\href{https://lists.sourceforge.net/lists/listinfo/genesis-sim-users}{\bf https://lists.sourceforge.net/lists/listinfo/genesis-sim-users}.

\item {\bf How do I email the genesis-sim-users mailing list?} \\
Send an email to:

\href{mailto:genesis-sim-users@lists.sourceforge.net}{\bf genesis-sim-users@lists.sourceforge.net}.

\item{\bf How do I contact the GENESIS Developers Team?} \\
You can contact us directly at:

\href{http://genesis-sim.org/contact}{\bf http://genesis-sim.org/contact}.

\item {\bf How do I generate a FAQ?} \\
The preferred method of generating a FAQ is to email the genesis-sim-users mailing list. As replies are received a FAQ response will be generated.

For further information see \href{../workflow-user-query/workflow-user-query.tex}{\bf Workflow\,User\,Question}.

\item {\bf Who should I be when installing GENESIS on my computer?} \\
GENESIS should not be installed as superuser/root/administrator or by using the {\it sudo} command. However, installation as a regular user will result in a prompt requesting the root password to be entered for each component installed. 

Testing an installation (recommended) as superuser (not recommended) via the command:
\begin{verbatim}
   neurospaces_check >/tmp/check.out 2>&1
\end{verbatim}
generates files that are owned by root. Subsequent use of GENESIS will fail as many files will not be owned by the user. For example, when running \href{../example-script1/example-script1.tex}{\bf Example\,Script\,1} in \href{../tutorial1/tutorial1.tex}{\bf Tutorial\,1}, an error message similar to the following will be generated:
\begin{verbatim}
   genesis > create cell /n
   . . .
   genesis > check /n
   Can't locate object method "new" via package "SSP" at /usr/local/glue/swig/perl/GENESIS3.pm line 1314.
   at /usr/local/bin/genesis-g3 line 40
           main::__ANON__('Can\'t locate object method "new" via package "SSP" at /usr/l...') called at /usr/local/glue/swig/perl/GENESIS3.pm line 1314
           GENESIS3::Commands::run('/n', 0) called at /usr/local/glue/swig/perl/GENESIS3.pm line 224
           GENESIS3::Commands::check('/n') called at (eval 89) line 1
           eval 'GENESIS3::Commands::check( \'/n\', )
   ;' called at /usr/local/bin/genesis-g3 line 146
           main::interprete('check /n') called at /usr/local/bin/genesis-g3 line 217
           main::loop() called at /usr/local/bin/genesis-g3 line 279
           main::main() called at /usr/local/bin/genesis-g3 line 339
   $Result = undef;
 \end{verbatim}
 If you have this problem a quick fix is to:
 \begin{enumerate}
    \item Delete all files in {\it /tmp} owned by root (find owner with ``{\tt ls -l}'') that were produced by {\it neurospaces\_check} with a command such as the following:
    \begin{verbatim}
root /tmp #  rm -Rf neurospaces/ morphology* file* text_* report_* heccer output* purk_test_soma* readcell_reset.txt  traub91_* text_gshell_1.* singlep.ssp 1.* current_schedule a1 a2
    \end{verbatim}
    
    \item If the previous step doesn't fix the problem, remove all {\it \_Inline} directories from your home directory/folder.
    
    \item Keep removing files owned by root in {\it /tmp} and the {\it \_Inline} directories in your home directory, until the problem is resolved.
 \end{enumerate}
 Although GENESIS is a system-wide install, currently only a single account owner or user should uniquely use a single copy of GENESIS on a machine.
\end{enumerate}

{\bf FAQ Acknowledgment:} Alexandre Gravier.

\section*{DEVELOPER FAQ}

\begin{enumerate}

\item {\bf How do I find the version numbers of the GENESIS components I have installed ?} \\
There are several ways to determine the version of the \href{../reserved-words/reserved-words.tex}{\bf GENESIS\,components} that are currently installed on your machine:
\begin{itemize}
\item To determine the version of the {\bf G-Shell}, at a command line prompt enter:

{\tt genesis-g3 --version}

\item The script {\it neurospaces\_versions} gives release labels, but the details
of the release procedure are currently not defined as we are pre-alpha release.  So the usefulness of information returned by this script is limited.

\item The script {\it neurospaces\_status} returns \href{../version-control/version-control.tex}{monotone\,version\,control} numbers.

\item Either {\it neurospaces\_versions} or {\it neurospaces\_status} can be combined with a ``{\tt --regex}'' argument to select
software components.  For example:

\begin{verbatim}
neurospaces_status --regex ns-sli
   /usr/local/bin/neurospaces_build: *** package ns-sli
   /usr/local/bin/neurospaces_build: package ns-sli [mtn ls missing] executing
   /usr/local/bin/neurospaces_build: package ns-sli [mtn ls unknown] executing
   /usr/local/bin/neurospaces_build: package ns-sli [mtn status] executing
   Current branch: 0
   Changes against parent d13a526e58e47dcfb6163dad38c59a2a7d39b723
      patched  tests/scripts/test-simplecell/simplecell-0.g
\end{verbatim}

The fact that no output followed the output lines
\begin{verbatim}
   /usr/local/bin/neurospaces_build: package ns-sli [mtn ls missing] executing
   /usr/local/bin/neurospaces_build: package ns-sli [mtn ls unknown] executing
\end{verbatim}
in response to the monotone commands ``{\tt mtn\,\,ls\,\,missing}'' and ``{\tt mtn\,\,ls\,\,unknown}'', confirms that documentation was neither missing nor unknown.

Output following the line
\begin{verbatim}
   /usr/local/bin/neurospaces_build: package ns-sli [mtn status] executing
\end{verbatim}
in response to the monotone ``{\tt mtn\,\,status}'' command shows that the {\bf NS-SLI} component is under monotone version control in (default) branch 0. The string {\tt d13a5...} is a monotone version identifier that is
unique to the current version of \href{../ns-sli/ns-sli.tex}{\bf NS-SLI}
under development on the machine where the query was made.

\item If you go to the {\bf NS-SLI} source directory and do
\begin{verbatim}
   mtn log | less
\end{verbatim}
you get a detailed cvs-style log of development progress.

\end{itemize}
\end{enumerate}

%\section*{DOCUMENTATION FAQ}
%
%\begin{enumerate}
%
%\item {\bf Why is developer documentation initiated in the wiki and then manually transferred to the GENESIS website ?} \\
%There are several reasons for this procedure. They all result in greater independence for GENESIS developers and users, e.g.
%
%\begin{itemize}
%\item It is easy to integrate with a separate GUI (this is an example of
%interfacing).  The \href{../gtube/gtube.tex}{\bf G-Tube} already has an interface to the
%documentation, such that a customized version of the {\bf G-Tube}, for example a
%user tutorial, has the correct documentation readily available for a
%naive user.
%
%\item It allows the definition of custom reader flows through the
%documents (the `related' headers at the beginning of each document are an
%example of this).
%
%\item The GENESIS Documentation System is a testbed for a distributed model publication system
%
%\item We want to give other groups the freedom to integrate their own
%documentation of those software components that they have integrated
%with GENESIS, without those groups becoming dependent on us (or having
%to feel that they are dependent).
%
%\item We want to allow people to install the documentation on their computer.
%
%\item We want to allow the inclusion of all kinds of media formats
%including math and give the ability to edit those formats in a
%flexible way (e.g. developers work on different computing platforms and use different
%environments to generate and edit documentation).
%
%\item We want to have serious version control (which none of the wikis
%currently provide).
%
%\item We incorporate external webpages that are not under our control
%  such as papers into the documentation.  This works with an http
%  redirect.
%
%\end{itemize}
%
%The GENESIS Documentation System is a custom written wiki, tailored to
%our own specific long-term design goals.  In summary, we want
%developers and documentation writers to be able to develop their
%documentation offline and independently, using tools of their choice,
%and then be able to easily add it to the GENESIS Documentation System
%when it is ready.
%
%See \href{../workflow-documentation/workflow-documentation.tex}{\bf Documentation\,Workflow} for further details.
%
%\end{enumerate}

\section*{BACKWARD COMPATIBILITY FAQ}

\begin{itemize}

\item {\bf  In G-2, field variables such as $V_m$ and $I_m$ that are calculated by the solver are not treated differently to fixed parameters such as $R_m$ and $C_m$ that are set with {\it setfield}.  (In fact you could set $V_m$ with {\it setfield}, but it will be changed at the next timestep, while {\it getfield} can be used on either type of variable.)  Evidently, G-3 does not show $V_m$ with {\it show\_parameter}, or make it available with {\it getfield}.  But, {\it getfield} should work on both kinds of field variables. What is the reason for this?} \\
The {\bf Model\,Container} registers solver information for each model being solved.  This allows the software to retrieve information from the solvers (e.g. {\bf Heccer}) about the model they are solving. Solved variables with common names such as $V_m$ or dedicated names such as $average_{V_m}$, and other information of potential interest to a user, such as what is the maximal time step so far or what is the simulation time, are local to the solver. Due to the separation of a model from its mathematical implementation, this is obviously very different from how G-2 works.

\item {\bf The G-2 {\it create} command creates and initializes all fields of a compartment when it is created.  In the {\bf G-Shell}, the field parameters don't seem to exist until they are initialized. Was that a design decision, and what do you think is best?} \\
Yes, this is a design decision.  The {\bf Model\,Container} allows the definition of
``incomplete'' models.  It is up to other software components to decide if a model is complete or not. In G-3 completeness for a mathematical solver
is quite different from completeness for a morphology analyzer.


\end{itemize}

\end{document}
