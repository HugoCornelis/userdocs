\documentclass[12pt]{article}
\usepackage{verbatim}
\usepackage[dvips]{epsfig}
\usepackage{color}
\usepackage{url}
\usepackage[colorlinks=true]{hyperref}

\begin{document}

\section*{GENESIS: Documentation}

{\bf Related Documentation:}
% start: userdocs-tag-replace-items related-do-nothing
% end: userdocs-tag-replace-items related-do-nothing

\section*{Frequently Asked Questions}

This page provides answers to frequently asked questions (FAQ). The answers to particular questions contain links to more detailed documentation. Questions and their answers are added to the lists as they are received by either the genesis-sim-users mailing list or the GENESIS Development Team.

The categories of FAQ are as listed below. They can be extended as needed or requested.

\section*{USER FAQ}

\begin{enumerate}

\item {\bf Why should I subscribe to the GENESIS mailing list?} \\
Although anyone can email a question or reply to the GENESIS mailing list, if you are not registered, you will only receive a reply if the person answering your question remembers to ``reply to all'', something that cannot be guaranteed.

\item {\bf How do I join the genesis-sim-users mailing list?} \\
Instructions can be found at:

\href{https://lists.sourceforge.net/lists/listinfo/genesis-sim-users}{\bf https://lists.sourceforge.net/lists/listinfo/genesis-sim-users}.

\item {\bf How do I email the genesis-sim-users mailing list?} \\
Send an email to:

\href{mailto:genesis-sim-users@lists.sourceforge.net}{\bf genesis-sim-users@lists.sourceforge.net}.

\item{\bf How do I contact the GENESIS Developers Team?} \\
You can contact us directly at:

\href{http://genesis-sim.org/contact}{\bf http://genesis-sim.org/contact}.

\item {\bf How do I generate a FAQ?} \\
The preferred method of generating a FAQ is to email the genesis-sim-users mailing list. As replies are received a FAQ response will be generated.

\end{enumerate}

\section*{DEVELOPER FAQ}

\begin{enumerate}

\item {\bf How do I find the version numbers of the GENESIS components I have installed ?} \\
There are several ways to determine the version of the \href{../reserved-words/reserved-words.tex}{\bf GENESIS\,components} that are currently installed on your machine:
\begin{itemize}
\item To determine the version of the {\bf G-Shell}, at a command line prompt enter:

{\tt genesis-g3 --version}

\item The script {\it neurospaces\_versions} gives release labels, but the details
of the release procedure are currently not defined as we are pre-alpha release.  So the usefulness of information returned by this script is limited.

\item The script {\it neurospaces\_status} returns \href{../version-control/version/control.tex}{monotone\,version\,control} numbers.

\item Either {\it neurospaces\_versions} or {\it neurospaces\_status} can be combined with a ``{\tt --regex}'' argument to select
software components.  For example:

\begin{verbatim}
neurospaces_status --regex ns-sli
   /usr/local/bin/neurospaces_build: *** package ns-sli
   /usr/local/bin/neurospaces_build: package ns-sli [mtn ls missing] executing
   /usr/local/bin/neurospaces_build: package ns-sli [mtn ls unknown] executing
   /usr/local/bin/neurospaces_build: package ns-sli [mtn status] executing
   Current branch: 0
   Changes against parent d13a526e58e47dcfb6163dad38c59a2a7d39b723
      patched  tests/scripts/test-simplecell/simplecell-0.g
\end{verbatim}

The string {\tt d13a5...} is a monotone version identifier that is unique to the currently installed version of \href{../nssli/nssli.tex}{\bf NS-SLI}.

\item If you go to the {\bf NS-SLI} source directory and do
\begin{verbatim}
   mtn log | less
\end{verbatim}
you get a detailed cvs-style log of progress.

\end{itemize}
\end{enumerate}

\section*{DOCUMENTATION FAQ}

\begin{enumerate}

\item {\bf Why is developer documentation initiated in the wiki and then manually transferred to the GENESIS website ?} \\
There are several reasons for this procedure. They all result in greater independence for GENESIS developers and users, e.g.

\begin{itemize}
\item It is easy to integrate with a separate GUI (this is an example of
interfacing).  The \href{../gtube/gtube.tex}{\bf G-Tube} already has an interface to the
documentation, such that a customized version of the {\bf G-Tube}, for example a
user tutorial, has the correct documentation readily available for a
naive user.

\item It allows the definition of custom reader flows through the
documents (the `related' headers at the beginning of each document are an
example of this).

\item The GENESIS Documentation System is a testbed for a distributed model publication system

\item We want to give other groups the freedom to integrate their own
documentation of those software components that they have integrated
with the G3, without those groups becoming dependent on us (or having
to feel that they are dependent).

\item We want to allow people to install the documentation on their computer.

\item We want to allow the inclusion of all kinds of media formats
including math and give the ability to edit those formats in a
flexible way (e.g. developers work on different computing platforms and use different
environments to generate and edit documentation).

\item We want to have serious version control (which none of the wikis
currently provide).

\item We incorporate webpages in the documentation that are not under our
control such as papers.  This works with an http redirect.

\item From our viewpoint, the GENESIS Documentation System is a custom
written wiki, tailored to our own specific long-term design goals.

\item In summary, we want developers and documentation writers to be able to develop their documentation offline and independently,
using tools of their choice, and then be able to easily add it to the GENESIS Documentation System when it is ready.

\end{itemize}

See \href{../documentation-workflow/documentation-workflow.tex}{\bf Documentation\,Workflow} for further details.

\end{enumerate}

\section*{BACKWARD COMPATIBILITY FAQ}

%\begin{enumerate}

%\item {\bf 

%\end{enumerate}

\end{document}
