\documentclass[12pt]{article}
\usepackage{verbatim}
\usepackage[dvips]{epsfig}
\usepackage{color}
\usepackage{url}
\usepackage[colorlinks=true]{hyperref}

\begin{document}

\section*{GENESIS: Documentation}

{\bf Related Documentation:}
% start: userdocs-tag-replace-items related-do-nothing
% end: userdocs-tag-replace-items related-do-nothing

\section*{De Schutter: Purkinje Cell Model}

\subsection*{VARIABILITY IN PURKINJE CELL RESPONSE PROPERTIES}

In slice preparations, different Purkinje cells often have quite
different firing patterns, especially as concerns the pattern
of interacting somatic and dendritic spikes (Fig.\,11 in \,\cite{R:1980ly}) 
and the slope of the $f-I$ curve.
However, a compartmental model is deterministic, i.e., always
producing the same output for a given input. Variability
in response properties comes about only as a result of
varying parameters in the model.

In the case of the current model, we found that all of the
reported variations in firing patterns could be obtained by
simply changing the density of K$^+$ conductances in the
soma and the main dendrite of the model. Figures\,\href{../pub-purkinje-deschutter1-fig-3/pub-purkinje-deschutter1-fig-3.tex}{\bf\,3} and \href{../pub-purkinje-deschutter1-fig-4/pub-purkinje-deschutter1-fig-4.tex}{\bf\,4}
compare the firing of a model with low K$^+$ conductance
(PM9) with a model with high K$^+$ conductance (PM 10).
Although these models were similar in all responses described
so far, they showed richly different patterns of somatic firing that encompassed most of the observed ones.
Compare, for example, the somatic responses toward the
end of the current injections in Figs.\,\href{../pub-purkinje-deschutter1-fig-3/pub-purkinje-deschutter1-fig-3.tex}{\bf\,3{\it B} and {\it C}}, and \href{../pub-purkinje-deschutter1-fig-4/pub-purkinje-deschutter1-fig-4.tex}{\bf\,4{\it D}}
and after the current step in Fig.\,\href{../pub-purkinje-deschutter1-fig-3/pub-purkinje-deschutter1-fig-3.tex}{\bf\,3{\it D}}. The higher K$^+$ channel
densities in model PM 10 made the soma also less excitable,
resulting in a shallower $f-I$ curve (\href{../pub-purkinje-deschutter1-fig-6/pub-purkinje-deschutter1-fig-6.tex}{\bf Fig. 6{\it A}}).

\subsection*{RESPONSE VARIABILITY}

Although any particular set of parameters
in the model generated an identical (deterministic)
output, we have also found that slight variations in
parameters could produce the kinds of subtle variations
seen in Purkinje cell recordings. Thus different levels of
current injection, small changes in the densities of outward
currents (PM9 versus PM10), or slight changes in morphology
generated subtle changes in model output. Under these
conditions the Purkinje cell model responded generally in
the same way but also showed small variations in the details
of its responses (e.g., in the sequence of action potentials).
We believe this variability in the model is important, because
the specific objective of this effort was to represent the
entire population of Purkinje cells rather than just one individual
cell\,\cite{Bower:1992vn}.

One of the variable aspects of Purkinje cells that has been
previously described involves the details of the $f-I$ curve\,\cite{R:1980ly}. Further, this curve can
change during long recordings of the same cell (D. Jaeger,
personal communication). It is interesting that changes in
the density of the delayed rectifier and noninactivating K$^+$
channels in the model alone could cause a lot of this variability,
as can be seen by comparing \href{../pub-purkinje-deschutter1-fig-3/pub-purkinje-deschutter1-fig-3.tex}{\bf Fig.\,3} with \href{../pub-purkinje-deschutter1-fig-4/pub-purkinje-deschutter1-fig-4.tex}{\bf Fig. 4} and in
the $f-I$ curves of \href{../pub-purkinje-deschutter1-fig-6/pub-purkinje-deschutter1-fig-6.tex}{\bf Fig. 6{\it A}}. The results from patch-clamp recordings
demonstrating that Purkinje cell Kdr channels are
under metabolic control by protein kinase C, which attenuates
Kdr current\,\cite{Linden:1992ys} could provide one
mechanism for short term changes inf-1 relationships. Further,
our finding that somatic action potentials only repolarized
completely when Kdr channels were added to the
main dendrite showed that the Kdr channel also controls
the coupling between the soma and more distal dendrite
that may influence thesef-I relationships. This distribution
of Kdr channels matches data obtained with voltage-sensitive
dye imaging, where a decoupling between the soma and
the more distal dendrite could be removed by blocking K$^+$
channels\,\cite{Knopfel:1990zr}.

\bibliographystyle{plain}
\bibliography{../tex/bib/g3-refs.bib}

\end{document}
