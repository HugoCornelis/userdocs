\documentclass[12pt]{article}
\usepackage{verbatim}
\usepackage[dvips]{epsfig}
\usepackage{color}
\usepackage{url}
\usepackage[colorlinks=true]{hyperref}

\begin{document}

\section*{GENESIS: Documentation}

{\bf Related Documentation:}
% start: userdocs-tag-replace-items related-do-nothing
% end: userdocs-tag-replace-items related-do-nothing

\section*{De Schutter: Purkinje Cell Model}

\subsection*{RESPONSES TO CHANNEL BLOCKERS}

A second measure of
model robustness is its ability to replicate physiological data
in response to standard channel blockers. In this case we
compared model results with\,\cite{R:1980ly} in response to both Na$^+$ and Ca$^{2+}$ channel
blockers while holding all other model parameters fixed.

\href{../pub-purkinje-deschutter1-fig-9/pub-purkinje-deschutter1-fig-9.tex}{\bf Figure 9{\it A}} shows a simulation of TTX block (the conductance
of the NaF and NaP channels was set to 0) during a
1.0\,nA current injection. Note that the somatic recording
showed only attenuated dendritic spikes in the soma, there
were no fast somatic spikes, and that there was a sustained
plateau potential after the end of the current injection.
Subthreshold current injection (0.1\,nA) only evoked the
plateau current. Similar to the results of\,\cite{R:1980ly} (their Fig.\,6), Na$^+$ channel blockers selectively
interfered with somatic spikes while having little effect on
dendritic spiking and plateau potentials. However, the dendritic
spikes in the model were smaller and less sharp than
those in experimental recordings. Also, contrary to experimental
results, the plateau potential did not decay.

From these simulations one can conclude that dendritic
spiking in the model was caused by the slow progressive
depolarization induced by the current injection, not by the
larger depolarizations during fast somatic spikes.

A simulation of Co$^{2+}$ block (the conductance of CaT and
CaP currents was set to 0) is shown in \href{../pub-purkinje-deschutter1-fig-9/pub-purkinje-deschutter1-fig-9.tex}{\bf Fig.\,9{\it B}}. Current
injection caused a slow depolarizing response that generated
fast somatic action potentials. With increasing amplitude
of currents there was less delay before the first action
potential, but the spikes also decreased in amplitude and
saturated at a sustained plateau depolarization of about
-30\,mV. These results are comparable with experimental
recordings, except that higher-amplitude current injections
were necessary in the model to obtain saturation of spiking.
As in the case of\,\cite{R:1980ly} (their Fig.\,8), Ca$^{2+}$ channel blockers selectively interfered with dendritic
spikes while leaving somatic plateau potentials and
somatic spiking intact. However, because the depolarization
could not evoke dendritic Ca$^{2+}$ spikes, there was also
no Ca$^{2+}$-activated hyperpolarization to repolarize the cell
and the model settled in sustained plateaus with high-intensity
current injection. The model thus confirmed the importance
of dendritic Ca$^{2+}$-activated K$^+$ conductances in
repolarizing the somatic depolarizing spike bursts.

\bibliographystyle{plain}
\bibliography{../tex/bib/g3-refs.bib}

\end{document}
