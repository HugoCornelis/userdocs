\documentclass[12pt]{article}
\usepackage[dvips]{epsfig}
\usepackage{url}
\usepackage[colorlinks=true]{hyperref}

\begin{document}

\section*{GENESIS: Documentation}

\subsection*{Introduction}

The GENESIS documentation system is fully automated with versioning under the control of \href{http://monotone.ca/}{monotone}. To learn more about monotone and its use by GENESIS see \href{../document-versionctrl}{Version Control} and to learn more about the levels of GENESIS documentation see \href{../document-overview}{Documentation Overview}. This document assumes that you are familiar with that material.

Here, we introduce some of the flexibility, features and requirements of the GENESIS documentation system. We describe the location of source files within this system and how to create a new document. To learn how to add a new document to the GENESIS version control system see \href{../document-versionctrl}{Version Control}. Once a document is under version control and identified with the correct tags (see below) it can be automatically published to be available for other members of the GENESIS community.

\subsection*{Location of GENESIS Documentation}

The documents that make up the GENESIS documentation system are located in your local workspace, which can be found at the default location
\begin{verbatim}
    ~/neurospaces_project/userdocs/source/snapshots/0
\end{verbatim}
The source materials for individual GENESIS documents can be found in uniquely named folders/directories located in your local workspace. One of the directories in the workspace should be named\,{\it NewDocument}. This directory and the files it contains provide a template for a generic GENESIS document. The files included by default are\,{\it descriptor.yml} and\,{\it NewDocument.tex}, and a sub-directory named\,{\it figures}.

\subsubsection*{The Default YML Descriptor File}

The default contents of the YML file\,{\it descriptor.yml} include:

\begin{verbatim}
---
description: Briefly describe file contents here. 
tags:
  - genesis3
  - level2
  - draft
\end{verbatim}

\begin{itemize}

\item {\tt ---} The three dashes on the first line of the file descriptor.yml are part of the Y Markup Language (YML) syntax for automated software engineering (see \href{http://fdik.org/yml/}{http://fdik.org/yml/}). {\bf Note:} The meaning of the white space/indents in the\,{\it descriptor.yml} file is defined for YML.  

\item {\tt description:} The default file description ``Briefly describe file contents here.'' should be replaced with a short description of the subject matter of the documentation file. It is recommended that this file description is less than 80 characters on a single line.

\item {\tt tags:} List of any tags used to identify the current document. The default tags include {\tt -\,genesis3}, {\tt -\,level2}, and {\tt -\,draft}. These tags indicate that the document is part of the GENESIS 3 documentation system at the level of User Guides and Documentation (level\,2), and that it is an as yet unpublished document.

For a document to be publishable, i.e. viewable with a browser, the {\tt -\,draft} tag must be replaced with the {\tt -\,published} tag. Note that a document cannot be published to be browser viewable in the absence of the {\tt -\,published} tag. New tags may be created just by adding them to this file, but see below for recommended tags.  Note that the tagging system is case insensitive to avoid tag duplication.

Importantly, each of the directories {\tt contents-level1}\,\ldots {\tt contents-level7} in your default workspace contains a list of the documentation to be found at the given level. These content indexes are generated automatically during a GENESIS installation or a {\tt userdocs} module installation or update. They provide direct links to all documents tagged for the given level of documentation.

\end{itemize}

\subsubsection*{The Default Documentation Source File}

Source files for the GENESIS documentation system are currently produced with the \LaTeX\,typesetting package (see \href{http://www.latex-project.org/}{www.latex-project.org}). It is planned that Microsoft Word ({\tt .doc} format) and OpenOffice ({\tt .odt}, ODF format) file formats will also be automatically recognized by the GENESIS documentation system.

The default \LaTeX\, documentation source file\,{\it NewDocument.tex} in the {\it NewDocument} folder can be compiled to produce a readable PDF file. It gives help and examples for colorizing text and generating hyperlinks to local and remote documentation.

\subsubsection*{The Figures Folder}

This directory contains any figures that your new document source file may include.
 
\section*{Create a New Document}

There are three simple steps to the creation of a new document within the GENESIS documentation system.

\begin{enumerate}

\item {\bf Make a new document template:} Copy the\,{\it NewDocument} folder to the name of the new document you want to make. Change the name of the \LaTeX\,\,\,file\,{\it NewDocument.tex} inside the renamed folder to the name of your new document. To do this replace the\,{\it NewDocument} part of the default source file name with the new name you have just given to the\,{\it NewDocument} folder.

This will result in both the documentation directory and the source file having the same name (with the exception of the {\tt .tex} extension of the source file). For example, the name of the directory containing this document is {\it document-create} and the name of the source file it contains is\,{\it document-create.tex}.

Importantly, the names of the descriptor file\,{\it descriptor.yml} and the directory\,{\it figures} located in the document folder are required by the GENESIS documentation system and should not be changed.

\item {\bf Update the descriptor.yml file:} Each directory/file in the GENESIS documentation system is tagged on the basis of its content, for example:

\begin{enumerate}

\item  {\bf The GENESIS version:} This tags the version of GENESIS that the content of the document refers to, e.g. {\tt -\,genesis3}.

\item {\bf Level of the GENESIS document:} This must be one of the seven levels of the GENESIS documentation system, which include:

\href{../contents-level1/contents-level1.pdf}{{\tt - level1}} (introductory documentation)

\href{../contents-level2/contents-level2.pdf}{{\tt - level2}} (user/developer documentation)

\href{../contents-level3/contents-level3.pdf}{{\tt - level3}} (autogenerated documentation)

\href{../contents-level4/contents-level4.pdf}{{\tt - level4}} (autogenerated technical documentation)

\href{../contents-level5/contents-level5.pdf}{{\tt - level5}} (autogenerated algorithm documentation)

\href{../contents-level6/contents-level6.pdf}{{\tt - level6}} (autogenerated api documentation)

\href{../contents-level7/contents-level7.pdf}{{\tt - level7}} (inline source code documentation)

For more on the levels of the GENESIS documentation system see the \href{../document-overview/document-overview.pdf}{Documentation Overview}.

\item {\bf Flag document status:} For a given document to be part of the documentation version control system, it must be in a documentation folder along with a {\tt descriptor.yml} file that contains one of either the {\tt -\,draft} or {\tt -\,published} tags. Absence of the {\tt -\,published} tag will prevent publication of the document to a browser.

Recommended tags for the descriptor file include:
\begin{verbatim}
  tags:
    - genesis2
    - genesis3
    - level1
    - level2
    - level3
    - level4
    - level5
    - level6
    - level7
    - meta
    - create
    - introduction
    - tutorial
    - document
    - draft
    - published
\end{verbatim}

Remember that the interpretation of the\,{\it descriptor.yml} file by the GENESIS documentation system is sensitive to white space and indents.
 
\end{enumerate}

\item {\bf Update the\,{\it figures} folder:} Place any figures or images, etc that your source document file contains in the\,{\it figures} directory.

\end{enumerate}

\subsection*{Document Version Tracking and Control}

To learn more about document version tracking and control in the GENESIS Documentation System, see \href{../document-versionctrl/document-versionctrl.pdf}{Document Version Control}.

\subsection*{Document Publication}

To learn more about the automated publication of documents in the GENESIS Documentation System, see \href{../document-publication/document-publication.pdf}{Document Publication}.


\end{document}
