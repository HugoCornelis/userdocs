\documentclass[12pt]{article}
\usepackage{verbatim}
\usepackage[dvips]{epsfig}
\usepackage{color}
\usepackage{url}
\usepackage[colorlinks=true]{hyperref}

\begin{document}

\section*{GENESIS: Documentation}

{\bf Related Documentation:}\\
\href{../pub-purkinje-deschutter1-estimates/pub-purkinje-deschutter1-estimates.tex}{\bf Parameter\,Estimation},
\href{../pub-purkinje-deschutter1-kinetics/pub-purkinje-deschutter1-kinetics.tex}{\bf Channel\,Kinetics},
\href{../pub-purkinje-deschutter1-densities/../pub-purkinje-deschutter1-densities.tex}{\bf Channel\,Densities}
% start: userdocs-tag-replace-items related-do-nothing
% end: userdocs-tag-replace-items related-do-nothing

\section*{De Schutter: Purkinje Cell Model}

\subsection*{Abstract}

The ionic channels were distributed differentially over three
zones of the model, with Na$^+$ channels in the soma, fast K$^+$ channels
in the soma and main dendrite, and Ca$^{2+}$ channels and Ca$^{2+}$-
activated K$^+$ channels in the entire dendrite. Channel densities in
the model were varied until it could reproduce Purkinje cell responses
to current injections in the soma or dendrite, as observed
in slice recordings.

\end{document}
