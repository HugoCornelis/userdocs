\documentclass[12pt]{article}
\usepackage{verbatim}
\usepackage[dvips]{epsfig}
\usepackage{color}
\usepackage{url}
\usepackage[colorlinks=true]{hyperref}

\begin{document}

\section*{GENESIS: Documentation}

{\bf Related Documentation:}
% start: userdocs-tag-replace-items related-tutorial
% end: userdocs-tag-replace-items related-tutorial

\section*{Convert a GENESIS\,2 Simulation to GENESIS\,3}

\subsection*{Background}

In \href{../tutorial1/tutorial1.tex}{\bf Tutorial\,1}, you learned to use the GENESIS Interactive Shell (\href{../gshell/gshell.tex}{\bf G-shell})
to create a simple cell, explore it, and run the simulation from the
{\bf G-shell} command line.

\href{../tutorial2/tutorial2.tex}{\bf Tutorial\,2} continues to show how to
use the G-shell to load and simulate predefined model neurons that come
bundled with the GENESIS distribution in the single cell model library.  As
with the Purkinje cell model used in the tutorial these have been saved in
the library in \href{../ndf-file-format/ndf-file-format.tex}{\bf
NDF\,format}.

Of course, as mentioned in Tutorial\,2, it would be very difficult to create a large multi-compartmental
cell model by entering commands in the {\bf G-shell}.

In the future, GENESIS will support import of models represented in neural exchange formats such as
\href{http://neuroml.org}{\bf NeuroML}, and from scripts that use a Python-based scripting language.
This tutorial is devoted to the process of loading existing GENESIS 2 simulation scripts into the
G-shell, running them, and converting them to GENESIS 3 NDF format.

In GENESIS 2, scripts are interpreted by the Script Language Interpreter (SLI), so these scripts are
in the ``SLI format''.  The process of creating a simulation in this format is described with
examples in the \href{http://genesis-sim.org/GENESIS/UGTD/Tutorials/genprog/genprog.html}{GENESIS
Modeling Tutorial} for GENESIS 2.  This is part of a larger package
\href{http://www.genesis-sim.org/GENESIS/UGTD.html}{\bf The Ultimate GENESIS Tutorial Distribution}
that is gradually being converted to GENESIS 3.  Some familiarity with the GENESIS Modeling Tutorial
section "Building a cell the easy way", is assumed in what follows below.

The simulation script used as an example in "Building a cell the easy
way" is called simplecell.g, and it uses the cell reader with the file
cell.p to create a simple two-compartment neuron.  The soma
compartment /cell/soma contains Hodgkin-Huxley type voltage-activated
sodium and potassium channels (implemented as tabchannels), and the single
dendrite compartment /cell/dend contains excitatory and inhibitory
synaptically activated channels (implemented as synchans).

Although many GENESIS 2 scripts may be run or imported with little or no
modification, there are a few differences between GENESIS 2 and GENESIS 3
that may require some changes.

In GENESIS 2, there is no distinction between objects that are used to
represent biological models (e.g. compartments and channels), device
objects that are used to provide input to the model (e.g. a pulsegen or
randomspike), those that provide output (e.g. {\it asc\_file} or {\it disk\_out}),
analysis objects (e.g. various modes of the {\it table} object, {\it interspike},
{\it autocorr}, or {\it peristim} objects), or graphical objects for control or
visualization.  All of these may be freely combined in any desired
manner in a GENESIS 2 script.  This accounts for the great versatility
and flexibility of scripted GENESIS simulations.

Although GENESIS 3 aims for nearly complete backwards compatibility
with GENESIS 2, this does not extend to the XODUS graphical objects and
commands, which are based on 1990's X window system widgets.  Modern
Python-based graphics have now been developed for GENESIS 3, which
allows the scriptable construction of custom GUIs by the user.
However, these do not attempt to reproduce the old XODUS
widgets and command syntax to the point of allowing automatic conversion of
existing scripts that use XODUS.  The scripting of XODUS-like GUIs for G-3
simulations in Python is taken up in the tutorial
\href{../tutorial-python-gui/tutorial-python-gui.html}{Creating a G-3 GUI
with Python}.

Also, the \href{../workflow-user/workflow-user.html}{GENESIS 3 User Workflow} makes a distinction
between these different functionalities, based on a modeling or experimental paradigm involving an
iterative process of model design, experimental design (e.g. providing stimuli to the model),
running the simulation, and output and analysis of results.  The
\href{../genesis-overview/genesis-overview.tex}{\bf Overview of the G-3 CBI Architecture}
describes how this separation of functionality is incorporated into the
internal representation of the simulation.  Commands given in the G-shell
allow these to be incorporated into G-3, while retaining the flexibility of
GENESIS 2.

For these reasons, it is best to separate these functional categories
by making changes to the SLI script which will facilitate the conversion.

Excluding the initial comments, the original {\it simplecell.g} script reads:
\begin{verbatim}
   // Create a library of prototype elements to be used by the cell reader
   include protodefs

   float tmax = 0.5                // simulation time in sec
   float dt = 0.00005              // simulation time step in sec
   setclock  0  {dt}               // set the simulation clock

   // include the graphics functions
   include graphics

   readcell cell.p /cell

   // make the control panel
   make_control

   // make the graph to display soma Vm and pass messages to the graph
   make_Vmgraph
   addmsg /cell/soma /data/voltage PLOT Vm *volts *red

   setfield /control/Injection value 0.5e-9
   set_inject /control/Injection  // set initial injection from Injection dialog

   check
   reset
\end{verbatim}

Fortunately, as in most GENESIS 2 scripts, the graphical commands have
been isolated to the included script {\it graphics.g}.   For use with GENESIS 3,
this file should not be included, and there will need to be commands
that write equivalent simulation results to files, instead of a graph.

The example script \href{figures/simplecell-g3.txt}{\bf simplecell-g3.g} illustrates how the orignal
script can be reorganized to run under either GENESIS 2 or 3.  When creating GENESIS 2 scripts that
will eventually be converted to run under GENESIS 3, the organization described below will help you
to avoid potential problems under GENESIS 3.

By defining some Boolean constants at the begining of the script,
you can easily change between running with GENESIS 2 and XODUS graphics,
and GENESIS 3 in batch mode with output to a file.
\begin{verbatim}
   int batch = 1        // if (batch) then run a default simulation
   int graphics = 0     // display control panel, graphs
   int file_out = 1     // write output to a file
\end{verbatim}
The script defines and uses the functions
\begin{itemize}
\item {\it make\_cell} Creates the cell, using {\it readcell} and the cell parameter file
     {\it cell.p}.  The {\it protodefs.g} file defines the prototype channels and
     other elements referenced by {\it cell.p}.

\item {\it make\_input} Provides input to the cell.  In this case it merely
     sets the {\tt /cell/soma inject} field to the desired value of injection
     current.

\item {\it make\_output} Provides output of the {\tt /cell/soma Vm}  to a file, using
     an {\it asc\_file} object.

\item {\it step\_tmax} Runs the simulation for a time {\tt tmax}, and optionally
     echoes information to the {\bf G-shell}, such as start and end times of a
     run.
\end{itemize}
These are used in the main section of the simulation, which reads:
\begin{verbatim}
 make_cell // creates a cell with a soma and a dend compartment
 make_input // provides input to the cell
 if (graphics)
     // include any graphics functions
     include graphics
     .... (other graphics commands not shown here)
 end

 if (file_out)
    make_output // provides output of the results to a file
 end

 check
 reset

 if (batch)
     step {tmax} -time
 end
\end{verbatim}

\subsection*{Running the simulation in backwards compatibility mode}

The first step in converting the simulation is to verify that
it will execute without errors under the G-shell using the {\it ns-sli}
module.  
% Note: ../ns-sli/ns-sli.tex ins "UNDER CONSTRUCTION" with no info
This runs the simulation under the backwards compatibility
mode.  This will be somewhat slower than in native G3 mode, and does not
provide the advanced features of G3, but is recommended, to be sure that
the model can be converted.  The steps are simply:
\begin{verbatim}
   $ genesis-g3

   Welcome to the GENESIS 3 shell
   genesis > sli_run simplecell-g3.g
\end{verbatim}
The output should be similar to that when running under GENESIS 2.  In fact,
it was not strictly necessary to isolate the code for providing input
and output to separate functions, in order for the simulation to
run properly in {\bf NS-SLI} mode.

\subsection*{Converting the cell model to NDF format}

To make full use of the capabilities of GENESIS, the cell model should
be converted into NDF format.  This can be done with the {\it sli\_load}
and {\it ndf\_save} commands that are available in the {\bf G-shell}:
\begin{verbatim}
   $ genesis-g3

   Welcome to the GENESIS 3 shell
   genesis > sli_load simplecell-g3.g
\end{verbatim}
To be sure that you remember the name that was assigned to your cell model
({\tt /cell}, in the {\it simplecell-g3.g} script) give the command:
\begin{verbatim}
   genesis > list_elements
   ---
   - /proto
   - /output
   - /library
   - /cell
\end{verbatim}
These are in fact the element trees that were created when the simulation
was run in GENESIS 2, or with {\it sli\_run} in the {\bf G-shell}.  If you like, you can
dig deeper with commands such as {\tt list\_elements /cell/soma}.


At this point you can explore the loaded model with some of the other tools
provided by the {\bf G-shell}.  Don't forget to use {\tt help commands} to obtain
a list of commands available in the current version of the {\bf G-shell}, in
addition to those described in Tutorials 1 and 2.

%[This could be filled out with more examples, e.g. how to use "explore".
% Allan or Hugo, do you have any suggestions?]

The final step is to save the cell representation to NDF format,
specifying the file name for the saved representation:
\begin{verbatim}
   genesis > ndf_save /cell/** simplecell-g3.ndf
   genesis > quit
\end{verbatim}

\subsection*{Running the converted cell model in the G-shell}

Now, we are ready to load and run the model in NDF format.
\begin{verbatim}
   $ genesis-g3
   Welcome to the GENESIS 3 shell
   genesis > ndf_load simplecell-g3.ndf
\end{verbatim}
At this point, you can explore the model as before, using the examples
given in Tutorial 2 for the {\tt /Purkinje} model.  The simplest
way to provide stimuli to the model and to output the results
is to use the built-in {\bf G-shell} tools for this, rather than scripting
your own input and output functions.  Some of these are also illustrated
in Tutorial 2.

Provide current injection:
\begin{verbatim}
   genesis > runtime_parameter_add /cell/soma INJECT 0.5e-9
\end{verbatim}

Provide output of the soma membrane potential to a file, specifying
a filename or path other than the default of {\tt /tmp/output}:
\begin{verbatim}
   genesis > output_add /cell/soma Vm
   genesis > output_filename simplecell-g3-ndf_Vm.out
\end{verbatim}

Heccer, the numerical solver used by G-3 for compartmental cell models and channels,
uses a default simulation time step of 20 microseconds.  To set the time step to the
50 microsecond value used in the original GENESIS 2 simulation, give the command:

\begin{verbatim}
    genesis > heccer_set_timestep 50.0e-6
\end{verbatim}

Finally, run the simulation on the model {\tt /cell} for 0.5 seconds:
\begin{verbatim}
    genesis > run /cell 0.5
\end{verbatim}

The \href{../g3plot/g3plot.html}{G3Plot} toolkit is a suite of Python-based tools under development
for visualizing and analyzing the output of GENESIS simulations.  Many of the tools are based on
\href{http://matplotlib.sourceforge.net/}{\bf Matplotlib} that allow
analysis capabilities similar to Matlab to be provided within GENESIS.  As
they are developed, they are being integrated into G-3 for direct use
without the necessity of creating files.

At present, if G3Plot has been installed, the results from {\it simplecell-g3.ndf} may be compared
with those from the GENESIS 2 run of {\it simplecell-g3.g} with the command
\begin{verbatim}
   $ G3Plot.py simplecell-g3-ndf__Vm.out Vm.out
\end{verbatim}

\subsection*{Converting user-defined input and output functions}

Although the G-shell will continue to gain more paradigms for model
input or output that will be sufficient for most single cell modeling,
it will sometimes be necessary to script custom functions in GENESIS 2
to provide unique capabilities.  This is particularly likely to be
true with network and subcellular level simulations.

Recent advances in the development of the G-3
\href{../SSPy/SSPy.html}{SSPy} module now include an Applications
Programming Interface (API) for scripting G-3 simulation in Python.  This
allows models to be interfaced with objects from the {\bf Experiment}
module, including a variety of output objects and input objects such as
pulse generators and voltage or current clamps.  Thus, there are no
immediate plans to extend the capabilities of the G-shell {\it sli\_load}
and {\it ndf\_save} commands to convert the {\it make\_input} and {\it
make\_output} functions that are used in the script
\href{figures/simplecell-g3.txt}{\bf simplecell-g3.g} into G-3.

The next tutorial in this series
\href{../tutorial-python-scripting/tutorial-python-scripting.html}
{Creating GENESIS 3 Simulations with Python} demonstrates how to do the
equivalent of these functions in Python, including the plotting of results.

\end{document}
