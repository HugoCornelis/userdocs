\documentclass[12pt]{article}
\usepackage{verbatim}
\usepackage[dvips]{epsfig}
\usepackage{color}
\usepackage{url}
\usepackage[colorlinks=true]{hyperref}

\begin{document}

\section*{GENESIS: Documentation}

\section*{GENESIS 2 Backward Compatibility Tests}

GENESIS is backward compatible with \href{http://genesis-sim.org/GENESIS/genesis-ftp/}{GENESIS Release 2.3} for all functionality of the original Purkinje cell research scripts that provide the basis of the \href{http://genesis-sim.org/GENESIS/illtuts/purkinje.html}{Tutorial 9--Purkinje Tutorial}. In doing this, backward compatibility supports the major functionalities of the GENESIS Script Language Interface.

Functionality for the graphical objects of the XODUS library is now provided by the \href{../studio/studio.tex}{Studio} and \href{../project-browser/project-browser.tex}{Project Browser} in conjunction with the new GENESIS \href{../gui/gui.tex}{GUI}.

Backward compatibility is provided by the \href{../nssli/nssli.tex}{Neurospaces Scripting Language Interface} (NSSLI).

For more details see \href{../backward-compatibility}{Backward Compatibility}.

\subsection*{Obtaining Test Specifications}

This \href{http://neurospaces.sourceforge.net/neurospaces_project/ns-sli/tests/html/index.html}{link} accesses all test specifications for the GENESIS 2 backward compatibility bridge for GENESIS (the NSSLI).

\subsection*{Test Documentation}

Backward compatibility test documentation includes:
\begin{itemize}
\item The command to start a given test within the NSSLI backward compatibility environment.
\item The input to the test.
\item The expected output of the test.
\end{itemize}
The tests give a good idea of what backward compatibility means, i.e. running GENESIS 2 scripts without modification. Testing is conducted at both the level of user experienced functionality and low-level unit tests. Integration tests between the GENESIS 2 SLI, the Model Container, and Heccer are also provided.

\subsection*{Scope of Tests}

Backward compatibility tests cover four major areas:
\begin{enumerate}
\item Specifications.
\item Recognized GENESIS 2 functions.
\item Heccer functionality in backward compatibility.
\item Tools.
\end{enumerate}

\end{document}
