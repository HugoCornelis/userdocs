\documentclass[12pt]{article}
\usepackage{verbatim}
\usepackage[dvips]{epsfig}
\usepackage{color}
\usepackage{url}
\usepackage[colorlinks=true]{hyperref}

\begin{document}

\section*{GENESIS: Documentation}

\section*{GENESIS 2 Backward Compatibility Regression Tests}

GENESIS is backward compatible with \href{http://genesis-sim.org/GENESIS/genesis-ftp/}{\bf GENESIS Release 2.3} for all functionality of the original Purkinje cell research scripts that provide the basis of the \href{http://genesis-sim.org/GENESIS/illtuts/purkinje.html}{\bf Tutorial 9--Purkinje Tutorial}. In doing this, backward compatibility supports the major functionalities of the GENESIS Script Language Interpreter.

Functionality for the graphical objects of the XODUS library is now provided by the \href{../studio/studio.tex}{\bf Studio} and \href{../project-browser/project-browser.tex}{\bf Project\,Browser} in conjunction with the new \href{../gui/gui.tex}{\bf GENESIS\,GUI}.

Backward compatibility is provided by the \href{../nssli/nssli.tex}{\bf Neurospaces\,Scripting\,Language\,Interpreter} (NS-SLI).

For more details see the \href{../backward-compatibility}{\bf Backward\,Compatibility} documentation.

\subsection*{Obtaining Regression Test Specifications}

This \href{http://neurospaces.sourceforge.net/neurospaces_project/ns-sli/tests/html/index.html}{\bf link} accesses all regression test specifications for the GENESIS 2 backward compatibility bridge for GENESIS (the NS-SLI).

\subsection*{Regression Test Documentation}

Backward compatibility regression test documentation includes:
\begin{itemize}
\item The command to start a given test within the NS-SLI backward compatibility environment.
\item The input to the test.
\item The expected output of the test.
\end{itemize}
The regression tests give a good idea of what backward compatibility means, i.e. running GENESIS 2 scripts without modification. Testing is conducted at both the level of user experienced functionality and low-level unit tests. Integration tests between the GENESIS 2 SLI, the \href{../model-container/model-container.tex}{\bf Model\,Container} and \href{../heccer/heccer.tex}{\bf Heccer} are also performed.

\subsection*{Scope of Regression Tests}

Regression tests cover the following major functionalities of backward compatibility:
\begin{itemize}
\item[]\href{http://neurospaces.sourceforge.net/neurospaces_project/ns-sli/tests/html/specifications/main.html}{\bf Specifications}

Purkinje cell model simplification.

\item[]\href{http://neurospaces.sourceforge.net/neurospaces_project/ns-sli/tests/html/specifications/core/main.html}{\bf GENESIS 2 Functions}

All available GENESIS objects; compartment creation, field setting and copying; simple startup and clock setting; simple functions containing ifelse control structures; simple if else statements.

\item[]\href{http://neurospaces.sourceforge.net/neurospaces_project/ns-sli/tests/html/specifications/heccer/main.html}{\bf Heccer}

Creation of a neutral object with a child compartment; channel and Nernst equation integration; gate tabulation and pool integration; synaptic equations.

\item[]\href{http://neurospaces.sourceforge.net/neurospaces_project/ns-sli/tests/html/specifications/tools/main.html}{\bf Tools}

The {\it asc\_file} object; Purkinje cell model; {\it readcell} parameters and coordinates; model simplification with the {\bf Model Container}.

\end{itemize}

\end{document}
