\documentclass[12pt]{article}
\usepackage[dvips]{epsfig}
\usepackage{color}
\usepackage{url}
\usepackage[colorlinks=true]{hyperref}

\begin{document}

\section*{GENESIS: Documentation}

{\bf Related Documentation:}
% start: userdocs-tag-replace-items related-do-nothing
% end: userdocs-tag-replace-items related-do-nothing

\section*{Introduction}

In this documentation we assume that the target computer has all
dependencies installed (see for
\href{../installation-developer/installation-developer.tex}{developers}
and \href{../installation-user/installation-user.tex}{users}). You can
then proceed by downloading the packages from
\href{http://sourceforge.net/project/showfiles.php?group_id=162899}{sourceforge}.
The download can be automated by the
\href{../developer-package/developer-package.tex}{\tt Developer
  Package}. Alternatively, you can checkout the code from the
\href{http://monotone.ca/}{monontone}
\href{../developers-intro/developers-intro.tex}{repositories}.

All GENESIS packages can be downloaded and installed by themselves,
based on \href{http://www.gnu.org/software/autoconf/}{autotools}
(``{\tt ./configure}'', ``{\tt make}'', ``{\tt sudo make install}'').
%There is a small issue with the {\tt Heccer} package as its
%compilation depends on the {\tt Model Container} being installed.

%\subsubsection*{Structure and Configuration of a GENESIS Target Computer}

%\begin{itemize}
%\item {\bf Required packages for a base installation:} Perl.
%\item TargetConfiguration in {\it /etc}.
%\item Follows autotools conventions (binaries, library archives, etc). 
%\end{itemize}

\section*{Installation Order of Packages}

The order of installation is:
\begin{enumerate}
\item \href{../developer-package/developer-package.tex}{\bf Developer
    Package:} An optional module that provides developer utilities
  complying with GENESIS development standards..
\item \href{../model-container/model-container.tex}{\bf NMC:} The {\tt
    Model Container} provides a highly optimized solver independent
  internal storage format for models.
\item \href{../heccer/heccer.tex}{\bf Heccer:} The numerical solver.
\item \href{../ssp/ssp.tex}{\bf SSP:} The Simple Scheduler in Perl is
  currently the standard scheduler for GENESIS.
%\item \href{../studio/studio.tex}{\bf Studio:} GUI front-end to the
%  {\tt Model Container} that enables browsing and visualization of a
%  model.
%\item \href{../project-browser/project-browser}{\bf Project Browser:}
%  Browse projects, inspect and compare simulation results or other
%  data over a web interface.
\item \href{../backward-compatibility/backward-compatibility.tex}{\bf
    GENESIS 2 Backward Compatibility Module:} is for running GENESIS 2
  SLI scripts.
\item \href{../gshell/gshell.tex}{\bf GENESIS 3 Shell:} is for running
  interactive GENESIS sessions.
\item \href{../documentation-overview/documentation-overview.tex}{\bf
    GENESIS Documentation System:} is a modular documentation system
  suited for maintaining geographically distribution software systems.
\end{enumerate} 

This order is followed by the {\it Developer Package}.

%\subsection*{For a user}

%Using the {\it install} script of the \href{../installer-package/installer-package.tex}{\tt InstallerPackage}:

%\begin{itemize}
%\item Install the common GENESIS packages (at this moment, the {\tt model container}, {\tt heccer}, {\tt ssp} and the {\tt studio}), from  \href{http://sourceforge.net/project/showfiles.php?group_id=162899}{sourceforge}, distribution {\tt python-7}\,\,\,({\it gshell\,})--all other modules {\tt python-5}.

%{\bf Important Note:} Please, modify {\tt python-n} to be the latest release available for download:
%\begin{verbatim}
%neurospaces_build --download-server downloads.sourceforge.net \
%   --src-tag python-7 --src-dir  /tmp/neurospaces/downloads \
%   --verbose --verbose --verbose
%\end{verbatim}
%Add a couple of {\tt --verbose} options to see what is going on. 

%\item To uninstall you can use the same command line, and use the {\tt --uninstall} option:
%\begin{verbatim}
%neurospaces_build --download-server downloads.sourceforge.net \
%   --src-tag build-37 --src-dir /tmp/neurospaces/downloads --verbose \
%   --verbose --verbose --uninstall --no-configure --no-compile \
%   --no-install
%\end{verbatim}
%{\bf Note:} You must add the {\tt --no-compile} option to avoid breakages due to compilation dependencies between packages. 

%\item If sources are still in your filesystem after a previous download:
%\begin{verbatim}
%neurospaces_build --no-compile --uninstall --src-tag python-7 \
%   --src-dir /tmp/neurospaces/downloads --verbose --verbose --verbose
%\end{verbatim}
%{\bf Note:} You must add the {\tt --no-compile} option to avoid breakages due to compilation dependencies between packages. 

%\item For each of the above command lines, it is possible to select individual packages using the {\tt --regex} options. The default value is {\tt --regex 'model-container|heccer|ssp|studio'}. 

%\end{itemize}

%\subsection*{For a developer}

%Using the {\it install} script of the {\tt InstallerPackage}:
%\begin{enumerate}
%\item Understand how and why the script works for a regular user.
%\item Use the following commands:

%\item[] {\bf Download, install, check:}
%\begin{verbatim}
%neurospaces_build --download-server downloads.sourceforge.net \
%   --check --regex 'model-container|heccer|ssp|studio' --src-tag python-7 \
%   --src-dir /tmp/neurospaces/downloads --verbose --verbose --verbose
%\end{verbatim}

%\item[]{\bf Uninstall:}
%\begin{verbatim}
%neurospaces_build --no-compile --uninstall \
%   --regex 'model-container|heccer|ssp|studio' --src-tag python-7 \
%   --src-dir /tmp/neurospaces/downloads --verbose --verbose \
%   --verbose
%\end{verbatim}

%\item[]{\bf Reinstall the things you were working on/developing:}
%\begin{verbatim}
%neurospaces_build --regex 'model-container|heccer|ssp|studio' \
%   --verbose --verbose --developer --verbose
%\end{verbatim}

%\end{enumerate}

\end{document}
