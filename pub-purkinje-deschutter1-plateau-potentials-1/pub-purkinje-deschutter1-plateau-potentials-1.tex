\documentclass[12pt]{article}
\usepackage{verbatim}
\usepackage[dvips]{epsfig}
\usepackage{color}
\usepackage{url}
\usepackage[colorlinks=true]{hyperref}

\begin{document}

\section*{GENESIS: Documentation}

{\bf Related Documentation:}
% start: userdocs-tag-replace-items related-do-nothing
% end: userdocs-tag-replace-items related-do-nothing

\section*{De Schutter: Purkinje Cell Model}

\subsection*{PLATEAU POTENTIALS}

Current injection steps are often followed
by a prolonged plateau potential. In our simulations
this phenomenon was always observed after short current
steps around firing threshold (\href{../pub-purkinje-deschutter1-fig-5/pub-purkinje-deschutter1-fig-5.tex}{\bf Fig.\,5}) and sometimes after
longer current steps also (\href{../pub-purkinje-deschutter1-fig-4/pub-purkinje-deschutter1-fig-4.tex}{\bf Fig.\,4{\it A}}). In the model these
plateau potentials resulted in depolarizations of between
-49 and -55\,mV. Similar prolonged plateau potentials
after the end of a current injection were found by\,\cite{R:1980pi} (compare their Fig. 5 D with our \href{../pub-purkinje-deschutter1-fig-3/pub-purkinje-deschutter1-fig-3.tex}{\bf Fig.\,3{\it A}}). However, in the experimental data these plateaus decay
over a time course of $\sim$\,100\,ms, whereas in the model
they often did not decay at all (\href{../pub-purkinje-deschutter1-fig-5/pub-purkinje-deschutter1-fig-5.tex}{\bf Fig.\,5}).

The model also generated longer duration plateau-type
potentials that have been shown to
exist in this neuron\,\cite{Jaeger:1991kh, R:1980pi, Llinas:1992rq}. 
It has been demonstrated physiologically
that current injection generates a plateau potential
in the soma of the cell that is dependent on Na+ conductances.
The model generated Na$^+$-dependent somatic
plateaus (Figs. \href{../pub-purkinje-deschutter1-fig-9/pub-purkinje-deschutter1-fig-9.tex}{\bf 9{\it B}} and \href{../pub-purkinje-deschutter1-fig-11/pub-purkinje-deschutter1-fig-11.tex}{\bf 11}). However, it is debated in the
literature whether such persistent Na$^+$ currents rely on a
special channel (NaP) or depend on the NaF channel itself\,\cite{Alzheimer:1993fk, C-R-French:1990uq, Kay:1990kx}.
In our model the somatic plateau potential was mainly
carried by NaF channels in the form of a so-called window
current, although the model also included the experimentally
demonstrated NaP channel\,\cite{Kay:1990kx}. The
plateau-related current in the model flowed through NaF
channels at a potential where the steady-state activation
and inactivation curves (\href{../pub-purkinje-deschutter1-conductance1-naf1/pub-purkinje-deschutter1-conductance1-naf1.tex}{\bf Fig. 2{\it A}}) overlap. Although this
mechanism was robust in the model, one argument against
the existence of such a window current is that it is based on
a wrong model of inactivation, because Na$^+$ channel inactivation
is probably not voltage dependent\,\cite{Aldrich:1987uq}. 
If this is in fact correct, then the term in our
equations for steady-state inactivation would be meaningless.
Although this possible inaccuracy in the Hodgkin-
Huxley model of inactivation\,\cite{hodgkin52:_quantitative_description} means that our results
are not conclusive, the model at least suggests that there
may not be a need for separate NaP channels to explain
somatic plateaus. In our simulations the NaP channel actually
primarily affected the $f-I$ curve.

The model also generated dendritic plateau potentials
(Figs.\,\href{../pub-purkinje-deschutter1-fig-9/pub-purkinje-deschutter1-fig-9.tex}{\bf 9} and\,\href{../pub-purkinje-deschutter1-fig-11/pub-purkinje-deschutter1-fig-11.tex}{\bf 11}) that have recently been described in more
detail\,\cite{Jaeger:1991kh} and that may be particularly
important during synaptic activation of the Purkinje
cell by peripheral stimuli\,\cite{Thompson:1991ac}. In
the model these dendritic plateau potentials were carried
largely by the CaP channels, because the CaT channel inactivated
too rapidly to play a major role. In this case again
the plateau response resulted from a window current-like
mechanism very similar to that found with the NaF channels
in the soma, because the dendritic plateau current
through the CaP channel occurred at a potential where the
CaP channel does not inactivate completely. As a consequence
these dendritic plateau potentials did not wane in
the model as they do in slice preparations.

\bibliographystyle{plain}
\bibliography{../tex/bib/g3-refs.bib}

\end{document}
