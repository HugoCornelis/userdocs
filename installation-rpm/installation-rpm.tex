\documentclass[12pt]{article}
\usepackage{verbatim}
\usepackage[dvips]{epsfig}
\usepackage{color}
\usepackage{url}
\usepackage[colorlinks=true]{hyperref}

\begin{document}

\section*{GENESIS: Documentation}

{\bf Related Documentation:}
% start: userdocs-tag-replace-items related-do-nothing
% end: userdocs-tag-replace-items related-do-nothing

\section*{Installing from RPMS}

\href{http://rpm.org}{\bf RPM (RPM Package Manager format)} is a well known software package format used in many Linux distributions. To make installation easier, the components of GENESIS 3 are available in RPM packages \href{http://repo-genesis3.cbi.utsa.edu/rpm/}{here}. To be able to run the \href{../gshell/gshell.tex}{\bf gshell} to go through \href{../tutorial1/tutorial1.tex}{tutorial 1}, you need to install the following software components:

\begin{itemize}
	\item[] {\bf heccer}
	\item[] {\bf model-container}
	\item[] {\bf ns-sli}
	\item[] {\bf ssp}
\end{itemize}

To install an rpm (in this case the model-container), you would use the following options with the {\bf rpm} command:

\begin{verbatim}
rpm -ivh model-container-0.0.0-alpha.x86_64.rpm
\end{verbatim}

A prompt will appear showing the installation progress. If an error occurs, such as a missing dependency, it will be printed on the screen. 


\section*{Viewing RPM contents}

If you wish to view the contents of an RPM package you can do so with the following flags:

\begin{verbatim}
rpm -qlp developer-0.0.0-alpha.noarch.rpm
\end{verbatim}

A snippet of the output shows the listing of all files and directories that are to be created on the target machine:

\begin{verbatim}
	/usr/local
	/usr/local/bin
	/usr/local/bin/mcad2doxy
	/usr/local/bin/mtn-ancestors
	/usr/local/bin/neurospaces_build
	/usr/local/bin/neurospaces_check
	/usr/local/bin/neurospaces_clean
	/usr/local/bin/neurospaces_clone
	/usr/local/bin/neurospaces_configure
	/usr/local/bin/neurospaces_countcode
	/usr/local/bin/neurospaces_create_directories
	/usr/local/bin/neurospaces_cron
	/usr/local/bin/neurospaces_dist
	/usr/local/bin/neurospaces_docs
	/usr/local/bin/neurospaces_harness
	/usr/local/bin/neurospaces_init
	/usr/local/bin/neurospaces_install
	/usr/local/bin/neurospaces_kill_servers
	/usr/local/bin/neurospaces_packages
	/usr/local/bin/neurospaces_pkgdeb
	/usr/local/bin/neurospaces_pkgrpm
	/usr/local/bin/neurospaces_pkgtar
\end{verbatim}


\end{document}
