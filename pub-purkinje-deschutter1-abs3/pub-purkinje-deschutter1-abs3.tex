\documentclass[12pt]{article}
\usepackage{verbatim}
\usepackage[dvips]{epsfig}
\usepackage{color}
\usepackage{url}
\usepackage[colorlinks=true]{hyperref}

\begin{document}

\section*{GENESIS: Publication}

{\bf Related Documentation:}
% start: userdocs-tag-replace-items related-do-nothing
% end: userdocs-tag-replace-items related-do-nothing

\section*{De Schutter: Purkinje Cell Model}

\subsection*{Abstract Item 3}

As in real Purkinje cells, the model generated two types of
spiking behavior. In response to small current injections the
model fired exclusively fast somatic spikes. These somatic spikes
were caused by Na$^+$ channels and repolarized by the delayed rectifier.
When higher-amplitude current injections were given, sodium
spiking increased in frequency until the model generated
large dendritic Ca$^{2+}$ spikes. Analysis of membrane currents underlying this 
behavior showed that these Ca$^{2+}$ spikes were caused by
the P-type Ca$^{2+}$ channel and repolarized by the BK-type Ca$^{2+}$-activated
K$^+$ channel. As in pharmacological blocking experiments,
removal of Na$^+$ channels abolished the fast spikes and removal of
Ca$^{2+}$ channels removed Ca$^{2+}$ spiking.

\end{document}
