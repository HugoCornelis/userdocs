\documentclass[12pt]{article}
\usepackage[dvips]{epsfig}
\usepackage{lineno}
\usepackage{color}
\usepackage{url}
\usepackage[colorlinks=true]{hyperref}

\begin{document}

\section*{GENESIS: Documentation}

{\bf Related Documentation:}

\href{../ndf-file-format/ndf-file-format.tex}{\bf The NDF File Format}

\href{../model-variables/model-variables.tex}{\bf Physical Processes and Model Variables}

\href{../ndf-procedural-description/ndf-procedural-description.tex}{\bf The NDF File Format and Procedural Model Descriptions}

% start: userdocs-tag-replace-items related-do-nothing
% end: userdocs-tag-replace-items related-do-nothing

\section*{Namespaces in the Model-Container and the Neurospaces Description Format}

\subsection*{Introduction}
\label{sec:introduction}


In scientific papers researchers often present a hierarchical top-down
description of their models.  The {\bf NDF File Format} supports the
integration of model libraries of prototype models and the
hierarchical description of models through its {\it namespaces}.  A
{\it namespace} is a labeled container of named model components and
disambiguates them from model components with the same name contained
in a different {\it namespace}.  Namespaces are indicated with the
{\tt ::} two character sequence.

The {\tt ::} notation is defined by the model component space of the
model-container (note: this notation is neither python nor perl
specific).  The Gshell, SSP and SSPy transparantly support this
notation.


Two examples of a hierarchical model description:

\begin{itemize}
\item In the G-2 version of the Purkinje cell model, the G-2 scripts
  construct prototypes of channels which are then mapped onto
  prototypes of dendritic sections which are then mapped onto the
  dendritic morphology.

\item In the G-2 version of the cerebellar cortex network model, the
  G-2 scripts construct prototypes of channels which are then mapped
  onto prototypes of either a single compartment Granule cell model or
  either a single compartment Golgi cell model, which are then mapped
  onto positions calculated by the createmap command.
\end{itemize}

The {\tt ::} supports the concept of referencing prototypes
independent of the level of the model, to facilitate the hierarchical
construction of models.

From the syntactical viewpoint the {\bf NDF File Format} supports
hierarchical model construction using file references with the {\tt
  IMPORT} command.  The {\tt IMPORT} command references a file and
associates a name with it, called the namespace.  When used in model
component references the name occuring after the {\tt ::} refers to a
namespace that is associated with a file in the {\tt IMPORT}s section
of an {\bf NDF} file.  For instance the snippet:

\begin{verbatim}
IMPORT
        FILE Purk_spine "segments/spines/purkinje.ndf"
END IMPORT
\end{verbatim}

defines {\tt ::Purk\_spine::} as a namespace.  After parsing every
model defined as a public model in the file {\it
  segments/spines/purkinje.ndf} can be accessed as {\tt
  ::Purk\_spine::<model-name>}.  The NDF file {\it
  segments/spines/purkinje.ndf} defines the models {\tt Purk\_spine}
and {\tt Purk\_spine2} (both are groups of segments defining two
different variants of Purkinje cell spines), thus after parsing these
two models become accessible as {\tt ::Purk\_spine::/Purk\_spine} and
{\tt ::Purk\_spine::/Purk\_spine2}

When referencing a model component in the model component space the
notation can be understood as follows:

\begin{itemize}
\item[] {\tt /} means {\it ... who has as child ...}
\item[] {\tt ->} means {\it ... who has as parameter ...}
\item[] {\tt =} means {\it ... with the value ...}

\item[]  and then

\item[] {\tt ::} means {\it ... who has a namespace ...}
\end{itemize}

In the gshell, the command

\begin{verbatim}
ndf_load_library rscell cells/RScell-nolib2.ndf
\end{verbatim}

defines the namespace {\tt ::rscell::} and associates it with the file
{\it cells/RScell-nolib2.ndf}.  All the (public) models in this file
are now accessible from this namespace.  The statement

\begin{verbatim}
insert_alias ::rscell::/cell /two_cells/1
\end{verbatim}

creates the model component {\tt /two\_cells/1} as a reference to {\tt
  /cell} underneath the namespace {\tt ::rscell::}.

\end{document}
