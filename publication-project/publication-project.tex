\documentclass[12pt]{article}
\usepackage{verbatim}
\usepackage[dvips]{epsfig}
\usepackage{color}
\usepackage{url}
\usepackage[colorlinks=true]{hyperref}

\begin{document}

\section*{GENESIS: Documentation}

{\bf Related Documentation:}
% start: userdocs-tag-replace-items related-do-nothing
% end: userdocs-tag-replace-items related-do-nothing

\section*{Project Publication}

At its core, the publication infrastructure is designed to facilitate sharing of scientific data and biophysical computational models and their annotations and by so doing provide a quantitative mechanism for the further development of computational neuroscience and our understanding of neurobiological systems as a whole.  It is specifically developed for the generation of what are commonly referred to as ``realistic models'', whose anatomical and biophysical structures are derived from the brain itself (see Bower, 1999 for discussion).  While clearly an ambitious undertaking, we believe that the time has come to design a mechanism for the ordered evaluation, sharing, and evolution of what we will refer to here as �Community Models� of components and parts of the nervous system. As outlined below, there is already early de facto movement in this direction as several models have already migrated from single laboratories to multiple and even competing laboratories.   One of these models, a cerebellar Purkinje cell model, will serve as the initial test case for the system we propose to develop.  However, while the technology of modeling has slowly been gaining acceptance in the field, like experimental science before it, this new approach requires the development of new technology to maximize its potential and value.  It is our view that the transition from purely descriptive experimental science, to investigations increasingly related to models, is, in essence, the first stage in establishing the first real quantitative theoretical base for understanding structure/function relationships in the nervous system.  Models themselves, properly constructed, in effect, represent our current understanding about what is functionally important in the structure of the brain and its components, as well as a tool to organize future experimental and modeling efforts to move that understanding forward (Bower, 1999).  However, this kind of success for modeling will require the development of new tools for the comparison and publication of models (Nickerson \& Buist, 2009; Wolkenhauer et al., 2009).

The core historical premise for the GENESIS Publication Paradigm is that, for most of its history, the greatest part of our efforts to communicate new neurobiological results has been based fundamentally on narrative story telling in the form of scientific journal articles (Gibson et al., 2004).  Accordingly, the usual first stage in constructing a publication is to develop a series of graphical figures around which a narrative story is then constructed.  It is common practice to construct these figures from``best case� results, which tell the story most convincingly (Cohen, 2009), or at least in a manner that will raise the least reviewer concerns (Sauerbrei and Blettner, 2009).  The short-term evaluation of the veracity of the new story is then performed by experts through the peer review process where the new story is aligned with existing and previously published stories (Brender \& Talmon, 2009; Foote 2009), and often, in practice, most especially with the beliefs of the reviewers (Perneger 2010; Roberts 2009). Although publication of a methods section, in principle, allows the possibility for the actual experimental replication of a new result, it is generally understood that methods descriptions are inadequate to replicate experiments, and with a strong emphasis on `new� stories, publication of pure replications is difficult or impossible (Wyness et al., 2009).  

In principle, model-based research provides a mechanism to overcome a number of these limitations and approach publishing, scientific communication, and even collaboration in a new way, because the models themselves can be published (Nickerson \& Buist 2009).  However, most efforts to date continue to be focused on either encouraging traditional publications to be receptive to modeling studies (Wolkenhauer et al 2009), our encouraging modelers to deposit models after publication of modeling results in a process independent of the publication process itself (Hines et al., 2004).  Constructing a publication system based on the model itself, however, provides many important advantages.  First, new stories generated by models can, at least in principle, be referred directly to mathematical structures from which the story was constructed.  At least some of these structures, such as for example basic biophysical parameters like membrane capacitance or resistance, can, in principle, be evaluated automatically as the first stage of publication of a new model, making some part of the review process more rigorous and objective.  Such an automatic evaluation process can also alert reviewers (as well as authors) to parameters that are well out of the norm.  Second, again in principle, instead of being handpicked from best-case data, figures can be constructed directly from the published model itself.  Thus a reviewer or reader would know exactly the conditions under which a particular figure or result was generated.  In effect, on-demand  figure generation is a form of result replication.  Finally, and perhaps most importantly, the model itself potentially becomes a forum to establish and structure an ongoing dialog, as well as a basis for the development of new modeling components. Models, used in this way, provide a common representation of what is known about a particular component of the nervous system, as well as a means for further elaboration and exploration of that component.  Such models would, in effect become ``Community Models� to which many could contribute.  They can also provide the structure of a common basis for training new researchers, educating students, including, for example, medical students, as well as a mechanism to clearly identify outstanding issues, for example the need for better or new experimental data. Models used in this way would greatly clarify communication (which should be a principle objective of scientific publication) and could also become a means to quantitatively identify and resolve scientific conflicts.  To do so, however, it will be necessary to construct a publication system that publishes models rather than textual description of models, as is currently the case.

\subsection*{The Current State}

While, in principle, model-based research has the potential to provide this kind of foundation for current and future neurobiological science, in fact, the majority of modeling studies today are based on models that remain unshared, or un-sharable, and which are built and associated with single laboratories.  As an example, at last year�s International Computational Neuroscience meeting in Berlin, results were presented from 12 different hippocampal network models from 12 different independent laboratories.  No effort has yet been made to integrate these models together into one common model.  This is typical for most (but not all, see below) computational models.

While there are clear sociological factors involved in this state of affairs (see discussion in later sections), we believe that the current form of scientific publication is also an important impediment.  For example, perhaps even more than is the case for the publication of experimental results, traditional methods sections are wholly inadequate in providing enough information to replicate a complex realistic model or its results.  While there have been important efforts to develop repositories for published models, with Model-DB at the Yale Sense lab being perhaps the most visible and successful example (Hines et al., 2004), models are added after the fact, and are not explicitly linked to the publication of the text or figures.  Post-publication sharing in general is known to have difficulties (Schofield et al., 2009). With respect to models, in particular, those deposited in a database as an after-thought have been notoriously difficult to extract and make sense of (Nordlie et al., 2009).  It is also well known, for example, that the relationship between the model deposited and the model that generated the results or figures in a particular paper is often tenuous, and, in fact, Model-DB only suggests that the deposited model replicates at least one of the published figures.  Deposited models also lack any documentation with respect to the execution of the model itself, its origins or antecedents.  It is the nature of the modeling process to generate multiple variations which, in the absence of a formal system for model versioning, can become very difficult to track.  It is the premise of this proposal that constructing a tool that integrates the modeling and publication process will produce a technical basis for the development of both higher fidelity publications, and the kind of community models that physics, for example, has depended on for several centuries.  

While the current state of modeling is more disparate than organized (Nordlie et al., 2009), there is clear evidence that community models are beginning to emerge.  Perhaps the best current example is based on a model of the cerebellar Purkinje cell, initially published in 1994 as a passive electrical model by Rapp et al. (1994). This model was then subsequently extended to include active conductances by our own research group (De Schutter and Bower, 1994a,b,c.) and made available to the public through the GENESIS data-bases (www.modelersworkspace.org).  Since 2001, four variations of this model have been published by four independent research groups (Chono et al., 2003; Coop and Reeke, 2001; Miyasho et al., 2001; Steuber et al., 2007).  In addition, numerous variations have been published by former members of the Bower laboratory now running their own research groups (De Schutter and Steuber 2009; Kreiner and Jaeger 2004; Solinas et al., 2006).  The Purkinje cell model was also the first full model of a neuron to be translated between the NEURON simulator (Hines and Carnevale, 1997) and GENESIS (Steuber et al., 2007) and has served as the basis for a funded collaboration between five different research groups in three different countries (Steuber et al., 2007).  The model has also now been incorporated into a network model of cerebellar cortex and, importantly, has been cited in a relatively large number of purely experimental studies (cf. Cavelier et al., 2002; Ogasawara et al., 2007; Pouille et al., 2000; Widmer et al., 2003; Williams et al., 2002; Willoughby and Schwiening, 2002). 

While the Purkinje cell model is well on the way to becoming one of the first ``Community Models� in neuroscience, the lack of model-based publication is even an issue for this model.  Like all models, the original continues to undergo a process of evolution that is more or less explicit. While the original model was based on a neuron morphology of a passive model developed by Rapp et al.(1994), other models have used different morphologies, but applied the same channel distribution and characteristics. Other versions of the model have used different values for conductance parameters (Chono et al., 2003).  While these model differences were explicitly called out in the corresponding publications, closer examination of the other published versions of this model also reveal sometimes subtle and sometimes not so subtle variations.  While this is potentially a concern in itself, the larger issue is that it is not at all clear how much of the original performance of the first model is preserved in subsequent models. For example, the original model showed an emergent property of dendritic amplification of excitatory post-synaptic signals (De Schutter and Bower, 1994c), while a more recent version shows an additional emergent property of amplification of inhibitory post-synaptic signals (Solinas et al., 2006). However, the updated channel characteristics in the latter model make it difficult to quantitatively compare the reported dendritic amplification characteristics of inhibitory and excitatory signals. This is an important issue because, in the current process of publication, it is often assumed that the current model has `inherited� the demonstrated characteristics of the original model.  No modern journal would accept the re-publication of 20 years worth of figures demonstrating that this is the case--but the question of whether the new version can still replicate the performance of the old version is a critical experimental and theoretical question that is currently simply not addressed.  The inability to replicate could either indicate that the old result was wrong or misinterpreted, or that the new version of the model has drifted away from the biological realism established in early modeling stages.  It will be necessary to develop new tools and a new publication system to address this important issue.

While in some sense, the Purkinje cell model represents one of the best current examples of a ``Community Model�, as just described, but it suffers, as do all modeling efforts (Nordlie et al., 2009), from the lack of set of standard tools for understanding and evaluating different models.  These include the following:

\subsubsection*{Current lack of model comparison tools}

With the exceptions of using Bayesian methods for evaluating compartmental models of neurons (Baldi et al., 1998) and the analysis of the structure of a model (Hines et al., 2007), there are few, if any, whole model comparison tools available.  While individual researchers are developing specialized tools that examine one or another feature of models (Cannon and D'Alessandro, 2006; Druckmann et al., 2007), there is as yet no infrastructure or software system to integrate these tools into a working whole.  None of these tools obviously have been included into the review process for publication of modeling results.

\subsubsection*{Current lack of model lineage inspection tools}

With the exception of tracing model lineage by hand from source publications, we know of no formal methodology or computational tools for identifying and/or tracking model lineage in computational neuroscience and biology. In these disciplines it is still difficult if not impossible to track either the micro-evolution of incremental changes made to a model during a single publication or the macro-evolution of a model across multiple publications.  This is a particularly relevant concern as it relates to the question of attribution, which is, in turn tied to evaluations for funding, promotion, and tenure.  In principle, tracking model lineage could provide a more quantitative and rigorous method for evaluating the impact of a particular researcher on the modeling community, as well as understanding the antecedents of a new model. In principle, combining model comparison tools with lineage tracking, one could also formally ask the question ``What has this new model contributed?� or ``How does it differ from an older model?�

\subsubsection*{Current lack of model verification}

Due to its micro-evolution, even with full access to the code of a given model, confirming that it was indeed the one used to produce the figures in a paper is not always possible. This greatly complicates model verification.

\subsubsection*{Current difficulty of result replication}

Because the validity of a model cannot currently be easily verified, replication of results (one of the corner stones of the scientific method) is greatly complicated.  This is particularly a problem with models, like the Purkinje cell model, which have evolved through several generations.  As already described, it will be important to know which of the previously published features of the model still hold true as the model parameters are modified with additional experimental data.

We propose that each of these issues can be best addressed by establishing a new system for model-based publication, however, we do not believe that such a system will simply emerge from the web as has been proposed by others (Nickerson \& Buist, 2009), instead it will require the careful and thoughtful development of new technology.

\subsection*{The CBI Federated Software Architecture}

The GUI and technology development here will be a component of the larger \href{../genesis-overview/genesis-overview.tex}{\bf CBI\,Federated\,Software\,Architecture} which has been developed to provide a modern modular framework for simulator technology. (Cornelis et al., 2008)  The CBI architecture is organized around the principle that a natural distinction can be drawn between data and algorithms.  In the context of computational research, the distinction between data and algorithms leads to a separation between model (data) and simulation control (control of data flows).

The highly modular approach to the development of simulator software provided by the CBI architecture allows for independent stand-alone software components that integrate on a just-in-time basis.  Importantly, this modularity also allows different simulation systems or analysis modules to be plugged into the entire system.  These different modules can contribute functionality to the workflow of model development, model exploration and analysis, model simulation, and (ultimately) data analysis and model publication.  This structure also allows for the addition of new modules and functions for model description as well as publication. A key innovation in the CBI architecture is the separation of the stimulus protocols from a given model cell and its numerical implementation. This is an innovation that allows both a model and a stimulation protocol to be combined dynamically at run time for a given simulation. Importantly, this separation also allows any stimulation protocol to be run with any model, a feature that greatly facilitates iterative model development and comparison. The isolation of a given biological model from the stimulation protocol also enables convenient exploration and quantification of model properties. For example, no special functions or scripting are required to quantify properties of a single cell model such as total or partial volumes or surface areas, the number of dendritic branches or branch points per dendritic tip, or the average somatopetal to dendritic tip distance. This new implementation gives fast and easy access to all the quantities that describe a model and its constituents giving the user, reviewer or publisher tight control over the processes of model verification and validation.

\subsection*{New Browser-based GUI}

While the CBI federated software architecture makes the publication system independent of any particular simulator, the  \href{../publication/publication.tex}{\bf Publication\,Workflow} will benefit from new software systems and features constructed as a result of our independent work to release a new version of the GENESIS simulation system (G-3) which is scheduled for open release in July 2010. Importantly, this new GUI is based on a workflow structure (Tiwari \& Sekhar 2007), that captures the process of conceiving, developing and applying model-based research to scientific questions.

Control and command for G-3 is implemented through a browser-based series of components which will also be available for use when developing scientific research projects. (a) The \href{../project-browser/project-browser.tex}{\bf Project\,Browser} is a developer contributed component providing a working prototype of a web server and database which allows different projects to be browsed and simulation results inspected and compared via a web browser. It allows the data from a single project to be distributed and synchronized over multiple computers. (b) The \href{../studio/studio.tex}{\bf Studio} is a second developer contributed component used to visualize a model stored by the {\bf Model\,Container} and allows a model to be queried from an interactive shell.  (c) The \href{../gtube/gtube.tex}{\bf G-Tube} can be evoked from either a terminal prompt or the interactive shell, and ultimately will be available as an icon on the desktop.  It provides the graphical interface to access a model, model configuration, stimulation protocol, and simulation configuration which is referred to as the \href{../workflow-user/eworkflow-user.tex}{\bf User\,Workflow}.

Perhaps the most important consequence of the CBI federated software architecture for the GUI is that the higher level interfaces present high-level biological concepts instead of mathematical ones.  It is now possible to considerably enhance the functionality of the GUI independent of the simulator itself.  This modular architecture has the advantage that it facilitates integration with existing infrastructures and internet based technologies. As an example, we have developed prototype code that interfaces the {\bf Model\,Container} with the \href{http://neuromorpho.org/neuroMorpho/index.jsp}{\bf NeuroMorpho.Org} morphology internet database. This code downloads a cell morphology from the database and generates both passive and active models according to a given configuration. A second example is a new database of computational stimulation protocols (e.g. different stochastic afferent synaptic activation, current, and voltage clamp) that allow models to be automatically validated.

\section*{System Contents}

\subsection*{G-Tube}

Traditional scientific publications come in the form of Introduction, Methods, Results, and Discussion. The proposed Publication System will be organized along similar lines, if for no other reason than this provides a familiar framework for the researcher or student. The specific context for the GUI development will be via the \href{../gtube/gtube.tex}{\bf G-Tube}, which will provide a browser-based and browsable menu of bullet points/narrative components that takes the place of the more traditional Abstract.  The critical enhancement provided by the {\bf G-Tube} is that it will provide an interface to both verification and validation of a model prior to publication submission and a direct link between the narrative, the figures and the simulations of the research project. The publication system moves the focus of a publication from the narrative text to figures and will allow a researcher to focus on the novel components, mechanisms, and/or behavior of a model first, and then delve into areas of specific interest second. It will allow a publication to be explored in depth from four menu items in the GUI:
\begin{itemize}
   \item{\bf Menu Item 1:} Links to an introduction and other relevant publications.
   \item{\bf Menu Item 2:} Explains the model in detail by supporting:
   \begin{itemize}
	\item {\bf (i)} The use of figures and tables to illustrate the functions and parameters of a model.
	\item {\bf (ii)} Browsing of the model lineage.
	\item {\bf (iii)} Browsing model structures including for example single neuron simulations exploring channel characteristics, and their distributions and kinetics.
	\item {\bf (iv)} Automated testing of the model, including structural analysis and validation simulations.
   \end{itemize}	
   \item{\bf Menu Item 3:} Gives access to all the simulations that have been run, and enables automatic reproduction of the figures that are important to the publication. The {\bf G-Tube} also allows manual figure production for figures that are not part of the publication.
   \item{\bf Menu Item 4:} Links to a conclusion that refers to the previous menu items and other publications.
\end{itemize}

\subsection*{Model Lineage}

In principle one of the values of mathematical modeling is that each model should extend one or more pre-existing model(s) or parts of a model. Formally capturing and tracking the incremental steps of model development and exploration is fundamental to a comprehensive publication process.  The internal storage format provided by the {\bf Model\,Container} has been designed around lineage tracking of models and their constituents and keeps a formal record of a model's hierarchy and its modifications.  These advanced features allow the history of both a model and the refinements made by different research groups to be tracked.  The multiple publications of the single Purkinje cell model described above, will be used to prototype and test this feature (Rapp and Yarom, 1994; De Schutter and Bower, 1994a,b,c; Chono et al., 2003; Coop and Reeke, 2001; Miyasho et al., 2001, Steuber et al., 2007; De Schutter and Steuber, 2009; Kreiner and Jaeger 2004; Solinas et al., 2006). 

\subsection*{The Model Lineage System and Attribution}

Perhaps one of the most important aspects of the model lineage system, is that it will allow a new form of attribution to be established (Iyengar et al., 2009), where one can track and quantify the contribution by a modeler through the influence  of their model or its components on a newly published model or one of its parts.  Related, the model lineage component will also allow readers to follow the progression of the work of a single investigator or lab.  This feature will likely be important in encouraging investigator cooperation with the new system (Thorisson, 2009).

\subsection*{Model Comparisons}

A very important component of this new form of publication will be the ability to quantitatively compare models. This applies both to the structural components of the model, as well as the parameters used to generate the reported modeling results. Because of the modularity of the CBI architecture, the system will be able to quickly identify what is the same and/or different between two models, both at the structural level, and also at the level of parameters. For example, a new Purkinje cell model might include a novel type of potassium channel, or might have significantly changed a parameter for a component common to two models. This kind of capability will allow readers to quickly identify and understand what is similar and different about models. 

\subsection*{Experimental Data}

It is not unusual for the same model to be used to explore different aspects of possible function, based on different types of data used or changes in the resolution (see for example, first three papers by De Schutter and Bower [1994a,b,c] on the Purkinje cell model). As an extension to model lineage we will also provide ways to include or link to the experimental data used to tune or test the model, thereby clearly distinguishing expected (tuned for) model behavior from emergent model behavior, and perhaps also starting to establish standard data sets for model validation.  The modularity of the system will allow us to also link to the large and growing number of projects involving experimental data storage and access (Kaspirzhny et al., 2002; Lucas et al., 2010; Neylon, 2009), providing a new mechanism for data extraction and annotation both serious issues in growing experimental databases (Cerrito and Cerrito, 2008).

\bibliographystyle{plain}
\bibliography{../tex/bib/pub-ref}

\end{document}
