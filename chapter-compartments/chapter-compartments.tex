\documentclass[12pt]{article}
\usepackage{verbatim}
\usepackage[dvips]{epsfig}
\usepackage{color}
\usepackage{url}
\usepackage[colorlinks=true]{hyperref}

\begin{document}

\section*{GENESIS: Documentation}

{\bf Related Documentation:}
% start: userdocs-tag-replace-items related-do-nothing
% end: userdocs-tag-replace-items related-do-nothing

\section*{References}

\href{http://en.wikipedia.org/wiki/Compartmental_modelling_of_dendrites}{Wikipedia: compartmental modelling.}

\href{http://web.cs.dal.ca/\~tt/fundamentals/slides/pdf slides/}{Fundamentals of Computational Neuroscience}


http://books.google.be/books?id=4PDsA1EVCx0C&pg=PA48&lpg=PA48&dq=computational+neuroscience+compartments&source=bl&ots=pyhHQFpIj7&sig=UENyi5TVUOrd-LE6t4WLrBP_zqQ&hl=nl&sa=X&ei=xsE4Ueb8C4LZOYqigKgH&ved=0CEUQ6AEwBA#v=onepage&q=computational%20neuroscience%20compartments&f=false


When neurons are not sufficiently electrotonically compact, a
single-compartment model is not adequate in most circumstances. In
this case, the response properties of neurons can be highly dependent
on the site of the applied synaptic input.


\bibliographystyle{plain}
\bibliography{../tex/bib/g3-refs.bib}

\end{document}
