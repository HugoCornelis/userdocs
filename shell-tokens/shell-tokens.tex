\documentclass[12pt]{article}
\usepackage[dvips]{epsfig}
\usepackage{color}
\usepackage{url}
\usepackage[colorlinks=true]{hyperref}

\begin{document}

\section*{GENESIS: Documentation}

\section*{GENESIS Interactive Shell Tokens}

The tokens recognized by the GENESIS shell fall into four broadly different categories:

\begin{enumerate}

\item {\bf Some tokens define sections in an NDF description file.} The most important are {\tt IMPORT}, {\tt PRIVATE\_MODELS}, and {\tt PUBLIC\_MODELS}. Models defined in one section of an NDF file can be forwarded to the next section with the {\tt ALIAS} token. To obtain a complete list of the {\it section} tokens enter ``{\tt list sections}''. To learn more about the NDF file format see \href{../ndf-file-format/ndf-file-format.pdf}{The Neurospaces Description Format}.

\item {\bf Many tokens have the semantics of a specific physical process.} Examples include, {\tt CHANNEL} for a Hodgkin-Huxley type conductance and {\tt SEGMENT} for a cylindrical cable equation. All these tokens map to a single hierarchical element tree (see for example, ???). This allows for grouping into parent--child data structures of model parameters. To obtain a complete list of the {\it physical} tokens enter ``{\tt list physical}''.

\item {\bf Some tokens have the semantics of a function.} This is a special case of the previous item. Examples include, {\tt RANDOMIZE} to randomize a value, and {\tt FIXED} to calculate a constant value that cannot be scaled. A special case is {\tt SERIAL} that returns the unique identifier  of the current symbol.  To obtain a complete list of the {\it  function} tokens enter ``{\tt list functions}''.

\item {\bf Other tokens exist to define the structure of a model.} For example, the {\tt CHILD} token adds a new (reference to a) child of an active symbol. The {\tt PARAMETERS} token defines attributes for symbols with (name, value) pairs. The {\tt BINDABLES} token allows variables to be shared between different components of a model. To obtain a complete list of the {\it structure} tokens enter ``{\tt list structure}''.

\end{enumerate}

\end{document}