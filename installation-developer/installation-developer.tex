\documentclass[12pt]{article}
\usepackage[dvips]{epsfig}
\usepackage{color}
\usepackage{url}
\usepackage[colorlinks=true]{hyperref}

\begin{document}

\section*{GENESIS: Introduction}

{\bf Related Documentation:} \\
\href{../genesis-system/genesis-system.tex}{\bf GENESIS\,System}
% start: userdocs-tag-replace-items related-do-nothing
% end: userdocs-tag-replace-items related-do-nothing

\section*{GENESIS Developer Installation}

\subsection*{Prerequisites}

There are several prerequisites to the construction of a GENESIS development environment. These include:

\begin{itemize}

\item {\bf Required tools:}\,{\it gdb},\,{\it gcc}, and\,{\it (x)Emacs} with the\,{\it elisp} packages. \\
{\bf Note:} The GENESIS \href{../ndf-file-format/ndf-file-format.tex}{\bf NDF\,file\,format} has a special\,{\it elisp} package for use with\,{\it (x)Emacs} that is installed as a part of the {\bf Model\,Container} package.
\item \href{../version-control/version-control.tex}{{\bf Version control:}} Monotone is used as a statically linked binary (currently version 0.45). As it is a distributed version control system, developers can use,\,{\it monotone} to maintain and exchange their code repositories and executables. Monotone installation is supported by the\,{\it Developer Package}. 
\item {\bf  The\,{\it DeveloperPackage}:} Can optionally be used to do initial directory layout. Importantly, the\,{\it DeveloperPackage} requires the correct directory layout. See the script {\it neurospaces\_create\_directories}.

\end{itemize}

{\bf Note:} Compiling from source will only produce executables and libraries for the host architecture. For example if you have a 64-bit system and 32-bit {\bf Perl}, the compiled {\bf SWIG} libraries will not load, giving an error citing "wrong architecture." So when compiling on a machine make sure that {\bf Perl}, {\bf Python} and all of their loadable modules and libraries are of the same architecture as the host machine.

\subsection*{Download Code for GENESIS Developer Installation}

Source code for GENESIS can be found at: \href{http://repo-genesis3.cbi.utsa.edu/src/}{\bf GENESIS package archive}.

The easiest way to get the latest version of the GENESIS source code is via the  \href{../developer-package/developer-package.tex}{\it DeveloperPackage}.

\subsection*{Software Dependencies}

Prior to installing GENESIS modules on your local computer you will need to install the necessary software dependencies. A general list of software dependencies can be found in \href{../genesis-dependencies/genesis-dependencies.tex}{\bf GENESIS\,Dependencies}. More specific dependencies are given in the installation documentation of the supported operating systems. For an overview see \href{../genesis-installation/genesis-installation.tex}{\bf GENESIS\,Installation}.

\subsection*{Prerequisite Operations}

Once you have software dependencies like {\bf monotone} installed you must set up some things on your machine in order to use the {\it Developer Package}. 

\subsubsection*{Setting up your monotone key}

In order to use monotone the user needs to create their monotone key using the command:

\begin{verbatim}
  mtn genkey email@myaddress.com
\end{verbatim}

 The email you use does not need to be a valid email address, it is simply an identifier that other developers using the repository will use to associate with your checkins. Even if you are not going to be checking in any code you still need a key to create and update your workspace. After entering the command it will prompt you for a passphrase, that will be used as a password whenever you perform any operations on the repository.
 After creating your key with your desired email identifier you will have a new directory in your home directory called {\it .monotone}. To keep from having to enter your passphrase every time you perform any monotone operation, you may automate it by creating a file in {\it .monotone} called {\it monotonerc}. In your {\it monotonerc} file place this code:
 
\begin{verbatim}
function get_passphrase(keypair_id)	
   return "mypassword"
end
\end{verbatim}

where the text {\it mypassword} is your passphrase, that you created earlier. 


\subsection*{Installing the {\it DeveloperPackage}}

Download the latest version of the {\it DeveloperPackage}, available from the \href{http://repo-genesis3.cbi.utsa.edu/src/}{\bf GENESIS package archive}.  It is called {\it developer-release-label.tar.gz}, where {\it release-label} is the current release identifier. ({\bf Note:} If downloading via your browser, do not unpack the package during the download as it will be placed into its own subdirectory).
\begin{enumerate}
   \item Change to the directory where you downloaded the file.
   \item Unpack archive: ``{\tt tar xfz developer-release-label.tar.gz}''.
   \item Change to the directory with the content of the archive: ``{\tt cd developer-release-label}''.
   \item Configure the {\it DeveloperPackage}: ``{\tt ./configure}''.
   \item Compile and install the {\it DeveloperPackage}: ``{\tt sudo make install}'' (requires administrator privileges).
\end{enumerate}

\subsection*{General Developer Installation Procedure}

Following manual installation of the \href{../developer-package/developer-package.tex}{\it DeveloperPackage}, the generalized scheme to download and install the available GENESIS \href{../reserved-words/reserved-words.tex}{\bf Components} is via the following commands:

\begin{enumerate}
\item {\bf Create the correct directory layout:}
\begin{verbatim}
	neurospaces_create_directories
\end{verbatim}
  
\item {\bf Pull all the code from the remote repositories:}
\begin{verbatim}
	neurospaces_pull
\end{verbatim}

\item {\bf Create workspace with up to date source code:}
\begin{verbatim}
	neurospaces_update
\end{verbatim}

\item {\bf Generate make files:}
\begin{verbatim}
	neurospaces_configure
\end{verbatim}

\item {\bf Clean all workspaces:} (Note: This step is optional, it is used when upgrading {\it DeveloperPackage}, or following a build to remove previously built files).
\begin{verbatim}
	neurospaces_clean
\end{verbatim}

\item {\bf Compile and install all the packages:}
\begin{verbatim}
	neurospaces_install
\end{verbatim}

\item {\bf Optionally check if the installation is successful:}
\begin{verbatim}
	neurospaces_check
\end{verbatim}

\end{enumerate}

\section*{Accessing the Version Control Repository from the {\it Developer Package}}

%The version control repository is accessible from the {\it DeveloperPackage}. Important for this feature to work is to have the correct directory layout on your developer machine.

%In summary, what you want to do is this: {\it pull} the set of default packages from the servers:
%\begin{verbatim}
%   neurospaces_build --repo-pull repo-genesis3.cbi.utsa.edu --repo-co \
%      --verbose --developer --directories-create --no-configure \
%      --no-compile --no-install
%\end{verbatim}
%This command has options that inhibit the default actions of configuration, compilation and installation. If you also want to compile and install in just one run, simply omit those options:

%\begin{verbatim}
%   neurospaces_build --repo-pull repo-genesis3.cbi.utsa.edu --repo-co \
%      --verbose --developer --directories-create
%\end{verbatim}

There are developer friendly frontends to the {\it neurospaces\_build} script in the {\it DeveloperPackage}:
\begin{itemize}
\item[]{\it neurospaces\_serve} Starts serving the source code repositories such that other people can {\it pull} and {\it sync} to your machine ({\bf Note} This locks all your databases).
\item[]{\it neurospaces\_pull}\,\,\,Download the source code from a repository.
\item[]{\it neurospaces\_status}\,\,\,Check for local source code modification (no network required).
\item[]{\it neurospaces\_sync}\,\,\,Synchronize local source code modification with a repository.
\item[]{\it neurospaces\_update}\,\,\,Update local source code using the repositories locally stored on your computer (so this is a local operation). 
\end{itemize}

\section*{Global Resources}

A GENESIS 3 developer keeps developer resources in the directory

\begin{verbatim}
   $HOME/neurospaces_project/
\end{verbatim}

In this directory we find a subdirectory for each of the
\href{../genesis-components/genesis-components.tex}{\bf software\,components} currently under development.

Besides the directories associated with software components, we also
find directories for local documentation ({\bf
  \$HOME/neurospaces\_project/docs}) and for source code repositories
for the \href{../version-control/version-control.tex}{\bf version\,control\,systems} 
that require a dedicated directory for this.  {\bf
  \$HOME/neurospaces\_project/\_MTN} is such a directory.

A personal text file {\bf TODO.txt} keeps track of items that wait
local implementation, but maybe are only indirectly related to GENESIS
3 development.  This file is commonly formatted according to the
\href{http://www.gnu.org/software/emacs/}{\bf Emacs}
\href{http://www.emacswiki.org/emacs/OutlineMode}{\bf outline\,major\,mode}.
The Emacs outline mode can be converted to latex and HTML when needed.


\end{document}

