\documentclass[12pt]{article}
\usepackage{verbatim}
\usepackage[dvips]{epsfig}
\usepackage{color}
\usepackage{url}
\usepackage[colorlinks=true]{hyperref}

\begin{document}

\section*{GENESIS: Documentation}

{\bf Related Documentation:}
% start: userdocs-tag-replace-items related-do-nothing
% end: userdocs-tag-replace-items related-do-nothing

\section*{De Schutter: Purkinje Cell Model}

\subsection*{CURRENTS IN THE MAIN DENDRITE}

The currents in the main dendrite were a mixture of somatic and dendritic
currents, and like the membrane potential they combined
the characteristics of both parts of the cell in their dynamic
changes (\href{../pub-purkinje-deschutter1-fig-11/pub-purkinje-deschutter1-fig-11.tex}{\bf Fig.\,11{\it B}}). Two aspects are noteworthy. The CaP
channel in the main dendrite was activated during the somatic
action potential and thus contributed to the broadened
base of these Na$^+$ spikes. Correspondingly, there were
small activations of the Ca$^{2+}$ -activated K$^+$ currents during
each somatic spike. However, the delayed rectifier was a
more important current in the main dendrite during the
depolarizing spike burst than the K2 current.

\bibliographystyle{plain}
\bibliography{../tex/bib/g3-refs.bib}

\end{document}
