\documentclass[12pt]{article}
\usepackage{verbatim}
\usepackage[dvips]{epsfig}
\usepackage{color}
\usepackage{url}
\usepackage[colorlinks=true]{hyperref}

\begin{document}

\section*{GENESIS: Documentation}

{\bf Related Documentation:}
% start: userdocs-tag-replace-items related-do-nothing
% end: userdocs-tag-replace-items related-do-nothing

\section*{De Schutter: Purkinje Cell Model}

\subsection*{Introduction}

The cerebellar Purkinje cell is one of the largest and most
complex neurons in the mammalian nervous system. Purkinje
cells have very active dendrites, generating massive
Ca2+ signals in response to synaptic input\,\cite{Miyakawa:1992ly, Sugimori:1990kx}. 
Further, the unusual isoplanar
anatomic organization of this cell�s dendrite has
been conserved to a remarkable degree through evolution
\,\cite{Ito:1984uq}, suggesting that this specific morphology is essential
to the function of the Purkinje cell.

In addition to its anatomic and physiological complexity
and uniqueness, Purkinje cell activity also constitutes the
sole output of the cerebellar cortex. Although the precise
computational role of the cerebellum is not yet known, it is
clear that understanding the physiology of the Purkinje cell
will be an essential part in unraveling the function of the
cerebellar cortex as a whole. Given the complexity of this
cell, we believe that computer modeling techniques are necessary
to explore and analyze completely the properties of
Purkinje cells.

In this paper we describe a large, detailed compartmental
model\,\cite{Perkel:1981vn, rall62:_theor, W:1964oq} of the Purkinje
cell based on real Purkinje cell morphology, which includes
10 active voltage-dependent ionic conductances that have
so far been demonstrated to exist in these neurons. Model
parameters were established using the results of several recent
voltage-clamp studies of conductances in Purkinje
cells\,\cite{Gahwiler:1989fk, Hirano:1989uq, Kaneda:1990ys, Regan:1991ly, Wang:1991bs}.
We tested whether the model was robust to changes in the
densities of individual channels. The model was then used
to explore the ionic mechanisms underlying the complex
response properties of Purkinje cells to current-clamp conditions
in vitro\,\cite{R:1980ly, R:1980pi}, which confirmed
several postulates made by\,\cite{Llinas:1992rq}. 
In addition, modeling results focus attention on the
importance of low-threshold Ca$^{2+}$-activated K$^+$ channels
in controlling dendritic excitability. Finally, the model was
used to explore the likely accuracy of whole-cell voltage
clamping experiments in this neuron.

Although several Purkinje cell models have previously
been described in the literature, most have not included
voltage-dependent conductances in the dendrites
\,\cite{Llinas:1976vn, Rapp-M:1992kx, Rapp-P:1994qf, P:1985mb}. 
Models that did include ionic conductances in the
dendrites either did not include all channels now known to
exist\,\cite{Pellionisz:1977zr} or were applied to a very
limited set of questions\,\cite{Bush:1991ly}. In addition
to exploring the mechanisms underlying this cells
response to current injection, the current model also lays
the groundwork for modeling and experimental studies of
Purkinje cell responses to synaptic input\,\cite{deschutter94:_purkin_ii}.

\bibliographystyle{plain}
\bibliography{../tex/bib/g3-refs}

\end{document}
