\documentclass[12pt]{article}
\usepackage[dvips]{epsfig}
\usepackage{color}
\usepackage{url}
\usepackage[colorlinks=true]{hyperref}

\begin{document}

\section*{GENESIS: Documentation}

\subsection*{Common Filename Suffixes used by GENESIS}

GENESIS knows about many different file formats.  These file formats
are commonly distinguished by appending a common suffix to the
filename.  GENESIS knows about the following suffixes:

\begin{enumerate}

\item {\bf .evl} A file that contains the time stamps of a series of events.
\item {\bf .g3p} A GENESIS 3 project file.
\item {\bf .g3} A file format that contains gshell commands.
\item {\bf .g} A file format that contains GENESIS 2 SLI commands.
\item {\bf .ndf} The Neurospaces Description Format used by the model-container to describe models.
\item {\bf .nms} A binary file format that contains the state of a simulation.
\item {\bf .nxf} An XML representation of the {\bf .ndf} format.  Both the {\bf .nxf} and the {\bf .ndf} formats describe the same type of information.
\item {\bf .out} A file with simulation output.
\item {\bf .per} A file format that describes the state of the G-Tube, describing a perspective on a simulation project.
\item {\bf .p} A morphology file of the GENESIS 2 SLI.
\item {\bf .ssp} A configuration file for the SSP scheduler.

\end{enumerate}

\subsection*{Common Filename Suffixes recognized by GENESIS}
As part of the GENESIS Documentation System, GENESIS recognizes a number of file name suffixes. If included as a tag in a {\it descriptor.yml} file, any files with the following suffixes will be incorporated into the documentation system if the source files are present in the top level documentation folder. The default is the {\bf .tex} \LaTeX\,\,format. See \href{../document-create/document-create.tex}{Document Creation} for further details.

\begin{enumerate}

\item {\bf .tex} GENESIS Documentation System default (requires no tag).
\item {\bf .eps} Encapsulated postscript.
\item {\bf .html} Hypertext markup language.
\item {\bf .pdf} Portable document format.
\item {\bf .png} Portable network graphics.
\item {\bf .ps} Postscript.

\end{enumerate}

\end{document}

%%% Local Variables: 
%%% mode: latex
%%% TeX-master: t
%%% End: 
