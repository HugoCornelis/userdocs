\documentclass[12pt]{article}
\usepackage{verbatim}
\usepackage[dvips]{epsfig}
\usepackage{color}
\usepackage{url}
\usepackage[colorlinks=true]{hyperref}

\begin{document}

\section*{GENESIS: Documentation}

{\bf Related Documentation:}
% start: userdocs-tag-replace-items related-do-nothing
% end: userdocs-tag-replace-items related-do-nothing

\section*{GENESIS Introduction}

The GEneral NEural SImulation System was designed from the outset
to support the biologically realistic modeling of neural systems at levels
of scale ranging from subcellular components and biochemical reactions to
complex models of single neurons, simulations of large networks, and
systems-level models.  GENESIS 3 (G-3) is a complete redesign of GENESIS,
with a modular plug-in architecture that facilitates the incorporation of
new types of model components at any level of scale.  This design allows
the direct real time interaction or embedding of models (local or remote)
at different levels of scale, and also provides a framework for
collaborative modeling and software tool development.

This page provides links to a variety of documentation that will help to get you started
using GENESIS 3. An overview of the GENESIS software platform design can be found here:

\begin{itemize}
\item \href{../genesis-overview/genesis-overview.tex}{\bf An Overview of GENESIS}
\item \href{../user-intro/user-intro.tex}{\bf Users Introduction to GENESIS}
\end{itemize}


\subsection*{The GENESIS Documentation System}
\begin{itemize}
\item \href{../documentation-homepage/documentation-homepage.tex}{\bf Documentation Home Page}
\item \href{../documentation-overview/documentation-overview.tex}{\bf Documentation Overview}
\item \href{../index.html}{\bf Listing of All Documents}
\item \href{../document-notation/document-notation.tex}{\bf Document\,Notation}
\item \href{../reserved-words/reserved-words.tex}{\bf Reserved\,Words}
\item \href{../common-suffixes/common-suffixes.tex}{\bf Filename Suffixes Recognized by GENESIS} 
\end{itemize}

\subsection*{The GENESIS Interactive Shell}

The following documents provide useful information about the GENESIS Interactive Shell or {\bf G-Shell}. They include:

\begin{itemize}
\item \href{../gshell/gshell.tex}{\bf G-Shell} (Introduction to the GENESIS Shell)
\item \href{../shell-tokens/shell-tokens.tex}{\bf G-Shell\,Tokens}
\item \href{../tutorial1/tutorial1.tex}{\bf Introductory\,G-Shell\,Tutorial (Tutorial 1)}
\item \href{../example-script1/example-script1.tex}{\bf Example\,of\,GENESIS\,Declarative\,Scripting}
\end{itemize}

The Python shell and API provided by the SSPy component of GENESIS
provides a way to interact using Python.  Useful documents include:

\begin{itemize}
\item \href{../sspy/sspy.tex}{\bf SSPy -- Simple Scheduler in Python}
\item \href{../g3-python/g3-python.tex}{\bf GENESIS 3 in Python}
\item \href{../tutorial-python-scripting/tutorial-python-scripting.html}{\bf
       Creating GENESIS 3 Simulations with Python}
\end{itemize}

\end{document}
