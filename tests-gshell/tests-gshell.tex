\documentclass[12pt]{article}
\usepackage{verbatim}
\usepackage[dvips]{epsfig}
\usepackage{color}
\usepackage{url}
\usepackage[colorlinks=true]{hyperref}

\begin{document}

\section*{GENESIS: Documentation}

{\bf Related Documentation:}
% start: userdocs-tag-replace-items related-do-nothing
% end: userdocs-tag-replace-items related-do-nothing

\section*{GENESIS Interactive Shell Regression Tests}

The GENESIS Interactive Shell, referred to as the {\bf G-Shell}, provides an interactive command line interface that replaces the GENESIS-2 Script Language Interpreter ({\bf SLI}).

For more details see the \href{../gshell/gshell.tex}{\bf GENESIS\,Interactive\,Shell} documentation.

\subsection*{Obtaining Regression Test Specifications}

This \href{http://www.neurospaces.org/neurospaces_project/gshell/tests/html/specifications/main.html}{\bf link} accesses all regression test specifications for the {\bf G-Shell}.

\subsection*{Regression Test Documentation}

{\bf G-Shell} regression test documentation includes:
\begin{itemize}
\item The startup command.
\item The input to the test.
\item The expected output of the test.
\end{itemize}
The regression tests give a good overview of {\bf G-Shell} functionality and test the integration between the \href{../gshell/gshell.tex}{{\bf G-Shell}}, \href{../model-container/model-container.tex}{\bf Model\,Container}, \href{../heccer/heccer.tex}{\bf Heccer}, and \href{../ssp/ssp.tex}{\bf SSP}.

\subsection*{Scope of Regression Tests}

Regression tests cover the following major functionalities of the {\bf G-Shell}:
\begin{itemize}
\item[]\href{http://www.neurospaces.org/neurospaces_project/gshell/tests/html/index.html}{\bf Specifications}

The NDF library; {\bf G-Shell} commands; help commands; errors and warnings; variables; element creation and deletion; model saving and loading; working with a multicompartment model; passive model simulation; the Purkinje cell model; checking and resetting models; voltage clamp models.

\end{itemize}

\end{document}
