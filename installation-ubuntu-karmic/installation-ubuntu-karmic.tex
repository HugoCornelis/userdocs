\documentclass[12pt]{article}
\usepackage{verbatim}
\usepackage[dvips]{epsfig}
\usepackage{color}
\usepackage{url}
\usepackage[colorlinks=true]{hyperref}

\begin{document}

\section*{GENESIS: Documentation}

{\bf Related Documentation:} \\
\href{../genesis-system/genesis-system.tex}{\bf GENESIS\,System}
% start: userdocs-tag-replace-items related-do-nothing
% end: userdocs-tag-replace-items related-do-nothing

\section*{Developer Installation Ubuntu (Karmic)}

Installing GENESIS on Ubuntu requires executing the following major steps. {\bf Note}: Installation requires administrator privileges.
\begin{itemize}
   \item[] Prepare and upgrade the system software.
   \item[] Download and install the {\it DeveloperPackage}.
   \item[] Install the software packages.
   \item[] Check if the installation was successful. 
\end{itemize}

\subsection*{Prepare and upgrade the system software}

\begin{itemize}
\item[] Install the following packages using the Start Menu -$>$
  System Settings -$>$ Add and Remove Software -$>$ Software
  Management menu.  Always follow the option to install additional
  dependencies.
      \begin{itemize}
         \item libyaml-perl
         \item libexpect-perl
         \item libinline-perl
         \item autoconf
         \item automake
         \item libc6-dev
         \item libncurses5-dev
         \item flex
         \item bison
         \item libperl-dev
         \item patch
         \item python-dev
         \item python-yaml
         \item swig
         \item libreadline5-dev
         \item libclone-perl
         \item libterm-readline-gnu-perl
         \item libxml-simple-perl
      \end{itemize}
   \item[] monotone-0.45 (available as a statically linked binary from \href{http://monotone.ca/}{http://monotone.ca/})
   \item[] Install the dependencies for running tests:
      \begin{itemize}
         \item perl -MCPAN -e ``{\tt install Test::More}''
         \item perl -MCPAN -e ``{\tt install Data::Utilities}''
         \item perl -MCPAN -e ``{\tt install File::Find::Rule}''
         \item perl -MCPAN -e ``{\tt install Digest::SHA}''
      \end{itemize}   
      {\bf Note:} The running of tests is optional, but strongly advised.
 \end{itemize}

\subsection*{Download and Install the {\it DeveloperPackage}}

\begin{enumerate}
   \item Download the latest version of the {\it DeveloperPackage}, available from \href{http://sourceforge.net/projects/neurospaces/files/}{Sourceforge}. It is called {\it developer-release-label.tar.gz}, where {\it release-label} is the current release identifier.
   \item Change to the directory where you downloaded the file.
   \item Unpack the archive by typing ``{\tt tar xfz developer-release-label.tar.gz}''.
   \item Change to the directory with the content of the archive by typing ``{\tt cd developer-release-label}''.
   \item Configure by typing ``{\tt ./configure}''.
   \item Compile by typing ``{\tt make}''.
   \item Install by typing ``{\tt make install}''. 
\end{enumerate}

\subsection*{Install software packages}

\begin{enumerate}
   \item Use the installer script to create the correct directory layout by typing ``{\tt neurospaces\_create\_directories}''.
   \item Pull the archives of the source code by typing ``{\tt neurospaces\_pull}''.
   \item Update the source code in the working directories by typing ``{\tt neurospaces\_update}''.
   \item Generate {\it make} files by typing ``{\tt neurospaces\_configure}''.
   \item Compile and install the software by typing ``{\tt neurospaces\_install}''.
\end{enumerate}

\subsection*{Check if the installation was successful}

This step is optional but strongly advised.

\begin{itemize}      
   \item[] Run tests of all the packages and save tester output to a file  by typing ``{\tt neurospaces\_check >/tmp/check.out 2>\&1}''.
   \item[] Check the output by typing ``{\tt less /tmp/check.out}''. Importantly, search for lines containing the string {\tt error\_count}.
\end{itemize}
    

\subsection*{Optional Dependencies}

\subsubsection*{G-Tube}

As the G-Tube is becoming more mature as the official G-3 GUI, it is
strongly recommendedthat you also install the dependencies of the G-Tube:

\begin{itemize}
\item mercurial
\item python-numpy
\item python-wxtools
\item python-wxgtk2.8
\item python-wxgtk2.8-dev
\item python-wxglade
\end{itemize}

\subsubsection*{Studio}

The following packages can optionally be installed to run the
graphical part of the \href{../studio/studio.tex}{\bf Studio}:
\begin{itemize}
\item libgtk2-perl
\item libgraphviz-perl
\item sdl-perl
\end{itemize}
After successful installation of the {\bf Studio}, issue the
shell command ``{\tt neurospaces cells/purkinje/edsjb1994.ndf --gui}'' to check that 
the graphical component works correctly.

\end{document}
