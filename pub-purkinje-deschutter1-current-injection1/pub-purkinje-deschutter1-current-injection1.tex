\documentclass[12pt]{article}
\usepackage{verbatim}
\usepackage[dvips]{epsfig}
\usepackage{color}
\usepackage{url}
\usepackage[colorlinks=true]{hyperref}

\begin{document}

\section*{GENESIS: Documentation}

{\bf Related Documentation:}
% start: userdocs-tag-replace-items related-do-nothing
% end: userdocs-tag-replace-items related-do-nothing

\section*{De Schutter: Purkinje Cell Model}

\subsection*{Tuning the model to replicate in vitro current injections}

The transitions between the firing of simple somatic spikes and the firing of dendritic spikes causes a break in the linearity of the frequency current ($f-I$) curve, with a second, shallower slope above this transition. Our $f-I$ curves (\href{../pub-purkinje-deschutter1-fig-6/pub-purkinje-deschutter1-fig-6.tex}{\bf Fig. 6$A$}) were remarkably similar to those published by\,\cite{R:1980ly} (their Fig. 5). In the model, dendritic spike firing frequency remained relatively constant (16--19\,Hz in both models) once the threshold was crossed and was rather insensitive to the current amplitude.

\subsection*{Replication of current injection results}

This paper principally concerns the simulation of Purkinje cell responses to current injection in the slice preparation\,\cite{Hounsgaard:1988nx, R:1980ly, R:1980pi, Llinas:1992rq}. We believe that overall the model does a good job of simulating these experimental results.

\subsubsection*{SOMATIC AND DENDRITIC FIRING PATTERNS}

As demonstrated in Figs. 3-7, both the somatic and dendritic firing patterns were reproduced well. In particular, at low-level current intensities the model fired only fast somatic spikes
after a delay. At higher current amplitudes these fast Na$^+$
spikes were interrupted by dendritic Ca$^{2+}$ spikes that were
caused by activation of the CaP channel. The model also
replicated the experimental $f-I$ curves well.

One aspect of Purkinje cells in slice preparations that we
have not addressed explicitly is the tendency for Purkinje
cells to become spontaneously active\,\cite{R:1980ly}. This could be achieved in the model by introducing
a bit of bias current. However, there are numerous ways
in which this could be done. For example, increasing the
amount of NaP or CaT channels would introduce such a
current, as would decreasing any of the K$^+$ currents or the
leak in the model. In other words, there might be many
ways in which Purkinje cells could become spontaneously
firing cells, because modulating the conductivity of any of
six different channels would be sufficient.

\bibliographystyle{plain}
\bibliography{../tex/bib/g3-refs.bib}

\end{document}
