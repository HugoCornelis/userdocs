\documentclass[12pt]{article}
\usepackage{verbatim,}
\usepackage[dvips]{epsfig}
\usepackage{color}
\usepackage{url}
\usepackage[colorlinks=true]{hyperref}

\begin{document}

\section*{GENESIS: Documentation}

{\bf Related Documentation:}
% start: userdocs-tag-replace-items related-do-nothing
% end: userdocs-tag-replace-items related-do-nothing

\section*{De Schutter \& Bower Purkinje Cell Model Lineages}

Here, we give the reported lineage of the various components in the De Schutter \& Bower Purkinje cell model:
\begin{itemize}
\item De Schutter E \& Bower JM (1994) An active membrane model of the cerebellar Purkinje cell I. Simulation of current clamp in slice. {\it Journal of Nerurophysiology}. {\bf 71}: 375--400.
\end{itemize}

\subsection*{Morphology}

\subsubsection*{Passive}

\begin{itemize}
   \item Llin\'as R \& Nicholson C (1976) Reversal properties of climbing fiber pontential in cat Purkinje cells: An example of a distributed synapse. {\it Journal of Neurophysiology}. {\bf 39}: 311-323.
   
   \item Rapp M, Segev I \& Yarom Y (1994) Physiology, morphology and detailed passive models of guinea-pig cerebellar Purkinje cells. {\it Journal of Physiology} (Lond.). {\bf 474}: 101--118.
   \begin{enumerate}
\item Purkinje cells (PCs) from guinea-pig cerebellar slices were physiologically characterized
using intracellular techniques. Extracellular caesium ions were used to linearize the
membrane properties of PCs near the resting potential. Under these conditions the average
input resistance, $R_N$, was 29 M�, the average system time constant, $\tau_0$, was 82\,ms and
the average cable length, $L_N$, was 0.59.
\item Three PCs were fully reconstructed following physiological measurements and staining
with horseradish peroxidase. Assuming that each spine has an area of 1$\mu$m$^2$ and that the
spine density over the spiny dendrites is ten spines per micrometre length, the total
membrane area of each PC is $\sim$150,000 $\mu$m$^2$, of which $\sim$100,000 $\mu$m$^2$ is in the spines.
\item Detailed passive cable and compartmental models were built for each of the three reconstructed
PCs. Computational methods were devised to incorporate globally the huge
number of spines into these models. In all three cells the models predict that the specific
membrane resistivity, $R_m$ of the soma is much lower than the dendritic $R_m$ (�500 and
�100,000 $\Omega$cm$^2$ respectively). The specific membrane capacitance, $C_m$, is estimated to be
1.5--2 $\mu$Fcm$^{-2}$ and the specific cytoplasm resistivity, $R_l$, is 250 $\Omega$cm.
\item The average cable length of the dendrites according to the model is 0.13$\lambda$, suggesting that
under caesium conditions PCs are electrically very compact. Brief somatic spikes, however,
are expected to attenuate 30-fold when spreading passively into the dendritic terminals.
A simulated 200\,Hz train of fast, 90\,mV somatic spikes produced a smooth 12\,mV steady
depolarization at the dendritic terminals.
\item A transient synaptic conductance increase, with a 1\,nS peak at 0.5 ms and a driving force
of 60\,mV, is expected to produce �20 mV peak depolarization at the spine head membrane.
This EPSP then attenuates between 200- and 900-fold into the soma. Approximately 800
randomly distributed and synchronously activated spiny inputs are required to fire the
soma.
\item The passive model of the PC predicts a poor resolution of the spatio-temporal pattern of
the parallel fibre input. An equally sized, randomly distributed group of 11\,\% of the
parallel fibres, activated within a time window of a few milliseconds, would result in
approximately the same composite EPSP at the soma.
   \end{enumerate}
   
   \item Shelton DP (1985) Membrane resistivity estimated for the Purkinje neuron by means of a passive computer model. {\it Neuroscience}. {\bf 14}: 111--131.
   
\end{itemize}
   
\subsubsection*{Active}

\begin{itemize}
   \item Pellionisz \& Llin\'as (1977)
   \item Bush \& Sejnowski (1991)
   \item De Schutter \& Bower (1994)
   \begin{enumerate}
\item A detailed compartmental model of a cerebellar Purkinje
   cell with active dendritic membrane was constructed. The model
was based on anatomic reconstructions of single Purkinje cells and
included 10 different types of voltage-dependent channels described
by Hodgkin-Huxley equations, derived from Purkinje cellspecific
voltage-clamp data where available. These channels included
a fast and persistent Na$^+$ channel, three voltage-dependent
K$^+$ channels, T-type and P-type Ca$^{2+}$ channels, and two types of
Ca$^{2+}$-activated K$^+$ channels.
\item The ionic channels were distributed differentially over three
zones of the model, with Na$^+$ channels in the soma, fast K$^+$ channels
in the soma and main dendrite, and Ca$^{2+}$ channels and Ca$^{2+}$-
activated K$^+$ channels in the entire dendrite. Channel densities in
the model were varied until it could reproduce Purkinje cell responses
to current injections in the soma or dendrite, as observed
in slice recordings.
\item As in real Purkinje cells, the model generated two types of
spiking behavior. In response to small current injections the
model fired exclusively fast somatic spikes. These somatic spikes
were caused by Na$^+$ channels and repolarized by the delayed rectifier.
When higher-amplitude current injections were given, sodium
spiking increased in frequency until the model generated
large dendritic Ca$^{2+}$ spikes. Analysis of membrane currents underlying
this behavior showed that these Ca$^{2+}$ spikes were caused by
the P-type Ca$^{2+}$ channel and repolarized by the BK-type Ca$^{2+}$-activated
K$^+$ channel. As in pharmacological blocking experiments,
removal of Na$^+$ channels abolished the fast spikes and removal of
Ca$^{2+}$ channels removed Ca$^{2+}$ spiking.
\item In addition to spiking behavior, the model also produced
slow plateau potentials in both the dendrite and soma. These
longer-duration potentials occurred in response to both short and
prolonged current steps. Analysis of the model demonstrated that
the plateau potentials in the soma were caused by the window
current component of the fast Na$^+$ current, which was much
larger than the current through the persistent Na$^+$ channels.
Plateau potentials in the dendrite were carried by the same P-type
Ca$^{2+}$ channel that was also responsible for Ca$^{2+}$ spike generation.
The P channel could participate in both model functions because
of the low-threshold K$_2$-type Ca$^{2+}$-activated K$^+$ channel, which
dynamically changed the threshold for dendritic spike generation
through a negative feedback loop with the activation kinetics of
the P-type Ca$^{2+}$ channel.
\item These model responses were robust to changes in the densities
of all of the ionic channels. For most of the channels, modifying
their densities by factors of 22 resulted only in left or right
shifts of the frequency-current curve. However, changes of $>$20\%
to the amount of P-type Ca$^{2+}$ channels or of one of the Ca$^{2+}$-activated
K$^+$ channels in the model either suppressed dendritic spikes
or caused the model to always fire Ca$^{2+}$ spikes. Modeling results
were also robust to variations in Purkinje cell morphology. We
simulated models of two other anatomically reconstructed Purkinje
cells with the same channel distributions and got similar
responses to current injections.
\item The model was used to compare the electrotonic length of
the Purkinje cell in the presence and absence of active dendritic
conductances. The electrotonic distance from soma to the tip of
the most distal dendrite increased from 0.57 X in a passive model
to 0.95 X in a quiet model with active membrane. During a dendritic
spike generated by current injection the distance increased
even more, to 1.57 X.
\item Finally, the model was used to study the probable accuracy
of experimental voltage-clamp data. Whole-cell patch-clamp conditions
were simulated by blocking most of the K$^+$ currents in the
model. The increased electrotonic length due to the active dendritic
membrane caused space clamp to fail, resulting in membrane
potentials in proximal and distal dendrites that differed critically
from the holding potential in the soma.
\end{enumerate}
\end{itemize}

\subsection*{Channels}

\begin{itemize}

\item {\bf Fast sodium current ($Na_F$):}
   \begin{itemize}
      \item Hodgkin AL \& Huxley AF (1952) A quantitative description of membrane current and its application to conduction and excitation in nerve. {\it Journal of Physiology} (Lond.). {\bf 117}: 500--544.
      \begin{quote}
       This article concludes a series of papers concerned with the flow of electric
current through the surface membrane of a giant nerve fibre (Hodgkin,
Huxley \& Katz, 1952; Hodgkin \& Huxley, 1952 a-c). Its general object is to
discu the results of the preceding papers (Part I), to put them into
mathematical form (Part II) and to show that they will account for conduction
and excitation in quantitative terms (Part III).
      \end{quote}
      
      \item G\"ahwiler BH \& Llano I (1989) Sodium and potassium conductances in somatic membranes of rat Purkinje cells from organotypic cerebellar cultures. {\it Journal of Physiology} (Lond.). {\bf 417:} 105--122.
      \begin{enumerate}
         \item The somatic voltage-gated conductances of Purkinje cells in organotypic
cultures (Giihwiler, 1981) were studied using the outside-out patch recording
configuration of the patch-clamp technique (Hamill, Marty, Neher, Sakmann \&
Sigworth, 1981).
         \item When activated by step depolarizations, the tetrodotoxin-sensitive voltagedependent
Na+ current presented two distinct phases: an initial surge of inward
current fluctuations which activates rapidly upon pulse onset and decays within
20-40\,ms, and a later phase in which discrete bursts of single-channel activity are
interspersed with silent periods.
         \item Ensemble fluctuation analysis of the current fluctuations during the early phase
of the Na$^+$ current and measurements of single channels during both early and late
phases indicate that a single type of Na$^+$ channel can account for both phases of the
Na$^+$ current. This channel has an elementary current amplitude of -2\,pA at
-40\,mV. This amplitude did not vary significantly between -60 and -20\,mV. The
mean open time depended on membrane potential, increasing by a factor of three
between -60 and -20\,mV.
         \item The early component of the Na$^+$ current activated at a threshold of -60\,mV
and reached its maximum amplitude at - 20, mid-point for the activation curve
being -40\,mV. Times-to-peak current decreased with membrane potential, from
3-5\,ms at -60\,mV to 0-3\,ms at 0\,mV. The decay phase of the current presented two
exponential components, with time constants of 1-5 and 10\,ms at -40\,mV. The
steady-state inactivation curve had a mid-point at -75\,mV.
         \item The late component of the Na$^+$ current was observed in the voltage range from
-60 to -20\,mV, with a maximum at -40\,mV. Its maximum amplitude
corresponded to approximately 1-7\,\% of the peak amplitude of the early component.
         \item Macroscopic potassium currents were observed upon step depolarizations above
a threshold of -30\,mV. The currents activated in a voltage-dependent fashion,
times-to-peak decreasing with depolarization, and partially inactivated during 40\,ms
         depolarizing steps. Peak current amplitudes at any given membrane potential were
decreased by depolarizing the holding potential. The macroscopic properties of the
K$^+$ current varied from patch to patch.
         \item Two types of single-channel K$^+$ currents were observed during steady-state
depolarizations. The unitary current amplitudes were 2-7 and 10-4\,pA at\,30 mV,
corresponding to chord conductances of 28 and 90\,pS respectively. The 28\,pS channel
was observed as well upon step depolarizations, the time course of activation for the
resulting averaged currents being voltage dependent. Both 28 and 90\,pS channels
were completely inhibited by external application of 1\,mM tetraethylammonium
chloride.
      \end{enumerate}
      
      \item Hirano T \& Hagiwara S (1989) Kinetics and distribution of voltage-gated Ca, Na, and K channels on the somata of rat cerebellar Purkinje cells. {\it Pfl\"ugers Archiv}. {\bf 413}: 463--469.
      \begin{quote}
      Voltage gated ion channels on the somatic membrane
of rat cerebellar Purkinje cells were studied in dissociated
cell culture with the combination of cell-attached
and whole-cell variation of patch clamp technique. The
method enables us to record local somatic membrane current
under an improved space clamp condition. Transient (fast-inactivating)
and steady (slow-inactivating) Ca channel currents,
Na current, transient (fast-inactivating) and steady
(slow-inactivating) K currents, were observed. Transient and
steady Ca channel currents were activated at test potentials
more positive than -40\,mV and -20\,mV, respectively (in
50\,mM external Ba). The transient current inactivated with
a half-decay time of 10-30\,ms during maintained
depolarizing pulses, while the steady current showed relatively
little inactivation. Na current was activated at more
positive potentials than -60\,mV, and inactivated with a
half-decay time of less than 5\,ms. Transient and steady K
outward currents were recorded at more positive potential
than - 20\,mV and - 40\,mV, respectively. The transient current
inactivated with a half-decay time of 2-8\,ms. Ca, Na
and K channels showed different patterns of distribution on
the somatic membrane. Steady Ca channels tended to cluster
compared with Na or K channels.
      \end{quote}
   \end{itemize}

\item {\bf Persistent sodium current ($Na_P$):}

\begin{itemize}

   \item French CR, Sah P, Buckett KJ \& Gage P (1990) A voltage-dependent persistent sodium current in mammalian hippocampal neurons. {\it Journal of General Physiology}. {\bf 95}: 1139--1157.
   \begin{quote}
   Currents generated by depolarizing voltage pulses were recorded in
neurons from the pyramidal cell layer of the CA1 region of rat or guinea pig hippocampus
with single electrode voltage-clamp or tight-seal whole-cell voltageclamp
techniques. In neurons in situ in slices, and in dissociated neurons, subtraction
of currents generated by identical depolarizing voltage pulses before and after
exposure to tetrodotoxin revealed a small, persistent current after the transient
current. These currents could also be recorded directly in dissociated neurons in
which other ionic currents were effectively suppressed. It was concluded that the
persistent current was carded by sodium ions because it was blocked by TIX,
decreased in amplitude when extraceUular sodium concentration was reduced, and
was not blocked by cadmium. The amplitude of the persistent sodium current varied
with clamp potential, being detectable at potentials as negative as -70\,mV and
reaching a maximum at $\sim$-40\,mV. The maximum amplitude at -40\,mV in 21
cells in slices was -0.34\,$\pm$\,0.05 nA (mean\,$\pm$\,1 SEM) and -0.21\,$\pm$\,0.05 nA in 10
dissociated neurons. Persistent sodium conductance increased sigmoidally with a
potential between -70 and -30\,mV and could be fitted with the Boltzmann equation,
$g = g_{max}\{1 + exp[(V' - V)/k)]\}$. The average $g_{max}$, was 7.8\,$\pm$\,1.1\,nS in the 21
neurons in slices and 4.4\,$\pm$\,1.6\,nS in the 10 dissociated cells that had lost their
processes indicating that the channels responsible are probably most densely
aggregated on or close to the soma. The half-maximum conductance occurred
close to -50\,mV, both in neurons in slices and in dissociated neurons, and the
slope factor ($k$) was 5-9\,mV. The persistent sodium current was much more resistant
to inactivation by depolarization than the transient current and could be
recorded at $>$50\,\% of its normal amplitude when the transient current was completely
inactivated.
Because the persistent sodium current activates at potentials close to the resting
membrane potential and is very resistant to inactivation, it probably plays an
important role in the repetitive firing of action potentials caused by prolonged
depolarizations such as those that occur during barrages of synaptic inputs into
these cells.
   \end{quote}
   
   \item Kay AR \& Wong RKS (1987) Calcium current activation kinetics in isolated pyramidal neurons of the CA1 region of the mature guinea-pig hippocampus. {\it Journal of Physiology} (Lond.). {\bf 392}: 603-616.
   \begin{enumerate}
      \item Neurones were isolated from the CAI region of the guinea-pig hippocampus and
subjected to the whole-cell mode of voltage clamping, to determine the kinetics of
voltage-gated Ca$^{2+}$ channel activation.
      \item Isolated neurones had an abbreviated morphology, having lost most of the
distal dendritic tree during the isolation procedure. The electrical compactness of the
cells facilitates voltage clamp analysis.
      \item Block of sodium and potassium currents revealed a persistent current activated
on depolarization above -40\,mV, which inactivated slowly when the intracellular
medium contained EGTA. The current was blocked by Co$^{2+}$ and Cd$^{2+}$, augmented by
increases in Ca$^{2+}$ and could be carried by Ba$^{2+}$, suggesting that the current is borne
by Ca$^{2+}$.
      \item Steady-state activation of the Ca$^{2+}$ current was found to be well described by
the Boltzman equation raised to the second power.
      \item The open channel's current-voltage ($I-V$) relationship rectified in the inward
direction and was consistent with the constant-field equation.
      \item The kinetics of Ca$^{2+}$ current onset followed $m^2$ kinetics throughout the range of
its activation. Tail current kinetics were in accord with this model. A detailed
Hodgkin-Huxley model was derived, defining the activation of this current.
      \item The kinetics of the currents observed in this regionally and morphologically
defined class of neurones were consistent with the existence of a single kinetic class
of channels.
   \end{enumerate}
\end{itemize}

\item {\bf T-type calcium current ($Ca_T$):}

\item {\bf P-type calcium current ($Ca_P$):}

\item {\bf Anomalous rectifier ($K_h$):}

\item {\bf Delayed rectifier ($K_{dr}$):}

\item {\bf Persistent potassium current ($K_m$):}

\item {\bf A current ($K_A$):}

\item {\bf Calcium-activated potassium current  ($BK$):}

\item {\bf Calcium-activated potassium current  ($K_2$):}

\item {\bf Leak current ($I_{leak}$):}

\end{itemize}

\end{document}
