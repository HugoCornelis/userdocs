\documentclass[12pt]{article}
\usepackage[dvips]{epsfig}
\usepackage{color}
%e.g.  \textcolor{red,green,blue}{text}
\usepackage{url}
\usepackage[colorlinks=true]{hyperref}
\usepackage{scrtime}
 
\begin{document}

\section*{NineML: Documentation}

\section{List of NineML Documents}

\begin{itemize}
\item {\em NineML Overview} Enumerates the essential and existential
  design goals of NineML, its history, its present community and its
  future.
\item {\em NineML Application Field Overview} Defines the scope of the
  NineML standard and has a list of applications that support NineML.
\item {\em NineML Document List} The list of documents directly
  associated with NineML project.
\item {\em NineML User Layer Specification} The user-layer
  specification defines the syntactical XML constructs available to
  user-applications that define model components, model libraries and
  complete runnable models.
\item {\em NineML Abstraction Layer Specification} The semantics of
  all the model component units in the user-layer available to
  user-applications are defined in the abstraction layer
  specification.
\item {\em NineML Developer's Tutorial} The NineML developer's
  tutorial provides examples of user-layer and abstraction-layer
  documents and explains the relationship between them.  This
  explanation also defines a workflow for developers who want to
  extend the NineML standard with new semantics.
\item {\em NineML User's Tutorial} The user tutorial provides NineML
  document examples of models explained in peer-reviewed papers.
\end{itemize}

\end{document}

%%% Local Variables: 
%%% mode: latex
%%% TeX-master: t
%%% End: 
