\documentclass[12pt]{article}
\usepackage{verbatim}
\usepackage[dvips]{epsfig}
\usepackage{color}
\usepackage{url}
\usepackage[colorlinks=true]{hyperref}

\begin{document}


%\section*{GENESIS: Documentation}

%{\bf Related Documentation:}
% start: userdocs-tag-replace-items related-do-nothing
% end: userdocs-tag-replace-items related-do-nothing

\section*{Using models to collaborate, communicate, and publish:  Using the GENESIS-3 user-workflow to organize databases in the computational neurosciences.}

The GENESIS Development Team will moderate a discussion about the
relationship between user-workflows and databases in the computational
neurosciences at one day workshop at the CNS*2011 meeting in
Stockholm.

\vspace{3mm}

The G-3 simulator is a modular reimplementation of the GEneral NEural
SImulation System software platform that is organized around a
comprehensive user-workflow.  Besides organizing the user-experience
during simulation construction, the G-3 user-workflow also connects
the simulator to different types of databases.

After a short introduction this workshop will define the five step
GENESIS user-workflow and compare and contrast it with other workflows
used for simulation construction.  For each of the steps in the
GENESIS user-workflow we will discuss the relationship with
databases such as morphology databases (for instance
www.neuromorpho.org and www.neuroml.org) and experimental protocol
databases (for instance www.brainml.org).

Speakers will be invited to present about both database design and
simulator implementation.  It is expected that this workshop will
facilitate discussion between experimental result database architects
and curators, and software designers implementing simulation software.


%\subsection*{Relevance of computational models}

%Introduce why are we using computational models?

%Introduce G-3.

%Introduce G-3 user workflow and publication workflow, experiment vs
%computational loop in sciences.

%Maybe introduce superficially concepts like cbi-architecture and its
%relationship with G-3 and interoperability.


%\subsection*{Organizing a database for each step of the user-workflow}

%Integrating all the G-3 posters:
%\begin{itemize}
%\item Models: structures, parameters.
%\item Experiments: structures, parameters.
%\item Simulations: identifiers for implementation used, parameters.
%\item Result Analysis and Visualization: external.
%\item Iterators: parameterized scripts.
%\end{itemize}

%\subsection*{Documentation and Publication System}

%\subsection*{Workshop schedule}

%\subsubsection*{Morning Session}

%Introductions.

%Models, Experiments and Simulations.

%\subsubsection*{Afternoon Session}

%Results, Iterators, Publications.


\paragraph*{Presenters:}
The GENESIS development team and other contributors and collaborators.

\paragraph*{Organizers:}
Hugo Cornelis, Allan D. Coop, Mando Rodriguez, David Beeman, and James M. Bower. \\

\noindent {\bf Additional information:} \\
{\bf G-3:} {\scriptsize \href{http://www.genesis-sim.org/userdocs/documentation-homepage/documentation-homepage.html}{\bf http://www.genesis-sim.org/userdocs/documentation-homepage/documentation-homepage.html}} \\
{\bf Workshop:} {\scriptsize \href{http://www.genesis-sim.org/userdocs/cns11-workshop/cns11-workshop.html}{\bf http://www.genesis-sim.org/userdocs/cns11-workshop/cns11-workshop.html}}

\end{document}


%%% Local Variables: 
%%% mode: latex
%%% TeX-master: t
%%% End: 
