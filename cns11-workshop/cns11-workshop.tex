\documentclass[12pt]{article}
\usepackage{verbatim}
\usepackage[dvips]{epsfig}
\usepackage{color}
\usepackage{url}
\usepackage[colorlinks=true]{hyperref}

\begin{document}

%\section*{GENESIS: Documentation}

%{\bf Related Documentation:}
% start: userdocs-tag-replace-items related-do-nothing
% end: userdocs-tag-replace-items related-do-nothing

\section*{Using models to collaborate, communicate, and publish:  Using the GENESIS-3 user-workflow to organize databases in the computational neurosciences.}

\subsection*{Relevance of computational models}

Introduce why are we using computational models?

Introduce G-3.

Introduce G-3 user workflow and publication workflow, experiment vs
computational loop in sciences.

Maybe introduce superficially concepts like cbi-architecture and its
relationship with G-3 and interoperability.


\subsection*{Organizing a database for each step of the user-workflow}

Integrating all the G-3 posters:
\begin{itemize}
\item Models: structures, parameters.
\item Experiments: structures, parameters.
\item Simulations: identifiers for implementation used, parameters.
\item Result Analysis and Visualization: external.
\item Iterators: parameterized scripts.
\end{itemize}

\subsection*{Documentation and Publication System}

\subsection*{Workshop schedule}

\subsubsection*{Morning Session}

Introductions.

Models, Experiments and Simulations.

\subsubsection*{Afternoon Session}

Results, Iterators, Publications.


\paragraph*{Presenters:}
Hugo Cornelis, Allan D. Coop, Mando Rodriguez, David Beeman, and James M. Bower. \\

\noindent {\bf Additional information:} \\
{\bf G-3:} {\scriptsize \href{http://www.genesis-sim.org/userdocs/documentation-homepage/documentation-homepage.html}{\bf http://www.genesis-sim.org/userdocs/documentation-homepage/documentation-homepage.html}} \\
{\bf Workshop:} {\scriptsize \href{http://www.genesis-sim.org/userdocs/cns11-workshop/cns11-workshop.html}{\bf http://www.genesis-sim.org/userdocs/cns11-workshop/cns11-workshop.html}}

\end{document}


%%% Local Variables: 
%%% mode: latex
%%% TeX-master: t
%%% End: 
