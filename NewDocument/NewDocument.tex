\documentclass[12pt]{article}
\usepackage[dvips]{epsfig}
\usepackage{color}
%e.g.  \textcolor{red,green,blue}{text}
\usepackage{url}
\usepackage[colorlinks=true]{hyperref}

\begin{document}

\section*{GENESIS: Documentation}

\section*{New Document}

The two above headings are included in all documentation by default. The {\bf Documetnation} identifier may be replaced by a descriptor of the type of documentation you have developed, for example {\bf Introduction} or {\bf Tutorial}, etc. The {\bf New Document} heading should be replaced with the title of your document. Note that in your document all text below the {\bf New Document} heading should be replaced with your own content.

\subsection*{Help with \LaTeX\,Documentation}

\subsubsection*{Text color}
\
Text in your document can be colorized  with the command
\begin{verbatim}
    \textcolor{red}{text}
\end{verbatim}
for example \textcolor{red}{\bf This is IMPORTANT}, green and blue are also recognized arguments.

\subsubsection*{Hyperlinks}

Local hyperlinks are produced with the command
\begin{verbatim}
    \href{../NewDocument/NewDocument.tex}{NewDocument}
\end{verbatim}
for example \href{../NewDocument/NewDocument.tex}{NewDocument} references this document.
Note that these links may not work in your \LaTeX\,{\tt pdf} file viewer, but should be testable with the \href{http://get.adobe.com/reader/}{Acrobat Reader} or \href{http://www.adobe.com/products/acrobatpro/tryout.html}{Acrobat Professional} packages.


Remote hyperlinks use the same command in a slightly different way
\begin{verbatim}
    \href{http://www.genesis-sim.org/documentation}{GENESIS Documentation}
\end{verbatim}
for example see \href{http://www.genesis-sim.org/documentation}{GENESIS Documentation}. Note that these links do not wrap automatically at the end of a line even outside of the {\tt verbatim} environment.

\end{document}
