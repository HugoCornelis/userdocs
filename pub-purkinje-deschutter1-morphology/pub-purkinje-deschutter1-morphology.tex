\documentclass[12pt]{article}
\usepackage{verbatim}
\usepackage[dvips]{epsfig}
\usepackage{color}
\usepackage{url}
\usepackage[colorlinks=true]{hyperref}

\begin{document}

\section*{GENESIS: Documentation}

{\bf Related Documentation:}
% start: userdocs-tag-replace-items related-do-nothing
% end: userdocs-tag-replace-items related-do-nothing

\section*{De Schutter: Purkinje Cell Model}

\subsection*{Morphology}

The morphology of the models was based on a detailed light
microscopic reconstruction of horseradish peroxidase-filled
guinea pig Purkinje cells by M. Rapp of the Hebrew University of
Jerusalem, Israel\,\cite{Rapp-M:1992kx, Rapp-P:1994qf}. All simulations were
performed with a model using the morphology of {\it cell I} of Rapp et
al.\,\cite{Rapp-P:1994qf} unless otherwise noted. We applied the same shrinkage
factor of 10\,\% as did Rapp.

The final model described here contained 1,600 compartments.
As is standard for compartmental modeling\,\cite{rall62:_theor, W:1964oq},
this number was determined by the morphology of the cell and by
simulation requirements for numerical accuracy. In particular,
when active channels are used, it has been previously shown that
the electrotonic length of compartments should be $<$\,0.05\,$\lambda$\,\cite{W:1966ve}.

In our model, the passive electrotonic length of single compartments
ranged from 0.009 to 0.05\,$\lambda$. Because these electrotonic
lengths were short, we were able to use more computationally
efficient asymmetric compartments (this GENESIS object corresponds
to a three-element segment, as described by\,\cite{Segev-I:1985kl}. 
Simulations done with and without asymmetric compartments
confirmed that the difference in input resistance ($R_{\mbox N}$) and
system time constant ($\tau_0$) for a passive membrane model was
$<$\,1\,\%.

A rat Purkinje cell is known to have $\sim$\,150,000 dendritic spines\,\cite{Harvey:1991xz}. 
However, because the morphological
data used to build the current model were obtained with light
microscopic techniques, the location and shapes of dendritic
spines for the reconstructed cell were not known. Accordingly,
spines were not simulated. Instead, membrane surface was added
to the spiny dendritic compartments (defined as dendrites with a
diameter $\sim$\,3.17\,$\mu$m) to compensate for missing 
spines\,\cite{R:1989cr, Rapp-M:1992kx}. On the basis of published
electron microscopic (EM) reconstructions of rat Purkinje cell
spines\,\cite{M:1988bh} we assumed a density of 13
spines per 1\,$\mu$m dendritic length, with a membrane surface for a
spine of 1.33 $\mu$m$^2$\,\cite{M:1988bh}.

\bibliographystyle{plain}
\bibliography{../tex/bib/g3-refs}

\end{document}
