\documentclass[12pt]{article}
\usepackage{verbatim}
\usepackage[dvips]{epsfig}
\usepackage{color}
\usepackage{url}
\usepackage[colorlinks=true]{hyperref}

\begin{document}

\section*{GENESIS: Documentation}

{\bf Related Documentation:} \\
% start: userdocs-tag-replace-items related-do-nothing
% end: userdocs-tag-replace-items related-do-nothing

\section*{Introduction}

The {\tt createprojection} command of the {\bf gshell} allows
convenient creation of {\bf projections} between two {\bf populations}
of multicompartmental {\bf neurons}.  A {\bf projection} can have one
or more {\bf connections}.

\section*{Implementation}

The {\tt createprojection} command is a user-friendly wrapper around
the low-level command {\tt volumeconnect}.  Functionally these
commands are the same and their implementation is based on the {\bf
  GENESIS-2} (G-2) commands for the creation of connections (G-2
\href{http://www.genesis-sim.org/GENESIS/UGTD/Tutorials/genprog/textDoc/planarconnect.doc.txt}{
  {\tt planarconnect}}), and setting their weights (G-2
\href{http://www.genesis-sim.org/GENESIS/UGTD/Tutorials/genprog/textDoc/planarweight.doc.txt}{
  {planarweight}}) and delays (G-2
\href{http://www.genesis-sim.org/GENESIS/UGTD/Tutorials/genprog/textDoc/planardelay.doc.txt}{
  {planardelay}}).  It is instructive to read the
\href{http://www.genesis-sim.org/GENESIS/UGTD/Tutorials/genprog/net-tut.html}{overview}
of these G-2 functions to develop an understanding of their
functionality.


\section*{Specification}

An example invocation of the {\tt createprojection}
command\footnote{Note: the {\tt createprojection} command is still
  under development.  What is written here is correct at moment of
  writing, but may change in the future.  We expect only minor changes
  at most.} is shown below:

\begin{verbatim}
createprojection
    (
     {
      root => '/RSNet',
      projection => {
                     name => '/RSNet/projection',
#                    source => '../population', # optional
#                    target => '../population', # optional
                    },
      probability => '1.0',
      random_seed => '1212.0',
      source => {
                 context => '/RSNet/population',
                 include => {
                             type => 'box', # type => 'all' would remove the need for the line below
                             coordinates => [ '-1e10', '-1e10', '-1e10', '1e10', '1e10', '1e10', ],
                            },
                },
      target => {
                 context => '/RSNet/population',
                 include => {
                             type => 'ellipse',
                             coordinates => [ 0, 0, 0, $SEP_X * 1.2, $SEP_Y * 1.2, $SEP_Z * 0.5, ],
                            },
                 exclude => {
                             type => 'box',
                             coordinates => [ - $SEP_X * 0.5, - $SEP_Y * 0.5, - $SEP_Z * 0.5, $SEP_X * 0.5, $SEP_Y * 0.5, $SEP_Z * 0.5, ],
                            },
                },
      synapse => {
                  pre => 'spike',
                  post => 'Ex_channel',
                  weight => {
#                            type => 'fixed', # not sure yet, weight distribution example needed
                             value => $syn_weight,
                            },
                  delay => {
#                           type => 'fixed', # not sure yet, variable delay example needed
                            value => $SEP_X / $cond_vel,
#                           velocity => 0.5, # alternative to fixed delay?
                           },
                 },
     },
    );
\end{verbatim}

We call the items before the arrow signs ($=>$) dictionary keys or
keys.  The invocation distinguishes four main of the parameterization
of the command:

\begin{itemize}
\item Periphery configuration includes setting of the name of the
  network ({\bf root}), the name of the projection that will be
  created as part of the model (the {\bf name} key inside the {\bf
    projection} key).

  The ({\bf probability} and {\bf random\_seed} keys
\begin{verbatim}
      root => '/RSNet',
      projection => {
                     name => '/RSNet/projection',
#                    source => '../population', # optional
#                    target => '../population', # optional
                    },
      probability => '1.0',
      random_seed => '1212.0',
\end{verbatim}
\item The {\bf source} key defines the model components that are
  selected as candidates for outgoing connections.  In this case all
  the neurons in the model component with name {\bf /RSNet/population}
  are selected (more precisely all the neurons in a box shaped volume
  with size $2e10$ meters).
\begin{verbatim}
      source => {
                 context => '/RSNet/population',
                 include => {
                             type => 'box', # type => 'all' would remove the need for the line below
                             coordinates => [ '-1e10', '-1e10', '-1e10', '1e10', '1e10', '1e10', ],
                            },
                },
\end{verbatim}
\item The {\bf target} key defines the model components that are
  selected as candidates to receive connections.  In this case for
  each neuron found in the source region, an attempt will be made to
  pair it with any of the neurons found by the {\bf include} key,
  except for the neurons that are excluded by the {\bf excluded} key.
\begin{verbatim}
      target => {
                 context => '/RSNet/population',
                 include => {
                             type => 'ellipse',
                             coordinates => [ 0, 0, 0, $SEP_X * 1.2, $SEP_Y * 1.2, $SEP_Z * 0.5, ],
                            },
                 exclude => {
                             type => 'box',
                             coordinates => [ - $SEP_X * 0.5, - $SEP_Y * 0.5, - $SEP_Z * 0.5, $SEP_X * 0.5, $SEP_Y * 0.5, $SEP_Z * 0.5, ],
                            },
                },
\end{verbatim}
\item The {\bf verbatim} key defines the properties of the individual
  connections.  The {\bf pre} and {\bf post} keys specify the names of
  the pre- and post-synaptic components in the model respectively.
  Other names are excluded from the pairing process that instantiates
  the connections.
\begin{verbatim}
      synapse => {
                  pre => 'spike',
                  post => 'Ex_channel',
                  weight => {
#                            type => 'fixed', # not sure yet, weight distribution example needed
                             value => $syn_weight,
                            },
                  delay => {
#                           type => 'fixed', # not sure yet, variable delay example needed
                            value => $SEP_X / $cond_vel,
#                           velocity => 0.5, # alternative to fixed delay?
                           },
                 },
\end{verbatim}
\end{itemize}


\section*{Terminology}

\begin{itemize}
\item[] {\bf projection}: contains a set of {\bf connections}.  {\bf
    Connections} can optionally be embedded in {\bf
    connection\_groups}.
\item[] {\bf population}: a set of {\bf neurons}.
\item[] {\bf neuron}: a neuron model with or without a realistic
  morphology.
\item[] {\bf connection}: an attachment between a pre- and
  post-synaptic component in a model.  A {\bf connection} can hold a
  weight and a delay.
\end{itemize}

\end{document}
