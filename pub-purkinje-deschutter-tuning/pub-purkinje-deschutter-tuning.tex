\documentclass[12pt]{article}
\usepackage{verbatim}
\usepackage[dvips]{epsfig}
\usepackage{color}
\usepackage{url}
\usepackage[colorlinks=true]{hyperref}

\begin{document}

\section*{GENESIS: Documentation}

{\bf Related Documentation:}
% start: userdocs-tag-replace-items related-do-nothing
% end: userdocs-tag-replace-items related-do-nothing

\section*{De Schutter: Purkinje Cell Model}

\subsection*{Source}

De Schutter E \& Bower JM (1994) An active membrane model of the cerebellar Purkinje cell II. Simulation of synaptic responses. {\it Journal of Nerurophysiology}. {\bf 71}: 401--419.

\subsection*{Tuning the model to replicate in vitro current injections}

The final stage of constructing a complex realistic model involves tuning model parameters to replicate well-described properties of the neuron in question\,\cite{S:1993dz}. Once the model is tuned, additional questions can be explored by holding model parameters fixed and applying different types of inputs (see\,\cite{deschutter94:_purkin_ii}).

Here we  describe the results of optimizing the Purkinje cell model to replicate the characteristic firing patterns of Purkinje cells in slices during current injection in the soma or dendrites. This optimization, or tuning, primarily involved changing the densities and distributions of the different voltage sensitive channels and the decay time constant of the Ca$^{2+}$ concentrations until the model generated biologically realistic output. Due primarily to the work of\,\cite{R:1980ly, R:1980pi}, considerable information is available concerning the intracellular response properties of cerebellar Purkinje cells to current injection in vitro in the slice preparation in the presence and absence of different channel blockers. Accordingly, all the following statements about normal Purkinje cell behavior refer to their work unless a different study is cited.

\subsubsection*{NaF SPIKES}

In intracellular recordings in brain slice preparations some Purkinje cells are silent; others fire spontaneously. Our model is quiet without stimulation, with a stable resting potential of -68\,mV. For intrasomatic current injections below the spiking threshold, no somatic spikes could be generated regardless of the length of the current pulses (simulation results not shown). Low-amplitude current injections in Purkinje cell somata above this threshold caused firing of fast somatic spikes, with the relatively high minimum spike frequency (30--40\,Hz) characteristic of these cells (Figs. 3$A$, 4$A$ and $B$, and 6$A$). Somatic firing frequency increased relatively rapidly with current amplitude (range 22--250\,Hz).

In in vitro intracellular recordings there is often a delay between the onset of the current injection and the occurrence of the first somatic spike, which decreases with increasing amplitude of current. This variable delay was also present in the simulations (Figs. 3--5 and 6$B$). Note also that in Fig. 4$D$ we reproduce another phenomenon observed in slice. When the injected current amplitude became too high, somatic spiking saturated at a depolarized level of about -30\,mV. This saturation could be reversed either spontaneously (at this current amplitude) or by shutting off the current.

\subsubsection*{CALCIUM SPIKES}

As the current amplitude injected in a Purkinje cell increases, dendritic Ca$^{2+}$ spikes become apparent as depolarizing spike bursts in the soma. Again, the model reproduced this behavior reasonably well (Figs. 3$B$ and $C$, and 4$C$ and $D$). In experimental somatic recordings the transition between simple spikes and depolarizing spike bursts is abrupt, whereas the model often showed one or two small bursts before a full dendritic spike became apparent.

The transitions between the firing of simple somatic spikes and the firing of dendritic spikes causes a break in the linearity of the frequency current ($f-I$) curve, with a second, shallower slope above this transition. Our $f-I$ curves (Fig. 6$A$) were remarkably similar to those published by\,\cite{R:1980ly} (their Fig. 5). In the model, dendritic spike firing frequency remained relatively constant (16--19\,Hz in both models) once the threshold was crossed and was rather insensitive to the current amplitude.

\subsubsection*{PLATEAU POTENTIALS}

Current injection steps are often followed by a prolonged plateau potential. In our simulations this phenomenon was always observed after short current steps around firing threshold (Fig. 5) and sometimes after longer current steps also (Fig. 4A). In the model these plateau potentials resulted in depolarizations of between -49 and -55\,mV. Similar prolonged plateau potentials after the end of a current injection were found by\,\cite{R:1980ly} (compare their Fig. 5D with our Fig. 3A). However, in the experimental data these plateaus decay over a time course of $\sim$\,100 ms, whereas in the model they often did not decay at all (Fig. 5).

\subsubsection*{COMPARISON OF SOMATIC AND DENDRITIC RECORDINGS}

Paired intracellular somatic and dendritic recordings in slice have demonstrated that somatic action potentials do not propagate far into the dendrite. Thus recordings of the main dendrites of Purkinje cells often reveal somatic action potentials as only small wavelets. When the membrane potentials in the soma and large proximal dendrites were compared in the model, a similar result was observed (Figs. 3, 4, and 7). Further, using the model it was possible to demonstrate that the action potential did not propagate into spiny dendrites (a location that electrophysiologists cannot record from), even those relatively close to the soma (Figs. 3, 4, and 10$A$).

Paired intracellular somatic and dendritic recordings have also suggested that the Ca$^{2+}$ spiking, responsible for the depolarizing spike bursts observed in the soma during higher-amplitude current injections, has a dendritic origin. As shown in Figs. 3$B$ and $C$, and 4$C$ and $D$, this was also the case in the model.

\subsubsection*{RESPONSE TO DENDRITIC CURRENT INJECTIONS}

The model was also able to replicate previously reported responses to direct current injection into Purkinje cell dendrites (Fig. 7). As in the physiological recordings, low-amplitude current injection resulted only in attenuated somatic spikes, whereas these spikes were interrupted by dendritic Ca$^{2+}$ spikes with larger current injections. Under these circumstances the onset of the bursting behavior was almost immediate as compared with somatic recordings. Note also the burst occurring after the end of the 2.0\,nA current injection.

\subsubsection*{VARIABILITY IN PURKINJE CELL RESPONSE PROPERTIES}

In slice preparations, different Purkinje cells often have quite different firing patterns, especially as concerns the pattern of interacting somatic and dendritic spikes (Fig. 11 in\,\cite{R:1980ly} and the slope of the $f-I$ curve. However, a compartmental model is deterministic, i.e., always producing the same output for a given input. Variability in response properties comes about only as a result of varying parameters in the model.

In the case of the current model, we found that all of the reported variations in firing patterns could be obtained by simply changing the density of K$^+$ conductances in the soma and the main dendrite of the model. Figures 3 and 4 compare the firing of a model with low K$^+$ conductance (PM\,9) with a model with high K$^+$ conductance (PM\,10). Although these models were similar in all responses described so far, they showed richly different patterns of somatic firing that encompassed most of the observed ones. Compare, for example, the somatic responses toward the end of the current injections in Figs. 3, $B$ and $C$, and 4$D$ and after the current step in Fig. 3$D$. The higher K$^+$ channel densities in model PM\,10 made the soma also less excitable, resulting in a shallower $f-I$ curve (Fig. 6$A$).

\subsection*{Replication of current injection results}

This project principally concerns the simulation of Purkinje cell responses to current injection in the slice preparation\,\cite{Hounsgaard:1988nx, R:1980ly, R:1980pi}. We believe that overall the model does a good job of simulating these experimental results.

\subsubsection*{SOMATIC AND DENDRITIC FIRING PATTERNS}

As demonstrated in Figs. 3--7, both the somatic and dendritic firing patterns were reproduced well. In particular, at low-level current intensities the model fired only fast somatic spikes after a delay. At higher current amplitudes these fast Na$^+$ spikes were interrupted by dendritic Ca$^{2+}$ spikes that were caused by activation of the CaP channel. The model also replicated the experimental $f-I$ curves well.

One aspect of Purkinje cells in slice preparations that we have not addressed explicitly is the tendency for Purkinje
cells to become spontaneously active\,\cite{R:1980ly}. This could be achieved in the model by introducing a bit of bias current. However, there are numerous ways in which this could be done. For example, increasing the amount of NaP or CaT channels would introduce such a current, as would decreasing any of the K$^+$ currents or the leak in the model. In other words, there might be many ways in which Purkinje cells could become spontaneously firing cells, because modulating the conductivity of any of six different channels would be sufficient.

\subsubsection*{PLATEAU POTENTIALS}

The model also generated longer duration plateau-type potentials that have been shown to exist in this neuron\,\cite{Jaeger:1991kh, R:1980ly, Llinas:1992rq}. It has been demonstrated physiologically that current injection generates a plateau potential in the soma of the cell that is dependent on Na$^+$ conductances. The model generated Na$^+$-dependent somatic plateaus (Figs. 9$B$ and 11). However, it is debated in the literature whether such persistent Na$^+$ currents rely on a special channel (Nap) or depend on the NaF channel itself\,\cite{Alzheimer:1993fk, C-R-French:1990uq, Kay:1987ar}. In our model the somatic plateau potential was mainly
carried by NaF channels in the form of a so-called window current, although the model also included the experimentally demonstrated NaP channel\,\cite{Kay:1990kx}. The plateau-related current in the model flowed through NaF channels at a potential where the steady-state activation and inactivation curves (Fig. 2$A$) overlap. Although this mechanism was robust in the model, one argument against the existence of such a window current is that it is based on a wrong model of inactivation, because Na+ channel inactivation is probably not voltage dependent\,\cite{Aldrich:1987uq}. If this is in fact correct, then the term in our equations for steady-state inactivation would be meaningless. Although this possible inaccuracy in the Hodgkin-Huxley\,\cite{L:1952fv} model of inactivation means that our results are not conclusive, the model at least suggests that there may not be a need for separate NaP channels to explain somatic plateaus. In our simulations the NaP channel actually primarily affected the $f-I$ curve.

The model also generated dendritic plateau potentials (Figs. 9 and 11) that have recently been described in more
detail\,\cite{Jaeger:1991kh} and that may be particularly important during synaptic activation of the Purkinje
cell by peripheral stimuli\,\cite{Thompson:1991ac}. In the model these dendritic plateau potentials were carried
largely by the CaP channels, because the CaT channel inactivated too rapidly to play a major role. In this case again the plateau response resulted from a window current-like mechanism very similar to that found with the NaF channels in the soma, because the dendritic plateau current through the CaP channel occurred at a potential where the CaP channel does not inactivate completely. As a consequence these dendritic plateau potentials did not wane in the model as they do in slice preparations.

\subsubsection*{INTERACTION BETWEEN IONIC CURRENTS}

One of the principle benefits of modeling neurons at this level of detail is that the interactions between different currents can be explored. All too frequently experimentalists assign very specific roles to particular channel types without taking into account the often complex interactions between different conductances that are actually responsible.

In the current model there are numerous examples of this type of interaction. One of the most striking involves the
interaction of the noninactivating K$^+$ channels and the Na$^+$ and Ca$^{2+}$ channels responsible for the plateaus. At the somatic level these K$^+$ conductances served to counteract the depolarizing effect of the Na$^+$ channels, with their balance determining the voltage of the plateau potential (Fig. 11$A$).

In the dendrite the Ca$^{2+}$ -activated K$^+$ channels played a critical role in the dual function of the CaP channel in generating Ca$^{2+}$ spikes as well as dendritic plateaus. Because the K2 channel is sensitive to small changes in Ca$^{2+}$ concentration it could effectively increase the threshold for Ca$^{2+}$ spike generation by counteracting the activation of small numbers of CaP channels. The involvement of the K2 channel in these two very different forms of dendritic response, and the presence of other mechanisms in the Purkinje cell dendrite that through changes in the Ca2� concentration\,\cite{Llano:1991kx, Takei:1992ac} could activate this channel, suggest interesting possibilities for regulation of more global dendritic response properties by the K2
channel and other Ca$^{2+}$-inactivated K$^+$ channels that were not included in the model (like the SK or afterhyperpolarization (AHP) current\,\cite{Lancaster:1991ye}.

The model supports the suggestion that the channels responsible for generating plateau potentials are different in
the soma and dendrite\,\cite{R:1980ly}. However, examination of the model also makes clear that these different plateaus are not physiologically isolated. A somatic plateau potential was always accompanied by a dendritic
plateau and vice versa, because any depolarization will spread throughout the cell.

Other interactions between channels, like the activation of the anomalous rectifier during prolonged hyperpolarizations, followed by an activation of CaT channels during the rebound spike, were not explored in detail. The anomalous rectifier did not affect the repetitive firing properties of the model because it was completely deactivated beyond spiking threshold.

\subsubsection*{RESPONSE VARIABILITY}

Although any particular set of parameters in the model generated an identical (deterministic) output, we have also found that slight variations in parameters could produce the kinds of subtle variations seen in Purkinje cell recordings. Thus different levels of current injection, small changes in the densities of outward currents (PM9 versus PM10), or slight changes in morphology generated subtle changes in model output. Under these conditions the Purkinje cell model responded generally in the same way but also showed small variations in the details of its responses (e.g., in the sequence of action potentials). We believe this variability in the model is important, because
the specific objective of this effort was to represent the entire population of Purkinje cells rather than just one individual cell\,\cite{Bower:1992vn}.

One of the variable aspects of Purkinje cells that has been previously described involves the details of the $f-I$ curve\,\cite{R:1980ly}. Further, this curve can change during long recordings of the same cell (D. Jaeger,
personal communication). It is interesting that changes in the density of the delayed rectifier and noninactivating K$^+$ channels in the model alone could cause a lot of this variability, as can be seen by comparing Fig. 3 with Fig. 4 and in the $f-I$ curves of Fig. 6$A$. The results from patch-clamp recordings demonstrating that Purkinje cell Kdr channels are under metabolic control by protein kinase C, which attenuates
Kdr current\,\cite{Linden:1992ys} could provide one mechanism for short term changes in $f-1$ relationships. Further, our finding that somatic action potentials only repolarized completely when Kdr channels were added to the main dendrite showed that the Kdr channel also controls the coupling between the soma and more distal dendrite that may influence these $f-I$ relationships. This distribution of Kdr channels matches data obtained with voltage-sensitive dye imaging, where a decoupling between the soma and the more distal dendrite could be removed by blocking K$+$ channels\,\cite{Knopfel:1990zr}.

\bibliographystyle{plain}
\bibliography{../tex/bib/g3-refs}

\end{document}
