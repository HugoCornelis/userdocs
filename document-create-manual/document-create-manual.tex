\documentclass[12pt]{article}
\usepackage{verbatim}
\usepackage[dvips]{epsfig}
\usepackage{color}
\usepackage{url}
\usepackage[colorlinks=true]{hyperref}

\begin{document}

\section*{GENESIS: Documentation}

{\bf Related Documentation:}
% start: userdocs-tag-replace-items related-do-nothing
% end: userdocs-tag-replace-items related-do-nothing

\section*{Manual Document Creation}

Copy the\,{\it NewDocument} template folder to the name of the new document you want to make. Change the name of the \LaTeX\,\,\,file\,{\it NewDocument.tex} inside the renamed folder to the name of your new document. To do this replace the\,{\it NewDocument} part of the default source file name with the new name you have just given to the\,{\it NewDocument} folder.

This will result in both the documentation directory and the source file having the same name (with the exception of the {\tt .tex} extension of the source file). This is a requirement of the GENESIS Documentation System. For example, the name of the directory containing this document is {\it document-create-manual} and the name of the source file it contains is\,{\it document-create-manual.tex}.

Importantly, remember that the names of the descriptor file\,{\it descriptor.yml} and the directory\,{\it figures} located in the document folder are required by the GENESIS Documentation System and should not be changed.

\end{document}
