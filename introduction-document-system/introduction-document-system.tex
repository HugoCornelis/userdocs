\documentclass[12pt]{article}
\usepackage[dvips]{epsfig}
\usepackage{color}
%e.g.  \textcolor{red,green,blue}{text}
\usepackage{url}
\usepackage[colorlinks=true]{hyperref}

\begin{document}

\section*{GENESIS: Documentation}

\section*{Introduction to the GENESIS Documentation System}

Installation and maintenance of the GENESIS documentation system is fully automated with \href{../document-versionctrl/document-versionctrl.pdf}{document versioning} under the control of \href{http://monotone.ca/}{monotone}. Here we introduce some of the system flexibility, features and requirements. We also describe how to create and add your documentation to your local workspace and how to publish documents so that they are available for other members of the GENESIS community.

\subsection*{Overview}

In its simplest form, the GENESIS documentation system can be seen as having three levels:
\begin{enumerate}

\item {\bf The local documentation workspace:} Exists at a specified location within the GENESIS developers installation on your local machine. The contents of this workspace are private to the local filesystem.

\item {\bf The private document repository:} Provides the first step in document publication. Documentation version control is initiated by checking documents in your local workspace into the repository on your local machine.

\item {\bf The public document repository:} Pushing documents from your local repository to the publicly accessible monotone repository makes all documentation in your local repository available (i) for publication on the GENESIS website and (ii) to become part of the GENESIS documentation system that can optionally be downloaded as part of a GENESIS installation.


\end{enumerate}

\end{document}
