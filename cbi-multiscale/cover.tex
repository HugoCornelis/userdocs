\documentclass[12pt]{article}
\usepackage{verbatim}
\usepackage[dvips]{epsfig}
\usepackage{color}
\usepackage{lineno}
\usepackage{url}
\usepackage[colorlinks=true]{hyperref}
\usepackage{enumitem} % [noitemsep,nolistsep]
\usepackage{setspace} 


\begin{document}

\section*{Cover Letter}

\begin{flushleft}
{\large
\textbf{Multi-Scale Modelling within the CBI Simulator Framework}
}
% Insert Author names, affiliations and corresponding author email.
\\
Cornelis, H.$^{1,^\ast}$, 
Coop, A. D.$^{2}$,
Rodriguez, A. L.$^{3}$, 
Beeman, D.$^{4}$,
Bower, J. M.$^{5}$.
\\
\vspace*{5mm}
\begin{small}
{1.} Cornelis H. Department of Neurophysiology, Catholic University of Leuven, Leuven, 3000, Belgium
\\
{2.} Coop A. D. Director of Flight Simulation, Merindah Energy.
\\
{3.} Rodriguez A. L. Research Imaging Institute, University of Texas Health Science Center at San Antonio, San Antonio, TX, United States
\\
{4.} Beeman D. Department of Electrical, Computer, and Energy Engineering, University of Colorado, Boulder, CO 80309
\\
{5.} Bower J. M. Barshop Institute for Longevity and Aging Studies, 15355 Lambda Drive, University of Texas Health Science Center, San Antonio, Texas  78245
\\
$^\ast$ E-mail: Corresponding Author Hugo.Cornelis@gmail.com
\end{small}
\end{flushleft}

\doublespacing


\subsection*{To the editor}

We recently published two papers about the foundations of the new
Genesis-3 neural simulator software architecture.  These papers
explain how the Genesis-3 simulator has been reconfigured to enhance
its intrinsic capacity for enhanced functionality.  In this paper we
explore the multi-scale capabilities of this architecture.  We have
put this exploration into a more general context that was previously
formulated by Marr, Churland and Sejnowski.  Within this context we
highlight possible implications of our work in the discussion.

\end{document}


%%% Local Variables: 
%%% mode: latex
%%% TeX-master: t
%%% End: 
