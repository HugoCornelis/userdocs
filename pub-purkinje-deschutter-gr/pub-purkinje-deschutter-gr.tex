\documentclass[12pt]{article}
\usepackage{verbatim}
\usepackage[dvips]{epsfig}
\usepackage{color}
\usepackage{url}
\usepackage[colorlinks=true]{hyperref}

\begin{document}

\section*{GENESIS: Documentation}

{\bf Related Documentation:}
% start: userdocs-tag-replace-items related-do-nothing
% end: userdocs-tag-replace-items related-do-nothing

\section*{De Schutter: Purkinje Cell Model}

\subsection*{Source}

De Schutter E \& Bower JM (1994) An active membrane model of the cerebellar Purkinje cell II. Simulation of synaptic responses. {\it Journal of Nerurophysiology}. {\bf 71}: 401--419.

\subsection*{Granule Cell Synapses}

\subsubsection*{SYNAPTIC LOCATIONS}

In the model we placed granule cell synapses
on all dendrites with diameters $\sim$3.17\,$\mu$m\,\cite{Palay:1974fk}.
These synapses are made on dendritic spines\,\cite{M:1988bh} 
that have been explicitly modeled for each
synaptic contact (see below).

\subsubsection*{SYNAPTIC PHARMACOLOGY AND KINETICS}

These excitatory inputs to the Purkinje cell are also mediated by AMPA receptors
\,\cite{Farrant:1991hc, Garthwaite:1989ij, Lambolez:1992bs}
as well as metabotropic receptors\,\cite{Blackstone:1989fu, Glaum:1992kl, Vranesic:2qa}. Accordingly,
we modeled this excitatory input with the same kinetics as
for the climbing fiber, again ignoring possible Ca$^{2+}$- associated effects.

\subsubsection*{SYNAPTIC CONDUCTANCE}

The maximum conductance of a
single parallel fiber synapseis also unknown but seems to be variable
\,\cite{Hirano:1986fv, Ito:1989dz}. We used a value of 0.7\,
nS for $\bar g$ that falls in the range of values (0.2--50\,nS) that have been
used in other models using glutamate synaptic transmission\,\cite{R:1989cr, W:1991qa, Miller:1985mi, Rapp-M:1992kx, Wehmeier:1989pi, Wilson:1989ff, Zador:1990lh}.
 This synapse also had a reversal
potential of 0\,mV\,\cite{Cull-Candy:1989oq, Mayer:1987kl}.

\subsubsection*{NUMBERS OF SYNAPSES}

It is known that each Purkinje cell receives
$\sim$150,000 granule cell synapses\,\cite{Harvey:1991xz}.
Given computing resources it was not possible in the
current simulation to model all these inputs. Accordingly, in most
of the simulationsp resentedin this paper, granule cell inputs were
delivered on 1,474 spines ( 1\,\% of real), i.e., one on each modeled
spiny dendritic compartment. This simplification has two consequences.
First, there are a large number of dendritic spines missing
from the model. This was compensated for by adding dendritic
membrane (see\,\cite{R:1989cr, Rapp-M:1992kx}).
Second, many synaptic inputs are missing. Under the conditions
of random, asynchronous inputs simulated here, we compensated
for this missing  input by increasing the firing rate of each synapse.
A similar approach has been taken by other Purkinje cell modelers\,\cite{Rapp-M:1992kx}.
Assuming a linear scaling and simulating 1\,\%
of the inputs, an asynchronous firing rate of 1\,Hz in the model
would thus correspond to an average firing rate of $\sim$-0.01\,Hz for
real parallel fibers. We will show in the RESULTS section that this
scaling was appropriate for the current model. All synaptic input
firing rates mentioned are unscaled unless explicitly stated.

\subsubsection*{SYNAPTIC SPECIALIZATIONS}

Granule cell synaptic inputs are
known to terminate on dendritic spines\,\cite{Harvey:1991xz, Palay:1974fk}.
On the basis of electron microscopic
(EM) reconstructions of rat Purkinje cell spines\,\cite{M:1988bh}
we assumed a density of 13 spines per 1\,$\mu$m
dendritic length, resulting in a total of 144,456 spines. Most of
these spines have narrow-diameter spine necks and it is likely that
they have a significant effect on the electrical effects of granule cell
synaptic inputs\,\cite{Rall:1990tg}. For this reason, when
granule cell synaptic effects were studied in the model, spines were
explicitly simulated using two compartments each. One compartment
represented a spherical spine head with a diameter of 0.54\,$\mu$m,
whereas the second compartment represented the cylindrical
spine neck with a diameter of 0.20\,$\mu$m and a length of 0.66\,$\mu$m\,\cite{M:1988bh}. These dimensions resulted in a membrane
surface per spine of 1.33\,$\mu$m$^2$.

Reducing the number of granule cell inputs in the model also
had the effect of reducing the total number of spines. Given the
enormous number of theses pines on a normal Purkinje cell, simply
leaving their membrane area out of the model would significantly
affect simulation results. In our previous efforts to model
the electrical (not synaptic) properties of the Purkinje cell\,\cite{deschutter94:_purkin_i}
the influence of these dendritic spines
was approximated by increasing the membrane surface of compartments
of the region of the dendrite on which spines occur\,\cite{R:1989cr, Jaslove:1992fu, Jaslove:1992fu}.
The same approach was used here when granule cell synaptic effects
were not being simulated. However, for those simulations that did
relate to granule cell input, the total membrane surface was kept
constant by subtracting the membrane surface of modeled spines
(i.e., 1.33\,$\mu$m$^2$ for each spine) from the membrane surface of the
dendritic compartment to which they were connected. As already mentioned,
when no granule cell inputs were modeled,
the simulation was run without spines (i.e., $N_s = 0$), as was
done in the preceding paper\,\cite{deschutter94:_purkin_i}.

\subsubsection*{SYNAPTIC ACTIVATION}

Parallel fiber synaptic input was modeled
as an asynchronous process\,\cite{Bernard:1991ye}.
Activation of each modeled synapse was triggered independently
by a random number generator, which generated a Poisson distribution
around a mean frequency of input.

\bibliographystyle{plain}
\bibliography{../tex/bib/g3-refs}

\end{document}
