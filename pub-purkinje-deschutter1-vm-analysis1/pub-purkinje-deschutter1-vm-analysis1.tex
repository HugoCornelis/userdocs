\documentclass[12pt]{article}
\usepackage{verbatim}
\usepackage[dvips]{epsfig}
\usepackage{color}
\usepackage{url}
\usepackage[colorlinks=true]{hyperref}

\begin{document}

\section*{GENESIS: Documentation}

{\bf Related Documentation:}
% start: userdocs-tag-replace-items related-do-nothing
% end: userdocs-tag-replace-items related-do-nothing

\section*{De Schutter: Purkinje Cell Model}

\subsection*{COMPARISON OF SOMATIC AND DENDRITIC RECORDINGS}

Paired intracellular somatic and dendritic recordings in
slice have demonstrated that somatic action potentials do
not propagate far into the dendrite. Thus recordings of the
main dendrites of Purkinje cells often reveal somatic action
potentials as only small wavelets. When the membrane potentials
in the soma and large proximal dendrites were compared
in the model, a similar result was observed (Figs. \href{../pub-purkinje-deschutter1-fig-3/pub-purkinje-deschutter1-fig-3.tex}{\bf 3},  \href{../pub-purkinje-deschutter1-fig-4/pub-purkinje-deschutter1-fig-4.tex}{\bf\,4},
and \href{../pub-purkinje-deschutter1-fig-7/pub-purkinje-deschutter1-fig-7.tex}{\bf\,7}). Further, using the model it was possible to demonstrate
that the action potential did not propagate into spiny
dendrites (a location that electrophysiologists cannot record
from), even those relatively close to the soma (Figs. \,\href{../pub-purkinje-deschutter1-fig-3/pub-purkinje-deschutter1-fig-3.tex}{\bf\,3},
 \href{../pub-purkinje-deschutter1-fig-4/pub-purkinje-deschutter1-fig-4.tex}{\bf\,4}, and  \href{../pub-purkinje-deschutter1-fig-10/pub-purkinje-deschutter1-fig-10.tex}{\bf\,10{\it A}}).
Paired intracellular somatic and dendritic recordings
have also suggested that the Ca$^{2+}$ spiking, responsible for
the depolarizing spike bursts observed in the soma during
higher-amplitude current injections, has a dendritic origin.
As shown in Figs.  \href{../pub-purkinje-deschutter1-fig-3/pub-purkinje-deschutter1-fig-3.tex}{\bf\,3{\it B}\,and\,{\it C}}, and \href{../pub-purkinje-deschutter1-fig-4/pub-purkinje-deschutter1-fig-4.tex}{\bf\,4{\it C}\,and\,{\it D}}, this was also
the case in the model.

\bibliographystyle{plain}
\bibliography{../tex/bib/g3-refs.bib}

\end{document}
