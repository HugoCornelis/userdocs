\documentclass[12pt]{article}
\usepackage[dvips]{epsfig}
\usepackage{color}
\usepackage{url}
\usepackage[colorlinks=true]{hyperref}

\begin{document}

\section*{GENESIS: Documentation}

{\bf Related Documentation:}
% start: userdocs-tag-replace-items related-do-nothing
% end: userdocs-tag-replace-items related-do-nothing

\section*{Introduction to the GENESIS Shell}

After GENESIS has completed its startup procedure you should see the command prompt, {\tt genesis $>$}, indicating that you are now in the GENESIS shell environment and running an interactive GENESIS session. Note that to provide backward compatibility, the GENESIS shell interfaces with the Script Language Interpreter (SLI) of previous versions of GENESIS. To learn more about that functionality see \href{../backward-compatibility/backward-compatibility.tex}{here}.
%It is possible to excute both UNIX and GENESIS shell commands from within the GENESIS shell. However, UNIX shell commands must use an appropriate Perl system call, e.g. ``{\tt system(`echo ``Hello, World''')}'' will return ``{\tt Hello,\,World!}''.

One feature of the GENESIS shell is that it is important to have only one model loaded during a GENESIS session. Currently, there are two ways to change a model:

\begin{enumerate}

\item Exit the GENESIS shell, start a new GENESIS session, then load the new model  via the {\tt ndf\_load} command. Importantly, don't forget that to save changes made to a model between GENESIS sessions, you must save the model prior to deleting it (via the {\tt ndf\_save} command--see \href{../tutorial1/tutorial1.tex}{Save Model} for details).

\item Alternatively, a faster and more convenient way to replace a model within a GENESIS session is to use the {\tt delete\_current\_model} command to remove the currently loaded model from the GENESIS shell. A new model can then be loaded via the {\tt ndf\_load} command (see \href{../tutorial1/tutorial1.tex}{Tutorial 1} for details).

\end{enumerate}

\subsection*{GENESIS Help from the Unix Command Line}
Several options are recognized when starting GENESIS. They include, ``{\tt --help}'', ``{\tt --verbose}'', and ``{\tt --version}''. For example, typing
\begin{verbatim}
    $ genesis-g3 --help
\end{verbatim}
returns a list of options that can be used when starting the GENESIS shell.
\begin{verbatim}
    options:
        --help     print usage information.
        --verbose  set verbosity level ( . . . ).
        --version  give version information.
\end{verbatim}
The four arguments to the ``{\tt --verbose}'' option flag (``{\tt errors}'', ``{\tt warnings}'', '`{\tt information}'', and ``{\tt debug}'') generate increasing levels of output by the GENESIS shell. They may be invoked at startup in the following way, for example,
\begin{verbatim}
    $ genesis-g3 --verbose warnings
\end{verbatim}
Note that ``{\tt warnings}'' defines the default level of output.

\subsection*{Controlling Output from the GENESIS Shell}

\subsubsection*{The {\tt verbose} Commands}

In its default state, with the exception of error messages, the GENESIS shell generates minimal output. This means that in the default state, if output occurs then something is wrong.

The amount of output generated by the GENESIS shell is controlled by the {\tt set\_verbose} option.  This option recognizes several arguments that generate increasing amounts of output. They include the default setting {\tt errors} which displays only error state messages, also {\tt warnings} which displays warning and error messages, {\tt information} which displays information, warning, and error messages, and the most prolific, {\tt debug} which is used for software development and also generates maintenance errors. To use the {\tt set\_verbose} option in the GENESIS shell, enter, e.g.
\begin{verbatim}
    genesis > set_verbose debug
\end{verbatim}
To return the shell to its default {\tt verbose} state, just enter ``{\tt set\_verbose warnings}''.  To check the state of the {\tt verbose} flag, enter
\begin{verbatim}
    genesis > show_verbose
        verbose_level: warnings
\end{verbatim}

\subsection*{The {\tt help} and {\tt list} Queries}
The GENESIS shell recognizes a {\tt help} query. In the absence of an argument, i.e. just entering ``{\tt help}'', an error message is generated that contains the recognized sub-topics, e.g.
\begin{verbatim}
    genesis > help
        synopsis: help 
        synopsis:  must be one of commands, components, variables, libraries
\end{verbatim}
Similarly, in the absence of an argument the {\tt list} query generates an error message that contains the recognized argument types:
\begin{verbatim}
    genesis > list
    synopsis: list
    synopsis: must be one of commands, components, functions,
              inputclass_templates, inputclasses, physical,
              sections, structure, verbose
\end{verbatim}
We can then find help on a specific topic, for example, {\tt command}
\begin{verbatim}
    genesis > help command
    synopsis: list
    description: help on a specific command
    synopsis: 'help command '
    all commands:
\end{verbatim}
To find help on a specific sub-topic the {\tt list} query can be used to find, for example, the commands available to the GENESIS shell, e.g.
\begin{verbatim}
    genesis > list commands 
        all commands:
            - add_input
            - add_inputclass
            - add_output
            - ce
            - check
            - create
            - delete
            - echo
            - help
            - list
            - list_elements
            - model_state_load
            - model_state_save
            - ndf_load
            - ndf_save
            - pwe
            - querymachine
            - quit
            - reset
            - run
            - set_model_parameter
            - set_runtime_parameter
            - set_verbose
            - sh
            - show_global_time
            - show_inputs
            - show_library
            - show_parameter
            - show_runtime_parameters
            - show_verbose
\end{verbatim}
We can then obtain help on a specific command with, for example
\begin{verbatim}
    genesis > help command add_output
        description: add a variable to the output file.
        synopsis: add_output
\end{verbatim}

\end{document}