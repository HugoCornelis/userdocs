\documentclass[12pt]{article}
\usepackage[dvips]{epsfig}
\usepackage{color}
\usepackage{url}
\usepackage[colorlinks=true]{hyperref}

\begin{document}

\section*{GENESIS: Introduction}

{\bf Related Documentation:}
% start: userdocs-tag-replace-items related-do-nothing
% end: userdocs-tag-replace-items related-do-nothing

\section*{The Prerequisite Script}

The purpose of this script is to install the independent software packages required to run the GENESIS system software.

The script installs a \href{http://www.centos.org/docs/5/html/yum/}{\it yum} repository for the \href{http://www.graphviz.org/}{graphviz} software, sets up the \href{http://dag.wieers.com/rpm/}{DAG} repositories, and pulls down the \href{http://monotone.ca/}{monotone} binary. The purpose of this script is for use in an automated testing environment. Currently, it has only been tested on CentOS 5 (i386). However, it should work on any recent Fedora derivative, including: Fedora, CentOS, and RedHat Enterprise 5.

Note that currently all values are hard coded, including architecture and distribution information.

\subsection*{Prerequisite Script Details}

\begin{verbatim}
#!/usr/bin/perl -w
use strict;
####################################################################
### install.pl
### GENESIS prereqisite installer
### pulls packages from various repositories online in preparation
### for installing the GENESIS neural simulator
####################################################################

my @mypackages = (
                'perl',
                'gcc',
                'SDL_gfx',
                'SDL_gfx-debuginfo',
                'SDL_gfx-demos',
                'SDL_gfx-devel',
                'SDL_image',
                'SDL_image-devel',
                'SDL_mixer',
                'SDL_mixer-devel',
                'SDL_net',
                'SDL_net-devel',
                'SDL_Perl',
                'SDL_ttf',
                'SDL_ttf-devel',
                'graphviz',
                'graphvis-perl',
                'wget');

my @modules = ( 
                'Bundle::CPAN',
                'Clone',
                'Expect::Simple',
                'YAML',
                'File::Find::Rule',
                'Digest::SHA',
                'Data::Utilities',
                'File::chdir',
                'ExtUtils::Depends',
                'ExtUtils::PkgConfig',
                'Glib',
                'Cairo',
                'Gtk2');

my $loginname = getlogin();
if ($< != 0) { die("This script should be run as root.\n"); }

print("Installing graphviz repo files.");

open( OUTFILE, '>/etc/yum.repos.d/graphviz-rhel.repo');
print OUTFILE qq(
[graphviz-stable]
name=Graphviz - RHEL \$releasever - \$basearch
baseurl=http://www.graphviz.org/pub/graphviz/stable/redhat/ \
   el\$releasever/\$basearch/os/
enabled=1
gpgcheck=0

[graphviz-stable-source]
name=Graphviz - RHEL \$releasever - Source
baseurl=http://www.graphviz.org/pub/graphviz/stable/SRPMS/
enabled=0
gpgcheck=0

[graphviz-stable-debuginfo]
name=Graphviz - RHEL - Debug
baseurl=http://www.graphviz.org/pub/graphviz/stable/redhat/el \
   $releasever/\$basearch/debug/
enabled=0
gpgcheck=0

[graphviz-snapshot]
name=Graphviz - RHEL \$releasever - \$basearch
baseurl=http://www.graphviz.org/pub/graphviz/development/redhat/el \
   $releasever/\$basearch/os/
enabled=0
gpgcheck=0

[graphviz-snapshot-source]
name=Graphviz - RHEL \$releasever - Source
baseurl=http://www.graphviz.org/pub/graphviz/development/SRPMS/
enabled=0
gpgcheck=0

[graphviz-snapshot-debuginfo]
name=Graphviz - RHEL - Debug
baseurl=http://www.graphviz.org/pub/graphviz/development/redhat \
   el$releasever/\$basearch/debug/
enabled=0
gpgcheck=0
);
print ("Installing DAG gpg key and repo files\n");
system("rpm -Uhv http://apt.sw.be/redhat/el5/en/i386/rpmforge/RPMS/ \
   rpmforge-release-0.3.6-1.el5.rf.i386.rpm");

#this next part is here because vmware restore seems to mess up yum metadata
system("yum clean all -y");

# I had problems installing gcc if the rest of the packages weren't up to date
#Should have just updated what it needed, but whatever...
print "Updating all current packages.\n";
system("yum update -y");

#install rpms first to make sure compiler is present for CPAN
for my $mypackage (@mypackages) {
        print "Installing RPM: $mypackage\n";
        system("yum install -y $mypackage");
}

use CPAN;
for my $mod (@modules) {
        print "Installing Module: $mod\n";
# Use this if you don't expect any test failures
#        system("perl -MCPAN -e \'install $mod \'");
# I usually get some test failures which can break the whole thing, so I use
        system("perl -MCPAN -e \'force install $mod \'");

        #CPAN docs say this should work, but it doesn't seem to
        #my $obj=CPAN::Shell->expand('Module',("$mod"));
        #$obj->install;
}

#Setup monotone repositories using the binary files from the monotone project

system("wget http://www.monotone.ca/downloads/0.41/mtn-0.41-linux-x86.bz2");
system("bzip2 -d mtn-0.41-linux-x86.bz2");
system("mv mtn-0.41-linux-x86 /usr/local/bin/mtn");
system("chmod u+x /usr/local/bin/mtn");
system("mkdir ~/mtn");
system("mtn db init --db ~/mtn/neurospaces.mtn");
system("mtn --db=~/mtn/neurospaces.mtn pull repo-genesis3.cbi.utsa.edu:4696 \"*\"");
system("mtn --db=~/mtn/neurospaces.mtn --branch=0 \
   co ~/neurospaces-installer-mtn");

chdir("$ENV{HOME}/neurospaces-installer-mtn");
system("./configure");
system("make");
system("make install");
system("sudo neurospaces_build --download-server downloads.sourceforge.net \
   --check --regex 'model-container|heccer|ssp' --src-tag python-5 --src-dir \
   /tmp/neurospaces/downloads --verbose --verbose --verbose >& output.out &");
system("tail -f output.out");
\end{verbatim}

\end{document}
