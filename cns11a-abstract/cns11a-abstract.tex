\documentclass[12pt]{article}
\usepackage[dvips]{epsfig}
\usepackage{color}
\usepackage{url}
\usepackage[colorlinks=true]{hyperref}

\begin{document}

\section*{GENESIS: Documentation}

\section*{CNS*11 Abstract 1}

\subsection*{Introduction}

An important feature of GENESIS 3 simulator (G-3) is that its use is
structured following an ideal user-workflow.  This workflow is
based on experimental paradigms rather than computational simulation.
As a consequence, the architecture of the G-3 simulator allows it to be connected
directly to databases that store models and experimental data.  We present and
explain the G-3 database interfaces and show examples of
data-driven modeling methods.


\subsection*{Methods}

The ideal user-workflow consists of 5 steps: (1) Construct model, (2) Design experiment, (3) Configure simulation, (4) Check output, and (5) Iterate.
In principle each of these steps provides the foundation for a single database. 

\subsection*{Results}

Here we provide three examples of the relationship between steps in the ideal user workflow and the databases that support them.

The G-3 software component that implements the first step in the ideal
user-workflow is called the Model-Container.  This component
natively supports a dedicated data format for neuroscience models
(NDF).  The model-container has a builtin database of NDF models.  We
show how to add computational models to this database.  The
model-container interfaces with NeuroML and NineML model databases.

The second step in the ideal user-workflow connects the G-3 simulator
to databases of experimental protocols.  The database produced by this
step stores generic protocols that can be adjusted by the user for
specific experimental protocols such as current clamp and voltage
clamp.  These protocols are stored independently of the specification
of the computational model.

The separation of the steps for biological model construction and
experiment design allows the modeler to build separate libraries of
biological models and experimental protocols.  In principle this
allows any model to be combined with any appropriate experimental
protocol without the requirement that the protocol was specifically
designed for the given model.  The automation of this process would be
useful to identify the strengths, weaknesses, or deficits of a
specific model.

Finally the {\tt Iterate} step can be used to define the differences
in configuration between simulations belonging to the same simulation
project, research project, educational tutorial or scientific
publication.

\subsection*{Conclusion}

In this presentation we will show the use of computational model
databases and experimental protocol databases to drive simulations.
When combined with appropriate simulation configurations, simulations
can be run from a single command line.


\end{document}

%%% Local Variables: 
%%% mode: latex
%%% TeX-master: t
%%% End: 
