\documentclass[12pt]{article}
\usepackage[dvips]{epsfig}
\usepackage{color}
\usepackage{url}
\usepackage[colorlinks=true]{hyperref}

\begin{document}

\section*{GENESIS: Documentation}

\section*{GENESIS 3.0 functionality enhanced by interface to multiple types of database}

\subsubsection*{Hugo Cornelis, Allan D. Coop, Mando Rodriguez, and James M. Bower}

\subsection*{Introduction}

An important feature of GENESIS 3 simulator (G-3) is that its use is
structured following an ideal user-workflow with a focus on actual
experimental paradigms rather than the process of computational
simulation by itself.  This organization reflects the likely growing
importance of linking models to real experiments and experimental
data.  As a consequence, the software architecture of the G-3
simulator allows it to be connected directly to independent databases
that store models, experimental protocols, libraries of numerical
solvers, repositories of results and aggregated analysis, and
ultimately, tutorials and publications, as well as real time
experimental protocols such as dynamic clamping procedures.  In this
paper, we present and explain the details of the database interfaces
that support this functionality and show examples of data-driven
modeling methods.


\subsection*{Methods}

The ideal user-workflow consists of 5 steps: (1) Construct model, (2)
Design experiment, (3) Run simulation, (4) Check and analyze results
including experimentally, and (5) Iterate. In principle, each of these
steps provides the foundation for databases that collectively define
the scope and content of a modeling project.

\subsection*{Results}

We provide three concrete and already implemented examples of the
relationship between steps in the ideal user workflow and the
databases that support them.

{\bf 1. Construct Model:} The G-3 software component that implements
this first step in the ideal user workflow is called the {\tt
  Model-Container}.  It natively supports a dedicated data format
(NDF) for computational models in neuroscience.  The {\tt
  Model-Container} has a built-in database of NDF models and also
interfaces with arbitrary external model representations, including
currently both NeuroML and NineML model databases.  We show how to add
new computational models to this database as well as how to interface
to new model databases.

{\bf 2. Design Experiment:} The second step in the ideal user workflow
connects the G-3 simulator to a database of generic experimental
protocols such as current injection and current or voltage clamp.
These protocols can be adjusted by a user to define specific
experimental procedures.  Note that these procedures are stored
independently of the specification of the computational model, and can
also, in principle, be obtained from any experimental source of data.

We will demonstrate how the separation of the steps for model
construction and experimental design allow, in principle, any model to
be combined with any experimental procedure without requiring the
procedure to be specifically designed for a given model.


{\bf 3. Iterate:} This final step in the ideal user workflow closes
the loop between output of results and model construction.  It can be
used to define the differences in configuration between simulations
belonging to the same simulation project, research project,
educational tutorial or scientific publication.  Further, this step
can be automated in order to identify strengths, weaknesses, or
deficits of a specific model.  As such it becomes a powerful tool for
pre-review of a model before its publication.

\subsection*{Conclusion}

In this presentation we will demonstrate how G-3 interfaces to a
variety of existing databases which are accordingly then linked to
model development and simulation.  In principle this capability should
accelerate a shift from purely procedural to data-driven modeling with
important consequences for computational modelling methodology, and
especially new forms of model publication and scientific
communication.

\subsection*{Acknowledgements}
This research was supported by a NSF grant (HRD-0932339) to the
University of Texas at San Antonio.  Hugo Cornelis is partially
supported by the CREA Financing program (CREA/07/027) of the
K.U.Leuven, Belgium, EU.  Hugo Cornelis, Armando L. Rodriguez and
Allan D. Coop are partially supported by NIH grant 3 R01 NS049288-06S1
to James M. Bower.


\subsection*{References}
1. Coop AD, Cornelis H, Rodriguez M, Bower JM: Using GENESIS 3 for single neuron modeling. BMC Neurosci 2009, 10(S1):52.

\end{document}

%%% Local Variables: 
%%% mode: latex
%%% TeX-master: t
%%% End: 
