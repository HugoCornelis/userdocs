\documentclass[12pt]{article}
\usepackage[dvips]{epsfig}
\usepackage{color}
\usepackage{url}
\usepackage[colorlinks=true]{hyperref}

\begin{document}

\section*{GENESIS: Documentation}

\section*{GENESIS 3.0 functionality enhanced by interface to multiple types of database}

\subsubsection*{Hugo Cornelis, Allan D. Coop, Mando Rodriguez, and James M. Bower}

\subsection*{Introduction}

An important feature of GENESIS 3 simulator (G-3) is that its use is
structured following an ideal user-workflow.  This workflow is based
on experimental paradigms rather than computational simulation.  As a
consequence, the software architecture of the G-3 simulator allows it to be
connected directly to independent databases that store models, experimental protocols, libraries of numerical solvers, repositories of results and
aggregated analysis, and ultimately, tutorials and publications.  We
present and explain the database interfaces that support this functionality and show examples of
data-driven modeling methods.

%As an example, consider a family of morphologies stored in a database.
%Passive model parameters are used to convert each of these
%morphologies to a computational model and apply a series of
%experiments to each model.  Both passive model parameters and
%experiments applies to the models are coming from databases.


\subsection*{Methods}

The ideal user-workflow consists of 5 steps: (1) Construct model, (2) Design experiment, (3) Run simulation, (4) Check and analyze output, and (5) Iterate. In principle, each of these steps provides the foundation for databases that collectively define the scope and content of a modeling project. 

\subsection*{Results}

We provide three examples of the relationship between steps in the ideal user workflow and the databases that support them.

{\bf 1. Construct Model:} The G-3 software component that implements this first step in the ideal
user workflow is called the {\tt Model-Container}.  It
natively supports a dedicated data format (NDF) for computational models in neuroscience.  The {\tt Model-Container} has a built-in database of NDF models and also interfaces with  external model databases such as NeuroML and NineML.  We
show how to add new computational models to this database.

{\bf 2. Design Experiment:} The second step in the ideal user workflow connects the G-3 simulator
to a database of generic experimental protocols such as current injection and current or voltage
clamp.  These protocols can be adjusted by a user to define specific experimental procedures. 
Note that these procedures are stored independently of the specification
of the computational model.

It is the separation of the steps for construction of a biological model and
experiment design that allows an investigator to build separate libraries of
biological models and experimental procedures.  This has the advantage of
allowing any model to be combined with any experimental
procedure without requiring the procedure to be specifically designed for a given model.  
%The 

{\bf 3. Iterate:} This final step in the ideal user workflow closes the loop between output of results and model construction.
It can be used to define the differences in configuration between simulations belonging to the same simulation
project, research project, educational tutorial or scientific publication. Automation of this process would be
useful for identifying the strengths, weaknesses, or deficits of a specific model.

\subsection*{Conclusion}

We show how the use of a variety of appropriately configured
databases can be used to drive model development and simulation.
When employed collectively, they have the advantage of allowing simulations
to be run from a single command line.

\end{document}

%%% Local Variables: 
%%% mode: latex
%%% TeX-master: t
%%% End: 
