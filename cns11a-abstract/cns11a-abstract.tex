\documentclass[12pt]{article}
\usepackage[dvips]{epsfig}
\usepackage{color}
\usepackage{url}
\usepackage[colorlinks=true]{hyperref}

\begin{document}

\section*{GENESIS: Documentation}

\section*{CNS*11 Abstract 1}

\subsection*{Introduction}

An important feature of GENESIS 3 simulator (G-3) is that its use is
structured following an ideal user-workflow.  This user-workflow is
based on experimental paradigms rather than computational simulations.
As a consequence the architecture of the G-3 simulator connects it
directly to databases storing experiment data.  This presentation
explains the G-3 database interfaces and shows examples of data-driven
modeling methods.


\subsection*{Methods}

The ideal user-workflow consists of 5 steps:

{\bf Construct model:} Simple models can be created directly within
the {\bf G-Shell} by entering commands. More complex models can be
imported into the {\bf G-Shell} from either the GENESIS model
libraries or from external model libraries. The model can also be
explored, checked, and saved.

{\bf Design experiment:} Set model parameter values specific to a
given simulation, the stimulus parameters for a given simulation run
or `experiment', and/or the variables to be stored for subsequent
analysis.

{\bf Run simulation:} Configure runtime options, check, run, reset
simulation, and save model state. The model state can be saved at any
simulation time step to allow it to be imported into a subsequent
GENESIS session. Output is flushed to raw result storage for
subsequent data analysis.

{\bf Output:} Check simulation output and the validity of results to
determine whether simulation output exists in the correct locations.
Output can be analyzed either within GENESIS or piped to external
applications.

{\bf Iterators:} Close the loop between output of results and model
construction in the GENESIS users workflow. Iterators connect
experimental results and model output and include for example,
automated construction of simulations and batch files, static
parameter searching, and active parameter searching using the dynamic
clamp.

\subsection*{Results}

The G-3 software component that implements the first step in the ideal
user-workflow is called the model-container.  The model-container
natively supports a dedicated data format for neuroscience models
(NDF).  The model-container has a builtin database of NDF models.  We
show how to add to this database.  The model-container interfaces with
NeuroML and NineML model databases.

The second step in the ideal user-workflow connects the G-3 simulator
to databases of experimental protocols.


\end{document}

%%% Local Variables: 
%%% mode: latex
%%% TeX-master: t
%%% End: 
