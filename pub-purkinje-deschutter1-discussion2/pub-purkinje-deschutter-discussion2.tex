\documentclass[12pt]{article}
\usepackage{verbatim}
\usepackage[dvips]{epsfig}
\usepackage{color}
\usepackage{url}
\usepackage[colorlinks=true]{hyperref}

\begin{document}

\section*{GENESIS: Documentation}

{\bf Related Documentation:}
% start: userdocs-tag-replace-items related-do-nothing
% end: userdocs-tag-replace-items related-do-nothing

\section*{Limitations of the model}

The current model is an advancement in modeling the
Purkinje cell, but as any computer model, it must be seen as
a limited representation of reality\,\cite{Bower:1992vn}.
Thus, although the model replicated the basic features of
the response of this cell to current injection quite well, it did
not simulate all details of the several generated response
properties perfectly. For example, the simulated dendrite
tended to be a bit too excitable, so that a small spike often
occurred at the beginning of a low-amplitude current injection,
followed by a plateau potential (\href{../pub-purkinje-deschutter1-fig-3/pub-purkinje-deschutter1-fig-3.tex}{\bf Figs.\,3{\it A}} and \href{../pub-purkinje-deschutter1-fig-4/pub-purkinje-deschutter1-fig-4.tex}{\bf 4{\it A,B}}).
In experimental data, low-amplitude current injections
do not result in this firing pattern\,\cite{R:1980ly, R:1980pi}. 
Similarly, at higher current amplitudes fullblown
dendritic spikes were often preceded by smaller Ca$^{2+}$
spikes (\href{../pub-purkinje-deschutter1-fig-3/pub-purkinje-deschutter1-fig-3.tex}{\bf Fig.\,3{\it B,C}}). Also, the prolonged plateau potentials
after current injections did not decay properly (\href{../pub-purkinje-deschutter1-fig-5/pub-purkinje-deschutter1-fig-5.tex}{\bf Fig.\,5}).

We believe that an important reason for limitations in
model performance is the lack of critical experimental data.
Because this is the first description of this model, the following
sections discuss the limitations posed by the lack of data
in some detail.

\subsection*{MORPHOLOGICAL DATA}

At the most basic level, the model
was based on a light microscopic reconstruction of the Purkinje
cell, effectively limiting the resolution of the model to
branches $\sim$\,1\,$\mu$m in diameter. However, it is known that
Purkinje cells have many branches smaller than this diameter\,\cite{Gundappa-Sulur:1990bs, Palay:1974fk}. 
On the basis of this information we estimate that
our model may be lacking $\sim$\,10\,\% of the total dendrite. However,
we also believe that for the purposes of the current
paper this inaccuracy is relatively unimportant. First, this
paper primarily concerns Purkinje cell responses to current
injections into the soma or the smooth dendrites. Second,
an analysis of previously published EM data that examined
the dendritic spines in detail\,\cite{M:1988bh, Harvey:1991xz} 
suggests that the inaccuracies in
the total spine surface area and in the shrinkage factor used
in the original reconstruction\,\cite{Rapp-M:1994qf} were of the
same order as the missing dendritic membrane.

\subsection*{DATA FROM DIFFERENT ANIMALS}

A second limitation of
the current model is that model parameters were based on a
mixing of data obtained from the two different species used
most commonly in cerebellar physiology and anatomy, rats
and guinea pigs. For example, the morphological reconstruction
on which the basic structure of the model is based
represents a guinea pig Purkinje cell\,\cite{Rapp-M:1994qf},
whereas the EM data on which we based the size and distribution
of spines were obtained in the rat\,\cite{M:1988bh}. 
Likewise, most of the voltage-clamp data came
from rat Purkinje cells\,\cite{Gahwiler:1989fk, Hirano:1989uq, Kaneda:1990ys, Regan:1991ly}, 
although we compared model outputs to current injection
data obtained in the guinea pig slice\,\cite{R:1980ly, R:1980pi}. 
Although in principle one would prefer to base
the entire model on data from a single species, the lack of
available data makes species mixing quite common in modern
modeling efforts (cf.\,\cite{bhalla92:_rallp, W:1991qa, McCormick:1992fk, Traub-R-D:1991mi}. 
Furthermore, in the current case there is
no evidence that these species differ substantially in the
properties of their Purkinje cells.

\subsection*{NaF CHANNELS}

As was pointed out in \href{../pub-purkinje-deschutter1-conductance1-naf/pub-purkinje-deschutter1-conductance1-naf.tex}{\bf METHODS}, voltage-
clamp data on Purkinje cell fast Na$^+$ channels are incomplete.
This resulted in a number of anomalies in our
equations for the NaF current that could not be resolved
without better experimental data. Because of the potential
importance of the window current in the generation of Na$^+$
plateau potentials (see above) a better description of the
NaF current would be useful.

First of all, the equations were accurate for voltages up to
-10\,mV, but beyond this level the activation time constant
became too fast (\href{../pub-purkinje-deschutter1-conductance1-naf/pub-purkinje-deschutter1-conductance1-naf.tex}{\bf Fig.\,2{\it A}}). However, because of the all-or none
nature of action potentials this discrepancy did not
affect modeling results.

Second, at rest (-68\,mV) 73\,\% of the NaF channels were
inactivated and during current injections 294\,\% of the
channels were always inactivated. This necessitated an unphysiologically
high density of NaF channels in the soma
(total conductance 7,500\,mS/cm$^2$). However, because of
the continuous inactivation, the real $\bar g$ was $\sim$\,2,000\,mS/cm$^2$, 
comparable with, for example, an Na$^+$ conductance
of $\sim$\,1,000\,mS/cm$^2$ (20$^\circ$\,C) for a rat node of 
Ranvier\,\cite{Neumcke:1982qa}, but much larger than the
15\,mS/cm$^2$ (22$^\circ$\,C) reported for the somata of freshly dissociated
hippocampal neurons\,\cite{Sah:1988fv}. This difference
can be explained by the absence of an axon initial
segment in the model. This was not included in the model
because the reconstruction of the Purkinje cell did not contain
the axon. During the tuning phase of the model some
simulations were performed with a model including an
axon initial segment\,\cite{Somogyi:1976dz} but
these showed no qualitative differences compared with the
standard model.

\subsection*{Ca$^{2+}$-ACTIVATED K$^+$ CHANNELS}

It is of more concern that
the Ca$^{2+}$-activated K$^+$ currents that are known to exist in
the Purkinje cell and that have a substantial influence on
model behavior have not been described adequately in this
cell with voltage-clamp techniques.

From experimental data it is quite likely that the Purkinje
cell membrane contains several types of Ca$^{2+}$-activated
K$^+$ channels\,\cite{Gruol:1991dz}. This is not surprising
considering that the BK channel alone seems to have
$>$\,100 expression variants\,\cite{Adelman:1992fu}. Without
more detailed Purkinje cell data, in the current model we
have attempted to cover the effects of these diverse channels
by including only two Ca$^{2+}$-activated K$^+$ channels, a
KC and a K2 channel. As described in\,\href{../pub-purkinje-deschutter1-conductance1-kc/pub-purkinje-deschutter1-conductance1-kc.tex}{\bf KC} and\,\href{../pub-purkinje-deschutter1-conductance1-k2/pub-purkinje-deschutter1-conductance1-k2.tex}{\bf K2}, these
channels have quite different activation characteristics,
with the KC channel requiring high Ca$^{2+}$ concentrations
and large depolarizations whereas the K2 channel activates
at low Ca$^{2+}$ concentrations and small depolarizations. By
using two relative extremes of what could very well be a
continuous distribution of slightly different Ca$^{2+}$-activated
K$^+$ channels\,\cite{Latorre:1989fu} we hoped to approximate
the behavior of the whole population.

In the absence of good Purkinje cell data on these channels
we initially borrowed the kinetic description for the BK
conductance (KC) from simulations of the bullfrog sympathetic
ganglion\,\cite{Yamada-W:1989bs}. However, in Purkinje
cell simulations these channels activated too quickly,
so that dendritic Ca$^{2+}$  channels could not generate full Ca$^{2+}$ 
spikes. Accordingly, we modified the channel description
equations to include an explicit time constant for Ca$^{2+}$ activation
that mimicked the experimentally observed delay in
onset of this conductance\,\cite{Ikemoto:1989lh}. The resulting
KC equation reproduced several characteristics of the
BK channel well, among them the two separate Ca$^{2+}$ activation
steps with a delay, the additional voltage-independent
open-closed transition with typical time constants,
and voltage threshold. However, some other reported characteristics
are not captured by our equations. For example,
a simple horizontal shift of the open probability versus voltage
($P_O/V$) curve on changes in Ca$^{2+}$ concentration has been
reported\,\cite{Gola:1990pi, Moczydlowski:1983qa}, 
whereas our equations only scale the amplitude of
the $P_O/V$ curve (\href{../pub-purkinje-deschutter1-conductance1-kc1/pub-purkinje-deschutter1-conductance1-kc1.tex}{\bf Fig.\,2{\it G}}). In addition, the BK channel also
seems to have a fast Ca$^{2+}$-related deactivation\,\cite{Ikemoto:1989lh}, 
whereas in our model Ca$^{2+}$ activation and Ca$^{2+}$ 
deactivation had the same time constants. The most substantial
difference between the data and our equations is
that the experimental data predict that the $\bar g$ does not depend
on Ca$^{2+}$ concentration, so that at very low Ca$^{2+}$ concentrations
extremely high depolarizations can still fully
open the channel and vice versa. In our equations g was
limited by the Ca$^{2+}$ concentration. However, because the
Purkinje cell dendrite never depolarized enough to open
BK channels at low Ca$^{2+}$ concentrations and because Ca$^{2+}$ 
never reached the saturating concentration for opening all
BK channels, the model operated in a region of parameter
space where the effective difference between equations and
experimental data is minimal.

The presence of the K2 channel was based on singlechannel
recordings\,\cite{Gruol:1991dz}, but few kinetic data
were available. We have therefore combined data from similar
channels in synaptosomal membranes\,\cite{Farley:1988tw, Reinhart1989:xe}. 
However, the experimental
data remain very incomplete. On the basis of our results,
we believe that a better characterization of this channel is
essential for a further refinement of the model.
There is also evidence in Purkinje cells for a K$^+$ channel
that causes slow afterhyperpolarizations (referred to as K7
by\,\cite{Gruol:1991dz}), but we did not model this conductance.
This channel resembles the SK channel or AHP conductance
that causes slow afterhyperpolarizations in other
neurons\,\cite{Lancaster:1991ye, Latorre:1989fu}. It is
believed to be sensitive to low Ca$^{2+}$ concentrations with
slow activation kinetics\,\cite{Pennefeather:1990kl}. We did
not include this channel in the model because the model
computed only quickly changing Ca$^{2+}$ concentrations, but
the SK channel would be expected to sense mainly slower
transients. The fact that the model could reproduce most of
the physiological characteristics of Purkinje cells leads us to
suggest that, if present, SK channels play a minor role in the
short-term response to current injections.

\bibliographystyle{plain}
\bibliography{../tex/bib/g3-refs}


\end{document}
