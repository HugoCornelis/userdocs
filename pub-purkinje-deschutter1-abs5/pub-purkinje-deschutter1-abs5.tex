\documentclass[12pt]{article}
\usepackage{verbatim}
\usepackage[dvips]{epsfig}
\usepackage{color}
\usepackage{url}
\usepackage[colorlinks=true]{hyperref}

\begin{document}

\section*{GENESIS: Publication}

{\bf Related Documentation:}
% start: userdocs-tag-replace-items related-do-nothing
% end: userdocs-tag-replace-items related-do-nothing

\section*{De Schutter: Purkinje Cell Model}

\subsection*{Abstract Item 5}

These model responses were robust to changes in the densities
of all of the ionic channels. For most of the channels, modifying
their densities by factors of $\geq$2 resulted only in left or right
shifts of the frequency-current curve. However, changes of $>$20\,\%
to the amount of P-type Ca$^{2+}$ channels or of one of the Ca$^{2+}$-activated
K$^+$ channels in the model either suppressed dendritic spikes
or caused the model to always fire Ca$^{2+}$ spikes. Modeling results
were also robust to variations in Purkinje cell morphology. We
simulated models of two other anatomically reconstructed Purkinje
cells with the same channel distributions and got similar
responses to current injections.

\end{document}
