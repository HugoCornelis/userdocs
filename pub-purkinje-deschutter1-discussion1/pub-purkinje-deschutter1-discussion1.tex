\documentclass[12pt]{article}
\usepackage{verbatim}
\usepackage[dvips]{epsfig}
\usepackage{color}
\usepackage{url}
\usepackage[colorlinks=true]{hyperref}

\begin{document}

\section*{GENESIS: Documentation}

{\bf Related Documentation:}
% start: userdocs-tag-replace-items related-do-nothing
% end: userdocs-tag-replace-items related-do-nothing

\section*{Discussion}

This paper describes a detailed compartmental model of the
cerebellar Purkinje cell. Although this cell has been the
subject of numerous previous modeling efforts (see below),
the current model includes important features of this neuron
that were not present in previous ones. For example,
several authors have explored the passive properties of the
Purkinje cell\,\cite{Llinas:1976vn, Rapp-M:1992kx, Rapp-M:1994qf, P:1985mb}. 
However, these models did not include
voltage-dependent conductances in the dendrites,
which are known to be extensive in Purkinje cells\,\cite{Llinas:1992rq} 
and which play an important role in
the response to both current injections and synaptic inputs
(most spectacular for the climbing fiber synapse). Two
models that do include ionic conductances in the dendrites
have been reported in the literature. The first one \cite{Pellionisz:1977zr}
was quite innovative in that it was
one of the first models with active membrane in the
dendrites ever published. Unfortunately, much less was known 
at that time about the dendritic conductances within this
cell, so that the model used fast Na$^+$ channels and a delayed
rectifier in the dendrites instead of the Ca$^{2+}$ and Ca$^{2+}$-activated 
K$^+$ channels that are now known to be present\,\cite{R:1980ly, R:1980pi}. 
The other active membrane model\,\cite{Bush:1991ly} used a more appropriate
set of ionic channels in the soma and dendrite and 
some data were presented indicating that it replicated 
responses to current injections. However, neither the model nor
its responses were described in any detail. Instead, the
main emphasis of the report was on describing a new set of
equations to simulate ionic channels in single-cell models.

\bibliographystyle{plain}
\bibliography{../tex/bib/g3-refs.bib}

\end{document}
