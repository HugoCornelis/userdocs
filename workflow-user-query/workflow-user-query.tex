\documentclass[12pt]{article}
\usepackage{verbatim}
\usepackage[dvips]{epsfig}
\usepackage{color}
\usepackage{url}
\usepackage[colorlinks=true]{hyperref}

\begin{document}

\section*{GENESIS: Documentation}

{\bf Related Documentation:}
% start: userdocs-tag-replace-items related-workflow
% end: userdocs-tag-replace-items related-workflow

\section*{Workflow for a GENESIS-Related Query}

\subsection*{Introduction}

The ``GENESIS development process'' can be initiated by either a
regular user or a software developer. The purpose of the workflow
below is to facilitate rapid high quality communication between users
and developers.

The preferred method for asking for help with the GENESIS
Neurosimulator is via the
\href{http://www.genesis-sim.org/forum}{\bf Forum}.  A link to the Forum
can be found at the top of the
\href{http://www.genesis-sim.org/}{\bf GENESIS\,Homepage}.  On the Forum
page, under the GENESIS Neurosimulator heading, there are two forums
listed, each with the following purpose:

\begin{enumerate}
\item {\bf Users Forum:} If you are using GENESIS in a research
  project, need technical support, or have questions about the
  simulator applications, please ask in this forum.  This forum can
  also be used for requests for new functions.
\item {\bf Development Forum:} This forum should be used for technical
  questions about software implementation issues, algorithms used, and
  the hardware infrastructure used to support GENESIS software
  development.
\end{enumerate}

As a member of the GENESIS User Community, contacting the
\href{http://www.genesis-sim.org/contact}{\bf GENESIS\,Development\,Team} or
asking a question on the GENESIS Forums carries some responsibility. The
following workflow outlines the ideal path from a user query to the
incorporation into the GENESIS Documentation System of any
documentation generated by a user or developer in response to the query.

\subsection*{User Query Workflow}

\begin{enumerate}
\item Query received.
\item An answer is formulated by a member of the User Community.
\item The answer is implemented (usually by the person who asked the
  original question).  Mostly this involves running simulations, and
  keeping track of how you configure the simulation platform (model used, stimulation protocol used, etc.).  (We are currently working
  on a tool to support the user tracking of a simulation
  configuration, check the detailed documentation of the {\it ssp\_save}
  command, and the documentation of the ``status'' key in the {\bf SSP}
  documentation).
\item A documentation writer (usually the person who developed the
  answer) takes the record/log and creates user documentation
  following the procedure outlined for
  \href{../document-create/document-create.tex}{\bf Document\,Creation}.
\item If the question results in software development, the record/log
  is used by a developer to create a (regression) test specification.
  The form of test documentation can be obtained from any of the
  pre-existing test documentation, e.g. for the
  \href{../tests-gshell/tests-gshell.tex}{\bf G-Shell\,regression\,tests}.
\end{enumerate}

\end{document}
