\documentclass[12pt]{article}
\usepackage[dvips]{epsfig}
\usepackage{url}
\usepackage[colorlinks=true]{hyperref}

\begin{document}

\section*{GENESIS: Documentation}

{\bf Related Documentation:}
% start: userdocs-tag-replace-items related-documentation-system
% end: userdocs-tag-replace-items related-documentation-system

\section*{Documentation Tools}

\subsection*{Introduction}

Here we describe a number of UNIX shell command line utilities to
query and maintain
\href{../documentation-overview/documentation-overview.tex}{the
  GENESIS Documentation System} as a filesystem based database of
documents.  Note that these tools are not used to search the contents
of the documents.  The \href{http://www.genesis-sim.org/}{GENESIS} and
\href{http://www.neurospaces.org/}{Neurospaces} websites provide this
type of user-functionality.


\subsection*{Available Tools}

Typing {\bf userdocs-} in a
\href{http://www.gnu.org/software/bash/}{bash system command shell}
followed by striking <tab> key twice will show all the available
commands to query and maintain the GENESIS Documentation System.

Each command has a help page that shows all the available options to
the invoked command.  This page isinvoked by the {\bf --help} argument
(for example {\bf userdocs-build --help}).

Here we summarize the function of all the commands.

\begin{itemize}
\item {\it userdocs-build}\,\,\,Builds a set of documents and
  publishes it to a website on the local computer or by email or both.
\item {\it userdocs-check}\,\,\,Checks a set of documents for errors.
\item {\it userdocs-create}\,\,\,Creates a new document.
\item {\it userdocs-pull}\,\,\,Pulls new changes from a remote
  documentation repository and updates the local copy.
\item {\it userdocs-rename}\,\,\,Renames a document.
\item {\it userdocs-snippet}\,\,\,Inserts snippets of automatically
  generated text into a document.
\item {\it userdocs-sync}\,\,\,Synchronizes the local documentation
  repository with a remote one and updates the local copy.
\item {\it userdocs-tag-filter}\,\,\,Displays the document set
  generated by a series of tags.  For example '{\bf
    userdocs-tag-filter level1}' displays a list of all documents in
  level 1.
\item {\it userdocs-tag-replace-items}\,\,\,Replaces a tag in a
  document with the alphabetical list of document descriptions
  generated by that tag with {\it userdocs-tag-filter}.
\item {\it userdocs-version}\,\,\,Displays version information about
  the locally installed documentation system.
\end{itemize}


\subsection*{Example}

This is an interactive tutorial with examples that show typical usage
of the tools mentioned above.

To learn about how to build individual documents:

\begin{verbatim}
$ userdocs-build --help


userdocs-build: build documentation and prepare them for web
publication.  Arguments on the command line are taken as documents
that need to be build.  If no arguments are given, all documents will
be build.

synopsis:
    $0 <document name1> <document name2> ...

options:
    --check           check documentation filesystem correctness.
    --dry-run         print what documents would be build without actually building them.
    --email           send the built document to this email address, repeat for each receiver.
                      NOTE: see http://www.amirwatad.com/blog/archives/2009/03/21/send-email-from-the-command-line-using-gmail-account/ as an example for local configuration of your MTA.
    --help            print usage information.
    --parse-only      only execute parse commands, without actually building documentation.
    --regex           selects documents by name (default is all).
    --report-tags     report the tags associated with the selected documents instead of building the documents.
    --tags            process the documents with these tags, multiple tags options may be given.
    --version         give version information.
    --v|verbose       tell what is being done, specify multiple times to get more feedback.

example usage:
    userdocs-build documentation-homepage
    firefox html/htdocs/neurospaces_project/userdocs/documentation-homepage/documentation-homepage.html

\end{verbatim}

To display all the tags currently in use:

\begin{verbatim}
$ userdocs-build --report-tags
---
all_processed_tags:
  - abstract
  - contents-level1
  - contents-level2
  - contents-level3
  - contents-level4
  - contents-level5
  - contents-level6
  - contents-level7
  - developer
  - do-nothing
  - genesis2
  - genesis3
  - gshell-interactive
  - introduction
  - level1
  - level2
  - level3
  - level5
  - pdf
  - poster
  - published
  - related-build-debian
  - related-developer-installation
  - related-do-nothing
  - related-documentation-system
  - related-genesis-systems
  - related-model-container
  - related-notes-dave-g2
  - related-pub-purkinje-deschutter1-conductance1
  - related-pub-purkinje-deschutter1-morphology
  - related-ssp
  - related-todo
  - related-tutorial
  - related-workflow
  - rst
  - rtf
  - rtxi
  - tutorial
  - userworkflow-do-nothing
  - wiki
  - workshop
\end{verbatim}

To learn what documents are shown on the contents page of level 1
documentation:

\begin{verbatim}
$ userdocs-build --tags contents-level1 --dry-run
---
all_documents:
  /home/cornelis/neurospaces_project/userdocs/source/snapshots/0/contents-level1: contents-level1
  /home/cornelis/neurospaces_project/userdocs/source/snapshots/0/developer-intro: contents-level1
  /home/cornelis/neurospaces_project/userdocs/source/snapshots/0/faq: contents-level1
  /home/cornelis/neurospaces_project/userdocs/source/snapshots/0/genesis-installation: contents-level1
  /home/cornelis/neurospaces_project/userdocs/source/snapshots/0/user-intro: contents-level1
\end{verbatim}

We can then build one specific document (for instance the user-intro
document) with:

\begin{verbatim}
$ userdocs-build user-intro
\end{verbatim}

Note that we do not prefix the name of the document with its
directory.

To build all the documents shown on the contents page of level 1
documentation:

\begin{verbatim}
$ userdocs-build --tags contents-level1
...
...
\end{verbatim}

To build all the documentation you issue the {\bf userdocs-build}
command by itself.  This will take some time to run.  The result is a
directory hierarchy with html documents in your home directory
(~/neurospaces\_project/userdocs/source/snapshots/0/html/htdocs/neurospaces\_project/userdocs/).

It is convenient to combine some of these options.  For instance to
learn which tags are associated with the {\bf user-intro} document:

\begin{verbatim}
$ userdocs-build --report-tags user-intro
---
all_processed_tags:
  - contents-level1
  - genesis3
  - level1
  - published
\end{verbatim}

To know which tags are associated with the documents in level 1
documentation:

\begin{verbatim}
$ userdocs-build --report-tags --tags level1
---
all_processed_tags:
  - contents-level1
  - developer
  - genesis3
  - gshell-interactive
  - introduction
  - level1
  - published
  - related-workflow
  - wiki
\end{verbatim}

To know which documents are still in {\bf draft} status and thus
unpublished:

\begin{verbatim}
$ userdocs-build --dry-run --tags draft
---
all_documents:
  /home/cornelis/neurospaces_project/userdocs/source/snapshots/0/genesis-integration-with-gshell: draft
  /home/cornelis/neurospaces_project/userdocs/source/snapshots/0/installation-studio: draft
  /home/cornelis/neurospaces_project/userdocs/source/snapshots/0/mercurial-testing: draft
  ...
\end{verbatim}



\end{document}

%%% Local Variables: 
%%% mode: latex
%%% TeX-master: t
%%% End: 
