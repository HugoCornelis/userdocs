\documentclass[12pt]{article}
\usepackage[dvips]{epsfig}
\usepackage{url}
\usepackage[colorlinks=true]{hyperref}

\begin{document}

\section*{GENESIS: Documentation}

{\bf Related Documentation:}
% start: userdocs-tag-replace-items related-documentation-system
% end: userdocs-tag-replace-items related-documentation-system

\section*{Documentation Tools}

\subsection*{Introduction}

Here we describe a number of UNIX shell command line utilities to
query and maintain
\href{../documentation-overview/documentation-overview.tex}{the
  GENESIS Documentation System} as a filesystem based database of
documents.  Note that these tools are not used to search the contents
of the documents.  The \href{http://www.genesis-sim.org/}{GENESIS} and
\href{http://www.neurospaces.org/}{Neurospaces} websites provide this
type of user-functionality.


\subsection*{Available Tools}

Typing {\bf userdocs-} in a
\href{http://www.gnu.org/software/bash/}{bash system command shell}
followed by striking <tab> key twice will show all the available
commands to query and maintain the GENESIS Documentation System.

Each command has a help page that shows all the available options to
the invoked command.  This page isinvoked by the {\bf --help} argument
(for example {\bf userdocs-build --help}).

Here we summarize the function of all the commands.

\begin{itemize}
\item {\it userdocs-build}\,\,\,Builds a set of documents and
  publishes it to a website on the local computer or by email or both.
\item {\it userdocs-check}\,\,\,Checks a set of documents for errors.
\item {\it userdocs-create}\,\,\,Creates a new document.
\item {\it userdocs-pull}\,\,\,Pulls new changes from a remote
  documentation repository and updates the local copy.
\item {\it userdocs-rename}\,\,\,Renames a document.
\item {\it userdocs-snippet}\,\,\,Inserts snippets of automatically
  generated text into a document.
\item {\it userdocs-sync}\,\,\,Synchronizes the local documentation
  repository with a remote one and updates the local copy.
\item {\it userdocs-tag-filter}\,\,\,Displays the document set
  generated by a series of tags.  For example '{\bf
    userdocs-tag-filter level1}' displays a list of all documents in
  level 1.
\item {\it userdocs-tag-replace-items}\,\,\,Replaces a tag in a
  document with the alphabetical list of document descriptions
  generated by that tag with {\it userdocs-tag-filter}.
\item {\it userdocs-version}\,\,\,Displays version information about
  the locally installed documentation system.
\end{itemize}


\subsection*{Example}

This is an interactive tutorial with examples that show typical usage
of the tools mentioned above.

To learn about how to build individual documents:

% start: userdocs-snippet help.tex
% filled in automatically
% end: userdocs-snippet help.tex

To display all the tags currently in use:

% start: userdocs-snippet tags.tex
% filled in automatically
% end: userdocs-snippet tags.tex

To learn what documents are shown on the contents page of level 1
documentation:

% start: userdocs-snippet contents-level1.tex
% filled in automatically
% end: userdocs-snippet contents-level1.tex

%\begin{verbatim}
%$ userdocs-build --tags contents-level1 --dry-run
%---
%all_documents:
%  /home/cornelis/neurospaces_project/userdocs/source/snapshots/0/contents-level1: contents-level1
%  /home/cornelis/neurospaces_project/userdocs/source/snapshots/0/developer-intro: contents-level1
%  /home/cornelis/neurospaces_project/userdocs/source/snapshots/0/faq: contents-level1
%  /home/cornelis/neurospaces_project/userdocs/source/snapshots/0/genesis-installation: contents-level1
%  /home/cornelis/neurospaces_project/userdocs/source/snapshots/0/user-intro: contents-level1
%\end{verbatim}

We can then build one specific document (for instance the user-intro
document) with:

\begin{verbatim}
$ userdocs-build user-intro
\end{verbatim}

Note that we do not prefix the name of the document with its
directory.

To build all the documents shown on the contents page of level 1
documentation:

\begin{verbatim}
$ userdocs-build --tags contents-level1
  ...
  ...
\end{verbatim}

To build all the documentation you issue the {\bf userdocs-build}
command by itself.  This will take some time to run.  The result is a
directory hierarchy with html documents in your home directory
(~/neurospaces\_project/userdocs/source/snapshots/0/html/htdocs/neurospaces\_project/userdocs/).

It is convenient to combine some of these options.  For instance to
learn which tags are associated with the {\bf user-intro} document:

% start: userdocs-snippet tags-user-intro.tex
% filled in automatically
% end: userdocs-snippet tags-user-intro.tex

To know which tags are associated with the documents in level 1
documentation:

% start: userdocs-snippet tags-level1.tex
% filled in automatically
% end: userdocs-snippet tags-level1.tex

To know which documents are still in {\bf draft} status and thus
unpublished:

\begin{verbatim}
$ userdocs-build --dry-run --tags draft
---
all_documents:
  /home/cornelis/neurospaces_project/userdocs/source/snapshots/0/genesis-integration-with-gshell: draft
  /home/cornelis/neurospaces_project/userdocs/source/snapshots/0/installation-studio: draft
  /home/cornelis/neurospaces_project/userdocs/source/snapshots/0/mercurial-testing: draft
  ...
\end{verbatim}



\end{document}

%%% Local Variables: 
%%% mode: latex
%%% TeX-master: t
%%% End: 
