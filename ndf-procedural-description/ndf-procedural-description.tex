\documentclass[12pt]{article}
\usepackage[dvips]{epsfig}
\usepackage{lineno}
\usepackage{color}
\usepackage{url}
\usepackage[colorlinks=true]{hyperref}

\begin{document}

\section*{GENESIS: Documentation}

{\bf Related Documentation:}

\href{../model-variables/model-variables.tex}{\bf Physical Processes and Model Variables}

% start: userdocs-tag-replace-items related-do-nothing
% end: userdocs-tag-replace-items related-do-nothing

\section*{The NDF File Format and Procedural Model Descriptions}

\href{../ndf-file-format/ndf-file-format.tex}{\bf The Neurospaces
  Description Format} or NDF file format is a declarative ASCII text
file format indicated by the {\tt .ndf} file name extension. NDF files
are purely declarative in nature.  Rather than giving direct access to
procedural language constructs, NDF files integrate with a library of
procedural algorithms.  Each algorithm in this library is a procedural
abstraction that has full access to the model, can query it, add new
components, and delete or modify existing components.  Alternatively
an algorithm can analyse the model structure and report the result to
the user.


\subsection*{Running an Algorithm}
\label{sec:running-an-algorithm}

At compilation time the library of algorithms is integrated with the
model-container.  Each algorithm has a unique name that can be used
from inside an NDF file to run the algorithm.  Some algorithms may
require the assignment of specific parameter values to determine a
deterministic outcome.  This assignment of parameter values is given
in the NDF file where the algorithm is referenced.  After the
algorithm has completed, the model-container stores the parameter
assignment and additional specific algorithm results as an algorithm
instance.  Such an algorithm instance can then be retrieved by the
user by name.


\subsection*{An Example}
\label{sec:an-example}

The following examples runs the algorithm with name 'Grid3D' and
assigns the name 'GranulePopulation' to the algorithm instance.  The
outcome of this NDF snippet will be 6240 neurons instantiated from the
prototype ``Granule\_cell'' and laid out in 3D space.

\begin{center}
% \begin{linenumbers}
 % \begin{boxedminipage}{13cm}
\begin{verbatim}
  POPULATION GranulePopulation
    ALGORITHM Grid3D
      GranuleGrid
      PARAMETERS
        PARAMETER ( PROTOTYPE = "Granule_cell" ),
        PARAMETER ( X_COUNT = 120.0 ),
        PARAMETER ( X_DISTANCE = 2.5e-05 ),
        PARAMETER ( Y_COUNT = 26.0 ),
        PARAMETER ( Y_DISTANCE = 1.875e-05 ),
        PARAMETER ( Z_COUNT = 2.0 ),
        PARAMETER ( Z_DISTANCE = 2e-5 ),
      END PARAMETERS
    END ALGORITHM
  END POPULATION
\end{verbatim}
%  \end{boxedminipage}
%\end{linenumbers}
\end{center}

\subsection*{Summary}

The model-container algorithm library complements the NDF file format
with procedural descriptions.  Algorithms in this library include
Grid3D (to define a 3D grid of other components), Spines (to add
spines to a morphology by calculating surface area corrections or by
attaching them given a prototype spine), ProjectionVolume and
ProjectionRandomized (to create projections and connections between
two populations of neurons) and Randomize (to randomize certain
parameters of a named model)


\end{document}

%%% Local Variables: 
%%% mode: latex
%%% TeX-master: t
%%% End: 
