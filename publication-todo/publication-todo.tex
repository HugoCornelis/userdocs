\documentclass[12pt]{article}
\usepackage[dvips]{epsfig}
\usepackage{color}
%e.g.  \textcolor{red,green,blue}{text}
\usepackage{url}
\usepackage[colorlinks=true]{hyperref}
\usepackage{scrtime}
 
\begin{document}

{\bf Related Documentation:}
% start: userdocs-tag-replace-items related-do-nothing
% end: userdocs-tag-replace-items related-do-nothing

\section*{GENESIS: Documentation}

\section{The TODO and the DONE Lists}

This list is specifically about the publication GUI.  See
\href{../project-todo/project-todo.tex}{The project-todo list} for an
explanation how to work with these lists.

This list needs further expansion by mapping the technical comments to
GUI todos.

\subsection{General Advantages of the Publication GUI}
\begin{itemize}
\item Paper version of story and figures.
\item Here is how it maps into the new system.
\item Talk about the capabilities once you do that.
\end{itemize}


\section{TODO List}

Example based on
\cite{deschutter94:_purkin_i}\cite{deschutter94:_purkin_ii}\cite{schutter94:_simul_purkin}

\subsection{Technically}

\subsubsection{loading the different Purkinje cell models into G-3}

\begin{enumerate}
\item The g-tube has a model name selection box.  It lists the names
  of all the models that can be loaded.
\item Both the g-tube and gshell currently work with one implicit
  workspace, so they can import only one model at a time.  The
  model-container uses 'namespaces' as an abstraction for user
  workspaces.  Gshell commands need to be implemented to access the
  interface to manage the model-container's namespaces.  The g-tube
  interfaces to the gshell over its standard I/O stream connection to
  use these new commands.
\item A namespace edit box will popup every time a new model is loaded
  to allow the user to edit the name of the namespace.  The edit box
  suggests a namespace for use.
\item The name of the namespace that was used to load the model, is
  visible in the model loader selection box.
\end{enumerate}

\subsubsection{Passive Segev Model}
\begin{itemize}
\item The Segev morphology is available from the
  neuromorpho database.  There are subtle differences to the soma
  dimensions with the edsjb1994 model.
\item The passive properties of this model are accurately reported in
  Segev paper.  The different versions of the Segev model must be
  added to the model library.
\item Both the segev model and the edsjb1994 model can be loaded under
  their own namespace.  The GUI displays the differences in:

  The following quantities are visible for both models simultaneously
  in a small table.  This table with morphology characteristics is
  available from a button in the menu 'Model Construction' -> 'Explore
  Model'.  (Todo: split this table in two: one for morphology, one for
  passive move).
  \begin{enumerate}
  \item soma dimensions (different between the two models)
  \item electrotonic length of the longest and shortest compartments (eds has a very very long compartment).
  \item total dendritic length, surface area and volume.
  \item Number of branch points, average branch order of dendritic tips.
  \item RM, CM, RA, ELEAK, number of spines, number of compartments.
  \item List of transmembrane currents.
  \end{enumerate}
\end{itemize}
The quantities mentioned above can be computed by the model container
and are made available as yaml text files to the GUI from the standard
I/O connection with the gshell.

\subsubsection{DeSchutter: 3 different geinea pigs dendritic morphologies}
\begin{itemize}
\item We only have a recompartmentalized version of one of the them.
  I don't know where to find the recompartmentalized versions of the
  other two.  The morphologies are available from the neuromorpho
  database, but naive recompartmentalization of the morphology changes
  computational behavior such that exact reproduction of the
  publication figures is not directly possible.  To be sorted out.
\item The GUI shows a list with all the transmembrane currents /
  channels found in the model.
  \begin{enumerate}
  \item The channel characteristics table shows the gate parameters
    and the reversal potential for each channel (table 1
    paper~\cite{deschutter94:_purkin_i}).
  \item The channel distribution table shows the current densities
    (table 2 paper~\cite{deschutter94:_purkin_i}).
  \item Clicking a button in the table with channel characteristics
    shows a plot of the steady-state and tau membrane potential
    relationship.  This relationship is calculated by heccer and made
    available to the GUI via the gshell 'tabulate' command.
  \item The voltage clamp current can be made visible too, using a
    library of SSP schedules (see below).
  \end{enumerate}
\end{itemize}


\subsubsection{Sergio\'s model}
\begin{itemize}
\item Different anamolous rectifier: I am in touch with Sergio about
  his model and will implement it in G3.  I am still waiting for his
  response, but if he cannot help us out, it should be relatively easy
  to reproduce his model starting from Arnd Roth's and Sergio's
  publications, because Sergio only replaced the anomalous rectifier
  (h-current).
\item The GUI shows that this model's morphology is identical to
  Erik's model's morphology by showing the table with morphology
  characteristics.
\item It would be nice to display the morphology graphically as well.
\end{itemize}

\subsubsection{Japan model}
\begin{itemize}
\item This is reported as a modification of the original EDS model,
  reimplemented in the NEURON simulator.  But: the morphology is taken
  from D.P. Shelton (but looks different from Shelton's), there is no
  report of different dendritic functional regions, and some channels
  have been modified, others removed, and new added.  It is basically
  a new model with some components shared between models.
\item There is currently no possibility to document the assumptions or
  hypotheses upon which the model is based, ie. comparing the
  subregions in the edsjb1994 dendrite with the assumption built into
  the Japanese model.
\item The channels are reported the same way as the EDS publications.
\item The publications are~\cite{miyasho01:_low_ca2_purkin,
    chono03:_purkin}.
\item Currently we cannot integrate this model with the publication
  system as we don't have its model scripts.
\end{itemize}

\subsubsection{Different species???}
Passive models: See Segev model above.


\subsubsection{Propagated to Neuron}
\begin{itemize}
\item Importing Neuron models can be made possible.  It assumed to
  take less time than the backward compatibility module because the
  Neuron script languages are not as sophisticated as the G-2 SLI.
\item From a user perspective, loading SLI models is the same as
  loading NDF models (and conversions are applied in the background).
\item Importing Neuron models will work in the same way, with
  conversions applied in the background.  The graphical part of the
  G-Tube does not need explicit support for NEURON files.
\end{itemize}

\subsubsection{Allan model ??}
Implementing Allan's model in G3 will require a profound investigation
for how to:
\begin{itemize}
\item Connect its dedicated solver to SSP.
\item How to do its model specification.
  \begin{enumerate}
  \item interfacing with the model-container?
  \item setup the solver from the model-container?
  \end{enumerate}
\item How such an entirely different model with entirely different
  concepts integrates with the GUI.  This is an interesting exercise
  we should do to check the flexibility and robustness of both
  high-level concepts and technical implementation.
\item The technical reports that document the implementation workflow
  can be found in the explanation for how to
  \href{../genesis-extend-functionality/genesis-extend-functionality.tex}{extend
    GENESIS functionality}.
\end{itemize}

\subsection{Lineage implementation / model comparison}

These three are detailed above.


\subsubsection{Differences in structure}
\begin{itemize}
\item Text.
\item visually / graphically.
\item plot time constants for the channel.
\end{itemize}

The G-Tube allows to show multiple morphology, passive models, channel
characteristics and channel distribution tables simultaneously.

The differences in structure are visually represented using different
colors (green color: same parameters / structure, red color:
differences in parameters / structure).

\subsubsection{Difference in behavior}
\begin{itemize}
\item Current injection (two traces one from each model).  A library
  of SSP schedules is used to setup the simulation required to produce
  the traces (see also below).
\item Is currently not possible: two animated PCs on either side with
  the data plot in middle.
\end{itemize}


\subsection{Attribution}
\begin{itemize}
\item The model-container will define a new parameter
  'PUBLICATION\_ATOM' that points to an external file.  The external
  file contains 'publication atoms' possibly including bibtex
  references to papers, tagged free text such as abstract and author
  comments.  The PUBLICATION\_ATOM parameter is available for all
  model types known by the model container (cell, channel, etc).
\item The full pathname of the external file is guaranteed to be
  unique in a distributed software system.
\item Based on the PUBLICATION\_ATOM parameters of a complex model,
  and by assigning 'attribution scores' to each model component, the
  attribution of one or more contributors to a model can be fully
  quantified.
\item Likewise the 'local' past and current importance of a model
  component can be quantified by counting its usage from the lineage
  tree.  In a centralized database with peer-reviewed publications the
  'local importance' provides an overall measure of the significance
  of a model component.  The algorithms to compute the importance of a
  model component shares principles with google algorithms to
  attribute scores to google search hits.  They are also related to
  spanning tree and coverage calculations in graph networks.
\item The gshell's show\_library command has extensions for file types
  such as 'ndf' files, 'g2' files, ao.  This command is already
  supported by the g-tube.  A new extension type 'cite' lists all the
  citations related to one model (channel, morphology, or other) and
  allows to browse its lineage.
\end{itemize}


\subsubsection{Purkinje cell model}
\begin{itemize}
\item Attributions.
\item Made like Sci citations Index.
\end{itemize}


\subsection{Reconstruct the figures in the papers from the model in real time}
\begin{itemize}
\item A library of SSP schedules with different stimulation paradigms
  is available.  The GUI instructs the SSP scheduler to load a
  schedule from the library and then runs the simulation.  This
  produces the output for one figure.  Tables and channel kinetic plot
  reconstruction may need a different mechanism.
\item Most of the SSP configurations required for this functionality
  are already available from different locations in the source code
  (eg. the regression tests).  After initial creation of this library,
  the buttons that implement functions for model comparison implicitly
  select one or more SSP configuration files and run a simulation.
\item Obviously figure reproduction can also be done manually from the
  g-tube menus.  For this, the G-Tube needs an interface to access and
  load schedules from the SSP library.  This interface is available
  from a menu that is part of the (still to be described) publication
  / research workflow.
\item Future suggestion: the G-Tube supports configuration files that
  after loading, popup two plot windows, instruct it to load an SSP
  (simulation) configuration for each plot window, and then run those
  simulations.
\end{itemize}


\subsubsection{DeSchutter papers}
\begin{itemize}
\item Identify the industry standard model measures
  \begin{enumerate}
  \item Passive
  \item Current injection / voltage clamp
  \item Synaptic input
  \item Physiological stimulation
  \end{enumerate}
\end{itemize}

\subsection{Tutorialize this}

\begin{enumerate}
\item PDF of standard paper.
\item Narrative components / Bullet points
  \begin{enumerate}
  \item Figures based
  \end{enumerate}
\item Points of deviation
  \begin{enumerate}
  \item Tells the authors in review
  \end{enumerate}
\end{enumerate}


\section{DONE List}


\bibliographystyle{abbrv}
\bibliography{../tex/bib/g3-refs.bib}

\end{document}


%%% Local Variables: 
%%% mode: latex
%%% TeX-master: t
%%% End: 
