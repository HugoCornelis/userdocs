\documentclass[12pt]{article}
\usepackage{verbatim}
\usepackage[dvips]{epsfig}
\usepackage{color}
\usepackage{url}
\usepackage[colorlinks=true]{hyperref}

\begin{document}

\section*{GENESIS: Documentation}

{\bf Related Documentation:}
% start: userdocs-tag-replace-items related-do-nothing
% end: userdocs-tag-replace-items related-do-nothing

\section*{Developer Installation Mac OSX}

The procedure for the installation of GENESIS on a Mac OSX is similar to that for a Linux build on a regular PC. The major difference is that it requires build tools for the Mac. These are somewhat hidden on disk 2 of the Mac OS installation disks. Installation requires administrator privileges. Note that  some installation must be done from the terminal. This can be found in the Applications --$>$ Utilities folder. For regular use of GENESIS it is convenient to place a Terminal in the Doc for easy access.
Installation requires executing the following major steps:
\begin{itemize}
   \item[] Prepare and upgrade the system software.
   \item[] Download and install the developer package.
   \item[] Install the software packages.
   \item[] Check if the installation was successful. 
\end{itemize}

\subsection*{Prepare and upgrade the system software}

Install the following packages:
\begin{itemize}
   \item[]\href{http://monotone.ca/}{\bf monotone-4.0:} (or higher).
   \item[]{\bf Xcode:} The Apple Development Toolset. Xcode is available on disk 2 of Mac OSX under optional install. It is also available from the \href{http://developer.apple.com/technology/xcode.html}{Apple Developer Website}. In the latter case  you will need a valid Apple ID to log in and access the developer page. Note: It is not recommended that the compilers available from macports are used for installation.
   \item[]Install the following perl dependencies via CPAN:
      \begin{itemize}
         \item YAML
         \item Inline
         \item RecDescent?
         \item Expect (and its dependency perl-IO)
         \item ExtUtils?
         \item Data::Comparator
         \item File::Find::Rule
      \end{itemize}
   \end{itemize}
   
\subsection*{Install the software packages}

Once the above dependencies are installed source code can be pulled from the GENESIS repository and a build performed with the following commands:
\begin{itemize}
   \item[]{\it neurospaces\_create\_directories}
   \item[]{\it neurospaces\_pull}
   \item[]{\it neurospaces\_update}
   \item[]{\it neurospaces\_configure}
   \item[]{\it neurospaces\_install} (requires admin password) 
\end{itemize}

\subsection*{Check if the installation was successful}

Once these have completed the installation can be tested with the following command:
\begin{itemize}
   \item[]{\it neurospaces\_check $ >$ /tmp/check.out 2 $>$ \&1}
\end{itemize}
This will perform a series of regression tests and save the output to file. The output can be checked by typing ``{\tt less /tmp/check.out}''. Importantly, search for lines containing the string {\tt error\_count} that will indicate any inconsistencies with the installation.

\end{document}
