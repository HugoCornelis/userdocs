\documentclass[12pt]{article}
\usepackage{verbatim}
\usepackage[dvips]{epsfig}
\usepackage{color}
\usepackage{url}
\usepackage[colorlinks=true]{hyperref}

\begin{document}

\section*{GENESIS: Documentation}

{\bf Related Documentation:} \\
\href{../genesis-system/genesis-system.tex}{\bf GENESIS\,System}
% start: userdocs-tag-replace-items related-do-nothing
% end: userdocs-tag-replace-items related-do-nothing


\section*{Developer Installation From Source On Mac OSX}

The procedure for the installation of GENESIS on a Mac OSX is similar to that for a Linux build on a regular PC. The major difference is that it requires build tools for the Mac. These are somewhat hidden on disk 2 of the Mac OS installation disks. Installation requires administrator privileges. ({\bf Note:} Some installation must be done from the Terminal). If you do not have administrator privileges you should contact your system administrator for help. The Terminal can be found in the Applications --$>$ Utilities folder. For regular use of GENESIS it is convenient to place a Terminal in the dock for easy access.
Installation requires executing the following major steps:
\begin{itemize}
   \item[] Prepare and upgrade the system software.
   \item[] Download and install the developer package.
   \item[] Install the software packages.
   \item[] Check if the installation was successful. 
\end{itemize}

GENESIS3 has been compiled and verified to work on the following OS versions:

\begin{itemize}
 \item[] \href{../installation-osx-leopard/installation-osx-leopard.tex}{Leopard}
 \item[] \href{../installation-osx-leopard/installation-osx-snow-leopard.tex}{Snow Leopard}
 \item[] \href{../installation-osx-leopard/installation-osx-lion.tex}{Lion}
\end{itemize}



\end{document}
