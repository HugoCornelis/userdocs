\documentclass[12pt]{article}
\usepackage[dvips]{epsfig}
\usepackage{color}
%e.g.  \textcolor{red,green,blue}{text}
\usepackage{url}
\usepackage[colorlinks=true]{hyperref}

\begin{document}

\section*{GENESIS: Tutorial 2}

{\bf Related Documentation:}
% start: userdocs-tag-replace-items related-tutorial
% end: userdocs-tag-replace-items related-tutorial

\section*{Using the GENESIS Shell for\\Multi-compartment Modeling and Simulation}

In this second of two introductory tutorials you will be guided through a brief session using the GENESIS shell (\href{../gshell/gshell.tex}{\bf G-Shell}). In this tutorial it is assumed that you are familiar with the contents of \href{../tutorial1/tutorial1.tex}{\bf Tutorial\,1} and the links it contains. You will be shown how to use basic features of the {\bf G-Shell} to create, explore, and run a realistic multi-compartment model neuron. For example:
\begin{verbatim}
    ndf_load cells/purkinje/edsjb1994.ndf
    run /Purkinje 0.1
\end{verbatim}

\subsection*{Import Model}

A large number of predefined model neurons come bundled with the GENESIS distribution. The command:
% start: userdocs-tag-replace-items related-command-library_show
% end: userdocs-tag-replace-items related-command-library_show
\begin{verbatim}
    genesis > library_show
\end{verbatim}
lists the available model libraries. Example library files can be found with:
\begin{verbatim}
    genesis > library_show ndf examples
\end{verbatim}
You can display the contents of the single cell library with:
\begin{verbatim}
    genesis > library_show ndf cells
\end{verbatim}
which returns in part:
\begin{verbatim}
    ---
    ndf_library:
     cells:
       ...
       - purkinje/
       ...
\end{verbatim}
This indicates the existence of a directory in the library where one or more Purkinje cell models are stored.
The documentation on \href{../models-library-additions/models-library-additions.html}{\bf Additions to the Models Library}
describes some other NDF files of models that have been converted from GENESIS 2.

The contents of the {\it purkinje} directory can be displayed with:
\begin{verbatim}
    genesis > library_show ndf cells/purkinje
    ---
    ndf_library:
     cells/purkinje:
       - edsjb1994.ndf
       - edsjb1994_partitioned.ndf
\end{verbatim}

To run one of these models the model description file (NDF) must first be loaded into the {\bf G-Shell}.  (To learn more
about the NDF format see \href{../ndf-file-format/ndf-file-format.tex}{\bf Introduction\,to\,the\,NDF\,File\,Format}.)
More details of cell model representation can be found in \href{../technical-guide-1/technical-guide-1.tex}
{\bf Technical Guide 1}.

For example, to load the multicompartment De Schutter and Bower Purkinje cell model referenced above, enter:

\begin{verbatim}
    ndf_load cells/purkinje/edsjb1994.ndf
\end{verbatim}
This command results in the complete Purkinje cell model being made available to the {\bf G-Shell}.

{\bf Note:} Currently, it is not advisable to load more than one model at a time into the {\bf G-Shell}. Although multiple models can be loaded, the command {\it ndf\_save} will save all models to the ndf file. To change to a different model you can either issue the {\it delete} command (see example below) or exit the {\bf G-Shell}, restart GENESIS, and load the new model (see the \href{../gshell/gshell.tex}{\bf Introduction\,to\,the\,GENESIS\,Shell}).

\subsection*{Explore Model}

To determine the name of a model that has been loaded and made available to the {\bf G-Shell} (i.e. the name located at the root node of the element hierarchy tree), enter, for example: 
\begin{verbatim}
    genesis > list_elements
    ---
    - /Purkinje
\end{verbatim}
We can then determine the child elements of {\it /Purkinje} with:
\begin{verbatim}
    genesis > list_elements /Purkinje
    ---
    - /Purkinje/segments
\end{verbatim}
A list of all segments in the model can then be generated with another {\it list\_elements} command:
\begin{verbatim}
    genesis > list_elements /Purkinje/segments
    ---
    - /Purkinje/segments/soma
    - /Purkinje/segments/main[0]
    . . .
    - /Purkinje/segments/b3s46[15]
\end{verbatim}
Choosing one of the segments at random (here, for example {\it /Purkinje/segments/b3s45[10]}), we can generate a list of the elements that are associated with the given segment either by using the full pathname:
\begin{verbatim}
    genesis > list_elements /Purkinje/segments/b3s45[10]
    ---
    - /Purkinje/segments/b3s45[10]/cat
    - /Purkinje/segments/b3s45[10]/cap
    - /Purkinje/segments/b3s45[10]/kc
    - /Purkinje/segments/b3s45[10]/k2
    - /Purkinje/segments/b3s45[10]/km
    - /Purkinje/segments/b3s45[10]/ca_pool
    - /Purkinje/segments/b3s45[10]/stellate
    - /Purkinje/segments/b3s45[10]/Purkinje_spine_0
\end{verbatim}
or by making the chosen element the current working element via the {\it ce} (change working element) command. We can then check what the current working element is with the {\it pwe} (print working element) command:
\begin{verbatim}
    genesis > ce /Purkinje/segments/b3s45[10]
    genesis > pwe
    /Purkinje/segments/b3s45[10]
    genesis > list_elements .
    ---
    - /Purkinje/segments/b3s45[10]/cat
    . . .
    - /Purkinje/segments/b3s45[10]/Purkinje_spine_0
\end{verbatim}
The ability to change the current working element allows us to either
check the value of the model parameters associated with the current
working element (such as {\tt RM} or {\tt CM}) via the command {\it
  model\_parameter\_show}, change default parameter values via the
{\it model\_parameter\_set} command or view the scaled value of the
parameter via the command {\it parameter\_scaled\_show}, for example:
\begin{verbatim}
    genesis > pwe
    /Purkinje/segments/b3s45[10]
    genesis > model_parameter_show .
    ---
       'parameter name': SURFACE
       type: number
       value: 1.54645e-10
       . . .
       'parameter name': ELEAK
       type: number
       value: -0.08
    genesis > parameter_show /Purkinje/segments/b3s45[10] RM
    value = 3    
    genesis > model_parameter_add /Purkinje/segments/b3s45[10] RM 2.0
    genesis > parameter_show /Purkinje/segments/b3s45[10] RM
    value = 2
    genesis > parameter_scaled_show /Purkinje/segments/b3s45[10] RM
    scaled value = 1.29328e+10
\end{verbatim}

{\bf Note:}

\begin{itemize}
%\item The commands {\it ce}, {\it set\_model\_parameter}, and {\it show\_parameter} currently only accept the full path name (e.g. {\it  /Purkinje/segments/b3s45[10]}) to a segment even if the parameter to be queried or changed is in the current working element. % or just the segment name itself (as given in the above example).  Parameter values associated with a given element can be changed using either the full path name or making the given segment the current working element (as with the above example).

\item In the {\bf G-Shell} the commands ``{\bf .}'' and ``{\bf ..}'' operate in the same way as when employed to traverse the Unix file system (see \href{../unix-linux/unix-linux.tex}{\bf Introduction\,to\,Unix\,and\,the\,Linux\,Graphical\,Desktop}). The ``{\bf ..}'' operator may be chained, e.g. ``{\bf ../../../}''.

\item Also remember that in the GENESIS documentation, an elipsis ($\ldots$) indicates missing lines.

\end{itemize}

Another Purkinje cell in the NDF library {\it cells/purkinje} is {\it edsjb1994\_partitioned.ndf}. This model is identical to edsjb1994.ndf with the exception that its morphology has been repartitioned. It is simple to delete the current model and load the repartitioned version:
\begin{verbatim}
    genesis > delete /Purkinje
    genesis > ndf_load cells/purkinje/edsjb1994_partitioned.ndf
\end{verbatim}
We can now ask for a summary of the morphology:
\begin{verbatim}
    genesis > morphology_summarize /Purkinje
\end{verbatim}
which generates the following output:
\begin{verbatim}
    Number of segments: 4548
    Number of segments without parents: 1
    Number of segment tips: 1474
\end{verbatim}
We can also list the segments in {\it /Purkinje/segments}:
\begin{verbatim}
    genesis > list_elements /Purkinje/segments
    ---
    - /Purkinje/segments/soma
    - /Purkinje/segments/main
    - /Purkinje/segments/branches
    - /Purkinje/segments/branchlets
\end{verbatim}
or find the nine segments in the main dendrite ({\it /Purkinje/segments/main}), for example:
\begin{verbatim}
    genesis > list_elements /Purkinje/segments/main
    ---
    - /Purkinje/segments/main/main[0]
    - /Purkinje/segments/main/main[1]
    . . . 
    - /Purkinje/segments/main/main[8]
\end{verbatim}
and so on for the branches (``{\tt list\_elements /Purkinje/segments/branches}'') and branchlets (``{\tt list\_elements /Purkinje/segments/branchlets}'').

It is also possible to list the location of the spine heads in the dendritic morphology:
\begin{verbatim}
    genesis > morphology_list_spine_heads /Purkinje
\end{verbatim}
This generates considerable output that locates each dendritic spine head in the element hierarchy:
\begin{verbatim}
    tips:
      name: Purkinje
      names:
        - /Purkinje/segments/b0s01[1]/Purkinje_spine_0/head
        - /Purkinje/segments/b0s01[2]/Purkinje_spine_0/head
        . . .
        - /Purkinje/segments/b3s46[15]/Purkinje_spine_0/head
\end{verbatim}

\subsection*{Save Model}

There are two ways to save a model in the {\bf G-Shell}.

\begin{enumerate}
   \item{\bf Explicitly:} A model can be saved by giving its name as loaded into the {\bf G-Shell} and an explicit path to the directory (including the name of the NDF file) where you want to save it:
   \begin{verbatim}
ndf_save /Purkinje /<directory path>/<file name>.ndf
   \end{verbatim}
   \item{\bf Generically:} Alternatively, if the name of the model loaded into the {\bf G-Shell} is not known, a wild card construct can be employed (although remember that a {\it list\_elements} command with no argument will report the name of the root element of the model hierarchy, in the example used here, {\it /Purkinje}):
   \begin{verbatim}
ndf_save /** /<directory path>/<file name>.ndf
   \end{verbatim}   
\end{enumerate}
{\bf Note:} In the absence of an explicit path to the directory where you would like to save the NDF file, it will be saved by default in the directory where you run the {\bf G-shell}.

\subsection*{Define Simulation Constants}

Here, for example, we set the run-time activity of the model neuron by finding and
setting the maximum conductance of the $Ca_T$ channels located at the soma. To find the settable parameters enter:
\begin{verbatim}
   genesis > list_elements
   ---
   - /Purkinje
   genesis > list_elements /Purkinje
   ---
   - /Purkinje/SpinesNormal_13_1
   - /Purkinje/segments
   genesis > list_elements /Purkinje/segments
   ---
   - /Purkinje/segments/soma
   - /Purkinje/segments/main
   - /Purkinje/segments/branches
   - /Purkinje/segments/branchlets
   genesis > list_elements /Purkinje/segments/soma
   ---
   - /Purkinje/segments/soma/km
   - /Purkinje/segments/soma/kdr
   - /Purkinje/segments/soma/ka
   - /Purkinje/segments/soma/kh
   - /Purkinje/segments/soma/nap
   - /Purkinje/segments/soma/naf
   - /Purkinje/segments/soma/cat
   - /Purkinje/segments/soma/ca_pool
\end{verbatim}
The value of model parameters for $Ca_T$ channels located at the soma can then be found with:
\begin{verbatim}
   genesis > model_parameter_show /Purkinje/segments/soma/cat
     -
       'parameter name': G_MAX
       type: number
       value: 5
    . . .  
\end{verbatim}

As discussed in \href{../tutorial1/tutorial1.tex}{\bf Tutorial 1}, this may also be treated as a run-time parameter.
The runtime value of {\tt G\_MAX} can then be set with the following command:
\begin{verbatim}
   genesis > runtime_parameter_add /Purkinje/segments/soma/cat G_MAX 4
\end{verbatim}
To check that the parameter value has been correctly set, enter:
\begin{verbatim}
   genesis > runtime_parameter_show /Purkinje/segments/soma/cat G_MAX
   ---
   runtime_parameters:
      - component_name: /Purkinje/segments/soma/cat
         field: G_MAX
         value: 4
         value_type: number
\end{verbatim}
{\bf Note:} Only the run time value of this parameter has been altered. The default value defined by the model is unaffected:
\begin{verbatim}
   genesis > model_parameter_show /Purkinje/segments/soma/cat
     -
       'parameter name': G_MAX
       type: number
       value: 5
\end{verbatim}

\subsection*{Define Simulation Inputs}

There are currently three ways to provide stimulation inputs to a single cell model. Once the model is loaded into the {\bf G-Shell}, either of the following input paradigms can be used to provide cell activation:

\subsubsection*{Current Injection}

A 2\,nA current injection can be applied to the soma with the following command:
\begin{verbatim}
   genesis > runtime_parameter_add /Purkinje/segments/soma INJECT 2e-9
\end{verbatim}

Information about these runtime parameters can be obtained with:
\begin{verbatim}
   genesis > runtime_parameter_show
   ---
   runtime_parameters:
      - component_name: /Purkinje/segments/soma
         field: INJECT
         value: 2e-9
         value_type: number
\end{verbatim}

\subsubsection*{Current or Voltage Clamp}

For details see the \href{../pclamp/pclamp.tex}{\bf Perfect\,Clamp} tutorial.

\subsubsection*{Dynamic Clamp}

One project being undertaken by GENESIS developers is the interfacing of the Real-Time eXperiment Interface (\href{http://www.rtxi.org/}{\bf RTXI}) with GENESIS to enable the realtime tuning of model parameters.

RTXI is a collaborative open-source software development project aimed at producing a real-time Linux based software system for hard real-time data acquisition and control applications in biological research. \\

\subsubsection*{Synaptic Activation}
\begin{enumerate}
   \item{\bf Endogenous Poissonian Synaptic Activation:} This provides the simplest way to activate synapses.
   \paragraph{Excitatory:} The following command applies a Poisson stream of event times at 25\,Hz to glutamate synapses on spine heads.
\begin{verbatim}
genesis > runtime_parameter_add spine::/Purk_spine/head/par FREQUENCY 25
\end{verbatim}
   \paragraph{Inhibitory:} The following command applies a Poisson stream of event times at 1\,Hz to GABAergic synapses on dendrites flagged as thick.
\begin{verbatim}
genesis > runtime_parameter_add thickd::gaba::/Purk_GABA FREQUENCY 1
\end{verbatim}
Information about the run time parameters that have been set can be obtained with:
\begin{verbatim}
genesis > runtime_parameter_show
\end{verbatim}
which for the endogenous excitatory and inhibitory Poissonian synaptic activation set above would return:
\begin{verbatim} 
runtime_parameters:
   - component_name: thickd::gaba::/Purk_GABA
      field: FREQUENCY
      value: 1
      value_type: number
   - component_name: spine::/Purk_spine/head/par
      field: FREQUENCY
      value: 25
      value_type: number
 \end{verbatim}
    
   \item{\bf Extrinsic Synaptic Activation:} This provides a much more flexible way to activate synapses than the endogenous activation method. Here, explicit synaptic activation times pre-generated outside of GENESIS (e.g. from a specified probability distribution using Matlab) are read from a file containing the required event list.
   
   After identifying which synapses should be activated, e.g.
   \begin{verbatim}
genesis > morphology_summarize /Purkinje
genesis > parameter_show /Purkinje/segments/branchlets/b1s06/b1s06[182] SOMATOPETAL_BRANCHPOINTS
   value = 23 
   \end{verbatim}
   The event list can then be attached to the spine heads of the specified dendritic segment:
   \begin{verbatim}
genesis > runtime_parameter_add /Purkinje/segments/branchlets/b1s06/b1s06[182]/Purkinje_spine_0/head/par/synapse EVENT_FILENAME event_data/events.yml
   \end{verbatim}   
\end{enumerate}

\subsection*{Define Simulation Outputs}

Outputs can be specified from any number of compartments and can consist of any variable known to the mathematical solvers. For example, the following commands will define simulation outputs from the soma for the membrane potential ($V_m$) and calcium concentration ($[Ca]_{in}$):
\begin{verbatim}
   genesis > output_add /Purkinje/segments/soma Vm
   genesis > output_add /Purkinje/segments/soma/ca_pool Ca
\end{verbatim}

\subsection*{Check Simulation}

The following command is used to check the simulation:
\begin{verbatim}
   genesis > check /Purkinje
\end{verbatim}

\subsection*{Run Simulation}

To run a simulation for 10\,s we can use the same {\it run} command as used for the simple one compartment model in \href{../tutorial1/tutorial1.tex}{\bf Tutorial\,1}
\begin{verbatim}
    genesis > run /Purkinje 10.00
\end{verbatim}

\subsection*{Reset Simulation}

A simulation can be returned to its initial state, where the time step of the simulation is set to zero and the initial values for all solved variables are loaded, with the following command:
\begin{verbatim}
    genesis > reset /Purkinje
\end{verbatim}

\subsection*{Save Model State}

The state of a simulation can be saved to a file name of your choosing, for example:
\begin{verbatim}
   genesis > model_state_save /Purkinje /<directory path>/<file name>.ndf
\end{verbatim}
This command saves the state of a model to a file at the given location following the last update time step of a run. In the absence of a directory path, the file is saved by default in the current directory, i.e. the directory from which the {\bf G-Shell} was initiated.

A simulation can then be initiated from this saved 
state with the command 
\begin{verbatim}
   genesis > model_state_load /<directory path>/<file name>.ndf 
\end{verbatim}
The next {\it run} command will then start the simulation from this reinitialized state.

\subsection*{Check Simulation Output}

Simulation output (in the default location) can be checked with the following command:
\begin{verbatim}
   genesis > sh cat /tmp/output
\end{verbatim}
This will print the values in {\it /tmp/output} to the screen.

{\bf Note:} Most common UNIX shell commands can be run in the {\bf G-Shell} if passed as arguments to the {\tt sh} command.

The next tutorial in this series (\href{../tutorial3/tutorial3.html}{\bf Tutorial 3}) is devoted to the process of
loading existing GENESIS 2 simulation scripts into the G-shell, saving the
models in NDF files, and running them in GENESIS 3.


\end{document}
