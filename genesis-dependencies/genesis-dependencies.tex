\documentclass[12pt]{article}
\usepackage{verbatim}
\usepackage[dvips]{epsfig}
\usepackage{color}
\usepackage{url}
\usepackage[colorlinks=true]{hyperref}

\begin{document}

\section*{GENESIS: Documentation}

{\bf Related Documentation:}
% start: userdocs-tag-replace-items related-do-nothing
% end: userdocs-tag-replace-items related-do-nothing

\section*{GENESIS Software Dependencies}

GENESIS-3 builds smoothly and without problems on most current
operating systems provided all software dependencies are installed
(see~\href{../developer-installation/developer-installation.tex}{developer
  installation documentation}).  This page summarizes the dependencies
required to build GENESIS-3.  It also provides details about known
problems on some operating systems.


\section*{Software Dependencies for User Installation}

Generally, there should be no software dependencies or configuration requirements for installing GENESIS as a user (c.f. a developer, see below).
%The exception is if you want to verify your installation. As installation testing is performed as a sequence of regression tests you must have specific Perl modules installed on your machine.

%CPAN as the standard location where Perl modules can be obtained. To install a Perl module from CPAN invoke, for example: 
%\begin{verbatim}
%   'perl -MCPAN -e ``install <required perl module>'' '
%\end{verbatim}

%If you don't feel comfortable doing this, you should get your systems administrator to install the following Perl modules required for user installation tests: 

%\begin{itemize}
%   \item[]Data::Comparator
%   \item[]Data::Transformator
%   \item[]YAML
%\end{itemize}

\section*{Software Dependencies for Developer Installation}

This is a general list of dependencies for configuring a computer for
GENESIS software development.  Also look at the list of \href{../tested-distributions/tested-distributions.tex}{\bf Tested\,Operating\,Systems} to see what distributions have actually
been tested.

Most of the required programs, libraries and modules are available as
packages specific for your Operating System.  This is the most convenient way to install them.

\subsection*{Required Programs}

\begin{itemize}
\item {\it automake}
\item {\it autoconf}
\item {\it make}
\item {\it gcc} (and dependencies)
\item {\it perl}
\item {\it bison}
\item {\it flex} (and libraries)
\item {\it swig}
\item {\it monotone}
\end{itemize} 

\section*{Required Libraries}

This section lists some libraries required by GENESIS-3 in order to compile from source or do development. Lists are separated by operating system. 

\subsection*{Libraries for Ubuntu}

To install these packages use the {\it apt-get} command via the command line, or the synaptic package manager. 

\begin{itemize}
\item {\it libc6-dev}
\item {\it lib(n)curses-dev}
\item {\it libperl-dev}
\item {\it libreadline5-dev}
\item {\it libhistory-dev} (Note: some linux builds and MAC OSX have
  {\it libhistory} built into another standard library such as {\it
    stdio} or {\it libedit})
\item {\it readline-dev}
\item {\it python-dev} (for your python version)
\end{itemize} 


\subsection*{Libraries for Fedora}

To install packages in Fedora use the {\it yum} utility from the command line. Since {\it yum} repositories typically contain different versions of the same package for various architectures, you must know your machines architecture type and insert it place of "arch" in the package name. The usual architecture types are x86\_64, i586 and i686.

\begin{itemize}
\item {\it ncurses-devel.arch}
\item {\it perl-devel.arch}
\item {\it readline-devel.arch}
\item {\it python-devel.arch}
\end{itemize}



\subsection*{Required Perl Modules}

The following Perl modules are required.  If you are using Linux, they
are likely available as packages for your Linux distribution or vendor.

\begin{itemize}
   \item CPAN
   \item YAML
   \item Parse::RecDescent
   \item Inline (which includes Inline::C)
   \item ExtUtils::Embed
   \item LWP::Simple
   \item Expect (including IO::Pty and IO::Tty)
   \item Test::More
   \item Clone
   \item Data::Utilities
   \item File::Find::Rule
   \item Digest::SHA
   \item Inline::Python
\end{itemize}


\subsection*{Note about Inline::Python}

The {\bf Perl Inline::Python} is compiled against the current installed version of {\bf Python}. If you upgrade {\bf Python}, then the {\bf Inline::Python} module must be updated as well. This can be done via {\bf CPAN} as before, only now, since the module has already been installed you may get this message:

\begin{verbatim}
	Inline::Python is up to date (0.36).
\end{verbatim}

In this case you may have to "force" the installation of the module on {\bf CPAN} like so:

\begin{verbatim}
cpan[6]> force install Inline::Python
\end{verbatim}

If there are still old temp files in the {\bf CPAN} .build directory, this may still give an error and say that it is already installed. In that case there is a source tarball for Inline-Python-0.36 available in the contrib directory of the \href{../developer-package/developer-package.tex}{\bf Developer Package} ($_{\widetilde{~}}$/neurospaces\_project/developer/source/snapshots/0/contrib/). Unzip the Inline-Python-0.36.tar.gz tarball with the {\bf tar} command:

\begin{verbatim}
tar -zxvf Inline-Python-0.36.tar.gz
\end{verbatim}

Change directories into Inline-Python-0.36 directory and then start the build process my typing:

\begin{verbatim}
perl Makefile.PL
\end{verbatim}

If CPAN proceeded without an "up to date" message you should wind up at this point, same as starting the make from the tarball. If there is more than one {\bf Python} executable in your path the next line is important. The Maker will prompt you for which {\bf Python} to build against:

\begin{verbatim}
	Found these python executables on your PATH:
	1. /Library/Frameworks/Python.framework/Versions/2.6/bin/python
	2. /Library/Frameworks/Python.framework/Versions/Current/bin/python
	4. /usr/bin/python
	5. /usr/local/bin/python
	Use which? [1] 
\end{verbatim}

You should check to see which is the default {\bf Python} that is called from your command line by performing:

\begin{verbatim}
which python
\end{verbatim}

It will print out the first {\bf Python} that is in your path. If the most recent version of {\bf Python} on your machine is not appearing, then prefix the most recent installation path to your PATH environment variable by a line similar to this in your .bashrc (or .profile if on Mac OSX):

\begin{verbatim}
	PATH="/Path/To/Most/Recent/Python/bin:${PATH}"
	export PATH
\end{verbatim}

After selecting the {\bf Python} to build against, the Maker will create a Makefile. From this point you can build and install:

\begin{verbatim}
make
sudo make install
\end{verbatim}

If you wish to check if the module is functioning properly there is a test script in the contrib directory called {\it pythontest}. Running this script will print out some information about the {\bf Python} environment:

\begin{verbatim}
The Inline Python caller is:  /usr/bin/perl
The Inline Python caller should be some derivative of perl, since it is being called via the perl Inline::Python module

Python version is  2.6.5 (r265:79359, Mar 24 2010, 01:32:55) 
[GCC 4.0.1 (Apple Inc. build 5493)] 

Python path is: 
['/Users/Abominous/neurospaces_project/developer/source/snapshots/0/contrib', '/usr/local/glue/swig/python', '/usr/local/lib/wxPython-unicode-2.8.11.0/lib/python2.6/site-packages', '/usr/local/lib/wxPython-unicode-2.8.11.0/lib/python2.6/site-packages/wx-2.8-mac-unicode', '/Library/Frameworks/Python.framework/Versions/2.6/lib/python26.zip', '/Library/Frameworks/Python.framework/Versions/2.6/lib/python2.6', '/Library/Frameworks/Python.framework/Versions/2.6/lib/python2.6/plat-darwin', '/Library/Frameworks/Python.framework/Versions/2.6/lib/python2.6/plat-mac', '/Library/Frameworks/Python.framework/Versions/2.6/lib/python2.6/plat-mac/lib-scriptpackages', '/Library/Frameworks/Python.framework/Versions/2.6/lib/python2.6/lib-tk', '/Library/Frameworks/Python.framework/Versions/2.6/lib/python2.6/lib-old', '/Library/Frameworks/Python.framework/Versions/2.6/lib/python2.6/lib-dynload', '/Library/Frameworks/Python.framework/Versions/2.6/lib/python2.6/site-packages', '/usr/local/lib/wxPython-unicode-2.8.11.0/lib/python2.6', '/usr/local/glue/swig/python']

The Python Path should match up with the proper python version of the executable.

This executable path /usr/bin

Performing an import check of installed GENESIS3 Python modules:

Done with Python check.
\end{verbatim}

If the module is functioning properly then the Inline Python caller should be your installation of {\bf Perl}, NOT your installation of {\bf Python}. Also check to see that the directories for {\bf Python} modules from the Python path match up with your current {\bf Python} version number. Again, your "executable path" should be the path to your {\bf Perl} executable. The final check it performs is a module check of all of the installed GENESIS3 {\bf Python} modules, giving a message if one is not installed or not in your path. 

\subsection*{Trouble Shooting}

Periodically the build of a Perl module can fail and requires a workaround to get it installed. A common error that occurs is a failure to produce a {\it Makefile} as in this example with {\tt Data::Utilities} :

\begin{verbatim}
	Checking if your kit is complete...
	Looks good
	Writing Makefile for Data::Utilities
	Use of uninitialized value in concatenation (.) or string at /usr/lib/perl5/5.10.0/CPAN.pm line 7652.
	Alert: no Build file available for 'make ' in cwd[/root/.cpan/build/Data-Utilities-0.04-MT2keD]. Danger, Will Robinson!
	Can't exec "./Build": No such file or directory at /usr/lib/perl5/5.10.0/CPAN.pm line 7698.
	  CORNELIS/Data-Utilities-0.04.tar.gz
	  ./Build -- NOT OK
	Running Build test
	  Can't test without successful make
	Running Build install
	  Make had returned bad status, install seems impossible
	Failed during this command:
	 CORNELIS/Data-Utilities-0.04.tar.gz          : make NO
\end{verbatim} 

In the event this happens it's often possible to manually build the module by entering the package directory and creating the {\it Makefile} via the module's Perl script. As shown in the output, CPAN downloaded and unzipped the package to the directory {\it /root/.cpan/build/Data-Utilities-0.04-MT2keD}. So to manually build the {\it Makefile} and module we perform this series of commands:

\begin{verbatim}
cd /root/.cpan/build/Data-Utilities-0.04-MT2keD
\end{verbatim}

Now create the {\it Makefile}:

\begin{verbatim}
perl Makefile.PL
\end{verbatim}

At this point a {\it Makefile} will be present, all that needs to be done is to build and install:

\begin{verbatim}
make
sudo make install
\end{verbatim}

The module should now be in your Perl library. 

\end{document}
