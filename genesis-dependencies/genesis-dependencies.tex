\documentclass[12pt]{article}
\usepackage{verbatim}
\usepackage[dvips]{epsfig}
\usepackage{color}
\usepackage{url}
\usepackage[colorlinks=true]{hyperref}

\begin{document}

\section*{GENESIS: Documentation}

{\bf Related Documentation:}
% start: userdocs-tag-replace-items related-do-nothing
% end: userdocs-tag-replace-items related-do-nothing

\section*{GENESIS Software Dependencies}

\section*{Software Dependencies for User Installation}

Generally, there should be no software dependencies or configuration requirements for installing GENESIS as a user (c.f. a developer, see below). The exception is if you want to verify your installation. As installation testing is performed as a sequence of regression tests you must have specific Perl modules installed on your machine.

CPAN as the standard location where Perl modules can be obtained. To install a Perl module from CPAN invoke, for example: 
\begin{verbatim}
   'perl -MCPAN -e ``install <required perl module>'' '
\end{verbatim}

If you don't feel comfortable doing this, you should get your systems administrator to install the following Perl modules required for user installation tests: 

\begin{itemize}
   \item[]Data::Comparator
   \item[]Data::Transformator
   \item[]YAML
\end{itemize}

\section*{Software Dependencies for Developer Installation}

This is a general list of dependencies for configuring a computer for
GENESIS software development.  Also look at the list of \href{../tested-distributions/tested-distributions.tex}{Tested Operating Systems} to see what distributions have actually
been tested.

Most of the required programs, libraries and modules are available as
packages specific for you OS.  This is the most convenient way to install them.

\subsection*{Required Programs}

\begin{itemize}
\item {\it automake}
\item {\it autoconf}
\item {\it make}
\item {\it gcc} (and dependencies)
\item {\it perl}
\item {\it bison}
\item {\it flex} (and libraries)
\item {\it swig}
\item {\it monotone}
\end{itemize} 

\subsection*{Required Libraries}

\begin{itemize}
\item {\it libc6-dev}
\item {\it lib(n)curses-dev}
\item {\it libperl-dev}
\item {\it libreadline5-dev}
\item {\it libhistory-dev} (Note: some linux builds and MAC OSX have
  {\it libhistory} built into another standard library such as {\it
    stdio} or {\it libedit})
\item {\it readline-dev}
\item {\it python-dev} (for your python version)
\end{itemize} 

\subsection*{Required Perl Modules}

The following Perl modules are required.  If you are using Linux, they
are likely available as packages for your Linux distribution or vendor.

\begin{itemize}
   \item CPAN
   \item YAML
   \item Parse::RecDescent
   \item Inline (which includes Inline::C)
   \item ExtUtils::Embed
   \item LWP::Simple
   \item Expect (including IO::Pty and IO::Tty)
   \item Test::More
   \item Clone
   \item Data::Utilities
   \item File::Find::Rule
   \item Digest::SHA
   \item Inline::Python
\end{itemize}

\end{document}
