\documentclass[12pt]{article}
\usepackage{verbatim}
\usepackage[dvips]{epsfig}
\usepackage{color}
\usepackage{url}
\usepackage[colorlinks=true]{hyperref}

\begin{document}

\section*{GENESIS: Documentation}

{\bf Related Documentation:}
% start: userdocs-tag-replace-items related-do-nothing
% end: userdocs-tag-replace-items related-do-nothing

\section*{GENESIS Software Dependencies}

\section*{Software Dependencies for User Installation}

Generally, there should be no software dependencies or configuration requirements for installing GENESIS as a user (c.f. a developer, see below). The exception is if you want to verify your installation. As installation testing is performed as a sequence of regression tests you must have specific Perl modules installed on your machine.

CPAN as the standard location where Perl modules can be obtained. To install a Perl module from CPAN invoke, for example: 
\begin{verbatim}
   'perl -MCPAN -e ``install <required perl module>'' '
\end{verbatim}

If you don't feel comfortable doing this, you should get your systems administrator to install the following Perl modules required for user installation tests: 

\begin{itemize}
   \item[]Data::Comparator
   \item[]Data::Transformator
   \item[]YAML
\end{itemize}

\section*{Software Dependencies for Developer Installation}

This is a general list of dependencies for configuring a computer for
GENESIS software development.  Also look at the list of \href{../tested-distributions/tested-distributions.tex}{\bf Tested\,Operating\,Systems} to see what distributions have actually
been tested.

Most of the required programs, libraries and modules are available as
packages specific for your Operating System.  This is the most convenient way to install them.

\subsection*{Required Programs}

\begin{itemize}
\item {\it automake}
\item {\it autoconf}
\item {\it make}
\item {\it gcc} (and dependencies)
\item {\it perl}
\item {\it bison}
\item {\it flex} (and libraries)
\item {\it swig}
\item {\it monotone}
\end{itemize} 

\section*{Required Libraries}

This section lists some libraries required by GENESIS3 in order to compile from source or do development. Lists are separated by operating system. 

\subsection*{Libraries for Ubuntu}

To install these packages use the {\it apt-get} command via the command line, or the synaptic package manager. 

\begin{itemize}
\item {\it libc6-dev}
\item {\it lib(n)curses-dev}
\item {\it libperl-dev}
\item {\it libreadline5-dev}
\item {\it libhistory-dev} (Note: some linux builds and MAC OSX have
  {\it libhistory} built into another standard library such as {\it
    stdio} or {\it libedit})
\item {\it readline-dev}
\item {\it python-dev} (for your python version)
\end{itemize} 


\subsection*{Libraries for Fedora}

To install packages in Fedora use the {\it yum} utility from the command line. Since {\it yum} repositories typically contain different versions of the same package for various architectures, you must know your machines architecture type and insert it place of "arch" in the package name. The usual architecture types are x86\_64, i586 and i686.

\begin{itemize}
\item {\it ncurses-devel.arch}
\item {\it perl-devel.arch}
\item {\it readline-devel.arch}
\item {\it python-devel.arch}
\end{itemize}



\subsection*{Required Perl Modules}

The following Perl modules are required.  If you are using Linux, they
are likely available as packages for your Linux distribution or vendor.

\begin{itemize}
   \item CPAN
   \item YAML
   \item Parse::RecDescent
   \item Inline (which includes Inline::C)
   \item ExtUtils::Embed
   \item LWP::Simple
   \item Expect (including IO::Pty and IO::Tty)
   \item Test::More
   \item Clone
   \item Data::Utilities
   \item File::Find::Rule
   \item Digest::SHA
   \item Inline::Python
\end{itemize}

\subsection*{Trouble Shooting}

Periodically the build of a Perl module can fail and requires a workaround to get it installed. A common error that occurs is a failure to produce a {\it Makefile} as in this example with {\tt Data::Utilities} :

\begin{verbatim}
	Checking if your kit is complete...
	Looks good
	Writing Makefile for Data::Utilities
	Use of uninitialized value in concatenation (.) or string at /usr/lib/perl5/5.10.0/CPAN.pm line 7652.
	Alert: no Build file available for 'make ' in cwd[/root/.cpan/build/Data-Utilities-0.04-MT2keD]. Danger, Will Robinson!
	Can't exec "./Build": No such file or directory at /usr/lib/perl5/5.10.0/CPAN.pm line 7698.
	  CORNELIS/Data-Utilities-0.04.tar.gz
	  ./Build -- NOT OK
	Running Build test
	  Can't test without successful make
	Running Build install
	  Make had returned bad status, install seems impossible
	Failed during this command:
	 CORNELIS/Data-Utilities-0.04.tar.gz          : make NO
\end{verbatim} 

In the event this happens it's often possible to manually build the module by entering the package directory and creating the {\it Makefile} via the module's Perl script. As shown in the output, CPAN downloaded and unzipped the package to the directory {\it /root/.cpan/build/Data-Utilities-0.04-MT2keD}. So to manually build the {\it Makefile} and module we perform this series of commands:

\begin{verbatim}
cd /root/.cpan/build/Data-Utilities-0.04-MT2keD
\end{verbatim}

Now create the {\it Makefile}:

\begin{verbatim}
perl Makefile.PL
\end{verbatim}

At this point a {\it Makefile} will be present, all that needs to be done is to build and install:

\begin{verbatim}
make
sudo make install
\end{verbatim}

The module should now be in your Perl library. 

\end{document}
