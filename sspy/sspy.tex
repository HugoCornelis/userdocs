\documentclass[12pt]{article}
\usepackage{verbatim}
\usepackage[dvips]{epsfig}
\usepackage{color}
\usepackage{url}
\usepackage[colorlinks=true]{hyperref}

\begin{document}

\section*{GENESIS: Documentation}

{\bf Related Documentation:}
% start: userdocs-tag-replace-items related-do-nothing
% end: userdocs-tag-replace-items related-do-nothing

\section*{Simple Scheduler in Python}

\section*{Introduction}

	GENESIS is composed of several independent software components, each of which has a presence in Python. It is possible via the API of components such as the \href{../model-container/model-container.tex}{Model Container}, \href{../heccer/heccer.tex}{Heccer}, and \href{../experiment/experiment.tex}{Experiment}, to script your simulations via Python. However this is not desirable since all simulations would be composed of code which contain their own control flow, that would often times require the user to understand all of the internals in order to expand on existing simulations. {\bf SSPy} is designed in the same vein as the \href{../ssp/ssp.tex}{SSP (Simple Scheduler in Perl)} in that it encapsulates the operations for loading and running a complete simulation while allowing for complete control of simulation parameters and simulator options via a declarative configuration file. It also provides an easy plugin framework so that new modeling services, experimental protocols and solvers can be dynamically loaded from a plugin directory, making them immediately available without the need to change any of the core code.  For more information on the plugin framework and extending SSPy, see the \href{../sspy-developer/sspy-developer.tex} {SSPy developer document}.


\section*{Prerequisites}

\begin{itemize}
\item In order to run {\bf SSPy} you must first install \href{http://pyyaml.org/}{PyYAML} to handle YAML files and dictionaries. .

\item Python version 2.5 or greater is required

\item Python SWIG bindings must be compiled for each component of GENESIS 3. This can be automatically handled via the \href{developer-installation/developer-installation.tex}{developer scripts} for your system installation.
\end{itemize}


\section*{Command Line}

	{\bf SSPy} has a top level executable file named 'sspy' that is used to run simulations on the command line. To run an existing schedule you simply type out the path to the sspy executable and pass it the schedule file as an argument. This example is performed when in the '0' directory of the SSPy package:
	
\begin{verbatim}

./sspy yaml/purk_test.yml

\end{verbatim}

This particular simulation loads a model, runs for 2500 steps and places the output in a file called {\it /tmp/output}.

\section*{Shell}


\section*{API}


\end{document}
