\documentclass[12pt]{article}
\usepackage{verbatim}
\usepackage[dvips]{epsfig}
\usepackage{color}
\usepackage{url}
\usepackage[colorlinks=true]{hyperref}

\begin{document}

\section*{GENESIS: Documentation}

{\bf Related Documentation:}
% start: userdocs-tag-replace-items related-do-nothing
% end: userdocs-tag-replace-items related-do-nothing

\section*{De Schutter: Purkinje Cell Model}

\subsection*{CALCIUM CHANNEL TYPES}
The obvious prominence of
Ca$^{2+}$ conductances in the dendrites of Purkinje cells has led
to much discussion, and some disagreement, concerning
their nature. In their original report Llinas and Sugimori\,\cite{R:1980pi}
explained the electroresponsiveness of the Purkinje
cell dendrite by suggesting that two Ca$^{2+}$ conductances
would be present, namely a plateau-generating one
and a spike-generating one. When several years later different
types of Ca$^{2+}$ conductances were identified in other
neural systems\,\cite{Fox:1987zr} and in the Purkinje cell\,\cite{Bossu:1989kl, Fortier:1991fk}
it was proposed that a
low-threshold Ca$^{2+}$ channel (T type) would be responsible
for generating the dendritic plateau. In the meantime, however,
Llinas et al.\,\cite{R:1980pi, R:1980pi}\ had discovered a new type of
high-threshold Ca$^{2+}$ conductance in the Purkinje cell,
which they named the P channel. Since then there has been
debate about whether other Ca$^{2+}$ channels than the P channel
are present in the Purkinje cell\,\cite{Fortier:1991fk, Llinas:1989uq, Usowicz:1992qf} and about the
physiological role of these channels in generating the dendritic
plateaus and spikes.

We believe that there is now ample experimental evidence
for the presence of both a CaP and a CaT channel in
the Purkinje cell. Voltage-clamp studies have shown the
CaT channel to be present in both young and adult animals,
with similar $I-V$ relations\,\cite{Kaneda:1990ys, Regan:1991ly}. 
Also, channel blocking studies with funnel web
spider toxin (FTX) have shown that the CaP channel constitutes
only $\sim$\,90\,\% of the total Ca$^{2+}$ conductance\,\cite{Mintz:1992kx}. 
This corresponds well to the relative channel
densities for CaP and CaT channels in the dendrites in our
model (\href{../pub-purkinje-deschutter1-table2/pub-purkinje-deschutter1-table2.tex}{\bf Table\,2}),
which were determined by trial and error
before these data became available.

Perhaps of more importance than the presence of these
channels, however, is their relative contribution to Purkinje
cell responses. Although it is generally accepted that the
CaP channel is responsible for the fast dendritic calcium
spikes\,\cite{R:1980ly}, it has been suggested that the
generation of the longer-duration plateau potentials requires
the slower kinetics of a channel like the CaT channel\,\cite{Fortier:1991fk}. 
Llinas and Sugimori\,\cite{Llinas:1992rq}, on the
other hand, have argued that the CaP channel is also capable
of producing these prolonged potentials, providing it
with a dual role. As described in detail above, our modeling
results clearly support the suggestion by Llinas and Sugimori.
Both the dendritic plateaus and spikes in the model
were carried by the CaP current. Recent Ca$^{2+}$ imaging results
also indicated a common mechanism for plateaus and
spiking\,\cite{Lev-Ram:1992vn}. Our model suggests that the
CaP channel can generate plateaus because it has a relatively
low threshold of activation\,\cite{Regan:1991ly, Usowicz:1992bh} 
compared with other high-threshold Ca$^{2+}$ channels\,\cite{Fox:1987zr} 
and because of its incomplete inactivation.
In our hands, conductance through the CaT channels
contributed only to the rebound spike generation after
hyperpolarizations.

\bibliographystyle{plain}
\bibliography{../tex/bib/g3-refs}

\end{document}
