% Template for PLoS
% Version 1.0 January 2009
%
% To compile to pdf, run:
% latex plos.template
% bibtex plos.template
% latex plos.template
% latex plos.template
% dvipdf plos.template

\documentclass[10pt]{article}

% amsmath package, useful for mathematical formulas
\usepackage{amsmath}
% amssymb package, useful for mathematical symbols
\usepackage{amssymb}

% graphicx package, useful for including eps and pdf graphics
% include graphics with the command \includegraphics
\usepackage{graphicx}

% cite package, to clean up citations in the main text. Do not remove.
\usepackage{cite}

\usepackage{color} 

% Use doublespacing - comment out for single spacing
\usepackage{setspace} 
\doublespacing

\usepackage{lineno}

% Text layout
\topmargin 0.0cm
\oddsidemargin 0.5cm
\evensidemargin 0.5cm
\textwidth 16cm 
\textheight 21cm

\begin{document}

\section*{A Federated Design for a Neurobiological Simulation Engine: 
The CBI federated software architecture.}
\subsection*{Cornelis H, Coop AD, and Bower JM}
\subsection*{Supplementary Materials 4:} 

We show a toy network example, a NEURON script to create a ring of cells (actual cell definitions are given in a separate file not shown here).

\linenumbers
\begin{verbatim}
load_file("nrngui.hoc")  // load the GUI and standard run libraries

//////////////////////////////////
// Step 1: Define the cell classes
//////////////////////////////////

load_file("cell.hoc")

//////////////////////////////////////////////////////////////
// Steps 2 and 3 are to create the cells and connect the cells
//////////////////////////////////////////////////////////////

NCELL = 20  // total number of cells in the ring network

objref cells, nclist  // will be Lists that hold all network cell
                      // and NetCon instances, respectively

proc mkring() {
  mkcells($1)  // create the cells
  connectcells()  // connect them together
}

// creates the cells and appends them to a List called cells
// argument is the number of cells to be created
proc mkcells() {local i  localobj cell
  cells = new List()
  for i=0, $1-1 {
    cell = new B_BallStick()
    cells.append(cell)
  }
}

// connects the cells
// appends the NetCons to a List called nclist
proc connectcells() {local i  localobj src, target, syn, nc
  nclist = new List()
  for i=0, cells.count-1 {  // iterating over sources
    src = cells.object(i)
    target = cells.object((i+1)%cells.count)
    syn = target.synlist.object(0)  // the first object in synlist
                                    // is an ExpSyn with e = 0, therefore an excitatory synapse
    nc = src.connect2target(syn)
    nclist.append(nc)
    nc.delay = 1
    nc.weight = 0.01
  }
}

mkring(NCELL)  // go ahead and create the net!

//////////////////////////////////////////////////
// Instrumentation, i.e. stimulation and recording
//////////////////////////////////////////////////

// stim will be an artificial spiking cell that generates a "spike" event
// that is delivered to the first cell in the net by ncstim
// in order to initiate network spiking.
// We won't bother including this "external stimulus source" or its NetCon
// in the network's lists of cells or NetCons.
objref stim, ncstim
proc mkstim() {
  stim = new NetStim()
  stim.number = 1
  stim.start = 0
  ncstim = new NetCon(stim, cells.object(0).synlist.object(0))
  ncstim.delay = 0
  ncstim.weight = 0.01
}

mkstim()

objref tvec, idvec  // will be Vectors that record all spike times (tvec)
                    // and the corresponding id numbers of the cells that spiked (idvec)
proc spikerecord() {local i  localobj nc, nil
  tvec = new Vector()
  idvec = new Vector()
  for i=0, nclist.count-1 {
    nc = cells.object(i).connect2target(nil)
    nc.record(tvec, idvec, i)
    // the Vector will continue to record spike times
    // even after the NetCon has been destroyed
  }
}

spikerecord()

/////////////////////
// Simulation control
/////////////////////

tstop = 100
run()

////////////////////////////
// Report simulation results
////////////////////////////

proc spikeout() { local i
  printf("\ntime\t cell\n")
  for i=0, tvec.size-1 {
    printf("%g\t %d\n", tvec.x[i], idvec.x[i])
  }
}

spikeout()

quit()
\end{verbatim}

\end{document}
