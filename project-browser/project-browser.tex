\documentclass[12pt]{article}
\usepackage[dvips]{epsfig}
\usepackage{color}
%e.g.  \textcolor{red,green,blue}{text}
\usepackage{url}
\usepackage[colorlinks=true]{hyperref}

\begin{document}

\section*{GENESIS: Documentation}

{\bf Related Documentation:}
% start: userdocs-tag-replace-items related-do-nothing
% end: userdocs-tag-replace-items related-do-nothing

\section*{The Project Browser}

The {\bf Project\,Browser} is a set of modules for the \href{http://www.webmin.com/usermin.html}{Usermin} web server. It allows you to browse projects, inspect, and compare simulation results or other data over a web interface.

The specific aims of the {\bf Project\,Browser} are introduced in the presentation given at the model sharing workshop at the \href{../cns07-poster/cns07-poster.pdf}{CNS*2007} meeting in Toronto.

The {\bf Project\,Browser} uses distributed version control mechanisms to physically distribute your project. The coupling to a web server allows you to inspect simulations and compare their results, regardless of location. You can take a look at some \href{../project-browser-screenshots/project-browser-screenshots.tex}{\bf Project\,Browser\,screenshots}. 

\subsection*{Usage}

The project browser comes equipped with commands for creating a project workspace as well as simulation modules. Commands are also provided to save and check work to a version control system.

\begin{itemize}
\item[] {\bf pb-create-project}: Creates a new project-browser directory structure for storing your work.
\item[] {\bf pb-create-module}: Creates a new neurospaces project-browser simulation module.
\item[] {\bf pb-pull}: pulls project information from a different machine.
\item[] {\bf pb-push}: pushes project information to a different machine.
\item[] {\bf pb-version}: Prints the current version. 
\end{itemize}

\subsection*{Directory Structure}

Upon creating a project, the {\bf Project Browser} creates a directory structure for storing all aspect of your simulation and results. 

\begin{itemize}
\item[] {\bf models}: project specific models (separate from the system wide library).

\item[] {\bf morphologies}: basically the same as above, it can be used as drop box to include externally generated morphologies.

\item[] {\bf morphology\_groups}: it should be possible to define groups of external resources.

\item[] {\bf colormaps}: definitions of color coding for other components, such as morphologies.  I was still looking for the right format to include this, the morphologies and morphology\_groups subdirectories when I stopped development on the project-browser.

\item[] {\bf scripts}: scripts that are not related to the functions of the project browser, but still part of the project.

\item[] {\bf modules}: simulation configurations and output. Inside this directory, you will find a numeric subdirectory for each module of the project. The name 'modules' should be replaced with 'iterations', keeping the terminology as close to the G-3 user workflow as possible.

\item[] {\bf pictures}: free format pictures.

\item[] {\bf narrative\_components}: summaries and pictures of result analysis.  The result analysis is done inside the modules/ directory.

\item[] {\bf papers}: unknown

\item[] {\bf replicator}: unknown 

\item[] {\bf summary}: unknown

\end{itemize}

YAML descriptor files are located in the top level directory as well as in each sub directory, allowing a user to edit all aspects of the project. By default project directories are created in:

\begin{verbatim}
/var/neurospaces/simulation_projects/
\end{verbatim}

however this can be configured to your liking.

\end{document}
