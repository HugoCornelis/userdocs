\documentclass[12pt]{article}
\usepackage{verbatim}
\usepackage[dvips]{epsfig}
\usepackage{color}
\usepackage{url}
\usepackage[colorlinks=true]{hyperref}

\begin{document}

\section*{GENESIS: Documentation}

\section*{Model\,Container Regression Tests}

The GENESIS {\bf Model\,Container} deals with biological entities and end-user concepts rather than the mathematical details of a model and its simulation which are handled by the solver (here, \href{../heccer/heccer.tex}{\bf Heccer}). It is optimized to store large models in a solver independent internal storage format.

For more details see the \href{../model-container/model-container.tex}{\bf Model\,Container} documentation.

\subsection*{Obtaining Regression Test Specifications}

This \href{http://www.neurospaces.org/neurospaces_project/neurospaces/tests/html/index.html}{\bf link} accesses all regression test specifications for the {\bf Model\,Container}.

\subsection*{Regression Test Documentation}

{\bf Model\,Container} regression test documentation includes:
\begin{itemize}
\item The startup command.
\item The input to the test.
\item The expected output of the test.
\end{itemize}
The regression tests give a good overview of {\bf Model\,Container} functionality.

\subsection*{Scope of Regression Tests}

Regression tests cover the following major functionalities of the {\bf Model\,Container}:
\begin{itemize}

\item[]\href{http://www.neurospaces.org/neurospaces_project/neurospaces/tests/html/specifications/main.html}{\bf Specifications}

Source code extensions; general syntax of NDF library files; file loading; command line switches; environment variables; error conditions;  algorithms, biolevels and biogroup definitions; meshing algorithms; partitioning; coordinates and coordinate caching;  segment linearization, morphology branchpoints, indexing, and structure analysis; internal data structures; parameter calculations on existing models; parameter operations and caching; reducing model parameters; networks and nesting of networks; projection query caching; context operations, wildcard matching and expansion; connections and solvers; model partitioning; automatically inferred and hard-coded channel type parameters;  generic pools and gates from multiple models; operations on symbol table components; model export in a variety of export formats.

\item[]\href{http://www.neurospaces.org/neurospaces_project/neurospaces/tests/html/specifications/algorithms/main.html}{\bf Algorithms}

Grid, inserter, and spine algorithms.

\item[]\href{http://www.neurospaces.org/neurospaces_project/neurospaces/tests/html/specifications/code/main.html}{\bf C Code}

Symbol allocation and alias creation; biolevels and biogroups, parameter and symbol types. 

\item[]\href{http://www.neurospaces.org/neurospaces_project/neurospaces/tests/html/specifications/convertors/main.html}{\bf Morphology convertsion utilities}

\item[]\href{http://www.neurospaces.org/neurospaces_project/neurospaces/tests/html/specifications/parameters/main.html}{\bf Segment parameter calculations and error signaling}

\item[]\href{http://www.neurospaces.org/neurospaces_project/neurospaces/tests/html/specifications/perl/main.html}{\bf Perl bindings}

\item[]\href{http://www.neurospaces.org/neurospaces_project/neurospaces/tests/html/specifications/python/main.html}{\bf Python bindings}

\end{itemize}

\end{document}
