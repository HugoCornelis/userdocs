\documentclass[12pt]{article}
\usepackage{verbatim}
\usepackage[dvips]{epsfig}
\usepackage{color}
\usepackage{url}
\usepackage[colorlinks=true]{hyperref}

\begin{document}

\section*{GENESIS: Documentation}

{\bf Related Documentation:}
% start: userdocs-tag-replace-items related-build-debian
% end: userdocs-tag-replace-items related-build-debian

\section*{Installing GENESIS}

The \href{../genesis-overview/genesis-overview.tex}{\bf GENESIS simulation system} is available in different `flavors' that are shaped by the requirements of the user community. This community can be characterized by how the system is used. For example:
\begin{itemize}
   \item[]{\bf Student:} Educational tutorials via the GUI.
   \item[]{\bf Research:} Collaborative research and development of educational tutorials and publications.
   \item[]{\bf Development:} Extension of GENESIS functionality for students and researchers.
\end{itemize}
Currently, there are two classes of installation:

\begin{itemize}
   \item{\bf User Installation:} For students and researchers. Installation is a simple automated process that employs an installer package. The following installer packages are available:
   \begin{itemize}
      \item[] \href{../installation-debian/installation-debian.tex}{\bf Debian Installer}
   \end{itemize}
   
   \item{\bf Developer Installation:} For collaborative research projects and GENESIS development. Installation guides are available for the following operating systems:
      \begin{itemize}
         \item \href{../installation-fedora10/installation-fedora10.tex}{\bf Fedora 10}
         \item \href{../installation-ubuntu-lennysid/installation-ubuntu-lennysid.tex}{\bf Ubuntu (Lenny/Sid)}
         \item \href{../installation-osx/installation-osx.tex}{\bf Mac OSX}
         
         
         
      \end{itemize}   

\end{itemize}

\end{document}
