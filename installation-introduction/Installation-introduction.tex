\documentclass[12pt]{article}
\usepackage{verbatim}
\usepackage[dvips]{epsfig}
\usepackage{color}
\usepackage{url}
\usepackage[colorlinks=true]{hyperref}

\begin{document}

\section*{GENESIS: Documentation}

{\bf Related Documentation:}
% start: userdocs-tag-replace-items related-build-debian
% end: userdocs-tag-replace-items related-build-debian

\section*{Installing GENESIS}

Currently, there are two classes of GENESIS installation.

\subsection*{A. User Installation}

For students and researchers. Installation is a simple automated process that employs an installer package. The following installer packages are available:
\begin{itemize}
   \item[] \href{../installation-debian/installation-debian.tex}{\bf Debian Installer}
\end{itemize}
   
\subsection*{B. Developer Installation}

There are three types of developer installation for collaborative research projects and GENESIS development.

\subsubsection*{1. Source Code Server Installation}

Installation guides are available for the following operating systems:
\begin{itemize}
   \item \href{../installation-fedora10/installation-fedora10.tex}{\bf Fedora 10}
   \item \href{../installation-ubuntu-lennysid/installation-ubuntu-lennysid.tex}{\bf Ubuntu (Lenny/Sid)}
   \item \href{../installation-osx/installation-osx.tex}{\bf Mac OSX}
\end{itemize}   

\subsubsection*{2. Documentation Server Installation}

\subsubsection*{3. Regression Tester Installation}

\begin{itemize}
   \item[]\href{../neurospaces\_cron/neurospaces\_cron.tex}{\it neurospaces\_cron}
\end{itemize}

\end{document}
