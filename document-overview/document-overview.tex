\documentclass[12pt]{article}
\usepackage[dvips]{epsfig}
\usepackage{color}
%e.g.  \textcolor{red,green,blue}{text}
\usepackage{url}
\usepackage[colorlinks=true]{hyperref}

\begin{document}

\section*{GENESIS: Documentation}

\section*{Documentation Overview}

There are currently seven levels of GENESIS documentation which can be
accessed from the following links.  These levels range from documentation required by the user to technical detail of a specific implementation.

Each level can include specifics about compatibility and
portability.  Because backward compatibility is
defined as a user requirement, the documentation for backward
compatibility falls under the first two levels.

\begin{enumerate}

\item \href{../contents-level1/contents-level1.pdf}{\bf \underline{Introductory Material}}\\
The first level of user documentation provides a comprehensive introduction to the use of the GENESIS software system. Documentation includes an introduction to Unix and the Linux Graphical Desktop, fundamentals of Computational Neuroscience, and basic tutorials on the concepts and use of GENESIS. Documentation at this level also guides the user
  in the choice of documentation necessary to implement
  research and educational models and simulations.  Tutorials at this level
  are short and contain links to the user documentation required for
  specific types of projects.  

\item \href{../contents-level2/contents-level2.pdf}{\bf \underline{User Guides and Documentation}}\\
The second level of user documentation contains in depth
  documentation about specific functions of GENESIS in the context of
  advanced research projects.

\item \href{http://neurospaces.sourceforge.net/tests-menu.html}{\bf \underline{Automated Use Cases}}\\
 {\bf UNDER CONSTRUCTION}\\
  This level of documentation provides common use cases and examples of workflows. It bridges between the functions visible to
  a user and the way those functions are implemented in the code.
  It also aims to guide technical developers through the code
  and define a framework for communication between users and developers.
  Currently, this level of documentation is automatically generated
  from test cases.
  
 \item \href{../contents-level4/contents-level4.pdf}{\bf \underline{Technical Guide Specification}}\\
 {\bf UNDER CONSTRUCTION}\\
  Technical outline
  and specification of the use of all GENESIS functions, commands and
  components.

  The tools related to software development such as the Neurospaces
  installer, can be documented via the \href{http://code.google.com/p/neurospaces/wiki/Index}{Neurospaces wiki}.

\item \href{../contents-level5/contents-level5.pdf}{\bf \underline{Algorithm Documentation}} \\
{\bf UNDER CONSTRUCTION}\\
  This type of documentation contains outlines of the
  algorithms used in an implementation, their theoretical analysis
  and behavior, their edge cases, and an outline of the expected
  application programmers interface (API).  Links to various publications
  and external documents are provided for most of the algorithms
  used.

  At a practical level, this documentation also includes the
  specification of basic unit tests.  At present, there are no plans to implement these tests due to the
  maintenance cost for the developer.

\item \href{../contents-level6/contents-level6.pdf}{\bf \underline{Algorithm API Documentation}}\\
{\bf UNDER CONSTRUCTION}\\
  This documentation includes source code
  documentation typically generated by \href{http://www.stack.nl/~dimitri/doxygen/}{Doxygen} or a similar
  documentation extraction system.  Documentation is focused on how to interface
  to the specific implementations of low-level algorithms. Doxygen
  also allows inclusion of implementation documentation.

\item \href{../contents-level7/contents-level7.pdf}{\bf \underline{Inline Source Code Documentation}}\\
{\bf UNDER CONSTRUCTION}\\
  Typically this documentation is provided by inline source code comments with a focus on how a function
  or algorithm is implemented.  It provides extensive cross referencing between
  implementation and the API of the same and other functions.
  % is an important requirement for the readability and quality of this documentation.
  Both the Doxygen system and \href {http://www.xref.sk/xrefactory/main.html}{Xrefactory} provide this
  functionality.
  
%\begin{itemize}

%\item {\bf Model Container}

%\item \href{http://neurospaces.sourceforge.net/neurospaces_project/heccer/html_source/heccer/source/c/snapshots/0/index.html}{\bf {\underline{Heccer}}}

%\item {\bf SSP}

%\item {\bf Neurospaces Studio}

%\item {\bf GENESIS Shell}

%\item {\bf GENESIS 2 Backwards Compatibility SLI}

%\end{itemize}

\end{enumerate}


\end{document}