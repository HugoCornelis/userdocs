\documentclass[12pt]{article}
\usepackage{verbatim}
\usepackage[dvips]{epsfig}
\usepackage{color}
\usepackage{url}
\usepackage[colorlinks=true]{hyperref}

\begin{document}

\section*{GENESIS: Documentation}

{\bf Related Documentation:}
% start: userdocs-tag-replace-items related-do-nothing
% end: userdocs-tag-replace-items related-do-nothing

\section*{De Schutter: Purkinje Cell Model}

\subsection*{Abstract}

When excitatory synapses were activated on the smooth
dendrites of the model, the model generated a complex dendritic
Ca$^{2+}$ spike similar to that generated by climbing fiber inputs. Examination
of the model showed that activation of P-type Ca$^{2+}$
channels in both the smooth and spiny dendrites augmented the
depolarization during the complex spike and that Ca$^{2+}$-activated
K$^+$ channels in the same dendritic regions determined the duration
of the spike. When these synapses were activated under simulated
current-clamp conditions the model also generated the characteristic
dual reversal potential of the complex spike. The shape
of the dendritic complex spike could be altered by changing the
maximum conductance of the climbing fiber synapse and thus the
amount of Ca$^{2+}$ entering the cell.

\end{document}
