\documentclass[12pt]{article}
\usepackage{verbatim}
\usepackage[dvips]{epsfig}
\usepackage{color}
\usepackage{url}
\usepackage[colorlinks=true]{hyperref}

\begin{document}

\section*{GENESIS: Documentation}

{\bf Related Documentation:}
% start: userdocs-tag-replace-items related-do-nothing
% end: userdocs-tag-replace-items related-do-nothing

\section*{Purkinje Cell Model--De Schutter}

\subsubsection*{Maximum conductances}

\begin{table}[!h]
\label{table:T2}
\caption{$\bar g$ for the voltage and Ca$^{2+}$-dependent channels in 2 different versions of the Purkinje cell model.}
\begin{tabular}{ l c c c c c c }  
                    &                     &                                          &                                       &                     &                            &                                                       \\
                    & \multicolumn{3}{ c} {\bf PM9 Model} &  \multicolumn{3}{ c} {\bf PM10 Model}                                                                                   \\
{\bf Name} & {\bf Soma} & {\bf Main Dendrite}       & {\bf Rest of Dendrite} & {\bf Soma} & {\bf Main Dendrite} & {\bf Rest of Dendrite}          \\
NaF            & 7,500          & 0.0                                   & 0.0                                & 7,500          & 0.0                     & 0.0                                                  \\
NaP            & 1.0              & 0.0                                    & 0.0                               & 1.0               & 0.0                     & 0.0                                                  \\
CaP            & 0.0              & 4.5                                    & 4.5                               & 0.0               & 4.0                     & 4.5                                                  \\
CaT            & 0.5              & 0.5                                    & 0.5                                & 0.5               & 0.5                     & 0.5                                                  \\
Kh               & 0.3              & 0.0                                    & 0.0                               & 0.3               & 0.0                     & 0.0                                                  \\
Kdr             & 600.0          & 60.0                                  & 0.0                               & 900.0           & 90.0                   & 0.0                                                  \\
KM              & 0.040         & 0.010                                & 0.013                          & 0.140           & 0.040                 & 0.013                                             \\
KA              &15.0             & 2.0                                     & 0.0                              & 15.0             & 2.0                      & 0.0                                                  \\
KC              & 0.0              & 80.0                                   & 80.0                            & 0.0               & 80.0                    & 80.0                                                \\
K2              & 0.0              & 0.39                                    & 0.39                            & 0.0               & 0.39                   & 0.39                                                \\
                   &                     &                                            &                                      &                     &                             &                                                         \\
\multicolumn{7}{l}{Extent of the main dendrite is shown in \href{../pub-purkinje-deschutter-fig-1/pub-purkinje-deschutter1-fig-1.tex}{\bf Figure\,1}.}                                                                                                                                 \\
\multicolumn{7}{l}{$\bar g$: maximum conductance mS/cm$^2$. For other abbreviations, see \href{../pub-purkinje-deschutter1-table1/pub-purkinje-deschutter1-table1.tex}{\bf Table\,1}.}                                               \\
\multicolumn{7}{l}{}                                                                                                                                                                                                                   \\
\end{tabular}
\end{table}

\subsection*{Channel Densities}

In the present model, simulations were started using initial
guesses for the $\bar g$ of the different channels in four zones of the cell,
i.e. the soma, the main dendrite, the smooth dendrites, and the
spiny dendrites (see Fig. 1). $\bar g$s were then altered until the model
could reproduce the characteristic firing behavior of Purkinje cells
during current injection in vitro\,\cite{R:1980ly, R:1980pi},
as described in Figs. 3--7. During this optimization
process it was not necessary to distinguish between
the smooth and spiny dendrites. In addition, optimal performance
required that the Kdr and A currents be present in the
main dendrite as well as in the soma.

Also, several combinations of $\bar g$ were found that could account for
the responses of Purkinje cells to current injection. For contrast,
versions PM9 and PM10 are presented, where PM10 represents
a Purkinje cell with higher K$^+$ channel densities, i.e., a leakier
cell.

\subsection*{Robustness to Changes in Channel Densities}

The principle parameters used to tune this
model to the current injection data were the distribution
and density of specific ion channels. Although Table\,\ref{table:T2} described
the final distribution for this model, it was also important
to determine how sensitive the modeling results
were to these particular density values.

A full search of the 10,000 dimensional parameter space
for this model would have been computationally prohibitive
(see\,\cite{S:1993dz}). Accordingly, the approach
we have taken involved examining the effect of
changing the density of a single-channel type on the physiological
responses to current injection in the soma. The results
showed, perhaps not surprisingly, that small-amplitude
currents like NaP and CaT currents, KA, or KM could
be completely removed with little effect, except for small
changes in firing frequency. The same was true for increasing
KA or KM by a factor of 2 or 3. Similar increases in
NaP or CaT currents, however, caused fast spiking in the
soma to saturate (as in \href{../pub-purkinje-deschutter1-fig-3/pub-purkinje-deschutter1-fig-3.tex}{\bf Fig.\,3D}) at lower current amplitudes
and could turn the cell into a spontaneously firing or
bursting neuron.

The model was more sensitive to changes of the conductances
involved in spike generation and repolarization.
Small changes resulted in a shift of the $f-I$ curve and of the
current amplitude at which dendritic spiking started.
Changes in the dendritic currents involved in spiking could
suppress all dendritic spiking (reducing CaP current by
220\,\%, increasing Kdr by $>$\,40\,\%, KC by 25\,\%, or K2 by
10\,\% ) or make dendritic bursting the unique mode of firing
of the cell (reducing NaF current by 50\,\%, increasing CaP
current by 50\,\%, or decreasing Kdr by 20\,\% or KC or K2 by
10\,\%). NaF current could be reduced by 70\,\% before NaF
spikes disappeared.

The model was thus sensitive to small changes in density
of CaP, KC, and K2 currents; in other words, the region of
parameter space that generated correct model responses
was not very large for these currents. Note, however, that
larger changes could be applied to these channel densities if
one of the other current densities was changed in the opposite
direction. The model was less sensitive to changes in the
other currents, like for example the densities of voltage-dependent
K+ currents (cf. Table\,\cite{table:T2}).

\subsection*{Channel Distributions}

The density and distribution of channels in the model were the main
uncontrolled variables in these simulations. Experimental
techniques do not yet exist to give detailed distribution and
density information for all ionic channels in a given cell.
Accordingly, by searching parameter space, detailed single cell
models can make predictions concerning this critical
information\,\cite{S:1993dz}. In the current case,
the initial channel distributions were primarily based on the
speculations of\,\cite{R:1980ly, R:1980pi}. The results
of the model largely confirm these predictions with a few
modifications.

\bibliographystyle{plain}
\bibliography{../tex/bib/g3-refs.bib}

\end{document}
