\documentclass[12pt]{article}
\usepackage[dvips]{epsfig}
\usepackage{color}
\usepackage{url}
\usepackage[colorlinks=true]{hyperref}

\begin{document}

\section*{GENESIS: Documentation}

\section*{GENESIS 3.0 functionality enhanced by interface}


Obviously the taxonomy figure plays a central role in the content of
the abstract and the poster, to explain the introduction and
categorize the examples.
We could use the topography
one to make the point that G-3 is now useable at multiple
levels of scale.

We can show examples of the different levels of scales supported by
G-3.  The existence of a single model-container for all levels within
the CBI simulator makes it 'transparant' for the user to run
multi-scale simulations.  So next we show a couple of examples of
multi-scale simulations.  We should specifically mention the CBI
communication component to upscale and downscale numerical variables
to anticipate questions.


The detailed part of the poster can have specific figures of the
different scales of models (including the one attached in Dave's
email, one of PC morphology comparison, and one of C-3, to be
produced).

Or, the attched RSnet2\_Vm\_0000.png could be used as an example of the
interfaces that we have developed for network modeling in G-3.
The figure illustrates the spreading excitation
in RSnet2 simulation that is developed in the GENESIS network
modelling tutorial.

I have a new verion of the viewer, that implements the single step
functionality, reads bzip2'ed asc\_file output with a header, and is
generally now a useful tool for analyzing my G-2 network simulations.
I am putting together a stripped-down version of the RSnet2 simulation
with no extraneous features, but with output for the viewer.  I should
have it ready by the time you have the basic g-shell interface in
place, and we can use it as a starting point for the G-3 version.  I
would like to try to generate the attached figure from G-3 instead of
G-2.


We could say that this poster introduces a series of tutorials on
subcellular, single cell, and network model in G-3 to go along with
its new capabilities.

In fact, a lot of the things mentioned in last year's poster could be
the basis of this year's abstract.

Perhaps we should specifically mention the workshop held in Luebeck.


The cognitive workflow figure in the recently submitted paper can be
used in the conclusion to provide a broad framework for the poster.
We can mention the cognitive workflow in the abstract without showing
the figure.  The cognitive workflow allows to make a distinction
between multi-scale, multi-level and multi-layered modeling.


\end{document}

%%% Local Variables: 
%%% mode: latex
%%% TeX-master: t
%%% End: 
