\documentclass[12pt]{article}
\usepackage{verbatim}
\usepackage[dvips]{epsfig}
\usepackage{color}
\usepackage{url}
\usepackage[colorlinks=true]{hyperref}

\begin{document}

\section*{GENESIS: Documentation}

{\bf Related Documentation:}
% start: userdocs-tag-replace-items related-do-nothing
% end: userdocs-tag-replace-items related-do-nothing

\section*{Documentation Workflow}

Developers use the \href{http://code.google.com/p/neurospaces/wiki/Index}{Google Neurospaces} site to write documentation.
\begin{enumerate}
\item Documentation supervisor visits the \href{http://code.google.com/p/neurospaces/w/list}{wiki pages} and identifies any that are new.
\item{\bf Create new document:} A \href{../NewDocument/NewDocument.tex}{NewDocument} is \href{../document-create/document-create.tex}{created} and the wiki documentation converted.
\item{\bf Document integration:} The new document is integrated into the GENESIS documentation system with the following steps:
\begin{itemize}
\item  After a wiki page has been converted, it must be given a ``{\tt G3doc}'' label.
\item The GENESIS Documentation file name the wiki page has been converted to must be placed at the head of the wiki page, e.g. ``{\tt G3 doc: <file-name>}''
\item A ``{\tt -\,wiki}'' tag must be inserted into the {\it descriptor.yml} file associated with the new GENESIS document.
\item The wiki page is deprecated by insertion of the ``{\tt Deprecated}'' label.
\end{itemize}
\end{enumerate}

{\bf Note 1:} Not all the converted pages need to be fully finished.

{\bf Note 2:} Changes to the documentation (but not the wiki) automatically propagate to the GENESIS website.

{\bf Note 3:} As the process of converting wiki pages to GENESIS documentation is manual, it is important to not add new content to a wiki page once either a ``{\tt G3 Doc: <file-name>}'' appears and/or the ``{\tt G3doc}'' label is added. Further additions/changes should only be made directly to the document once it appears in the GENESIS Documentation System.
\end{document}
