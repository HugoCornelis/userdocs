\documentclass[12pt]{article}
\usepackage{verbatim}
\usepackage[dvips]{epsfig}
\usepackage{color}
\usepackage{url}
\usepackage[colorlinks=true]{hyperref}

\begin{document}

\section*{GENESIS: Documentation}

{\bf Related Documentation:}
% start: userdocs-tag-replace-items related-do-nothing
% end: userdocs-tag-replace-items related-do-nothing

\section*{De Schutter: Purkinje Cell Model}

\subsection*{Passive Membrane Parameters}

initially used the same parameters as previously published by \,\cite{Rapp-M:1992kx}. 
Accordingly, membrane capacitance was set at
1.64\,pF/cm$^2$, membrane resistance ($R_{\mbox{m}}$) was 0.44\,k$\Omega\cdot$cm$^2$ in the
soma and 110\,k$\Omega\cdot$cm$^2$ in the dendrites, and axial resistance was
given a value of 250\,k$\Omega\cdot$cm. With these parameters we obtained a
$\tau_)$ of 46\,ms and an $R_{\mbox{m}}$ of 12.6 M$\Omega$, which is almost identical to the
values of 46\,ms and 12.9 M$\Omega$ reported by\,\cite{Rapp-M:1992kx}.
However, initial simulations with active membrane showed that it
was not possible to reproduce the characteristic firing pattern of
Purkinje cells with these membrane parameters. In particular, the
low $R_{\mbox{m}}$ value in the soma caused a huge current sink, so that the
model could not fire somatic Na$^+$ spikes. For this reason the
model presented here had an $R_{\mbox{m}}$ of 10\,k$\Omega\cdot$cm$^2$ in the soma and 
30 k$\Omega\cdot$cm$^2$ in the rest of the cell, which are comparable with values
for $R_{\mbox{m}}$ in other neuron models\,\cite{R:1992ys}. This
model had an $R_{\mbox{m}}$of 19.6\,M$\Omega$ 
under conditions of simulated external
Cs$^+$ block (i.e., Kdr, KM, and Kh channels blocked;\,\cite{Hille:1991zr}) 
which is a realistic $R_{\mbox{m}}$ value for Purkinje cells in slice\,\cite{R:1980ly}.

\bibliographystyle{plain}
\bibliography{../tex/bib/g3-refs}

\end{document}
