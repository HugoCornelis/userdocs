\documentclass[12pt]{article}
\usepackage[utf8]{inputenc}
\usepackage[T1]{fontenc}
\usepackage[dvips]{epsfig}
%\usepackage{graphicx}
\usepackage{url}

\begin{document}

%\title{Python as a Federation Tool for GENESIS 3.0}

%\maketitle

\pagenumbering{arabic}
\section*{Python as a Federation Tool for GENESIS 3.0}

\vspace{1mm}
\begin{center}
  Cornelis H. \hspace{2mm} Coop A. D. \hspace{2mm} Bower J. M. \\
  {\small University of Texas Health Science Center at San Antonio}
\end{center}
\vspace{1mm}

The first Python enabled implementation of a neural simulator was
attempted by the GENESIS (GEneral NEural SImulation System)
development group in the late 1990's\,\cite{martelli06:_python_nutsh,
  vanier97:_genes_python}.  Python was chosen to interface to GENESIS
and replace the existing GENESIS script language, but the project was
not completely successful due to the sophistication of the GENESIS
platform and the relative immaturity of Python.  Since that time the
Python user community has matured and a rich set of freely available
open source libraries has been
developed\,\cite{langtangen04:_python_scrip_comput_scien}.  With a
clean dynamic object-oriented design producing highly readable code,
Python is now widely employed in specialized areas of systems
integration (e.g\,.~\cite{thiruvathukal01:_web_progr_python}).

Currently, there are many technologies that simplify the binding of a
simulator to Python libraries.  In the case of the GENESIS 3 simulator
we have employed SWIG, the Simplified Wrapper Interface
Generator\,\cite{08:_simpl_wrapp_inter_gener}.  SWIG examines an
application API and makes it available to a scripting language.
However, neither the API of a low-level application, nor the API
generated by SWIG, possess the high-level functionality required to
complete this binding.  Additional code is required to transform the
low-level functionality of an application coded in C or C++ to the
high-level functionality of a scripting
language\,\cite{08:_swig_python}.  While this can be achieved in
different ways, we have chosen to use self-query and dynamic
compilation techniques to minimize the maintenance cost of existing
code and simplify the addition of new functionality.

An important feature of the GENESIS 3 simulator is that it decomposes
into self-contained software modules, referred to as the CBI simulator
architecture\,\cite{cornelis08:_cbi_archit_comput_simul_realis}.  This
federated architecture allows us to define separate bindings for the
GENESIS 3 solvers and the GUI, as well as for other necessary software
modules.  The scheduler is a control module that ultimately integrates
these modules to run a given
simulation\,\cite{cornelis08:_feder_desig_neurob_simul_engin}.

We illustrate our approach with two examples: (1) a set of simulations
that apply various stimulus paradigms to a model Purkinje
neuron\,\cite{schutter94:_simul_purkin}, and (2) the application of
Python bindings to connect the GENESIS 3 simulator to Blender, an open
source 3D content creation suite.  We use these bindings to visualize
3D models based on electron microscopy and convert them to
computational models\,\cite{cornelis08:_model_neuros_genes}.

%In this paper we describe in more detail the Python bindings of the
%GENESIS 3 simulator, based on the implementation of the Neurospaces
%project.  The Neurospaces project implements software modules for
%neuroscience simulator according to the CBI paradigm.  The CBI
%paradigm separates biological data from numerical data, and
%distinguishes between model data and simulation control.
%Consequently, we describe the Python bindings of the GENESIS 3
%simulator in these categories.



\pagenumbering{roman}
\bibliographystyle{abbrv}
\bibliography{informatics,neuroscience,www}

\end{document}

