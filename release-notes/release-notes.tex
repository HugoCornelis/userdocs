\documentclass[12pt]{article}
\usepackage[dvips]{epsfig}
\usepackage{color}
%e.g.  \textcolor{red,green,blue}{text}
\usepackage{url}
\usepackage[colorlinks=true]{hyperref}

\setlength{\textwidth}{15.7cm}
\setlength{\textheight}{22cm}

\setlength{\oddsidemargin}{0pt}

\setlength{\topmargin}{0cm}
\setlength{\headheight}{1cm}
\setlength{\headsep}{0.5cm}
\setlength{\topskip}{0cm}

\begin{document}

\section*{GENESIS: Documentation}

{\bf Related Documentation:}
% start: userdocs-tag-replace-items related-do-nothing
% end: userdocs-tag-replace-items related-do-nothing

\section*{Release Notes}

This document gives an overview of priorities for official releases of
the GENESIS 3 simulator.  Items required for proper functioning of the
simulator appear in this document as TODO items, and they disappear
when they have been implemented.  We are currently working on the
first alpha release of GENESIS 3.0.

Feedback is always appreciated.


\section{Outline of Priorities}

The first alpha release of GENESIS 3 will have a focus on single cell
modeling.  Single cell simulations include simulations based on
complex morphologies, active channels, current injection, voltage and
dynamic clamp, synaptic and endogenous stimulation.  Output can be
membrane potentials, currents or ion concentrations.  The G-Tube, the
official GENESIS 3 GUI, is an integral part of the alpha release.
Other GUIs for model exploration and result visualization are not
included, and undocumented (e.g. the Neurospaces studio and the
Neurospaces project-browser)\footnote{These modules are candidates to
  be incorporated into the G-Tube for a next release.}.

The most important functions of GENESIS 3 for single cell modeling
are:

\begin{itemize}
\item The GENESIS 3 shell for user interaction with model and
  simulation, including functions such as model construction, model
  exploration, input and output configuration, and running a
  simulation.
\item The G-Tube that offers the same functions as the GENESIS 3 shell
  from a GUI.
\item A morphology structure analyzer accessible from the project
  browser and the command line.
\item Installation documentation, and documentation for single neuron
  simulations.
\end{itemize}

To foster community building, also included are:
\begin{itemize}
\item User and developers documentation.
\item A twitter page that describes events important to GENESIS
  developments.
\item A blog and wiki to reach out to users and developers, and
  describes procedures for other developers to step in.
\end{itemize}


\section{Functionality for the alpha release}

\subsection{Two Scripting Interfaces}

GENESIS 3 comes with two shells.  One offers backward compatibility.
The second shell replaces the first one and has the advantages that it
offers better performance and integrates with the GENESIS 3 GUI.

\subsubsection{NS-SLI}
The GENESIS 2 backward compatibility module of GENESIS 3 (NS-SLI)
currently includes:
\begin{itemize}
\item A framework that allows to fill in most common core GENESIS 2
  functions.  These functions include the creation of model elements,
  seting their fields, and adding messages to connect model elements.
\item NS-SLI supports simulating complex morphologies.
\item NS-SLI supports tabulated channels.
\item NS-SLI supports calcium pools and Nernst functions.
\item NS-SLI supports the ascii out element.
\end{itemize}

The publicly availabe Purkinje cell scripts of the
De~Schutter~and~Bower model are supported by the NS-SLI module.

Models created from GENESIS 2 scripts can be exported to the GENESIS 3
NDF modeling format.


\subsubsection{GENESIS 3 Shell}

The GENESIS 3 command shell is a simple replacement for the GENESIS 2
SLI.  The GENESIS 3 command shell allows the user to define simple
models from scratch and save them to declarative model definition
files.  Building complex models is supported by glueing together
existing declarative model definition files.

The GENESIS 3 command shell to be shipped with the alpha
release, has all functions needed to instantiate and run single cell
models.  The shell talks directly to the {\bf Model\,Container} and {\bf SSP}, and
the {\bf Model\,Container} and {\bf SSP} talk to the compartmental solver ({\bf Heccer})
as required.

The GENESIS 3 shell can import and combine models from GENESIS 2
scripts, NDF files, SWC morphologies and Perl or Python scripts.

It is also possible to export the models to NDF files, and to export
configurations to {\bf SSP} files.  This facilitates batch simulations for
research projects.


\section{Included Documentation}

The included documentation introduces GENESIS 3, describes tutorials, gives technical
descriptions of commands, and documents the layout and APIs of the
source code.


\subsection{Documentation For Users}

Allan, can you fill this in?

\begin{itemize}
\item Basic tutorials: Tutorials for the {\it rallpacks} (passive models and
  axon models).
\item Basic tutorials: Tutorials for single cell model simulations.
\item FAQ: to explain what GENESIS 3 is, its relationship with the CBI
  architecture, with MOOSE, Neuroconstruct, and Neurospaces.  To
  explain what the discriminators are with other simulators, and that
  it is still backwards compatible with GENESIS 2.
\end{itemize}

The tutorials are based on the GENESIS 2 tutorials and demonstrate the
use of the NS-SLI to run simulations based on GENESIS 2 SLI scripts.
They also show how the model programmed into the script can be
exported to declarative NDF files for use with the more advanced
features of GENESIS 3 such as the project browser.

Some features of the compartmental solver are not documented yet, and
for completeness we briefly mention them here.  These include:
\begin{itemize}
\item Calculation of the total instantaneous cumulative currents of
  user-tagged channels of a dendritic subtree.
\item Calculation of the total instantaneous inhibitory and excitatory
  current.
%\item Soma-dendrite current during action potentials.
%\item Average membrane potential of a dendritic subtree.
\end{itemize}


\subsection{GENESIS 3 technical documentation}
The technical documentation describes the use of the GENESIS 3
simulator from a UNIX shell command line and from the GENESIS 3 shell.
It also outlines the NDF file format.  Examples are given for a simple
Hodgkin-Huxley model and for a model with Ca pools.  Technical
documentation bridges between user documentation and developer
documentation.

\subsection{Documentation For Developers}

The developer documentation is meant for people who are interested in
software development or other technical aspects of the GENESIS 3
simulator.  This documentation is not included in the alpha release,
but instead made available over the internet.

\begin{itemize}
\item A blog is updated on a daily to weekly basis.  The blog can be
  found at http://neurospaces.blogspot.com/.  The blog details ideas
  and vision and it shows developers and users the parts of the
  GENESIS 3 simulator we are working on, with a demonstration of
  progress.
\item The Neurospaces wiki
  (http://code.google.com/p/neurospaces/wiki/Index) documents a number
  of technical aspects of the software.  Currently the installation
  and test procedure is documented from a developers viewpoint.
\item The APIs of software modules such as {\bf Heccer} and the model
  container are documented in the source code.  Doxygen is a tool to
  convert such inline documents to HTML.  These HTML pages are
  published on the GENESIS website
  (http://www.neurospaces.org/doxygen-menu.html).
\item The latest version of the source code and its full history are
  made available on a daily basis.  The Neurospaces wiki documents how
  to use this source code repository.  The technology used allows
  developers to make changes independently, even from a disconnected
  laptop.  Specialized freely available technology is used to
  synchronize such distributed software development.
\end{itemize}

An experimental feature is the use of a community tool to document
user requirements and user workflows.

\begin{itemize}
\item A project planning tool documents user workflows through the
  software system, and establishes a relationship between them and the
  technical aspects of their implementation.  Xplanner (available from
  www.xplanner.org) will be used for this purpose.
\end{itemize}


\section{Installation and Testing}

The GENESIS 3 simulator uses tools of professional quality for
automated installation, testing and reporting.  An outline of the use
of these tools can be found on the Neurospaces wiki.  Briefly, the
installation tool enable and simplify installation and upgrade of source code, as well as initiation of your own development projects that (optionally) can be
integrated with GENESIS 3.

The test framework is a higly dynamic environment for stand-alone
software component testing and software component integration testing.
It compares test specifications and expected output with the output
produced by the software applications.

The test specifications are included in the alpha release and can be
exported to HTML and mirrored on a web server.  The Neurospaces
website has an HTML version of all tests.  The testing tool is
integrated with the installation procedure.  The Neurospaces wiki
documents how to setup a testbot based on the installation scripts
({\it neurospaces\_build} and {\it neurospaces\_cron}).

%The CBI hosts an functional simulator server accessible over a web
%browser.  The server hosts a simulation project that demonstrates the
%capabilities of the project browser using the Purkinje cell model.
%The simulator server is linked through the GENESIS website.


\section{Next Release}

The next release is planned for October 2009.

%\subsection{Threading}

%Parallelization is built into the design of GENESIS 3.  The next
%release of GENESIS 3 will have support for multi threading for
%enhanced simulation performance on multi core CPUs.

%\subsection{Cluster computing support}

%Running neuroscience simulations on clusters requires the use of MPI
%technology.  We are currently evaluating the Music API for MPI based
%implementation of the GENESIS 3 simulator such that simulations can be
%run on cluster computers.

%\subsection{Python}

%The current implementation has minimal support for the Python
%scripting language.  The Python functions are available over the
%GENESIS 3 shell, but are not mature.  In the next release we will
%provide improved Python bindings for better integration with other
%simulators.



\end{document}

%%% Local Variables: 
%%% mode: latex
%%% TeX-master: t
%%% End: 
