\documentclass[12pt]{article}
\usepackage[dvips]{epsfig}
\usepackage{color}
\usepackage{url}
\usepackage[colorlinks=true]{hyperref}

\begin{document}

\section*{GENESIS: Documentation}

\section*{Introduction to GENESIS for Developers}

The documentation linked to here is for GENESIS users who want to contribute to the development of extended functionality of the GENESIS simulation platform, rather than for those users who want to do scientific simulations. (Although a single individual may want to do both.) Developer documentation provides links to more technical documentation supporting software development. It is assumed that you are familiar with the monotone version control software used by GENESIS (for more information see \href{../document-versionctrl/document-versionctrl.tex}{Version Control}). Links to the more general user documentation can be found at the \href{../document-homepage/document-homepage.tex}{documentation home page}. 
%The documentation linked to here is for the use by people who are involved in the development and extension of GENESIS functionality.

\subsection*{Becoming a GENESIS Developer}

In principle anyone can become a GENESIS developer, all that is required is a desire to focus on tool and documentation development rather than model building and simulation. There are at least two ways to be a GENESIS  developer:

\begin{itemize}
	\item {\bf Member of the GENESIS Developers Federation:} To become a GENESIS developer send a request to 
	\href{mailto:genesis@genesis-sim.org}{genesis@genesis-sim.org}. You will be registered as a GENESIS developer 
	and receive a key that will allow you to pull and push documentation and code to the GENESIS repositories. To learn more 
	about these GENESIS repositories and version control see \href{../document-versionctrl/document-versionctrl.tex}{here}.
	
	\item {\bf Independent Developer:} GENESIS is structured in such a way that it is possible to download the code from:\\
	\href{\href{http://sourceforge.net/project/showfiles.php?group_id=162899}{http://sourceforge.net/project/showfiles.php?group
	\_id=162899}}{\href{http://sourceforge.net/project/showfiles.php?group_id=162899}{http://sourceforge.net/project/showfiles.php?group\_id=162899}}).\\
	You can then proceed entirely independently with your own code and documentation development. To do this see the 
	\href{../introduction-installation-developer}{developer installation} documentation. It describes how to set up your own 
	choice of version control and associated file repositories.

\end{itemize}

\section*{Developing Sources}

Here we describe the directory structure used to do GENESIS development.

GENESIS contains several software components useful for neural simulations using models based on experimental data:
\begin{enumerate}
\item {\bf Each software component can have implementations in several programming languages:} For example, there are java and C based implementations of the {\it model-container} component.
\item {\bf Each software component has several resources:} For example, source code, documentation, additional tools etc.
\item {\bf Each software component can have experimental features that are not part of an official distribution:} We call this a branch of the source code (it is irrelevant if such a branch is present in a version control system or not). 
\end{enumerate}

\subsection*{Source Locations}

\begin{itemize}
\item {\bf The source code of a software component is normally put in the directory:}
\begin{verbatim}
$HOME/neurospaces_project/<package-name>/source/ \
   <programming language>/snapshots/<branch name>/
\end{verbatim}

\item {\bf Other resources are put in other directories:} For example, documents that are not part of the distribution can be put under
\begin{verbatim}
$HOME/neurospaces_project/<package-name>/docs/
\end{verbatim}

\item {\bf Archived patches are located in:}
\begin{verbatim}
$HOME/neurospaces_project/<package-name>/source/ \
   <programming language>/patches/ 
\end{verbatim}

\end{itemize}
The {\tt InstallerPackage} has built in support for this directory structure when working in developer mode (using its {\tt --developer} option).

\subsection*{Configuration}

Configuration is located in the {\it /etc/neurospaces/} directory.

Every software component or tool has its own subdirectory. E.g. the {\it morphology2ndf} convertor, part of the {\it model-container}, has its default configuration in {\it /etc/neurospaces/morphology2ndf}. Configuration files are specified in the \href{http://www.yaml.org/}{YAML format}, because scripting languages like \href{http://www.perl.org/}{perl} and \href{http://www.python.org/}{python} have built in support to process YAML, and because YAML is more readable than XML.

Configuration is normally processed as follows:
\begin{enumerate}
\item {\bf Every tool has built in default configuration:} Every tool must be able to work in an intuitive way when no other configuration is found except the default built in configuration.
\item {\bf Every tool reads its configuration from {\it /etc/neurospaces/$<$tool-name$>$/}:} This is merged with default configuration, possibly overriding default configuration.
\item {\bf Every tool accepts additional configuration from the user:} This is via command line options or otherwise, again possibly overwriting configuration settings generated in the previous step. 
\end{enumerate}
The merging algorithm is defined in the \href{http://search.cpan.org/dist/Data-Utilities/}{Data-Utilities package}, available from \href{http://www.cpan.org/}{CPAN}.

There is a default configuration available in the configurator package (UNDER CONSTRUCTION). 

\section*{Creating a New Software Component}

GENESIS contains many software components. The source code of the most important ones are publicly available from a central repository. (The server {\tt virtual2 at cbi} is coded as the default server in the {\it InstallerPackage}. This default can be overwritten using command line options.) Other software components can be made available from other sources. The installer package, when configured correctly, automatically incorporates software components from geographically distributed sources.

New software components can be added to the configuration of the {\it neurospaces\_build} script of the {\it InstallerPackage}. All the other tools of the {\it InstallerPackage}, such as tools to compile and install, tools to synchronize the source code with remote servers, and tools to generate and publish documentation on a website will work with the new configuration.

For smooth integration with the GENESIS installer, it is a requirement that the top level source directory of the new component contains a {\it configure} script, and a {\it Makefile} with the targets {\it clean}, {\it check}, {\it dist}, {\it distcheck}, {\it install}, {\it uninstall}, {\it docs}, {\it html-upload-prepare}, {\it html-upload}, and {\it dist-keywords}.

\subsubsection*{Creating a new software component}

The following steps should be performed on your developer machine each time you wish to start development of a new software component.

Use the command line arguments ``{\tt --enable your-software --regex your-software}'' to limit the operations given below to only the software component named, here as an example we use ``{\tt your-software}''. 

\begin{itemize}
	\item {\bf Add the new software component to the configuration of the {\it neurospaces\_build} script:} As an example, for the component named {\tt your-software}, you would add the following code block to the configuration file
	\begin{footnotesize}
	\begin{verbatim}
	'your-software' => {
	   './configure' => ['--with-delete-operation'],
	   directory => "$ENV{HOME}/neurospaces_project/your-software/source/snapshots/0",
	   disabled => 0,
	   order => 1,
	   target_name => 'your-software',
	   version_control => {
	      port_number => 4693,
	      repository => "$ENV{HOME}/neurospaces_project/MTN/your-software.mtn",
	   },
},
	\end{verbatim}
	\end{footnotesize}
	This code block includes the directory name where sources are to be found, the build order, and version control information 
	(note that the server port number cannot be changed at anytime).

	\item {\bf Create the correct directory layout:}
	\begin{verbatim}
	neurospaces_create_directories
	\end{verbatim}
	
	\item {\bf Create the monotone repository:}
	\begin{verbatim}
	neurospaces_build --enable your-software \
	   --regex your-software --developer --repo-co
	\end{verbatim}

	\item {\bf Populate the project with files:} Use monotone to {\it checkin} your modifications.

	\item {\bf Configure your new monotone server:} Use the information in the configuration of the new software component.
	
	\item {\bf Synchronize your local monotone repository:} Use {\it neurospaces\_sync} to synchronize with the server repository.
	
\end{itemize} 

\subsubsection*{Pushing a new software component to other computers}

\begin{itemize}

	\item {\bf Get a version of the installer package that contains configuration for the new software component.}
	
	\item {\bf Create the correct directory layout:}
	\begin{verbatim}
	neurospaces_create_directories
	\end{verbatim}

	\item {\bf Pull the repository from the server:} You can also create a new repository locally if needed.
	\begin{verbatim}
	neurospaces_pull --enable your-software \
	   --regex your-software
	\end{verbatim}
	
	\item {\bf Update local source code with the latest source in the local repository, keeping local changes if any:}
	\begin{verbatim}
	neurospaces_update --enable your-software \
	   --regex your-software
	\end{verbatim}
	
	\item {\bf Recompile all software components, and link them against each other:} 
	\begin{verbatim}
	neurospaces_install --configure
	\end{verbatim}
	The ``{\tt --configure}'' argument enables configuration, which is needed for a new software component.
\end{itemize}

\section*{Using the GENESIS repository for Development}

For the federated development of the software components of GENESIS, a software buildbot machine houses a \href{http://monotone.ca/}{monotone} server that is connectible over the internet. Source code changes can be properly merged using the functions available on monotone, regardless of the location where the changes have been made (at home, in your office, on the airplane). The build mechanism will always build the most up to date version of the code.

\subsection*{Convenient Access to the Source Code}

The easiest way to get the latest version of the source code is via the Neurospaces installer. See the section  NeurospacesRepositoryAndTheInstaller for more information.

We now explain how to {\it checkout} code manually when not using the installer.

\subsubsection*{Serving your source code}

If you want someone else to {\it sync} their code with what you have on your laptop, you can conveniently convert your laptop to a source code server using the command {\it neurospaces\_serve} (see the \href{../installerpackage/installerpackage.tex}{InstallerPackage}). Other people can now connect to your laptop using, for example
\begin{verbatim}
   neurospaces_sync --repo-sync <ip-address-of-your-laptop>
\end{verbatim}
or
\begin {verbatim}
   neurospaces_pull --repo-pull <ip-address-of-your-laptop>
\end{verbatim}
All these commands accept a ``{\tt --regex $<$package-regex$>$}'' to select packages to {\it serve}, {\it sync} and {\it pull} respectively. To interrupt the server and kill all the server processes, use ``{\tt neurospaces\_kill\_servers}''.

\subsection*{Checking out Code Manually}

To check out the latest version of the GENESIS code you must first initialize a local monotone repository via this command:
\begin{verbatim}
   mtn --db=myrepo.mtn db init 
\end{verbatim}
There are monotone instances running for different parts of GENESIS on different ports. To target the correct component you need to use the following arguments for the remote server. The server address and port number are accessed by, for example, for the backward compatibility layer
\begin{verbatim}
   repo-genesis3.cbi.utsa.edu:4692
\end{verbatim}

\vspace{3mm}
\begin{footnotesize}
\begin{centering}
\begin{tabular}{| l | l | c | l |}
\hline
{\bf Description}                                               & {\bf Package name}   & {\bf Port \#} & {\bf Repository Name} \\ \hline
GENESIS backward compatability layer    & ns-genesis-SLI           & 4692           & ns-gen.mtn \\ \hline
Internal Storage for Models                           & model-container        & 4693           & model-container.mtn \\ \hline
Single neuron solver                                      & heccer 	                        & 4694           & heccer.mtn \\ \hline
Scheduler                                                         & ssp                                & 4695           & ssp.mtn \\ \hline
Installer                                                             & installer                        & 4696           & neurospaces-developer.mtn \\ \hline
Project Browser                                               & project-browser          & 4697           & neurospacesweb.mtn \\ \hline
Incomplete GUI 	                                      & studio                            & 4698           & studio.mtn \\ \hline
GShell 	                                                         & gshell                            & 4699          & gshell.mtn \\ \hline
User documentation                                       & userdocs                      & 4700          & userdocs.mtn \\ \hline
Gui development                                             & gui                                 & 4701          & gui.mtn \\ \hline
\end{tabular}
\end{centering}
\end{footnotesize}

\noindent Code can be pulled from the monotone server into your local working repository with the command
\begin{verbatim}
   mtn --db=myrepo.mtn pull virtual2.cbi.utsa.edu:<port#> ``*''
\end{verbatim}
The ``*'' argument pulls all branches from the server into your local repository. To {\it pull} a specific branch see the following section. It's important that each server be synchronized with it's own local repository. Once the code has been pulled you can checkout one of the branches into your working directory with the command
\begin{verbatim}
   mtn --db=myrepo.mtn --branch=<Branch ID> co source_directory 
\end{verbatim}

\subsection*{Synchronizing Databases}

To synchronize your monotone repository to the remote server you'll need to use the {\it sync} command with the proper address of the server (as listed in the above table).
\begin{verbatim}
   mtn --db=myrepo.mtn sync virtual2.cbi.utsa.edu:<port#> "*" 
\end{verbatim}
Now all changes, including logs will be on both the client and server repositories. Note in order to perform this operation you need to be given write access to the repositories on the GENESIS server repository {\tt virtual2}.

\subsection*{Branches}

You must know the desired branch when pulling a repository. All branches can be downloaded simultaneously via the ``*'' argument. We currently use numeric branches that start at zero ({\tt 0}). After pulling a repository you can generate a list of branches available for check out from your local repository via the command
\begin{verbatim}
   mtn --db=myrepo.mtn list branches 
\end{verbatim}

\subsection*{Committing Changes}

A monotone repository is an annotated database and every change that is made is signed with keys. To be able to commit changes to your local monotone repository you need to generate a key pair. To do this execute the command
\begin{verbatim}
   mtn genkey email@address.com 
\end{verbatim}
This stores a your private key in {\it ~/.monotone}. To enable this, you will be asked to enter a password which is used to verify your identity when you check in changes. To see all the keys monotone has stored you can do a
\begin{verbatim}
   mtn list keys 
\end{verbatim}
and also an
\begin{verbatim}
   mtn --db=myrepo.mtn list keys 
\end{verbatim}
To commit the changes you have made to your own database repository, {\tt cd} to the appropriate source directory and execute either
\begin{verbatim}
   mtn ci
\end{verbatim}
or
\begin{verbatim}
   mtn commit 
\end{verbatim}

\subsection*{Repository Access}

To get write access to a repository which you did not initiate, it must be signed with your public key. To export your public key perform the following command
\begin{verbatim}
   mtn --db=repo.mtn pubkey my@email.com > me.pubkey 
\end{verbatim}
Send the repository owner your public key for write access. To perform this you must have monotone read in the key from {\it stdin} with the read command
\begin{verbatim}
   cat me.pubkey | mtn --db=repo.mtn read 
\end{verbatim}
If this is successful it will indicate that a packet has been read. You can store several keys in a file this way with the number of successful packets read in being given. 

\section*{Accessing the Neurospaces Repository from the {\it InstallerPackage}}

The \href{../genesis-repository/genesis-repository.tex}{GENESIS Repository} is accessible from the \href{../installerpackage/installerpackage.tex}{InstallerPackage}. Important for this feature to work is to have the correct directory layout on your \href{../developermachine/developermachine.tex}{DeveloperMachine}.

In summary, what you want to do is pull the set of default packages from the server(s):
\begin{verbatim}
   neurospaces_build --repo-pull virtual2.cbi.utsa.edu \
      --repo-co --verbose --developer --directories-create \
      --no-configure --no-compile --no-install
\end{verbatim}
This command has options that inhibit the default actions of configuration, compilation, and installation. If you also want to compile and install in just one run, simply omit those options:
\begin{verbatim}
   neurospaces_build --repo-pull virtual2.cbi.utsa.edu \
      --repo-co --verbose --developer --directories-create
\end{verbatim}
There are developer friendly frontends to the {\it neurospaces\_build} script in the \href{../installerpackage/installerpackage.tex}{InstallerPackage}:
\begin{itemize}
\item {\it neurospaces\_serve}\,\,\,Starts serving the source code repositories such that other people can {\it pull} and {\it sync} to your machine (note that this locks all your databases).
\item {\it neurospaces\_pull}\,\,\,Download the source code from a repository.
\item {\it neurospaces\_status}\,\,\,Check for local source code modification (no network required).
\item {\it neurospaces\_sync}\,\,\,Synchronize local source code modification with a repository.
\item {\it neurospaces\_update}\,\,\,Make the local source code up to date using the repositories locally stored on your PC (so this is a local operation). 
\end{itemize}

\subsection*{Details}

\begin{itemize}
\item {\bf Pull to your local repository:}
\begin{verbatim}
neurospaces_build --repo-pull virtual2.cbi.utsa.edu \
   --no-configure --no-install --no-compile --verbose --developer \
   --regex 'heccer|model-container|ssp'
\end{verbatim}

\item {\bf Checkout from your local repository:}
\begin{verbatim}
neurospaces_build --repo-co --no-configure \
   --no-install --no-compile --verbose --developer \
   --regex 'heccer|model-container|ssp'
\end{verbatim}

\item {\bf When you start from scratch it is useful to create the workspace directories:}
\begin{verbatim}
neurospaces_build --repo-co --no-configure \
   --no-install --no-compile --verbose --developer \
   --regex 'heccer|model-container|ssp' --directories-create
\end{verbatim}

\item {\bf Combine everything for the set of default packages, including compilation and installation:}
\begin{verbatim}
neurospaces_build --repo-pull virtual2.cbi.utsa.edu \
   --repo-co --verbose --developer --directories-create
\end{verbatim}

\end{itemize}

\end{document}
