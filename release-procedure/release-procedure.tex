\documentclass[12pt]{article}
\usepackage[dvips]{epsfig}
\usepackage{color}
\usepackage{url}
\usepackage[colorlinks=true]{hyperref}

\begin{document}

\section*{GENESIS: Documentation}

{\bf Related Documentation:}
% start: userdocs-tag-replace-items related-do-nothing
% end: userdocs-tag-replace-items related-do-nothing

\section*{Release Procedure}

A release is a bundle with a consistent set of software packages
required to use the GENESIS simulator.  Internal releases are built on
a daily basis, public release are built on a monthly to yearly basis.

A public release has the guarantee that it passes all the regression
tests.  Internal releases are tested to, but can show minor problems.

Every release is identified with a tag.  The tag (see below for a
definition) allows to reconstruct the release source code from the
repositories using the developer package.  This helps in solving
problems reported by users.

\begin{itemize}
\item {\bf Define the tag}
  \begin{itemize}
  \item In principal, developer tags follow the naming convention:
\begin{verbatim}
   <milestone-number>
\end{verbatim}   
    So, for example, for the current release, the tag is {\tt des-10}, which is the tenth subrelease after starting development of the {\tt des} milestone.
  \item Milestone labels point to an important event for development.
  \item Tag numbers are monotonically increased per milestone.
  \item Some packages may skip some releases, milestone labels, and/or tag numbers.
  \item Internally, the development system refers to the milestone labels as `majors', the numbers as `minor's', and for future use, there is also a `micro' number possible.
  \end{itemize}
  
\item {\bf Set the tag using {\it neurospaces\_build}:} Use the options {\tt --tag}, e.g. {\tt neurospaces\_build --tag des-10}. Use additional options to select individual packages if needed.
\item {\bf Build the tarballs:}
  \begin{itemize}
  \item Use the UNIX shell command {\tt neurospaces\_dist}.
  \item Use additional options such as {\tt --regex} and {\tt
      --enable} to select individual packages if required.
  \item Use the {\tt --certification-report} to annotate the {\it
      monotone} revision with the output of the build (this step is
    optional).
  \end{itemize}

\item {\bf Build the OS specific packages:}
  \begin{itemize}
  \item Use the UNIX shell command {\tt neurospaces\_release}.
  \item Use additional options such as {\tt --regex} and {\tt
      --enable} to select individual packages if required.
  \item Use the {\tt --certification-report} to annotate the {\it
      monotone} revision with the output of the build (this step is
    optional).
  \end{itemize}

\item {\bf Put the files on an file server.}
%  \begin{itemize}
%  \item Currently, only {\it ftp} is supported.
%  \item If uploading to the \href{http://sourceforge.net/projects/neurospaces/}{sourceforge} server, you have to fill the release notes manually. 
%  \end{itemize}
%  Note that the build script by default installs the packages on the developer machine. Use {\tt --no-install} if this is not intended.

\item For examples, see the \href{../developer-package/developer-package.tex}{\tt DeveloperPackage} 

  It might be useful to do ``{\tt make html-upload}'' for some packages after the release, to synchronize the content of the website with the content of the downloadable packages. 
\end{itemize}


\end{document}
