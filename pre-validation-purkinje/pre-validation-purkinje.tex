\documentclass[12pt]{article}
\usepackage{verbatim}
\usepackage[dvips]{epsfig}
\usepackage{color}
\usepackage{url}
\usepackage[colorlinks=true]{hyperref}

\begin{document}

\section*{GENESIS: Documentation}

{\bf Related Documentation:}
% start: userdocs-tag-replace-items related-do-nothing
% end: userdocs-tag-replace-items related-do-nothing

\section*{Pre-validation of the Purkinje cell before model submission}

\subsection*{Introduction}

Model validation can be automated to a large extent.  This document
show specific capabilities of the GENESIS 3 publication system related
to automated model validation of single neuron models.  The neuron
model developed in~\cite{De-Schutter-E:1994vn},~\cite{E:1994hc} and
studied in~\cite{schutter94:_simul_purkin} is used as an example.

\subsection*{What is included?}

Automated validation of single neuron models includes simulations for
the following sections.


\subsubsection*{From~\cite{De-Schutter-E:1994vn}}

\begin{itemize}
\item Static model checks:
  \begin{itemize}
  \item Morphology validation: are all the branches connected via an
    electric pathway?
  \item Cable discretization validation:
    \begin{itemize}
    \item Are the constraints to electrotonic length met?
    \item Are all electrotonic compartments of similar lengths?
    \end{itemize}
  \item Equations for different membrane and synaptic channel types
    and their kinetics?  Are these common equations or new ones?
  \item Model parameter values: are all parameter values in the
    physiological range?
    \begin{itemize}
    \item Passive parameters: capacitance, axial and membrane
      resistance, resulting membrane time constant?
    \item Channel maximal conductances and channel densities?
    \item Channel time constants?
    \item Concentration value time constants?
    \end{itemize}
  \end{itemize}
\item Dynamic response to specific protocols:
  \begin{itemize}
  \item Current injection: 0.2nA, 0.4nA, 0.6nA, 0.8nA, 1nA, 2nA, 3nA,
    4nA,
  \item Current injection with sodium blocker (TTX): 1nA, 0.1nA
  \item Current injection with Ca$^{2+}$ blocker (Co$^{2+}$,
    Cd$^{2+}$): 4nA, 3.5nA, 3nA, 2.5nA, 2nA, 1nA
  \item Fig10: False color representation of membrane potential and Ca$^{2+}$
    concentration in the complete model during a 2.0nA current
    injection.
  \item Fig 11: model variable plots during somatic and dendritic
    spikes, 2nA current injection.
  \item Fig 12: complicated
  \item Fig 13: Simulation of voltage clamp steps: -100mV, +40mV
  \end{itemize}
\item Dynamic response to additional common protocols:
  \begin{itemize}
  \item Voltage clamp with various blockers.
  \end{itemize}

\end{itemize}

\subsubsection*{From~\cite{E:1994hc}}

\begin{itemize}
\item Comparison of model parameters to model
  of\cite{De-Schutter-E:1994vn}.  Report of changes and differences.
\item Application of validation procedures
  of\cite{De-Schutter-E:1994vn}.
\item Validation of new model parameters.
  \begin{itemize}
  \item Passive structure of spines.
  \item For each of the following synapses:
    \begin{itemize}
    \item Locations (visualization).
    \item Maximal synaptic conductance.
    \item Reversal Potentials.
    \item Other (nernst equations, Mg blocking?).
    \end{itemize}
  \item Parallel fiber synapses.
  \item Climbing fiber synapse locations.
  \item Stellate cell synapses.
  \item Basket cell synapses.
  \end{itemize}
\item Selected figures:
  \begin{itemize}
  \item Figure 6: asynchronous excitation alone, 5Hz, 3Hz, 1Hz, 0.5Hz,
    0.1Hz.
  \item Figure 7: 8 simulations with asynchronous excitation 100Hz,
    60Hz, 40Hz, 30Hz, 26Hz, 24Hz, 22Hz, 20Hz and with inhibition 1Hz.
  \item Figure 8, left panel (right is experimental data): unclear
    what the simulations were, to be reconstructed.
  \item Figure 9
    \begin{itemize}
    \item Panel A: inhibition 0Hz, 7 simulations.  Inhibition 0.5Hz, 6
      slow + 22 fast firing.  Inhibition 1.0Hz, 7 slow + 22 fast
      firing.  Inhibition 1.5Hz, 6 slow + 14 fast firing.  Inhibition
      2.0Hz, 10 slow + 12 fast firing.
    \item Panel B is model PM10
    \item Panel C: excitation 10.4Hz and inhibition 0.5Hz, excitation
      23.5Hz and inhibition 1Hz, excitation 37Hz and inhibition 1.5Hz,
      excitation 50Hz and inhibition 2Hz.
    \end{itemize}
  \item Figure 10.
  \end{itemize}
\end{itemize}

\subsubsection*{From~\cite{schutter94:_simul_purkin}}

\begin{itemize}
\item Comparison of base model parameters to model of\cite{E:1994hc}.
\item Application of validation procedures of\cite{De-Schutter-E:1994vn}.
\item Checks on changed parameters
  \begin{itemize}
  \item Passive soma.
  \end{itemize}
\item Selected figures:
  \begin{itemize}
  \item Figure 1
    \begin{itemize}
    \item Twice the same set of simulations.
    \item 9 different stimulation protocols: unitary synaptic
      stimulation applied to 8 different branchlets individually and
      to the 8 same branches simultaneously.
    \item Passive model.
    \item Active model.
    \end{itemize}
  \item Figure 2: 10 data points, averaged over 40 simulation runs
    (events), 4 different model variations.
  \item Figure 4: Detail of two selected simulations.
  \item Figure 5: 200 simulations, distributed, proximal and distal
    stimulation.
  \end{itemize}
\end{itemize}

\subsection*{Results}



\bibliographystyle{plain}
\bibliography{../tex/bib/g3-refs.bib}

\end{document}


