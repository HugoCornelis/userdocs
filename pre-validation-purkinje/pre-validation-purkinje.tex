\documentclass[12pt]{article}
\usepackage{verbatim}
\usepackage[dvips]{epsfig}
\usepackage{color}
\usepackage{url}
\usepackage[colorlinks=true]{hyperref}

\begin{document}

\section*{GENESIS: Documentation}

{\bf Related Documentation:}
% start: userdocs-tag-replace-items related-do-nothing
% end: userdocs-tag-replace-items related-do-nothing

\section*{Pre-validation of the Purkinje cell before model submission}

\subsection*{Introduction}

This document shows specific capabilities of the GENESIS 3 publication
system for automated model validation and electronic model publication
of single neurons.  The examples used in this document include the
neuron model developed in~\cite{De-Schutter-E:1994vn},~\cite{E:1994hc}
and studied in~\cite{schutter94:_simul_purkin}, the comparative
extension of these studies
in~\cite{bower09:_compar_evolut_comput_analy_cereb} and finally the
theoretical study of the model in~\cite{coop09:_using_purkin}.

\subsection*{What is included?}

Single neuron models can be validated by analysis of their
morphological and mathematical structure.  Static checks include
soundness checks on the morphology such as range checks on dendritic
lengths and number of branch points, passive cable parameters, maximal
channel conductances and channel kinetics.  Behaviour checks are
implemented with simulations that apply common experimental paradigms
to the neuron model.  Simulations for parameter searches are not
included.


\subsubsection*{From~\cite{De-Schutter-E:1994vn}}

\begin{itemize}
\item Static model checks:
  \begin{itemize}
  \item Morphology validation: are all the branches connected via an
    electric pathway?
  \item Cable discretization validation:
    \begin{itemize}
    \item Are the constraints to electrotonic length met?
    \item Are all electrotonic compartments of similar lengths?
    \end{itemize}
  \item Equations for different membrane and synaptic channel types
    and their kinetics?  Are these common equations or new ones?
  \item Model parameter values: are all parameter values in the
    physiological range?
    \begin{itemize}
    \item Passive parameters: capacitance, axial and membrane
      resistance, resulting membrane time constant?
    \item Channel maximal conductances and channel densities?
    \item Channel time constants?
    \item Concentration value time constants?
    \end{itemize}
  \end{itemize}
\item Dynamic response to specific protocols:
  \begin{itemize}
  \item Current injection (fig 5, 0.2nA, 0.4nA, 0.6nA, 0.8nA) and
    current clamp (fig 3, 0.5nA, 1nA, 2nA, 3nA).
  \item Current injection with sodium blocker (TTX): 1nA, 0.1nA
  \item Current injection with Ca$^{2+}$ blocker (Co$^{2+}$,
    Cd$^{2+}$): 4nA, 3.5nA, 3nA, 2.5nA, 2nA, 1nA
  \item Fig10: False color representation of membrane potential and Ca$^{2+}$
    concentration in the complete model during a 2.0nA current
    injection.
  \item Fig 11: model variable plots during somatic and dendritic
    spikes, 2nA current injection.
  \item Fig 12: complicated
  \item Fig 13: Simulation of voltage clamp steps: -100mV, +40mV,
    -40mV.
  \end{itemize}
\item Dynamic response to additional common protocols:
  \begin{itemize}
  \item Voltage clamp with various blockers.
  \end{itemize}

\end{itemize}

\subsubsection*{From~\cite{E:1994hc}}

\begin{itemize}
\item Comparison of model parameters to model
  of~\cite{De-Schutter-E:1994vn}.  Report of changes and differences.
\item Application of validation procedures
  of~\cite{De-Schutter-E:1994vn}.
\item Static validation of new model parameters.
  \begin{itemize}
  \item Passive structure of spines.
  \item For each of the following synapses:
    \begin{itemize}
    \item Locations (visualization).
    \item Maximal synaptic conductance.
    \item Reversal Potentials.
    \item Other (nernst equations, Mg blocking?).
    \end{itemize}
  \item Parallel fiber synapses.
  \item Climbing fiber synapse locations.
  \item Stellate cell synapses.
  \item Basket cell synapses.
  \end{itemize}
\item Selected figures:
  \begin{itemize}
  \item Figure 1: climbing fiber response characterization in vitro.
  \item Figure 3: climbing fiber response characterization in vitro,
    active vs passive model.
  \item Figure 6: asynchronous excitation alone, 5Hz, 3Hz, 1Hz, 0.5Hz,
    0.1Hz.
  \item Figure 7: 8 simulations with asynchronous excitation 100Hz,
    60Hz, 40Hz, 30Hz, 26Hz, 24Hz, 22Hz, 20Hz and with inhibition 1Hz.
  \item Figure 8, left panel (right is experimental data): unclear
    what the simulations were, to be reconstructed.
  \item Figure 9
    \begin{itemize}
    \item Panel A: inhibition 0Hz, 7 simulations.  Inhibition 0.5Hz, 6
      slow + 22 fast firing.  Inhibition 1.0Hz, 7 slow + 22 fast
      firing.  Inhibition 1.5Hz, 6 slow + 14 fast firing.  Inhibition
      2.0Hz, 10 slow + 12 fast firing.
    \item Panel B is model PM10
    \item Panel C: excitation 10.4Hz and inhibition 0.5Hz, excitation
      23.5Hz and inhibition 1Hz, excitation 37Hz and inhibition 1.5Hz,
      excitation 50Hz and inhibition 2Hz.
    \end{itemize}
  \item Figure 10: inhibition always 1Hz, excitation (1) 21Hz at 1nS,
    (2) 30Hz at 0.7nS, (3) 60Hz at 0.35nS, (4) 210Hz at 0.1nS.
    Simulation runs until 1600 events have been collected.
  \end{itemize}
\end{itemize}

\subsubsection*{From~\cite{schutter94:_simul_purkin}}

\begin{itemize}
\item Comparison of base model parameters to model of~\cite{E:1994hc}.
\item Application of validation procedures
  of~\cite{De-Schutter-E:1994vn}.
\item Checks on changed parameters
  \begin{itemize}
  \item Passive soma.
  \end{itemize}
\item Selected figures:
  \begin{itemize}
  \item Figure 1
    \begin{itemize}
    \item Twice the same set of simulations.
    \item 9 different stimulation protocols: unitary synaptic
      stimulation applied to 8 different branchlets individually and
      to the 8 same branches simultaneously.
    \item Passive model.
    \item Active model.
    \end{itemize}
  \item Figure 2: 10 data points, averaged over 40 simulation runs
    (events), 4 different model variations.
  \item Figure 3: Detail of two selected simulations.
  \item Figure 4: one selected simulation, false color images of the
    dendrite.
  \item Figure 5: 200 simulations, distributed, proximal and distal
    stimulation.
  \end{itemize}
\end{itemize}


\subsubsection*{From~\cite{bower09:_compar_evolut_comput_analy_cereb}}

\begin{itemize}
\item 9 species, stimulation protocols (static dendritic voltage
  clamp, static somatic voltage clamp, dynamic somatic voltage
  clamp, unitary synaptic stimulation)
\end{itemize}

\subsubsection*{From~\cite{coop09:_using_purkin}}


\subsection*{Benchmarking}

\subsubsection*{Introduction}

\begin{itemize}
\item Queries to the system for static model checks must be
  configured.  These checks complete 'instantly' (< 1s).
\item For simple simulations benchmarking on a passive and an active
  version of the Purkinje cell has been done.  Details related to the
  stimulation protocol or simulation such as an active model with a
  passive soma or addition of a single synapse for unitary synaptic
  stimulation are currently ignored.
\item Simulation construction time for an active version of the
  purkinje cell model is about 4 seconds and for a passive version 0.5
  seconds.  All simulations were run on a 2 CPU 8 core intel 64bit at
  1.60GHz with 6MB level 2 cache.  The simulations were neither run in
  parallel nor multi-threaded.  All simulations of the Purkinje cell
  model were run with each of the 1474 spines explicitly modeled with
  two compartments.
\item Estimates for more complicated analysis with results from many
  simulations such as figure~10 in~\cite{E:1994hc} are acquired by
  extrapolation of the estimates of the previous item.
\end{itemize}


\subsubsection*{Results}

\begin{itemize}
\item From~\cite{De-Schutter-E:1994vn}

  \begin{itemize}
  \item Static model checks: complete instantly.
  \item Parameter search is not included.
  \item Dynamic response to specific protocols:
    \begin{itemize}
    \item Current injection (fig 5, 4 simulations of 0.5s, 36s
      run-time each) and current clamp (fig 3, 4 simulations of 1s,
      72s run-time each).
    \item Current injection with sodium blocker (TTX), fig 9A, 2
      simulations of 1s, 72s run-time each.
    \item Current injection with Ca$^{2+}$ blocker (Co$^{2+}$,
      Cd$^{2+}$), fig 9B, 6 simulations of 1.2s, 86s run-time each.
%    \item Fig 10: False color representation of membrane potential and
%      Ca$^{2+}$ concentration in the complete model during a 2.0nA
%      current injection.
    \item Fig 11: model variable plots, 2nA current injection, one
      simulation of 1s, 72s run-time.
    \item Fig 13: Simulation of voltage clamp steps: -100mV, +40mV,
      -40mV, 3 simulations of 0.2s, 15s run-time each.
    \end{itemize}
%  \item Dynamic response to additional common protocols:
%    \begin{itemize}
%    \item Voltage clamp with various blockers.
%    \end{itemize}

  \item Total of 144s + 288s + 144s + 516s + 72s + 45s = 1209s total
    run-time.
  \end{itemize}


\item From~\cite{E:1994hc}

  \begin{itemize}
  \item Comparison of model parameters to model
    of~\cite{De-Schutter-E:1994vn}.  Report of changes and
    differences.

  \item Application of validation procedures
    of~\cite{De-Schutter-E:1994vn}.
  \item Static validation of new model parameters completes instantly.
  \item Selected figures:
    \begin{itemize}
    \item Figure 1: climbing fiber response characterization in vitro,
      3 simulations of 0.15s, 13s run-time each.
    \item Figure 3: climbing fiber response characterization in vitro,
      active vs passive model, 2 simulations of 0.15s with 13s
      run-time each, active model, passive model 2.5s run-time.
    \item Figure 6: asynchronous excitation alone, 5Hz, 3Hz, 1Hz, 0.5Hz,
      0.1Hz, 5 simulations of 50ms, each 7s run-time.
    \item Figure 7: 8 simulations with asynchronous excitation 100Hz,
      60Hz, 40Hz, 30Hz, 26Hz, 24Hz, 22Hz, 20Hz and with inhibition
      1Hz, simulations of 50ms, each 7s run-time.
    \item Figure 8, left panel (right is experimental data): unclear
      what the simulations were, to be reconstructed.  We assume 24Hz
      excitation, 1Hz inhibition, about 100s simulation, about 7100s
      run-time\marginpar{why again 100s?}.
    \item Figure 9
      \begin{itemize}
      \item Panel A: inhibition 0Hz, 7 simulations.  Inhibition 0.5Hz,
        6 slow + 22 fast firing.  Inhibition 1.0Hz, 7 slow + 22 fast
        firing.  Inhibition 1.5Hz, 6 slow + 14 fast firing.
        Inhibition 2.0Hz, 10 slow + 12 fast firing.  Total is 7 + 6 +
        22 + 7 + 22 + 6 + 14 + 10 + 12 = 106 simulations.  Assuming a
        minimum of 100 events per simulation and a total average
        firing frequency of 100Hz gives about 7200s run-time per
        simulation.
      \item Panel B is model PM10
      \item Panel C: excitation 10.4Hz and inhibition 0.5Hz,
        excitation 23.5Hz and inhibition 1Hz, excitation 37Hz and
        inhibition 1.5Hz, excitation 50Hz and inhibition 2Hz.  Each
        simulation of 800s has a run-time of 800 * 68s run-time =
        54400s (15hours).  Total of 4 simulations.
      \end{itemize}
%    \item Figure 10: inhibition always 1Hz, excitation (1) 21Hz at 1nS,
%      (2) 30Hz at 0.7nS, (3) 60Hz at 0.35nS, (4) 210Hz at 0.1nS.
%      Simulation runs until 1600 events have been collected.
    \end{itemize}
  \item Total of 39s + 31s + 35s + 56s + 7100s + 763200s + 217600s =
    988061s (274.46 hours) total run-time.


  \end{itemize}

\item From~\cite{schutter94:_simul_purkin}

  \begin{itemize}
  \item Comparison of base model parameters to model
    of~\cite{E:1994hc}.
  \item Application of validation procedures
    of~\cite{De-Schutter-E:1994vn}.
  \item Checks on changed parameters
    \begin{itemize}
    \item Passive soma.
    \end{itemize}
  \item Selected figures:
    \begin{itemize}
    \item Figure 1
      \begin{itemize}
      \item Twice the same set of simulations.
      \item 9 different stimulation protocols: unitary synaptic
        stimulation applied to 8 different branchlets individually and
        to the 8 same branches simultaneously.  Simultaneously
        asynchronous excitation of 28Hz and inhibition of 1Hz.
      \item Passive model, 9 simultaneously of 50ms, each 1.5s
        run-time.
      \item Active model, 9 simulations of 50ms, each 7s run-time.
      \end{itemize}
    \item Figure 2: 10 data points, averaged over 40 simulation runs
      (events), 4 different model variations.  10 x 40 x 4
      simulations, each 7s run-time.  11200s total run-time (3 hours).
%    \item Figure 3: Detail of two selected simulations.
%    \item Figure 4: one selected simulation, false color images of the
%      dendrite.
    \item Figure 5: 200 simulations, distributed, proximal and distal
      stimulation.  600 simulations total, each 7s run-time.  Total of
      4200s run-time.
    \end{itemize}
  \item Total of 76.5s + 11200s + 4200s = 15476.5s (4.3 hours) total
    run-time.
  \end{itemize}

\item From~\cite{bower09:_compar_evolut_comput_analy_cereb}
  \begin{itemize}
  \item static dendritic voltage clamp
    \begin{itemize}
    \item fish1: simulation construction time takes about 2s, a
      simulation of 0.5s takes about 5.5s run-time for each of the
      2187 simulations.
    \item gp\_pc1: simulation construction time takes about 1s, a
      simulation of 0.5s takes about 4s run-time for each of the 2066
      simulations.
    \item gp\_pc2: simulation construction time takes about 1s, a
      simulation of 0.5s takes about 3s run-time for each of the 1520
      simulations.
    \item gp\_pc3: simulation construction time takes about 1s, a
      simulation of 0.5s takes about 3s run-time for each of the 1381
      simulations.
    \item rat\_pc1: simulation construction time takes about 2.5s, a
      simulation of 0.5s takes about 8s run-time for each of the 3885
      simulations.
    \item rat\_pc2: simulation construction time takes about 1.5s, a
      simulation of 0.5s takes about 5s run-time for each of the 2331
      simulations.
    \item rat\_pc3: simulation construction time takes about 1.5s, a
      simulation of 0.5s takes about 5.5s run-time for each of the
      2671 simulations.
    \item turtle1: simulation construction time takes about 1.5s, a
      simulation of 0.5s takes about 6s run-time for each of the 2554
      simulations.
    \item turtle2: simulation construction time takes about 3.5s, a
      simulation 0.5s takes about 11s run-time for each of the 4507
      simulations.
    \item turtle3: simulation construction time takes about 1.5s, a
      simulation of 0.5s takes about 4.5s run-time for each of the
      1991 simulations.
    \end{itemize}
  \item static somatic voltage clamp
    \begin{itemize}
    \item fish1: simulation construction time takes about 2s, a
      simulation of 0.5s takes about 5.5s run-time.
    \item gp\_pc1: simulation construction time takes about 1s, a
      simulation of 0.5s takes about 4s run-time.
    \item gp\_pc2: simulation construction time takes about 1s, a
      simulation of 0.5s takes about 3s run-time.
    \item gp\_pc3: simulation construction time takes about 1s, a
      simulation of 0.5s takes about 3s run-time.
    \item rat\_pc1: simulation construction time takes about 2.5s, a
      simulation of 0.5s takes about 8s run-time.
    \item rat\_pc2: simulation construction time takes about 1.5s, a
      simulation of 0.5s takes about 5s run-time.
    \item rat\_pc3: simulation construction time takes about 1.5s, a
      simulation of 0.5s takes about 5.5s run-time.
    \item turtle1: simulation construction time takes about 1.5s, a
      simulation of 0.5s takes about 6s run-time.
    \item turtle2: simulation construction time takes about 3.5s, a
      simulation 0.5s takes about 11s run-time.
    \item turtle3: simulation construction time takes about 1.5s, a
      simulation of 0.5s takes about 4.5s run-time.
    \end{itemize}
  \item dynamic somatic voltage clamp: CPU load is the same as for
    static somatic voltage clamp.
  \item unitary synaptic stimulation: CPU load is the same as for
    static dendritic voltage clamp.
  \end{itemize}
\item From~\cite{coop09:_using_purkin}

\end{itemize}


\bibliographystyle{plain}
\bibliography{../tex/bib/g3-refs.bib}

\end{document}


