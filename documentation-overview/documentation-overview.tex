\documentclass[12pt]{article}
\usepackage[dvips]{epsfig}
\usepackage{color}
%e.g.  \textcolor{red,green,blue}{text}
\usepackage{url}
\usepackage[colorlinks=true]{hyperref}

\begin{document}

\section*{GENESIS: Documentation}

{\bf Related Documentation:}
% start: userdocs-tag-replace-items related-do-nothing
% end: userdocs-tag-replace-items related-do-nothing

\section*{Documentation Overview}

There are currently seven levels of GENESIS documentation which can be
accessed from the following links.  These levels range from
documentation required by the user to technical detail of a specific
implementation.

Each level can include specifics about compatibility and portability.
Because backward compatibility is defined as a user requirement, the
documentation for backward compatibility falls under the first two
levels.

\begin{itemize}

\item[]\href{../contents-level1/contents-level1.tex}{\bf \underline{Level 1: Introductory Material}}\\
Provides a comprehensive
  introduction to the use of the GENESIS software system.
  Documentation includes an introduction to Unix and the Linux
  Graphical Desktop, fundamentals of Computational Neuroscience, and
  basic tutorials on the concepts and use of GENESIS. Documentation at
  this level also guides the user in the choice of documentation
  necessary to implement research and educational models and
  simulations.  Tutorials at this level are short and contain links to
  the user documentation required for specific types of projects.

\item[]\href{../contents-level2/contents-level2.tex}{\bf \underline{Level 2: User Guides and Documentation}}\\
 In depth
  documentation about specific functions of GENESIS in the context of
  advanced research projects.

\item[]\href{http://www.neurospaces.org/tests-menu.html}{\bf \underline{Level 3: Automated Use Cases}}\\
  Common use cases and examples
  of workflows. Provides a bridge between the functions visible to a user and
  the way those functions are implemented in the code.  This level of documentation also guides technical developers through the code and defines a
  framework for communication between users and developers.
  Currently, this level of documentation is automatically generated
  from test cases.
  
The following GENESIS software components have output defined for regression testing. Most of these tests are defined by use cases.
\begin{itemize}
\item[]\href{../tests-gshell/tests-gshell.tex}{\bf G-Shell} (Interactive Command Line Shell)
\item[]\href{../tests-experiment/tests-experiment.tex}{\bf Experiment} (Experimental Protocols)
\item[]\href{../tests-heccer/tests-heccer.tex}{\bf Heccer} (Compartmental Solver)
\item[]\href{../tests-exchange/tests-exchange.tex}{\bf Exchange} (NeuroML and NineML Interchange)
\item[]\href{../tests-model-container/tests-model-container.tex}{\bf  Model\,Container}
\item[]\href{../tests-g2-backward-compatibility/tests-g2-backward-compatibility.tex}{\bf NS-SLI} (GENESIS 2 Backward Compatibility SLI)
\item[]\href{../tests-ssp/tests-ssp.tex}{\bf SSP} {(Simple\,Scheduler\,in\,Perl)}
\item[]\href{../tests-studio/tests-studio.tex}{\bf Studio} (Visual Model Explorer)
\item[]\href{http://neurospaces.sourceforge.net/neurospaces_project/userdocs/tests/html/index.html}{\bf Documentation\,System}
\end{itemize}
  
\item[]\href{../contents-level4/contents-level4.tex}{\bf Level 4: Technical Guide Specification}\\
  Technical outline and specification of the use of all GENESIS
  functions, commands and components.

  The tools related to software development such as the Neurospaces
  installer, can be documented via the
  \href{http://code.google.com/p/neurospaces/wiki/Index}{Neurospaces
    wiki}.

\item[]\href{../contents-level5/contents-level5.tex}{\bf Level 5: Algorithm Documentation} \\
  This type of documentation contains outlines of the algorithms used
  in an implementation, their theoretical analysis and behavior, their
  edge cases, and an outline of the expected application programmers
  interface (API).  Links to various publications and external
  documents are provided for most of the algorithms used.

  At a practical level, this documentation also includes the
  specification of basic unit tests.  At present, there are no plans
  to implement these tests due to the maintenance cost for the
  developer.

\item[]\href{http://www.neurospaces.org/doxygen-menu.html}{\bf Level 6: Algorithm API Documentation}
  This documentation includes source code documentation typically
  generated by \href{http://www.stack.nl/~dimitri/doxygen/}{Doxygen}
  or a similar documentation extraction system.  Documentation is
  focused on how to interface to the specific implementations of
  low-level algorithms. Doxygen also allows inclusion of
  implementation documentation.

\item[]\href{http://www.neurospaces.org/cxref-menu.html}{\bf Level 7: Inline Source Code Documentation}
  Typically this documentation is provided by inline source code
  comments with a focus on how a function or algorithm is implemented.
  It provides extensive cross referencing between implementation and
  the API of the same and other functions.
  % is an important requirement for the readability and quality of this documentation.
  Both the Doxygen system and \href {http://www.xref.sk/xrefactory/main.html}{Xrefactory} provide this
  functionality.
  
\end{itemize}

\end{document}