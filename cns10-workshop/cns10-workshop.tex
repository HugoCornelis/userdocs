\documentclass[12pt]{article}
\usepackage{verbatim}
\usepackage[dvips]{epsfig}
\usepackage{color}
\usepackage{url}
\usepackage[colorlinks=true]{hyperref}

\begin{document}

%\section*{GENESIS: Documentation}

%{\bf Related Documentation:}
% start: userdocs-tag-replace-items related-do-nothing
% end: userdocs-tag-replace-items related-do-nothing

\section*{Using models to collaborate, communicate, and publish:  An introduction to GENESIS 3.0 and the future of computational neurobiology.}

The GENESIS Development Team announces a one day workshop to introduce the new GENESIS 3.0 (G-3) simulator.

The G-3 simulator is a modular reimplementation of the GEneral NEural SImulation System software platform and provides both substantial functional enhancement over earlier versions of GENESIS and a fundamentally new ``modern'' architecture for modeling software. Implemented software components include realistic modeling solvers, a storage system for models, a flexible run-time scheduler, modules to implement various experimental designs, and a GUI interface for users that supports model-based communication including publication. 

G-3 has been designed around a simple and comprehensive user workflow. We will use this  during the workshop as a framework to show how G-3 will impact scientific communication and its attribution, model lineage, and model comparison. An initial version of the G-3 GUI will be introduced to demonstrate the relationships between the user workflow and the different functions of the simulator. During the workshop we will show how to install both User and Developer versions of G-3 and introduce the G-3 Documentation System. We will demonstrate how to successfully run G-2 scripts in the new software, how to convert existing G-2 models to G-3, and how to build a new model in G-3 by importing components from model databases. Model queries will be used to design experimental protocols that can be applied to the model during simulation. Simulation results will then be inspected using a project browser intended to also facilitate scientific communication and collaboration. G-3 will be demonstrated in the context of two recent research projects, one involving a comparative study of the cerebellar Purkinje cell across multiple species.

\paragraph*{Presenters:}
Hugo Cornelis, Allan D. Coop, Mando Rodriguez, David Beeman, and James M. Bower. \\

\noindent {\bf Additional information:} \\
{\bf G-3:} {\scriptsize \href{http://www.genesis-sim.org/userdocs/documentation-homepage/documentation-homepage.html}{\bf http://www.genesis-sim.org/userdocs/documentation-homepage/documentation-homepage.html}} \\
{\bf Workshop:} {\scriptsize \href{http://www.genesis-sim.org/userdocs/cns10-workshop/cns10-workshop.html}{\bf http://www.genesis-sim.org/userdocs/cns10-workshop/cns10-workshop.html}}

\end{document}
