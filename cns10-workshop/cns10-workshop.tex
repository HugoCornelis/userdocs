\documentclass[12pt]{article}
\usepackage{verbatim}
\usepackage[dvips]{epsfig}
\usepackage{color}
\usepackage{url}
\usepackage[colorlinks=true]{hyperref}

\begin{document}

\section*{GENESIS: Documentation}

{\bf Related Documentation:}
% start: userdocs-tag-replace-items related-do-nothing
% end: userdocs-tag-replace-items related-do-nothing

\section*{CNS*10 GENESIS 3.0 WORKSHOP}

The GENESIS Development Team announces a one day
workshop to introduce the new GENESIS 3.0 (G-3) simulator.

The G-3 simulator is a modular reimplementation of the GEneral NEural SImulation System
simulator software platform and provides substantial functional enhancement over earlier versions of GENESIS.
Implemented software modules include a fast single neuron solver, a
storage system for neuronal models, a flexible run-time scheduler,
modules to implement various experimental designs, and a GUI interface for users that supports model publication.

G-3 has been designed around a simple and comprehensive user
workflow and provides explicit support for scientific communication and its attribution, model lineage, and model comparison.
During the workshop we will introduce the G-3 Documentation System and show how to install both User and Developer versions of G-3, how to successfully run G-2
scripts in the new software, and how to convert them to G-3
models.  We will then show how to build a model with G-3 by
importing components from a database.  Model queries will be used to design
experimental protocols that can be applied to the model during
simulation.  An initial version of the G-3 GUI will be used to
demonstrate the relationships between the user workflow and the
different functions of the simulator.  Simulation results will be
inspected using a project browser that also facilitates scientific
communication.  All simulations and results are based on two recent
research projects that were undertaken with G-3 software.

More information about the GENESIS 3 workshop will be made available
during the CNS meeting.

\end{document}
