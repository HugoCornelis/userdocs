\documentclass[12pt]{article}
\usepackage{verbatim}
\usepackage[dvips]{epsfig}
\usepackage{color}
\usepackage{url}
\usepackage[colorlinks=true]{hyperref}

\begin{document}

\section*{GENESIS: Documentation}

{\bf Related Documentation:}
% start: userdocs-tag-replace-items related-do-nothing
% end: userdocs-tag-replace-items related-do-nothing

\section*{CNS*10 GENESIS 3.0 WORKSHOP}

The GENESIS Development Team announces a one day workshop to introduce the new GENESIS 3.0 (G-3) simulator.

The G-3 simulator is a modular reimplementation of the GEneral NEural SImulation System simulator software platform and provides substantial functional enhancement over earlier versions of GENESIS and representing a fundamentally new "modern" architecture for modeling software. Implemented software modules include realistic modeling solvers, a storage system for models, a flexible run-time scheduler, modules to implement various experimental designs, and a GUI interface for users that supports model-based communication including publication.

G-3 has been designed around a simple and comprehensive user workflow and provides explicit support for scientific communication and its attribution, model lineage, and model comparison. During the workshop we will introduce the G-3 Documentation System and show how to install both User and Developer versions of G-3, how to successfully run G-2 scripts in the new software, and how to convert existing G-2 models to G-3. We will then show how to build a new model in G-3 by importing components from model databases. Model queries will be used to design experimental protocols that can be applied to the model during simulation. An initial version of the G-3 GUI will be introduced to demonstrate the relationships between the user workflow and the different functions of the simulator. Simulation results will be inspected using a project browser intended to also facilitate scientific communication and collaboration. G-3 will be demonstrated in the context of two recent research projects, one involving a comparative study of the cerebellar Purkinje cell across multiple species.

Additional information regarding G-3 as well as the workshop can be obtained from \href{http://www.genesis-sim.org/userdocs/cns10-workshop/cns10-workshop.html}{\bf http://www.genesis-sim.org/userdocs/cns10-workshop/cns10-workshop.html}.

\end{document}
