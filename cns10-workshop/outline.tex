\documentclass[12pt]{article}
\usepackage{verbatim}
\usepackage[dvips]{epsfig}
\usepackage{color}
\usepackage{url}
\usepackage[colorlinks=true]{hyperref}

\begin{document}

%\section*{GENESIS: Documentation}

%{\bf Related Documentation:}
% start: userdocs-tag-replace-items related-do-nothing
% end: userdocs-tag-replace-items related-do-nothing

\section{General Introduction}

\subsection{Why Modeling?}

What is so special about modeling and why do we think we need
computational modeling in neuroscience?  What is the relationship with
experimental neuroscience?


\subsection{History of GENESIS}

\begin{enumerate}
\item G-2 architecture.
\item Models and software extensions (technologies).
%  \begin{enumerate}
%  \item Olfactory model 1: Matt Wilson
%  \item Olfactory model 2: Mike Vanier
%  \item Purkinje Model.
%  \end{enumerate}
\item Purkinje cell model lineage.
\item How do we advance the formal records of neuroscience?  What is
  required for the next generation simulator?  What impact will this
  have on scientific communication and publication?
\end{enumerate}


\subsection{Simulator Archetypes}

\begin{enumerate}
\item Technical Axis
  \begin{itemize}
  \item Open vs Closed \& Extensibility: how easy is to contribute to
    the simulator code?
  \item Future orientation: who easy is it to interface with future
    technologies?
  \end{itemize}

\item Community Axis
  \begin{itemize}
  \item Level of support for Biological modeling as opposed to purely
    mathematical modeling?
  \item Workflow Orientation: is model development well structured ?
    how easy is it to disassemble model descriptions?  how clear is
    the separation between the model of the biology and periphery
    code?
  \item Model Sharing: how open-minded / protective is the
    user-community?
  \end{itemize}
\end{enumerate}

%\subsection{Simulators}

%\begin{enumerate}
%\item Description of Simulator Archetypes.
%  \begin{itemize}
%  \item Open vs Closed.
%    \begin{itemize}
%    \item G-2 has architecture description, closed by gradient of
%      proficiency, few able to add new solvers.
%    \item Neuron closed: no architecture description, only limited
%      developer's documentation. % ask Dave
%    \item Nest?
%    \item Other: mostly single lab artefacts, very limited
%      documentation available.
%    \end{itemize}
%  \item Workflow orientation.
%    \begin{itemize}
%    \item Interactive, scripted, batch oriented.
%    \item Clearly suggested workflow vs workflow independent.
%    \end{itemize}
%  \item Biological conceptualization vs mathematical
%    conceptualization.
%  \item Extensibility.
%    \begin{itemize}
%    \item G-2 shows limited extensibility: two types of extensions:
%      function extensions and object extensions.
%    \item Neuron extension by an inner circle under supervision of
%      Michael Hines.
%    \end{itemize}
%  \item Model Sharing and Model Documentation.
%    \begin{itemize}
%    \item At the mathematical / technical level: all simulators.
%    \item With tutorials / introductory materials: only few tutorials
%      available.
%    \item With publications: few simulators with many publications.
%    \end{itemize}
%  \item Digital Universe Integration and The Next Generation Student.
%    \begin{itemize}
%    \item Level of interaction with community tools?
%    \item Level of embeddedness, eg. simulator behind a web server to
%      enable cluster computing, connections with community tools,
%      youtube, blogging.
%    \end{itemize}
%  \end{itemize}
%\end{enumerate}

\section{Basic Concepts}

\begin{enumerate}
\item Using \href{../workflow-user/workflow-user.tex}{the G-3 User
    Workflow} for
  \href{../tutorial1/tutorial1.tex}{single-compartment} and
  \href{../tutorial2/tutorial2.tex}{multi-compartment} modeling and
  simulation.
\item \href{../tutorial3/tutorial3.tex}{Advanced use}:
  \begin{enumerate}
  \item Querying a model.
  \item Setting up the simulation.
  \item Running batch simulations.
  \end{enumerate}
\end{enumerate}

\section{What is currently in G-3?}

See release notes.


\section{Setting up Collaborations}

\subsection{An Architecture for a Heterogeneous Community}

See \href{../genesis-overview/genesis-overview.tex}{CBI Architecture}

\begin{enumerate}
\item New Software Components as
  \href{../genesis-addto-component-developerpackage/genesis-addto-component-developerpackage.tex}{CBI
    Architecture Extensions} allow third parties to develop new
  components independently.
\item Coarsely grained extension:
  \href{../genesis-add-feature-ssp/genesis-add-feature-ssp.tex}{SSP
    plugins} and
  \href{../genesis-add-object-solver/genesis-add-object-solver.tex}{simulation
    objects}.
\item Finely grained extensions:
  \begin{enumerate}
  \item
    \href{../genesis-extend-model-container/genesis-extend-model-container.tex}{Extending
      the model-container} (for more details see
    \href{../genesis-extend-model-container-detail/genesis-extend-model-container-detail.tex}{here})
  \item
    \href{../genesis-add-object-solver/genesis-add-object-solver.tex}{Extending the compartmental solver}
  \item \href{../genesis-add-object-solver/genesis-add-object-solver.tex}{Extending the G-3 Shell}
  \end{enumerate}
\end{enumerate}



\subsection{The Documentation and Publishing System}

\begin{itemize}
\item Levels of Documentation
\item Documentation separation for separate software systems /
  components.  Third party software and documentation can be
  maintainted independently, and integrated into the G-3 software and
  documentation system by correct configuration of G-3.
\item Documentation Integration
  \begin{itemize}
  \item Dynamic Integration
  \item External documents are declared using document descriptors
  \end{itemize}
\end{itemize}


\section{The Graphical User Interface}


\paragraph*{Presenters:}
Hugo Cornelis, Allan D. Coop, Mando Rodriguez, David Beeman, and James M. Bower. \\

\noindent {\bf Additional information:} \\
{\bf G-3:} {\scriptsize \href{http://www.genesis-sim.org/userdocs/documentation-homepage/documentation-homepage.html}{\bf http://www.genesis-sim.org/userdocs/documentation-homepage/documentation-homepage.html}} \\
%{\bf Workshop:} {\scriptsize \href{http://www.genesis-sim.org/userdocs/cns10-workshop/cns10-workshop.html}{\bf http://www.genesis-sim.org/userdocs/cns10-workshop/cns10-workshop.html}}

\end{document}

%%% Local Variables: 
%%% mode: latex
%%% TeX-master: t
%%% End: 
