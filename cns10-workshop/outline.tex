\documentclass[12pt]{article}
\usepackage{verbatim}
\usepackage[dvips]{epsfig}
\usepackage{color}
\usepackage{url}
\usepackage[colorlinks=true]{hyperref}

\begin{document}

%\section*{GENESIS: Documentation}

%{\bf Related Documentation:}
% start: userdocs-tag-replace-items related-do-nothing
% end: userdocs-tag-replace-items related-do-nothing

\section{General Introduction}

\begin{enumerate}
\item Description of Simulator Archetypes.
  \begin{itemize}
  \item Open vs Closed.
    \begin{itemize}
    \item G-2 has architecture description, closed by gradient of
      proficiency, no-one able to add new solvers.
    \item Neuron closed: no architecture description, no developer's
      documentation
    \end{itemize}
  \item Workflow orientation.
    \begin{itemize}
    \item Everything scripted, interactive.
    \item Batch oriented.
    \item Clearly suggested workflow vs workflow independent.
    \item Workflows and GUI support?
    \end{itemize}
  \item Supports biological concepts vs purely mathematical.
  \item Extensibility.
    \begin{itemize}
    \item G-2 shows limited extensibility: two types of extensions:
      function extensions and object extensions.
    \item Neuron extension by an inner circle under supervision of
      Michael Hines.
    \end{itemize}
  \item Documentation and model sharing.
    \begin{itemize}
    \item At the mathematical level only?
    \item With tutorials?
    \item With publications?
    \end{itemize}
  \item Digital Universe Integration.
    \begin{itemize}
    \item Level of interaction with community tools?
    \item Level of embeddedness, eg. simulator as web server.
    \end{itemize}
  \end{itemize}
\end{enumerate}

\section{Basic Concepts}

\begin{enumerate}
\item \href{../workflow-user/workflow-user.tex}: The G-3 User Workflow.
\item \href{../tutorial1/tutorial1.tex}{Tutorial 1}: Basic use following to the user workflow (1).
\item \href{../tutorial2/tutorial2.tex}{Tutorial 2}: Basic use following to the user workflow (2).
\item \href{../tutorial3/tutorial3.tex}{Tutorial 3}: Advanced use,
  Mutual Information Processing of a Purkinje cell.
  \begin{enumerate}
  \item Querying a model.
  \item Setting up the simulation.
  \item Running batch simulations.
  \end{enumerate}
\end{enumerate}

\section{Setting up Collaborations: Community Based Architecture}

See \href{../genesis-overview/genesis-overview.tex}{CBI Architecture}

\begin{enumerate}
\item New Software Components:
  \href{../genesis-addto-component-developerpackage/genesis-addto-component-developerpackage.tex}{CBI
    Architecture Extensions}
\item Coarsely grained extension:
  \href{../genesis-add-feature-ssp/genesis-add-feature-ssp.tex}{CBI
    Architecture Extensions} and
  \href{../genesis-add-object-solver/genesis-add-object-solver.tex}{simulation
    objects}.
\item Finely grained extensions:
  \begin{enumerate}
  \item
    \href{../genesis-extend-model-container/genesis-extend-model-container.tex}{1}
    and
    \href{../genesis-extend-model-container-detail/genesis-extend-model-container-detail.tex}{2}
    and
    \href{../genesis-extend-functionality/genesis-extend-functionality.tex}{3}
    and extend heccer
  \end{enumerate}
\end{enumerate}



\section{Setting up Collaborations: The Documentation and Publishing System}

Separate documentation for separate software components.

Separate documentation for separate parts of the project.

Inclusion of external documents via descriptors.



\section{The Graphical User Interface}


\paragraph*{Presenters:}
Hugo Cornelis, Allan D. Coop, Mando Rodriguez, David Beeman, and James M. Bower. \\

\noindent {\bf Additional information:} \\
{\bf G-3:} {\scriptsize \href{http://www.genesis-sim.org/userdocs/documentation-homepage/documentation-homepage.html}{\bf http://www.genesis-sim.org/userdocs/documentation-homepage/documentation-homepage.html}} \\
{\bf Workshop:} {\scriptsize \href{http://www.genesis-sim.org/userdocs/cns10-workshop/cns10-workshop.html}{\bf http://www.genesis-sim.org/userdocs/cns10-workshop/cns10-workshop.html}}

\end{document}

%%% Local Variables: 
%%% mode: latex
%%% TeX-master: t
%%% End: 
