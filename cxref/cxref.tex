\documentclass[12pt]{article}
\usepackage{verbatim}
\usepackage[dvips]{epsfig}
\usepackage{color}
\usepackage{url}
\usepackage[colorlinks=true]{hyperref}

\begin{document}

\section*{GENESIS: Documentation}

\section*{Xrefactory}

\href{http://www.xref.sk/xrefactory/main.html}{Xrefactory} is a development tool used used with {\it emacs} and {\it xemacs} for source browsing and refactoring of C and C++ code. It can be used to browse and restructure original source code and convert the source code to browsable html pages.

The procedure for generating browsable html source code is now outlined.

\subsection*{xrefrc}

A `recipe' file is typically used to set up a project in Xrefactory. It consists of formatted blocks relevant to a particular project. Each block contains a set of xref flags with their complimentary arguments. 

\begin{verbatim}
[model-container]
  //  input files and directories (processed recursively)
  /local_home/local_home/hugo/neurospaces_project/model-container/source/snapshots/0
  //  directory where tag files are stored
  -refs /local_home/local_home/hugo/.xrefs/neurospaces/source/snapshots/0/model-container
  //  split tag files using first letter
  -refalphahash
  //  include directories
  -I ~/neurospaces_project/model-container/source/snapshots/0
  -I ~/neurospaces_project/model-container/source/snapshots/0/modules/symbol
  -I ~/neurospaces_project/model-container/source/snapshots/0/modules/event
  -I /usr/lib/perl5/5.8.5/i386-linux-thread-multi/CORE
  -I /usr/lib/perl/5.8.7/CORE
  -I /
  //  setting for Emacs compile and run
  -set compilefile "cc %s"
  -set compiledir "cc *.c"
  -set compileproject "
        cd /local_home/local_home/hugo/neurospaces_project/model-container/source/snapshots/0
        make
        "
  -set run1 "a.out"
  -set run5 ""  // an empty run; C-F8 will only compile
  //  set default to run1
  -set run ${run1}
  //  HTML configuration
  -htmlroot=/local_home/local_home/hugo/HTML
  -htmlgxlist -htmllxlist -htmldirectx -htmllinenums
  -htmltab=8 -htmllinenumcolor=000000
  //  pre-processor macros and passes specification
  -DPRE_PROTO_TRAVERSAL
  -DUSE_READLINE
  // found in smart browsing toggle help
  -storelocalxrefs
  -prune neurospacesread-*
\end{verbatim}  
  
\subsection*{Generating HTML}

Here is an example of Xrefactory being run from the {\bf NS-SLI} command line showing all the required flags:

\begin{verbatim}
xref -xrefactory-II -o ~/ns-sliHTML.log -encoding=european -html -htmlgxlist -htmllxlist -htmlfunseparate -htmllinenums -errors -xrefrc ~/.xrefrc -p ns-sli -user 1
\end{verbatim}

The {\tt -html} tags are required to insure that source code referenced more than once does not produce multiple output pages. The {\tt -xrefrc} file references the recipe file set up for a project (as illustrated above).

The {\bf -p} flag allows you to select a project block from the {\it xrefrc} file. The referenced file contains three projects, so it must be explicitly told which project to build the documents for.

\end{document}
