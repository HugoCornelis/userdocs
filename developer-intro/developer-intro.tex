\documentclass[12pt]{article}
\usepackage[dvips]{epsfig}
\usepackage{color}
\usepackage{url}
\usepackage[colorlinks=true]{hyperref}

\begin{document}

\section*{GENESIS: Documentation}

{\bf Related Documentation:}
% start: userdocs-tag-replace-items related-do-nothing
% end: userdocs-tag-replace-items related-do-nothing

\section*{Introduction to GENESIS for Developers}

The documentation linked to here is for GENESIS users who want to contribute to the development of extended functionality of the GENESIS simulation platform, rather than for those users who want to do scientific simulations. (Although a single individual may want to do both.) Developer documentation provides links to more technical documentation supporting software development. It is assumed that you are familiar with the monotone version control software used by GENESIS (for more information see \href{../version-control/version-control.tex}{\bf Version\,Control}). Links to the more general user documentation can be found at the \href{http://www.genesis-sim.org/userdocs/contents-level1/contents-level1.html}{\bf Level\,1\,Documentation\,Contents\,Index}. 
%The documentation linked to here is for the use by people who are involved in the development and extension of GENESIS functionality.

The federated development of GENESIS software components proceeds via a software buildbot machine that houses a \href{http://monotone.ca/}{monotone} server that is connectible over the internet. This allows source code changes to be properly merged using the functions available in\,{\it monotone}, regardless of the location where the changes have been made (at home, in your office, on the airplane). The build mechanism will always build the most up to date version of the code.

\section*{Becoming a GENESIS Developer}

In principle anyone can become a GENESIS developer, all that is required is a desire to focus on tool and documentation development rather than model building and simulation. There are at least three ways to be a GENESIS  developer:

\subsection*{Member of the GENESIS Core Development Team}

Core development of the GENESIS software platform is undertaken by the members of the GENESIS Core Development Team.

\subsection*{Member of the GENESIS Developers Federation}

To become a GENESIS developer send a request to \href{mailto:genesis@genesis-sim.org}{\bf genesis@genesis-sim.org} with the subject ``New Developer Request''. You will be registered as a GENESIS developer and receive a key that will allow you to pull and push documentation and code to the GENESIS repositories.

To learn more about the GENESIS repositories and version control see \href{../version-control/version-control.tex}{\bf Version\,Control}.
	
To learn more about installation on your local machine as a GENESIS developer see \href{../installation-developer/installation-developer.tex}{\bf Developer\,Installation}.

\subsection*{\bf Independent Developer}

GENESIS is structured in such a way that it is possible to download the code from \href{http://sourceforge.net/projects/neurospaces/files/}{\bf http://sourceforge.net/projects/neurospaces/files/}. You can then proceed entirely independently with your own code and documentation development. To do this see the \href{../installation-independent/installation-independent.tex}{\bf Independent\,Installation} documentation. It describes how to set up your own choice of version control and associated file repositories independently of the \href{http://www.genesis-sim.org/}{\bf GENESIS website}.

%The \href{../developer-package/developer-package.tex}{\it DeveloperPackage} has built in support to create and maintain this directory layout.

% structure when
%working in developer mode (using its {\tt --developer} option).
%There is a default configuration available in the {\it ConfiguratorPackage} (UNDER CONSTRUCTION). 

%\section*{Build Procedure}


\end{document}
