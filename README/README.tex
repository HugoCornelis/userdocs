\documentclass[12pt]{article}
%\usepackage[utf8]{inputenc}
%\usepackage[T1]{fontenc}
\usepackage[dvips]{epsfig}
%\usepackage{graphicx}
%\usepackage{lineno}
\usepackage{url}
\usepackage[colorlinks=true]{hyperref}

\begin{document}

\section*{GENESIS: Documentation}

\subsection*{README}


This directory structure contains GENESIS documentation source material. Each uniquely named folder/directory contains a single file of the level 1-7 documents in the GENESIS documentation database. Each directory also contains a subdirectory ``{\tt figs}'' that holds any figures the given source document may contain.

New tags may be created just by adding them to the {\tt descriptor.yml} file in a document directory. 

Pre-existing tags include:
\begin{verbatim}
  tags:
    - genesis3
    - level1
    - level2
    - level3
    - level4
    - level5
    - level6
    - level7
    - meta
    - introduction
    - tutorial
    - document
    - draft
    - published
\end{verbatim}
Each directory/file is tagged on the basis of content:

\begin{enumerate}

\item The GENESIS version that the content of the documentation system refers to, e.g.
\begin{enumerate}
\item  {\tt -genesis3}
\end{enumerate}

\item The level of GENESIS documentation a directory contains. This includes:

	\begin{enumerate}

	\item \href{../contents-level1/contents-level1.pdf}{{\tt -level1}} (tutorials)

	\item \href{../contents-level2/contents-level2.pdf}{{\tt -level2}} (userdocs)

	\item \href{../contents-level3/contents-level3.pdf}{{\tt -level3}} (autogenerated)

	\item \href{../contents-level4/contents-level4.pdf}{{\tt -level4}} (technical)

	\item \href{../contents-level5/contents-level5.pdf}{{\tt -level5}} (algorithm)

	\item \href{../contents-level6/contents-level6.pdf}{{\tt -level6}} (api)

	\item \href{../contents-level7/contents-level7.pdf}{{\tt -level7}} (inlinesrcode)

	\end{enumerate}

\item Flag whether a document is a {\tt -draft} or is {\tt -published}. For a document to be part of the documentation version control system, a given documentation folder must contain a {\tt descriptor.yml} file that contains one of either the {\tt -draft} or {\tt -published} tags.

\end{enumerate}

The {\tt contents-level1} \ldots {\tt contents-level7} directories each contain a list of the documentation to be found at the given level. These content indexes also provide direct links to all documents tagged for the given level of documentation.

\subsection*{Make a New Document}

The directory {\tt NewDocument} can be found in the parent directory of the directory that contains this document. It contains the files required to generate a new document within the GENESIS documentation system.

\begin{enumerate}

\item To make a new document, copy and paste the {\tt NewDocument} directory into the directory where you found it and rename it with an unique identifier. The directory and the document source file it contains should be given the same name, e.g. the directory containing this document is called {\tt README} and the documentation file it contains is a \LaTeX\, file (this file) called {\tt README.tex}.

\item A descriptor file (descriptor.yml) contains tags that identify and define the status of the documentation file. The default contents of this file include:
\begin{verbatim}
---
description: Briefly describe file contents here. 
tags:
  - genesis3
  - level2
  - draft
\end{verbatim}

\begin{itemize}

\item {\tt ---} The three dashes on the first line of the file descriptor.yml are part of the Y Markup Language (YML) syntax for automated software engineering (see \href{http://fdik.org/yml/}{http://fdik.org/yml/}). {\bf Note:} The meaning of the white space/indents in the descriptor.yml file is defined for YML.  


\item {\tt description:} The default file description ``Briefly describe file contents here.'' should be replaced with a short description of the subject of the documentation file. It is recommended that this file descriptor is less than 80 characters on a single line.

\item {\tt tags:} List of any tags used to identify the current document. Default tags include {\tt -genesis3}, {\tt -level2}, and {\tt -draft}. These tags indicate that the document is part of the GENESIS 3 documentation system at the level of User Guides and Documentation (level 2), and that it is an as yet unpublished draft document. For a document to be publishable, i.e. viewable with a browser, this tag must be changed to {\tt -published}.

\end{itemize}

\end{enumerate}

\subsection*{Document Version Tracking and Control}

To learn more about document version tracking and control in the GENESIS Documentation System, see \href{../document-versionctrl/document-versionctrl.pdf}{here}.

\subsection*{Document Publication}

To learn more about document publication in the GENESIS Documentation System, see \href{../document-publication/document-publication.pdf}{here}.

\end{document}
