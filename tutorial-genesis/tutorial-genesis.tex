\documentclass[12pt]{article}
\usepackage{verbatim}
\usepackage[dvips]{epsfig}
\usepackage{color}
\usepackage{url}
\usepackage[colorlinks=true]{hyperref}

\begin{document}

\section*{GENESIS: Documentation}

{\bf Related Documentation:}
% start: userdocs-tag-replace-items related-tutorial
% end: userdocs-tag-replace-items related-tutorial

\section*{Tutorials}

The following tutorials provide ``howto'' documentation and examples for
GENESIS modeling and simulation:

\begin{itemize}

   \item[]\href{../tutorial1/tutorial1.html}
          {\bf Tutorial 1} shows how to use the G-Shell to create, run,
          save, and explore the output of a simple single compartment model neuron.

   \item[]\href{../tutorial2/tutorial2.html}
          {\bf Tutorial 2} shows how to load and simulate predefined model
    neurons that come bundled with the GENESIS distribution in the single
    cell model library. As with the Purkinje cell model used in this
    tutorial, these have been saved in the library in
    \href{../ndf-file-format/ndf-file-format.tex}{\bf NDF} format files.

   \item[]\href{../tutorial3/tutorial3.html}
          {\bf Tutorial 3} covers the process of loading existing GENESIS 2
   simulation scripts into the G-shell, saving the models in NDF files, and
   running them in GENESIS 3. 

   \item[]\href{../tutorial-python-scripting/tutorial-python-scripting.html}
          {\bf Creating GENESIS 3 Simulations with Python}
    is an introduction to creating G-3 simulation
    scripts in Python, using the SSPy component of G-3.

   \item[]\href{../sspy/sspy.tex}{\bf Loading and running simulations with the
 Simple Scheduler in Python (SSPy)}

   \item[]\href{../pclamp/pclamp.tex}{\bf Using the Perfect\,Clamp Module to Implement Voltage Clamp and Current Clamp Protocols}
   \item[]\href{../rtxi-injector-validation/rtxi-injector-validation.tex}{\bf RTXI\,{\it injector}\,Validation}
\end{itemize}
\end{document}
