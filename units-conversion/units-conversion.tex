\documentclass[12pt]{article}
\usepackage{verbatim}
\usepackage[dvips]{epsfig}
\usepackage{color}
\usepackage{url}
\usepackage[colorlinks=true]{hyperref}

\begin{document}
\section*{GENESIS: Documentation}

{\bf Related Documentation:}
% start: userdocs-tag-replace-items related-do-nothing
% end: userdocs-tag-replace-items related-do-nothing
\href{../technical-guide-1/technical-guide-1.html}{\bf Technical Guide 1}

\section*{Conversion From Physiological to SI Units in GENESIS 3}


This documentation describes procedures for converting between
the physiological units often used in papers describing the results of
electrophysiological experiments, and the SI units used in GENESIS 3.
It also explains the relationship between common parameterized descriptions
of channel gate rate functions and corresponding parameters in the
\href{../ndf-file-format/ndf-file-format.tex}{\bf Neurospaces Description
Format} (NDF).

\section{Introduction}

Section 3 of the GENESIS 3 Technical Guide 1
\href{../technical-guide-1/technical-guide-1.html}{\bf Technical Guide 1}
explains the representation of a simple cell model containing
Hodgkin-Huxley channels in the Neurospaces Model Container (NMC).
Equations (15) and (16), restated below, give the forward and backward rate
function for the kinetics of the squid axon potassium channel $n$ gate
activation, as fit to their experimental data by Hodgkin and Huxley.

\begin{equation}
\alpha_n(V) = \frac{0.01 (10-V)}{exp(\frac{10-V}{10}) - 1}
\label{eq1}
\end{equation}

\begin{equation}
\beta_n(V) = 0.125\; exp(-V/80).
\label{eq2}
\end{equation}
 
These are in {\em physiological units}, commonly used in experimental
neuroscience, where voltage is measured in millivolts ($mV$) and time in
milliseconds ($msec$).  

In GENESIS 3, the Model Container uses SI (meter, kilogram, second) units
for stored parameters.  Voltages are expressed in Volts, and currents in
Amperes.  Unfortunately, quantities typical of neurons tend to
have either very large or very small values when expressed in SI units.
For this reason, many people prefer to use physiological units.

Technical Guide 1 shows the {\tt PARAMETERS} block of the NDF represention
of the forward rate of Eq. (\ref{eq1}) as:

\begin{verbatim}
      PARAMETERS
        PARAMETER ( HH_AB_Add_Num = -600.0 ),
        PARAMETER ( HH_AB_Mult = -10000 ),
        PARAMETER ( HH_AB_Factor_Flag = -1.0 ),
        PARAMETER ( HH_AB_Add_Den = -1.0 ),
        PARAMETER ( HH_AB_Offset_E = 60e-3 ),
        PARAMETER ( HH_AB_Div_E = -10.0e-3 ),
      END PARAMETERS
\end{verbatim}

This documentation gives further details of the procedures for converting
between physiological and SI units, and the relationship between rate equations
such as Eq. (\ref{eq1}) and the parameters in the NDF {\tt PARAMETERS}
block above.

\subsection{Some Remarks on Units}

Any inconsistency in the units that are used can result in confusion as
well as incorrect results.  One way to keep confusion to a minimum is to
stick to SI units.  This is the approach taken in GENESIS 3, and with the GENESIS 2
{\em Neurokit\/} program and its associated prototype libraries of channels
and other cell components.  Although we are moving away from GENESIS 2
SLI scripting for new simulations, the GENESIS 2 prototype libraries will
remain an important source of components to be used in GENESIS 3
simulations.

Table~\ref{tab1} presents a summary of
the units used for common quantities ocurring in neural models and
the conversion between physiological and SI units.

\begin{table}[thb]
\label{tab1}
\centering
\begin{tabular}{lll} \hline
{\em Quantity}   & {\em SI units}       & {\em physiological units} \\
\hline
  resistance   & ohm ($\Omega$)  & kilohm ($K\Omega = 10^3 \Omega$) \\
  capacitance  & farad ($F$)     & microfarad ($\mu F = 10^{-6} F$) \\
  voltage      & volt ($V$)      & millivolt ($mV = 10^{-3} V$) \\
  current      & ampere ($A$)    & microampere ($\mu A = 10^{-6} A$) \\
  time         & second ($sec$)  & millisecond ($msec = 10^{-3} sec$) \\
  conductance & siemen ($S = 1/\Omega$) & millisiemen ($mS = 10^{-3} S$) \\
  length       & meter ($m$)     & centimeter ($cm = 10^{-2} m$) \\
\hline
\end{tabular}
\caption{Correspondence between SI and physiological units for common
quantities used in neural modeling.}
\end{table}


\subsubsection{Passive Cell Parameters}

For a compartment of length $l$ and diameter $d$ the passive parameters
in farads and ohms are related to the size-independent {\em specific} membrane
capacitance $C_M$, membrane resistance $R_M$, and axial resistance $R_A$ by

\begin{equation}
C_{m} =  \pi l d C_M
\end{equation}

\begin{equation}
R_{m} = \frac{R_M}{\pi l d}
\end{equation}

\begin{equation}
R_{a}= \frac{4 l R_A}{\pi d^{2}}
\end{equation}
 
From the units given in Table \ref{tab1} one can obtain the units and
conversion factors given in Table \ref{tab2}.

\begin{table}[thb]
\label{tab2}
\centering
\begin{tabular}{llll} \hline
{\em Quantity} & {\em SI units} & {\em physiological units} & {\em SI/physiol} \\
\hline
  $C_M$  & $F/m^2$ &$\mu F/cm^2$  & $0.01$  \\
  $R_M$  & $\Omega \cdot m^2$ & $K \Omega \cdot cm^2$ & $0.1$ \\
  $R_A$  & $\Omega \cdot m$   & $K \Omega \cdot cm$ & $10$   \\
\hline
\end{tabular}
\caption{Correspondence between SI and physiological units for passive
cell parameters.}
\end{table}

\subsection{Converting the Rate Functions}

Equations (\ref{eq1}) and (\ref{eq2}) have another difference from modern
notation, in that the voltages were measured with respect to the resting
potential of the squid axon ($-70~mV$), rather than relative to the
extracellular potential.  GENESIS defines the default extracellular
potential as $0~V$.  GENESIS 2 has traditionally used the global variable
{\em EREST\_ACT} as an offset to give more flexibility when implementing
functions to create a channel.  {\em EREST\_ACT} is the nominal resting
potential of the cell, which is $-0.07~V$ in this case.  By replacing $V$
by $V - EREST_{-}ACT$ in the rate equations, we may use the channel with
any assumed resting potential.  This can be useful, for example, when using
a Hodgkin-Huxley squid axon channel in a mammalian neuron model with a
resting potential of $-60~mV$.  When converting Eqs. (\ref{eq1}) and
(\ref{eq2}) to SI units, the voltages would be converted with

\begin{equation}
V(SI) = 0.001 (V(pysiol) + E_{rest}) - EREST_{-}ACT
\label{eq6}
\end{equation}

where $E_{rest}$ is the resting potential in $mV$, which in this case
was defined to be zero.  In other cases, $E_{rest}$ is likely to be
the the same as {\em EREST\_ACT} when expressed in $mV$,  and would give a
simple scaling of voltage rather than a scaling plus offset.

In addition to converting the voltages from {\em mV\/} to volts, the rate
functions need to be converted from phyiological units ($1/msec$) to the SI
units of $1/sec$.  This means that the rate functions should be scaled by a
factor of 1000.  As a result, Eqs. (\ref{eq1}) and (\ref{eq2}) become

\begin{equation}
\alpha_n(V) = \frac{10.0 (0.08 - V)}{exp(\frac{0.08 - V}{0.01}) - 1}
\label{eq7}
\end{equation}

\begin{equation}
\beta_n(V) = 125.0\; exp(-(V - 0.07)/0.080).
\label{eq8}
\end{equation}
 
From these, one can determine the values to be used in Eq. (12) of
Technical Guide 1.

\subsection{Parameterizing the Rate Constants}
\label{sec4}

This conversion becomes much easier if the rate function can be expressed
in one of the standard parameterized forms that are used in GENESIS.  This
is the case with Eqs. (\ref{eq1}) and (\ref{eq2}), and many other channels
for which someone has already analyzed the experimental data and fitted the
rate functions $\alpha (V)$ and $\beta (V)$ to some equations.  The squid
channels and many of the other channels in the GENESIS channel library and
the {\em neurokit/prototypes\/} directory have been put in such a form.
For other cases, it is necessary to have the function that creates the
channel use a {\em for\/} loop to evaluate the equation and to fill each
gate table.

GENESIS 2 and the backwards compatibility module {\em ns-sli} have a
command, {\em setupalpha} that can be used to create and fill the $A$ and
$B$ tables when a gate has rate equations of the general form

\begin{equation}
y = \frac{A + B V}{C + \exp((V + D) / F))}
\label{eq9}
\end{equation}

Note that this is a generalization of the three basic Hodgkin-Huxley forms
$EXPONENTIAL$, $SIGMOID$, and $LINOID$.  Eqs.~(\ref{eq7}) and (\ref{eq8})
are of this form, so one can set up the tables for the $n$ gate with a
GENESIS 2 or ns-sli command of the form

\begin{verbatim}
        setupalpha chan gate AA AB AC AD AF BA BB BC BD BF
\end{verbatim}

where the parameters {\em AA\/}--{\em AF\/} correspond to those in
Eq.~(\ref{eq9}) when it represents $\alpha(V)$ and the parameters {\em
BA\/}--{\em BF\/} correspond to those for $\beta(V)$.  

The Hodgkin-Huxley squid axon $n$ gate described by Eqs.~(\ref{eq1}) and
(\ref{eq2}), and used as an example in Technical Guide 1, is defined along
with the sodium $m$ and $h$ gates in the GENESIS 2 channel prototype file
{\em neurokit/prototypes/hh\_tchan.g}.  It contains the command

\begin{verbatim}
          setupalpha {chanpath} X {10e3*(0.01 + EREST_ACT)} -10.0e3  \
              -1.0 {-1.0*(0.01 + EREST_ACT)} -0.01 125.0 0.0 0.0  \
              {-1.0*EREST_ACT} 80.0e-3
\end{verbatim}

to set up the $A$ and $B$ tables for the potassium $n$ gate, which is
represented here as $X$ in GENESIS 2 notation.

The parameterized gate description used in the NMC is given by Eq. (12) in
Technical Guide 1

% \begin{equation}
%  A(V), B(V) = \frac{\mathrm{HH\_AB\_Scale} + \mathrm{HH\_AB\_Offset\_M} -
%  \mathrm{HH\_AB\_Mult} * V}
%  { \mathrm{HH\_AB\_Add} + e ^ {\frac{ V +
%  \mathrm{HH\_AB\_Offset\_E} }{ \mathrm{HH\_AB\_Tau} } } }
% \end{equation}

\begin{equation}
  \label{eq10}
  A(V), B(V) =  \frac{ 
 \mathrm{HH_{-}AB_{-}Add_{-}Num} +  \mathrm{HH_{-}AB_{-}Mult} *  V }
   { \mathrm{HH_{-}AB_{-}Add_{-}Den} +   e ^ {\frac { V + 
  \mathrm{HH_{-}AB_{-}Offset_{-}E} }{ \mathrm{HH_{-}AB_{-}Div_{-}E}}}}
\end{equation}

which applies when the constant {\tt HH\_AB\_Factor\_Flag} has its default
value of -1. The constants appearing in Eq. \ref{eq10} are used in the
NDF {\tt PARAMETERS} block, and are related to those in Eq. (\ref{eq9})
by the table

\begin{table}[thb]
\label{tab3}
\centering
\begin{tabular}{lll} \hline
{\em NMC parameter}	 & {\em setupalpha parameter} \\
\hline
{\em NMC parameter}	 & {\em setupalpha parameter} & {\em SI Units} \\
\hline
 HH\_AB\_Add\_Num  & $A$ & $1/\mathrm{sec}$ \\
 HH\_AB\_Mult & $B$ & $1/(\mathrm{sec} \mathrm{Volts})$ \\
 HH\_AB\_Add\_Den   & $C$ & dimensionless \\
 HH\_AB\_Offset\_E & $D$ & $\mathrm{Volts}$ \\
 HH\_AB\_Div\_E & $F$  & $\mathrm{Volts}$ \\
\hline
\end{tabular}
\caption{Correspondence between NMC and {\em setupalpha} parameters.}
\end{table}

\subsubsection{Conversion of Parameterized Constants to SI Units}
\label{sec4-1}

If the rate equations are in physiological units, and can be expressed
in the form of Eq. (\ref{eq9}), possibly with a different resting potential,
then Eq. (\ref{eq6}) can be used to translate the voltages.  If the lower case
letters $a$, $b$, $c$, $d$, and $f$ are used to represent the parameters
in physiological units, then the SI parameters are:

$A = 1000 a - 10^6 b (EREST_{-}ACT - 0.001 E_{rest})$ \\
$B = 10^6 b$ \\
$C = c$ \\
$D = 0.001 d - (EREST_{-}ACT - 0.001 E_{rest})$ \\
$F = 0.001 f$

As an example, these relations may be applied to the forward rate function
in Eq. (\ref{eq1}), using $E_{rest} = 0 ~mV$ and $EREST_{-}ACT = -0.07 ~V$
to give the results for the parameters:

\begin{table}[thb]
\label{tab4}
\centering
\begin{tabular}{clcl} \hline
{\em physiol parameter } & {\em value}  & {\em SI parameter} & {\em value} \\
\hline

$a$  & 0.1 & $A$ & -600 \\
$b$  & -0.01 & $B$ & -10000 \\
$c$  & -1 & $C$ & -1 \\
$d$  & -10 & $D$ & 0.06 \\
$f$  & -10 & $F$ & -0.01 \\
\hline
\end{tabular}
\caption{Correspondence between SI and physiological {\em setupalpha}
parameters that represent Eq. (\ref{eq1}).}
\end{table}

Note the correspondence between the values for the SI parameters in Table
\ref{tab4} and the NDF {\tt PARAMETERS} block given in the Introduction.

\subsubsection{The {\em setuptau} Parameterized Form}
\label{sec4-2}

Sometimes, gates are described in terms of the voltage dependent steady state
activation $n_\infty$ and time constant $\tau_n$.  These functions are more
directly related to the results of voltage clamp experiments, and can be
expressed in terms of the rate functions as

\begin{equation}
n_\infty(V) = \frac{ \alpha_n(V) }{ \alpha_n(V) + \beta_n(V) }
\label{eq11}
\end{equation}
\begin{equation}
\tau_n(V) = \frac{1}{\alpha_n(V) + \beta_n(V)}
\label{eq12}
\end{equation}


Some GENESIS 2 scripts for creating channels use the {\em setuptau} command
to fill the gate $A$ and $B$ tables by specifying the time constant and
activation with an equation in the form of Eq. (\ref{eq9}) where the
parameters $AA$ -- $AF$ correspond to the time constant and $BA$ -- $BF$
correspond to the steady state activation.

GENESIS 3 does not currently provide a parameterized representation in
this form.  In the future the {\em sli\_load} command should  translate
{\em setuptau} commands into a NDF representation with a {\tt PARAMETERS}
block containing the table values, or other equivalent representation.

The conversion from these parameters in pysiological units to SI units
is a slighty different from the {\em setupalpha} form, because the
conversion equations for the time constant and steady state activation
have different units.   For the time constant parameters $A$ -- $F$,
where $\tau$ is in seconds,

$A = 0.001 a - b (EREST_{-}ACT - 0.001 E_{rest})$ \\
$B = b$ \\
$C = c$ \\
$D = 0.001 d - (EREST_{-}ACT - 0.001 E_{rest})$ \\
$F = 0.001 f$

The steady state activation is a dimensionless quantity, and the
parameters $A$ -- $F$ are

$A = a - 1000 b (EREST_{-}ACT - 0.001 E_{rest})$ \\
$B = 1000 b$ \\
$C = c$ \\
$D = 0.001 d - (EREST_{-}ACT - 0.001 E_{rest})$ \\
$F = 0.001 f$

\subsubsection{Simplified Form of Gate Equations}

Often, detailed voltage clamp data is not available for a channel to be
modeled.  In this case, it is common to roughly fit the steady state
activation of gate to a sigmoid function, with a fixed
voltage-independent value of the time constant $\tau$.  In this case,

\begin{equation}
n_\infty(V) = \frac{ 1 }{ 1 + \exp((V - V_{0}) / K) }
\label{eq13}
\end{equation}

and $\tau$ is a constant.  Here, $V_{0}$ is the voltage for 50\% activation,
and $K$ is related to the slope of $n_\infty(V)$ at its midpoint 
$(\frac{d n_\infty(V)}{d V})_{V=V_0}$ by

\begin{equation}
K =  \frac{ -1 }{ 4 (\frac{d n_\infty(V)}{d V})_{V=V_0} }
\label{eq14}
\end{equation}

Equation (\ref{eq13}) can be put in the {\em setuptau} form of Eq.
(\ref{eq9}), using $A = 1$, $B = 0$, $C = 1$, $D = -V_0$, and $F = K$.
However, the voltage-dependent exponential term in the denominator
presents a problem if $\tau$ is to be a constant.  One way to deal with
this is to use an approximation with $A = \tau$ , $B = \tau / F$, $C = 0$,
$D = 0$, and $F = 10^6$.

However, it is possible to put these in the {\em setupalpha} form.
Equations (\ref{eq11}) and (\ref{eq12}) can be used to calculate the
expressions for the forward rate $\alpha(V)$ and the backward
rate $\beta(V)$ in terms of $\tau$ and $n_\infty(V)$.  Normally, one would
not expect the resulting equations to be in the form of Eq.  (\ref{eq9}),
but in this simple case, a little algebra yields the surprisingly simple
result

\begin{equation}
\alpha_n(V) = \frac{ 1 }{\tau (1 + \exp((V - V_0) /K)))}
\label{eq15}
\end{equation}


\begin{equation}
\beta_n(V) = \frac{ 1 }{\tau (1 + \exp((-V + V_0) /K)))}
\label{eq16}
\end{equation}

Thus, $\alpha_n(V)$ and $\beta_n(V)$ are in the form of Eq. (\ref{eq9}),
with the parameters for $\alpha_n(V)$: $A = 1/\tau$, $B = 0$, $C = 1$,
$D = -V_0$, and $F = K$.  For $\beta_n(V)$, they are the same, except that
$F = -K$.

If $\tau$, $V_0$, and $K$ are given in physiological units, possibly with a
different reference point for $E_{rest}$, the equations in Sec.~\ref{sec4-1}
may be used to convert them to SI units, and Table \ref{tab3}
may be used to convert them to the parameters used in the NMC parameterized
notation.

\end{document}

