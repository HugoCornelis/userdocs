\documentclass[12pt]{article}
\usepackage{verbatim}
\usepackage[dvips]{epsfig}
\usepackage{color}
\usepackage{url}
\usepackage[colorlinks=true]{hyperref}

\begin{document}

\section*{GENESIS: Documentation}

{\bf Related Documentation:}
% start: userdocs-tag-replace-items related-do-nothing
% end: userdocs-tag-replace-items related-do-nothing

\section*{Fidelity of voltage clamps of Purkinje cells}

Constructing a realistic
model of a neuron relies on the availability of high quality
data, especially from voltage-clamp experiments. However,
once constructed, the model itself can be used to explore
the likely value of the voltage-clamp experiments themselves.
For example, using voltage-clamp techniques, Regehr\,\cite{Regehr:1992tg} have recently proposed that Na+ channels
are present in the (distal) dendrite as well as the soma of the
Purkinje cell. They based this claim on the fact that they
could measure fast inward currents with whole-cell patch
clamps of axotomized Purkinje cells under conditions of
good voltage control in the soma. However, it has been
suggested that in these experiments the soma itself may not
have been adequately clamped\,\cite{Sugimori:1992hc}.

Using the model, we have found that the presence of
active dendritic membrane had a dramatic effect on the
electrotonic length of the Purkinje cell and thus the ability
to achieve an adequate space clamp\,\cite{Rall:1985ys}.
Passive membrane models predict that Purkinje cells are
electrotonically compact, with a length of $\approx$ 0.3\,X\,\cite{P:1985mb}. 
Rapp et al.\,\cite{Rapp-M:1992kx} have pointed out that the large
synaptic input the Purkinje cell receives could increase this
length by a factor of 2.4. Our model shows that during a
dendritic spike the electrotonic distance to the top of the
dendrite increased by a factor of 3 compared with a passive
membrane dendrite. Between spikes the length was also increased,
but less. Thus the state of activation of the dendritic
channels also made a significant difference in the
electrotonic length of the dendrite. Similarly, activation of
dendritic channels also caused a very bad space clamp, especially
in the depolarizing direction. We have assumed that
the application of Cs$^+$ externally and EGTA internally
blocked most K$^+$ currents completely\,\cite{Hille:1991zr}. If this
was not the case, K$^+$ currents would contribute only a small
conductance compared with the CaP conductance and similar
results would be obtained.

Our results suggest that space clamp is extremely bad in
adult Purkinje cells, with a significant divergence from the
holding potential at only 50\,$\mu$m from the soma for depolarizing
potentials. Therefore it is possible that the Na$^+$
currents recorded by Regehr et al.\,\cite{Regehr:1992tg} are carried by
Na$^+$ channels located in the very proximal parts of the
main dendrite. In our present model, we have confined
Na$^+$ channels to the soma, because several other studies,
including TTX binding\,\cite{Llinas:1992rq}, dendritic
patch recordings (ibid.), and imaging with a fluorescent
Na$^+$ indicator\,\cite{Lasser-Ross:1992ij} demonstrated
Na$^+$ channels only on the soma and axon of the
Purkinje cell.

Using the model, we can predict that space clamps could
be somewhat improved if Ca$^{2+}$ channels were also blocked,
but this was not the case in most reported whole-cell patchclamp
studies\,\cite{Kaneda:1990ys, Llano:1991qf, Regan:1991ly, Regehr:1992tg}. 
Note that if the CaP channel
does not inactivate, as has been reported by\,\cite{Usowicz:1992qf}, 
one cannot remove the Ca$^{2+}$ conductance by
steady depolarization as has been claimed in some studies\,\cite{Llano:1991qf}. 
In dissociated Purkinje cells, where
large parts of the dendritic tree may be amputated, space
clamp can also assumed to be better. Thus the ``whole-cell''
patch-clamp method may be appropriate to measure ionic
currents in soma and proximal dendrite in dissociated, cultured
Purkinje cells. Equations for several channels used in
our model were based on this approach\,\cite{Hirano:1989uq, Kaneda:1990ys, Regan:1991ly}.

Finally, our ultimate objective in constructing this model
was to explore synaptic influences on Purkinje cell firing
patterns.

\bibliographystyle{plain}
\bibliography{../tex/bib/g3-refs.bib}

\end{document}
