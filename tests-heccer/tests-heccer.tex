\documentclass[12pt]{article}
\usepackage{verbatim}
\usepackage[dvips]{epsfig}
\usepackage{color}
\usepackage{url}
\usepackage[colorlinks=true]{hyperref}

\begin{document}

\section*{GENESIS: Documentation}

{\bf Related Documentation:}
% start: userdocs-tag-replace-items related-do-nothing
% end: userdocs-tag-replace-items related-do-nothing

\section*{Heccer Regression Tests}

Heccer is a fast compartmental solver, that is based on {\it hsolve} of the GENESIS simulator.

For more details see the \href{../heccer/heccer.tex}{\bf Heccer} documentation.

\subsection*{Obtaining Regression Test Specifications}

This \href{http://www.neurospaces.org/neurospaces_project/heccer/tests/html/index.html}{\bf link} accesses all regression test specifications for {\bf Heccer}.

\subsection*{Regression Test Documentation}

{\bf Heccer} regression test documentation includes:
\begin{itemize}
\item The startup command.
\item The input to the test.
\item The expected output of the test.
\end{itemize}
The regression tests give a good overview of {\bf Heccer} functionality and test code gives a good idea of how to use {\bf Heccer} from the Unix command line. Regression tests consist mainly of user experienced functionality (i.e. not low level unit tests). Single executables are run with a hard-coded model and the generated output is checked.

See the \href{../ssp/ssp.tex}{\bf SSP} package for integration tests with GENESIS.

\subsection*{Scope of Regression Tests}

The first two suites of regression tests check C code bindings to {\bf Heccer}, whereas, the second two test suites check the equivalent Perl bindings to {\bf Heccer}. Regression tests cover the following major functionalities:
\begin{itemize}
\item[]\href{http://www.neurospaces.org/neurospaces_project/heccer/tests/html/specifications/main.html}{\bf Model specification}

Internal and external addressing of variables and external callouts; check for illegal parameters; options; serialization; synaptic channels; pool integration; table sharing and gate tabulation; Nernst potentials; passive models; voltage and current clamp circuitry.

\item[]\href{http://www.neurospaces.org/neurospaces_project/heccer/tests/html/specifications/active/main.html}{\bf Active single compartment and channel interactions.}

\item[]\href{http://www.neurospaces.org/neurospaces_project/heccer/tests/html/specifications/glue/swig/perl/main.html}{\bf Active model}

Perl bindings to {\bf Heccer} for both pool integration and gate tabulation for a passive model.

\item[]\href{http://www.neurospaces.org/neurospaces_project/heccer/tests/html/specifications/glue/swig/perl/active/main.html}{\bf Perl bindings to {\bf Heccer} for a single active compartment}
\end{itemize}

\end{document}

\end{document}
