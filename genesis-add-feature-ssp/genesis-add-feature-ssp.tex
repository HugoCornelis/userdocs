\documentclass[12pt]{article}
\usepackage{verbatim}
\usepackage[dvips]{epsfig}
\usepackage{color}
\usepackage{url}
\usepackage[colorlinks=true]{hyperref}

\begin{document}

\section*{GENESIS: Documentation}

{\bf Related Documentation:}
% start: userdocs-tag-replace-items related-add-functionality
% end: userdocs-tag-replace-items related-add-functionality

\section*{Add the PulseGen\{\} Object to SSP}

\subsection*{Reading Parameters}

To integrate a solver object (here, we use the example \href{../genesis-add-object-solver/genesis-add-object-solver.tex}{\it PulseGen\{\}}) with \href{../ssp/ssp.tex}{\bf SSP}, {\bf SSP} must first be made aware of the parameters of the object. {\it PulseGen\{\}} takes parameters for {\it Level1}, {\it Width1}, {\it Delay1}, {\it Level2}, {\it Width2}, {\it Delay2}, {\it BaseLevel\{\}}, and {\it TriggerMode\{\}}, so {\bf SSP} must be able to read in a value for each as an argument.

First, variables are declared for the optional arguments near the top of the {\it bin/ssp} file:
\begin{verbatim}
   my $option_pulsegen_level1;
   my $option_pulsegen_width1;
   my $option_pulsegen_delay1;
   my $option_pulsegen_level2;
   my $option_pulsegen_width2;
   my $option_pulsegen_delay2;
   my $option_pulsegen_baselvel;
   my $option_pulsegen_triggermode;
\end{verbatim}
Then in the Perl subroutine {\it read\_cmd\_line} we add the options to be read in to the call to {\it GetOptions()}:
\begin{verbatim}
   "pulsegen-level1=s" => \$option_pulsegen_level1,
   "pulsegen-width1=s" => \$option_pulsegen_width1,
   "pulsegen-delay1=s" => \$option_pulsegen_delay1,
   "pulsegen-level2=s" => \$option_pulsegen_level2,
   "pulsegen-width2=s" => \$option_pulsegen_width2,
   "pulsegen-delay2=s" => \$option_pulsegen_delay2,
\end{verbatim}
This allows the values to be read from the command line and placed into memory. They must then be passed on to {\it PulseGen\{\}}.

\subsection*{Passing data to {\it PulseGen\{\}}}

Now that {\it PulseGen\{\}} can read in values, each parameter must be checked and passed on to the {\it PulseGen\{\}} object. We do this by using a simple {\it if} statement that checks whether a variable has been defined and then passes on a value to {\bf SSP}. A block similar to the following must be made for each parameter to be accepted by{\it PulseGen\{\}}. Here is an example for setting {\it width1} in {\it PulseGen\{\}}:
\begin{verbatim}
   if (defined $option_pulsegen_width1)
   {
      my $pulsegen_inputclassname = "pulsegen";

      my $inputclass
         = {
               module_name => "Experiment",
               options => {
                  command => $option_pulsegen_width1,
                  name => "pulsegen set to $option_pulsegen_width1",
               },
               package => "Experiment::Pulsegen",
            };

      my $inputs
         = [
            {
               component_name => "$model_root/segments/soma",
               field => "WIDTH1",
               inputclass => $pulsegen_inputclassname,
            },
         ];

      $scheduler->{inputclasses}->{$pulsegen_inputclassname} = $inputclass;

         if (!defined $scheduler->{inputs})
         {
            $scheduler->{inputs} = [];
         }

         push @{$scheduler->{inputs}}, @$inputs;
   }
\end{verbatim}

|\subsection*{Creating a SSP specification}

For testing purposes, a {\bf SSP} specification is created to check the functionality of the {\it PulseGen\{\}} object. A test will load a model, create a {\it PulseGen\{\}} solver object, then clamp the {\it PulseGen\{\}} output to the model's membrane potential ($V_m$) at the soma. Although not biologically correct, this provides a good base case for a test.

First, a file {\it pulsegen\_freerun.yml} is created as a large YAML hash. The simulation runs for 200 steps, similar to the examples in \href{../genesis-create-test-heccer/genesis-create-test-heccer.tex}{\bf Create a Heccer Test} and the associated GENESIS 2 script :
\begin{verbatim}
--- !!perl/hash:SSP
apply:
  results:
    - arguments:
        - commands:
            - perl -e 'my $result = [ `cat /tmp/output`, ]'
      method: shell
  simulation:
    - arguments:
        - 200
      method: steps
\end{verbatim}
Here, the {\it PulseGen\{\}} object is created as an input class and all of the parameters previously used in the \href{../genesis-create-test-heccer/genesis-create-test-heccer.tex}{\bf Create a Heccer Test} and GENESIS 2 scripts are assigned to it.
\begin{verbatim}
inputclasses:
  pulsegen:
    module_name: Experiment
    options:
      level1: 50.0
      width1: 3.0
      delay1: 5.0
      level2: -20.0
      width2: 5.0
      delay2: 8.0
      baselevel: 10.0
      triggermode: 0
    package: Experiment::PulseGen
\end{verbatim}
The  {\it PulseGen\{\}} object is then "clamped" to the model's soma.
\begin{verbatim}
inputs:
  - component_name: /Purkinje/segments/soma
    field: Vm
    inputclass: pulsegen
models:
  - modelname: /Purkinje
    solverclass: heccer
name: purkinje cell pulsegen
\end{verbatim}
An output object {\it double\_2\_ascii} is then created that writes $V_m$ to the file {\it /tmp/output}.
\begin{verbatim}
outputclasses:
  double_2_ascii:
    module_name: Experiment
    options:
      filename: /tmp/output
    package: Experiment::Output
outputs:
  - component_name: /Purkinje/segments/soma
    field: Vm
    outputclass: double_2_ascii
\end{verbatim}
Here, the \href{../model-container/model-container.tex}{\bf Model\,Container} is invoked to load the desired model. Parameters for the {\bf Model\,Container} can also be set in this block:
\begin{verbatim}
services:
  model_container:
    initializers:
      - arguments:
          - filename: cells/purkinje/edsjb1994.ndf
            no-use-library: 1
          -
            - ssp configuration for edsjb1994 using pulsegen
            - -R
            - -A
        method: load
    module_name: Neurospaces
\end{verbatim}
Finally, \href{../heccer/heccer.tex}{\bf Heccer} is invoked. The output granularity is set to 1 and the step size to 0.5 seconds. Again, in line with the previous examples:
\begin{verbatim}
solverclasses:
  heccer:
    constructor_settings:
      configuration:
        reporting:
          granularity: 1
          tested_things: 6225920
      dStep: 0.5
      options:
        iOptions: 4
    module_name: Heccer
    service_name: model_container
\end{verbatim}
After saving the specification, the simulation is started with the command:
\begin{verbatim}
   ssp pulsegen_freerun.yml
\end{verbatim}
The expected output is then written to the file {\it /tmp/output} as declared in the specification. 

\end{document}
