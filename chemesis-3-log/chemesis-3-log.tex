\documentclass[12pt]{article}
\usepackage{verbatim}
\usepackage[dvips]{epsfig}
\usepackage{color}
\usepackage{url}
\usepackage[colorlinks=true]{hyperref}

\begin{document}

\section*{GENESIS: Documentation}

{\bf Related Documentation:}
% start: userdocs-tag-replace-items related-chemesis-3
% end: userdocs-tag-replace-items related-chemesis-3
% start: userdocs-tag-replace-items related-add-functionality
% end: userdocs-tag-replace-items related-add-functionality

\section*{Chemesis-3 and G-3}

This document outlines the steps of implementation of {\bf
  Chemesis-3}.

{\bf Chemesis} is a library of biochemical reaction objects for
modeling calcium
concentration\cite{blackwell02:_neuros_datab_pract_guide}, originally
contributed by Avrama Blackwell (avrama@gmu.edu) to G-2.  The
\href{../genesis-overview/genesis-overview.tex}{G-3} implementation of
{\bf Chemesis}, called {\bf Chemesis-3}, is based on the G-2
functionality, the code is mostly rewritten from scratch, with small
snippets copied pasted over.

\section*{Creation of {\bf Chemesis-3} for G-3}

The global outline of adding new functionality to G-3 is defined as:
\begin{enumerate}
\item Implementation of the code of the new functionality.
\item Incorporation of this new code into the maintenance loop of
  documentation and tests.
\item Interfacing the new code with the rest of the system.
\end{enumerate}

The core of this process is in the first step.  A software developer
can be fully focussed on the new mathematics and algorithms without
being distracted or feeling restricted by overhead of G-3 maintenance.
The other these steps are semi-automatic.  It is possible but not
required to start with step 2 and 3 before finishing step 1.

The steps required for the creation and the implementation of the {\bf
  Chemesis-3} library are (1) implementation of the independent solver
objects, (2) definition in the
\href{../developer-package/developer-package.tex}{\bf
  DeveloperPackage}, (3) initial creation of the workspace under
version control, (4) configuration of
\href{model-container/model-container.tex}{the {\bf model-container}}
for the new {\bf Chemesis-3} tokens, (5) connecting the {\bf
  model-container}'s {\bf Chemesis-3} tokens with the solver objects via
the {\bf model-container API}, (6) connecting the solvers with the
scriptable software components of G-3 such as
\href{../sspy/sspy.tex}{\bf SSPy}, \href{../ssp/ssp.tex}{\bf SSP} and
the \href{../gshell/gshell.tex}{\bf gshell}, and (7) implementation of
documentation for the new functionality.

\begin{enumerate}
\item The implementation of the independent solver objects is specific
  to {\bf Chemesis-3} functionality.  The development of these solvers
  can be started independent of G-3.
\item The following configuration snippet in the file {\bf
    /etc/neurospaces/developer/build.yml} defines the {\bf Chemesis-3}
  library for the
  \href{../developer-package/developer-package.tex}{\bf
    DeveloperPackage}:
\begin{verbatim}
all_packages:
  chemesis-3:
    order: 2.5
    version_control:
      port_number: 4706
\end{verbatim}
  The order key defines the build order of the {\bf Chemesis-3} library.
  The {\bf Chemesis-3} library is built after
  \href{../heccer/heccer.tex}{\bf Heccer} (build order 2) and before
  \href{../ssp/ssp.tex}{\bf SSP} (build order 3).  The command
\begin{verbatim}
neurospaces_build --help-build-order
\end{verbatim}
  reveals the details of the software component build order.
\item For the creation of the initial workspace the {\bf
    DeveloperPackage} provides the command {\it
    neurospaces\_new\_component}.  The option {\tt --package-name}
  defines the name of the new software component, in this case {\bf
    Chemesis-3}.  In full the command is:
\begin{verbatim}
neurospaces_new_component --package-name chemesis-3
\end{verbatim}
  This command creates a new workspace under control of the {\bf
    DeveloperPackage} and populates it with configuration scripts and
  makefiles.  After the creation of the workspace, it can be populated
  with the source code that was created in the first step and the
  makefiles can be updated with compilation and installation targets
  as necessary.
\item The {\bf model-container} already recognizes the {\tt POOL}
  token for calcium concentration ion pools.  The {\bf Chemesis-3}
  library also needs a definition of the {\tt REACTION} token that
  recognizes two parameters {\tt FORWARD\_RATE} and {\tt
    BACKWARD\_RATE}.  The configuration snippet\footnote{The
    configuration snippet should be further simplified.} below should
  be installed to {\tt /etc/neurospaces/model-container/symbols/}.  It
  defines the new token {\tt REACTION} and for simplicity maps it to
  the implementation of the {\tt GROUP} token that is already
  available.  This reduces the implementation overhead.  Later on,
  additional implementation may be useful or necessary.
\begin{verbatim}
---
class_hierarchy:
  group:
    grammar:
      components:
        - Reaction
  reaction:
    allows:
      create_alias: group
    annotations:
      piSymbolType2Biolevel: BIOLEVEL_MECHANISM
    description: a reaction between different molecule concentration pools
    dimensions:
      - mechanism
    grammar:
      components:
        - Group
        - Pool
      specific_allocator: GroupCalloc
      specific_token:
        class: reaction
        lexical: TOKEN_REACTION
        purpose: physical
      typing:
        base: phsle
        id: pidin
        spec: pgrup
        to_base: ->bio.ioh.iol.hsle
    isa: group
    name: symtab_Group
    parameters:
       FORWARD_RATE: forward reaction rate, from substrate(s) to product.
       BACKWARD_RATE: backward reaction rate, from product to substrate(s).
\end{verbatim}
\item Connection with the {\bf model-container API}.
\item Making the new functionality scriptable.
\item Implementation of integration tests and development of user
  documentation.
\end{enumerate}


\section*{Making {\bf Chemesis-3} Available for Others}

Setting up a repository server is what is required to make {\bf
  Chemesis-3} available for others.  To initialize the server
repository, implement steps (2) and (3) explained in the previous
section on the server PC.  Then use the {\it neurospaces\_serve}
command to start the server.  For an explanation of permissions and
other security aspects, see
\href{../installation-debian-server/installation-debian-server.html}{the
  server installation documentation}.


\section*{Installation and Testing of {\bf Chemesis-3} on Other Computers}

To incorporate the new software component in existing and new
installations, the configuration snippet in the file {\bf
  /etc/neurospaces/developer/build.yml} defines the {\bf Chemesis-3}
library for the \href{../developer-package/developer-package.tex}{\bf
  DeveloperPackage}:
\begin{verbatim}
all_packages:
  chemesis-3:
    order: 2.5
    version_control:
      port_number: 4706
\end{verbatim}
Then the {\it neurospaces\_upgrade} command can be used to automate
the process of importation and installation of the new component.



\bibliographystyle{plain}
\bibliography{../tex/bib/g3-refs.bib}

\end{document}
