\documentclass[12pt]{article}
\usepackage{verbatim}
\usepackage[dvips]{epsfig}
\usepackage{color}
\usepackage{url}
\usepackage[colorlinks=true]{hyperref}

\begin{document}

\section*{GENESIS: Documentation}

{\bf Related Documentation:}
% start: userdocs-tag-replace-items related-do-nothing
% end: userdocs-tag-replace-items related-do-nothing

\section*{De Schutter: Purkinje Cell Model}

\subsection*{Source}

De Schutter E \& Bower JM (1994) Simulated responses of cerebellar Purkinje cells are independent of the dendritic location of granule cell synaptic inputs. {\it Proceedings of the National Academy of Sciences (USA).} {\bf 91}: 4736--4740.

\subsection*{Abstract}

\begin{enumerate}
   \item Cerebellar Purkinje cell responses to granule cell synaptic inputs were examined with a computer model
including active dendritic conductances.

   \item In the active model, dendritic P-type Ca$^{2+}$ channels amplified postsynaptic responses when the model was firing at a physiological rate. Small synchronous excitatory inputs applied distally on the large dendritic tree resulted in somatic responses of similar size to those generated by more proximal inputs.

   \item In contrast, in a passive model the somatic postsynaptic potentials to distal inputs were 76\,\% smaller. The
model predicts that the somatic firing response of Purkinje cells is relatively insensitive to the exact dendritic location of synaptic inputs.

   \item We describe a mechanism of Ca$^{2+}$-mediated synaptic amplification, based on the subspiking threshold recruitment of P-type Ca$^{2+}$ channels in the dendritic branches surrounding the input site.
\end{enumerate}

\end{document}
