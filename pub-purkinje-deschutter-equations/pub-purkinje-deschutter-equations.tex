\documentclass[12pt]{article}
\usepackage{verbatim}
\usepackage[dvips]{epsfig}
\usepackage{color}
\usepackage{url}
\usepackage[colorlinks=true]{hyperref}

\begin{document}

\section*{GENESIS: Documentation}

{\bf Related Documentation:}
% start: userdocs-tag-replace-items related-do-nothing
% end: userdocs-tag-replace-items related-do-nothing

\section*{De Schutter: Purkinje Cell Model}

\subsection*{Source}

De Schutter E \& Bower JM (1994) An active membrane model of the cerebellar Purkinje cell I. Simulation of current clamp in slice. {\it Journal of Nerurophysiology}. {\bf 71}: 375--400. \\
De Schutter E \& Bower JM (1994) An active membrane model of the cerebellar Purkinje cell II. Simulation of synaptic responses.  {\it Journal of Nerurophysiology}. {\bf 71}: 401--419.

\subsection*{Equations}

The channel conductance was
determined by the product of voltage-dependent activation ($m$)
and inactivation ($h$) gates, and for the Ca$^{2+}$-activated channels
a Ca$^{2+}$-dependent activation gate ($z$)

\begin{equation}
   G(V,[Ca^{2+}], t) = \bar gm(V, t)^p z([Ca^{2+}], t)^r, \mbox{        (units: mV, $\mu$M, ms)}
\end{equation}

Equations describing the voltage-dependent gates were described from the classic Hodgkin-Huxley\,\cite{L:1952fv} scheme

\begin{equation}
   \frac{\partial m}{\partial t} = \alpha_m(1-m) - \beta_mm, \mbox{        idem for $h$}
\end{equation}

\begin{equation}
   \alpha_m(V,t) = \frac{A}{B+\exp^{V+C)/D}}, \mbox{        }\beta_m(V,t) = \frac{E}{F+\exp^{(V+G)/H}}, \mbox{        idem for $\alpha_h$ and $\beta_h$}
\end{equation}

Activation rates for Ca$^{2+}$-dependent gates were determined by a dissociation constant $A$ and a time constant $B$

\begin{equation}
   \frac{\partial z}{\partial t} = \frac{z_\infty - z}{\tau_z}
\end{equation}

\begin{equation}
   z_\infty = \frac{1}{1+\frac{A}{\mbox{[Ca$^{2+}$]}}} \mbox{        }\tau_z = B
\end{equation}

For the Ca$^{2+}$ channels the Nernst potential\,\cite{B:1991zr} was
computed continuously. Rectification of
Ca$^{2+}$ channels was not modeled using the Goldman-Hodgkin-Katz (GHK) equation\,\cite{B:1991zr}
because dendritic membrane potentials in this
study stayed within a range where Ca$^{2+}$  channels can be considered
ohmic (i.e., below -20\,mV; Fig. 4.15 in\,\cite{B:1991zr} ). Using
the simulation results from the final model, we estimate that using
the GHK equation with an appropriately scaled maximum conductance
($\bar g$) to compensate for differences in driving force would
cause only small changes in the amplitude of dendritic Ca$^{2+}$
currents (mean difference 0.7\,\%, maximum 4.5\,\%).


\bibliographystyle{plain}
\bibliography{../tex/bib/pub-ref}

\end{document}
