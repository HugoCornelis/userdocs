\documentclass[12pt]{article}
\usepackage{verbatim}
\usepackage[dvips]{epsfig}
\usepackage{color}
\usepackage{url}
\usepackage[colorlinks=true]{hyperref}

\begin{document}

\section*{GENESIS: Documentation}

{\bf Related Documentation:}
% start: userdocs-tag-replace-items related-do-nothing
% end: userdocs-tag-replace-items related-do-nothing

\section*{GENESIS Components}

The GENESIS software platform is based on the \href{../genesis-overview/genesis-overview.tex}{\bf CBI federated software architecture}. This provides a modular paradigm that places stand-alone software components into logical relationships. The following GENESIS \href{../reserved-words/reserved-words.tex}{\bf Components} have been implemented and tested and are currently compliant with the CBI architecture.

\begin{enumerate}
   \item[]\href{../gshell/gshell.tex}{\bf G-Shell}: Interactive GENESIS shell environment.
   \item[]\href{../nssli/nssli.tex}{\bf NS-SLI}: GENESIS 2 backward compatibility.
   \item[]\href{../model-container/model-container.tex}{\bf Model\,Container}: Solver-independent model storage.
   \item[]\href{../heccer/heccer.tex}{\bf Heccer}: Mathematical solver.
   \item[]\href{../ssp/ssp.tex}{\bf SSP}: Simple Scheduler in Perl.
   \item[]\href{../project-browser/project-browser.tex}{\bf Project\,Browser}: Inspect and compare simulation output.
   \item[]\href{../studio/studio.tex}{\bf Studio}: Visualize models in the {\bf Model\,Container}.
   \item[]\href{../gtube/gtube.tex}{\bf G-Tube}: GENESIS GUI.
\end{enumerate}

\end{document}
