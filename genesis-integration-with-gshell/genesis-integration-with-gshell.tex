\documentclass[12pt]{article}
\usepackage{verbatim}
\usepackage[dvips]{epsfig}
\usepackage{color}
\usepackage{url}
\usepackage[colorlinks=true]{hyperref}

\begin{document}

\section*{GENESIS: Documentation}

{\bf Related Documentation:}
% start: userdocs-tag-replace-items related-add-functionality
% end: userdocs-tag-replace-items related-add-functionality

\section*{Integrating new functionality into the G-Shell}

Previously we created a solver object called PulseGen\{\}. In this document we will outline the process for adding our objects functionality to the gshell. Up to this point, the PulseGen\{\} Object should:

\begin{itemize}
\item[] Have a symbol complete with parameters in the Model Container as outlined \href{../genesis-extend-model-container/genesis-extend-model-container.tex}{\bf here.}
\item[] Have a corresponding \href{../genesis-add-object-solver/genesis-add-object-solver.tex}{\bf solver object} with \href{../genesis-create-test-heccer/genesis-create-test-heccer.tex}{\bf tests} to verify it's working correctly.
\item[] Have a \href{../genesis-add-swigbinding-heccer/genesis-add-swigbinding-heccer.tex}{\bf SWIG binding} to make the solver useable from Perl.
\item[] Be callable from \href{../genesis-add-feature-ssp.tex}{\bf SSP}.
\end{itemize} 

Now to make our solver object useable from the {\bf ghsell} we must add code to some sections of the {\it GENESIS.pm} Perl module located in the directory:


\begin{itemize}
\item[] {\it \~/neurospaces\_project/gshell/source/snapshots/0/perl/}
\end{itemize}


\subsection*{Add the PulseGen\{\} Object to the Input Classes}

Since the PulseGen\{\} is a standalone solver object, it must be imported into the {\bf gshell} as an input class. To make the {\bf gshell} aware of the presence of our solver object, we must add an entry for it to the module variable {\it \$all\_inputclass\_templates} located in the {\it GENESIS.pm} file. Here we add a hash entry for our pulsegen object to the inputclass template variable. 


\begin{verbatim}

our $all_inputclass_templates 
  = {
    perfectclamp => {
                      module_name => 'Experiment',
                      options => {
                                  name => 'name of this inputclass',
                                  command => 'command value',
                                 },
                      package => 'Experiment::PerfectClamp',
			       },
    pulsegen => {
                  module_name => 'Experiment',
				  options => {
					    name => 'name of this inputclass',
					    width1 => 'First pulse width',
					    level1 => 'First pulse level',
					    delay1 => 'First pulse delay',
					    width2 => 'Second pulse width',
					    level2 => 'Second pulse level',
					    delay2 => 'Second pulse delay',
					    baselevel => 'The pulse base level',
					    triggermode => 'The pulse triggermode, 0 - freerun, 1 - ext trig, 2 - ext gate',
				  },
                  package => 'Experiment::PulseGen',
                },
    };
\end{verbatim}

Our entry is named {\bf pulsegen} and should have three keys:

\begin{itemize}
\item[] {\bf module\_name}: Since our PulseGen\{\} simulation object was compiled into the {\bf Experiment} package we list it for the module name.
\item[] {\bf options}: This set of subkeys are options to pass to our object to set parameters for the inputclass. They are the same parameters outlined in {\bf SSP} for initializing a Pulsegen\{\} object.
\item[] {\bf package}: This key gives the callable object from the SWIG binding so the {\bf gshell} can utilize it correctly.
\end{itemize}

You can check to see if the {\bf gshell} properly reads the template by typing {\it list template\_classes} on the {\bf gshell} prompt:

\begin{verbatim}
	genesis > list inputclass_templates                                                          
	all input class templates:
	  perfectclamp:
	    module_name: Experiment
	    options:
	      command: command value
	      name: name of this inputclass
	    package: Experiment::PerfectClamp
	  pulsegen:
	    module_name: Experiment
	    options:
	      baselevel: The pulse base level
	      delay1: First pulse delay
	      delay2: Second pulse delay
	      level1: First pulse level
	      level2: Second pulse level
	      name: name of this inputclass
	      triggermode: 'The pulse triggermode, 0 - freerun, 1 - ext trig, 2 - ext gate'
	      width1: First pulse width
	      width2: Second pulse width
	    package: Experiment::PulseGen

\end{verbatim}


\end{document}
