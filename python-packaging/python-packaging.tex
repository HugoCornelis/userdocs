\documentclass[12pt]{article}
\usepackage{verbatim}
\usepackage[dvips]{epsfig}
\usepackage{color}
\usepackage{url}
\usepackage[colorlinks=true]{hyperref}

\begin{document}

\section*{GENESIS: Documentation}

{\bf Related Documentation:}
% start: userdocs-tag-replace-items related-do-nothing
% end: userdocs-tag-replace-items related-do-nothing

\section*{Python Packaging}

This document contains information for building {\bf Python} modules, eggs and other distributable packages using the built in setuptools and distutils frameworks.  



\section*{Setup script}

	Each {\bf GENESIS3} component with a {\bf Python} interface has a {\it setup.py} script in the {\it glue/swig/python} directory. The script is responsible for compiling the python source into a module, generating Python eggs, installing compiled modules, and registering and pushing eggs to PyPi. 


\subsection*{Building a module}

	To build a module you simply run the command:
	
\begin{verbatim}
	python setup.py build
\end{verbatim}

The result of running this command will generate a 'build' directory that contains a 'lib' directory, which contains the importable module. And directory that contains any executable scripts. On some systems it may also contain a directory that has any pre-build data. After building the module you can import from build/lib so long as you add the path vis the sys.path variable. 



\subsection*{Installing a module}


\subsection*{Uninstalling a module}


\subsection*{Registering on PyPi}


\subsection*{Pushing an egg to PyPi}

\end{document}
