\documentclass[12pt]{article}
\usepackage[dvips]{epsfig}
\usepackage{color}
\usepackage{url}
\usepackage[colorlinks=true]{hyperref}

\begin{document}

\section*{GENESIS: Introduction}

\section*{Workflow Overview}

The typical workflow within GENESIS has four basic steps. It is followed, for example, by the \href{../gshell/gshell.tex}{\bf G-Shell} and tutorial documentation.

\begin{enumerate}

\item {\bf Model operations:} Simple models can be created directly within the {\bf G-Shell} by entering commands. More complex models can be imported into the {\bf G-Shell} from either the GENESIS model libraries or from external model libraries. The command line based \href{../studio/studio.tex}{\bf Studio} can then be used to explore, check, and and the model.

\item {\bf Characterize simulation:} For example,  the \href{../project-browser/project-browser.tex}{\bf Project\,Browser} can be used to configure the simulation update time step or the model parameter values specific to a given simulation, the stimulus parameters for a given simulation run or  `experiment', and/or the variables to be stored for subsequent analysis.

\item  {\bf Simulate:} Check, run, reset simulation, and save model state. Flush output to raw result storage for subsequent data analysis. A model can be saved at any simulation time step to allow it to be imported into a subsequent GENESIS session for further development and exploration.

\item {\bf Process output:}  Check simulation output and the validity of results to determine whether simulation output exists in the correct locations. Output can be analyzed either within GENESIS or piped to external applications such as \href{http://www.mathworks.com/}{Matlab}, \href{http://www.mathworks.com/}{xmgrace}, or \href{http://www.wolfram.com/}{Mathematica}.

\end{enumerate}

\end{document}