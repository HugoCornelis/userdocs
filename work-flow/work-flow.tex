\documentclass[12pt]{article}
\usepackage[dvips]{epsfig}
\usepackage{color}
\usepackage{url}
\usepackage[colorlinks=true]{hyperref}

\begin{document}

\section*{GENESIS: Introduction}

\section*{Workflow Overview}

An outline of the typical workflow within GENESIS, and the one followed by tutorial documentation, has four basic steps.

\begin{enumerate}

\item {\bf Create/import, explore, and save model:} Simple models can be created directly within the GENESIS shell by entering commands. More complex models can be imported into the GENESIS shell from either the GENESIS model libraries or from other model libraries. The command line based \href{../studio/studio.pdf}{Neurospaces Studio} can then be used to explore and check the model.

\item {\bf Define simulation constants, inputs, and outputs:} For example,  the \href{../project-browser/project-browser.pdf}{Project Browser} can be used to configure the simulation update time step or the model parameter values specific to a given simulation, the stimulus parameters for a given simulation run or  `experiment', and/or variables to be stored for subsequent analysis.

\item  {\bf Check, run, reset simulation, and save model state:} Flush output to raw result storage for subsequent data analysis. A model can be saved at any simulation time step to allow it to be imported into a subsequent GENESIS session for further development and exploration.

\item {\bf Check simulation output:}  Check the validity of results and determine whether simulation output exists in the correct locations.

\end{enumerate}

\end{document}