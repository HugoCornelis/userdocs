\documentclass[12pt]{article}
\usepackage{verbatim}
\usepackage[dvips]{epsfig}
\usepackage{color}
\usepackage{url}
\usepackage[colorlinks=true]{hyperref}

\begin{document}

\section*{GENESIS: Documentation}

\section*{Tester Configuration}

The GENESIS tester needs to know certain things about your machine before tests can be executed. These things are in the tester configuration file, named {\it tests.config}, located in your current directory.

It is normal that a software package has multiple {\it tests.config} files, e.g. in several directories.

The config file defines a perl hash (we also plan to support JSON). In the test specifications, this hash is available in the variable {\tt main::config}. So to access the {\tt core\_directory} entry in the hash, you use:
\begin{verbatim}
$main::config->{core\_directory}
\end{verbatim}

Here is an example script:
\begin{verbatim}
#!/usr/bin/perl -w

use strict;

my $package_name = "heccer";
my $package_label = "build-8";
my $package_version = "build-8-0";

my $monotone_id = `mtn automate get_current_revision_id`;
chomp $monotone_id;

my $config
   = {
      core_directory => './',
      description => 'Configure the tester when run from this directory',
      outputs_dir => './tests/html',
      package => {
         label => $package_label,
         name => $package_name,
         version => $package_version,
         version_control_id => $monotone_id,
         },
      tests_directory => './tests/specifications',
      };

return $config;
\end{verbatim}

\subsection*{Definitions}

\begin{itemize}
   \item[]{\tt core\_directory:} The directory that contains the source code of your package. Tests can use this directory to read files.
   \item[]{\tt description:} Short (few words) description of the tests.
   \item[]{\tt outputs\_dir:} Defines the directory where the {\bf htmlified} test specifications (generated using {\it tests\_2\_html}) are located.
   \item[]{\tt package:} Contains package version information.
   \item[]{\tt tests\_dir:} Contains the directory to be searched for test specifications with the {\bf .t} suffix. 
\end{itemize}
\end{document}
