\documentclass[12pt]{article}
\usepackage[dvips]{epsfig}
\usepackage{color}
%e.g.  \textcolor{red,green,blue}{text}
\usepackage{url}
\usepackage[colorlinks=true]{hyperref}

\begin{document}

\section*{GENESIS: Documentation}

{\bf Related Documentation:}
% start: userdocs-tag-replace-items related-do-nothing
% end: userdocs-tag-replace-items related-do-nothing


\section*{The Studio}

The {\bf Studio} has a GUI front-end to the \href{../model-container/model-container.tex}{\bf Model Container} and allows browsing and visualization of a model. Note that the {\bf Studio} is not a graphical editor or construction kit. External applications such as \href{http://www.physiol.ucl.ac.uk/research/silver\_a/neuroConstruct/}{neuroConstruct} should be employed for these purposes.

Additionally, the {\bf Studio} comes with a shell command that uses the {\bf Model Container} \href{http://www.swig.org/}{Swig} bindings to access the model stored by the {\bf Model Container}.

\subsubsection*{Features}

\begin{itemize}

\item Visualizes models stored by the {\bf Model Container}.

\item  Allows a model to be queried from the \href{../gshell/gshell.tex}{\bf G-Shell} command line.

\end{itemize}
The {\bf Studio} is a necessary part of the \href{../project-browser/project-browser.tex}{\bf Project Browser}.

\end{document}
