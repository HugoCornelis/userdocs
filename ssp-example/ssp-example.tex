\documentclass[12pt]{article}
\usepackage[dvips]{epsfig}
\usepackage{color}
%e.g.  \textcolor{red,green,blue}{text}
\usepackage{url}
\usepackage[colorlinks=true]{hyperref}
\usepackage{scrtime}
 
\begin{document}

{\bf Related Documentation:}
% start: userdocs-tag-replace-items related-ssp
% end: userdocs-tag-replace-items related-ssp

\section*{GENESIS: Documentation}

\section{EMail: Description of an SSP Example Configuration FIle}

In the example below I try to give an impression of what it looks like
to work with the *new* features of the current Neurospaces based
implementation of the GENESIS 3 simulator.  Note that this does not
preclude backwards compatibility, something we are working on right
now.

The example is taken from the purkinje cell comparison project, which
was presented at the society for neuroscience meeting in San Diego.
There are a couple of things I would like to add before starting: (1)
this is a technical and dense explanation, that goes from input and
model over simulation to output and output processing, giving examples
of the file formats along the way.  (2) The explanation is mainly
about an \href{../ssp/ssp.tex}{SSP} configuration file that was
generated automatically from a template (see also attachment).  The
SSP config generator is currently part of the Neurospaces project
browser.  The NS project browser is still under active development, I
will probably make a first pre-alpha release available for the CNS
meeting.  (3) Everything below is about new functions of the
simulator.  Everything explained below, is part of the automatic
regression tests (see the neurospaces website for the output of these
tests).  (4) It took me some time to write this all down, I hope to
receive technical and conceptual feedback.

In this example, I want to stimulate a passive model of a morphology
of a purkinje neuron at a known dendritic location.  The name of the
stimulated dendritic compartment is b1s20[14], it will be stimulated
with a unitary synaptic event.

\begin{enumerate}
\item The model of the neuron will be loaded by the neurospaces model
  container, and is defined using a Genesis 2 morphology file
  (Purk2M9s.p).  The model is an active model, ie it includes
  dendritic ion channels (see below for how the active channels are
  removed from the simulation).  The SSP configuration to load the
  model is:\label{sec:email:-ssp-example}

\begin{verbatim}
services:
 model_container:
   initializers:
     - arguments:
         - filename: /var/neurospaces/simulation_projects//purkinje-comparison2/morphologies/lab/gp/Purk2M9s.p
       method: load
   module_name: Neurospaces
\end{verbatim}

  This section of the SSP file instructs SSP to load the perl module
  {\bf 'Neurospaces'}, and to call the method {\bf 'load'} with the
  argument to read a GENESIS morphology file from the file system.
  For SSP, the internal reference to the resulting object is {\bf
    'model\_container'}.  This {\bf 'model\_container'} is a service
  that can present the model to other internal SSP objects, e.g. a
  solver.  It is possible to use other model containers, in which case
  Neurospaces.pm and its shared libraries will not be loaded (but
  others will).

\item The input event list is accessible in a JSON / YAML formatted
  file called {\bf 'event\_data/events.yml'}.  The content of this
  file looks like this:

\begin{verbatim}
# an event file
events:
   - 0.0001
\end{verbatim}

  So this JSON / YAML file encodes a single event at time 0.1ms in SI
  units.  The notion of time is implicit, and can be overridden.  I
  use this file as an encoding of a synaptic event.  If I would like
  to have more events in this file, it could look like this:

\begin{verbatim}
# an event file
events:
   - 0.0001
   - 0.0002
   - 0.0003
\end{verbatim}

\item I would like to simulate the model using a compartmental solver.
  The compartmental solver is available in the perl module {\bf
    'Heccer.pm'}.  The SSP encoding to load the solver is:

\begin{verbatim}
solverclasses:
 compartmental_solver:
   constructor_settings:
     options:
       iOptions: 4
   module_name: Heccer
   service_name: model_container
\end{verbatim}

  This section of the SSP file instructs SSP to load the perl module
  {\bf 'Heccer'}.  This perl module will automatically load the C code
  of the compartmental solver, and the perl module gets attached to
  the internal object with name {\bf 'compartmental\_solver'}.  When
  the solver gets instantiated, it will be linked to the internal SSP
  service object with name {\bf 'model\_container'}, this is the
  service object that was defined in
  point~\ref{sec:email:-ssp-example}.  The option setting, {\bf
    'iOptions'}, is currently a cryptic way to put heccer in passive
  mode, such that it does not simulate membrane potential dependent
  ionic conductances (it does simulate synaptic channels however).
  This is also the place to specify a time step, which is here by
  default 2e-5s (this default value is coming from Heccer, not from
  SSP).  The exact names of the options are specific to the solver,
  and are opaque to SSP.

  However, SSP does not automatically initialize the compartmental
  solver with a model.  It is possible to modify the structure of the
  model, or put it in a particular mode (eg. in-vivo mode), before
  feeding it to the rest of the system.  For initialization of the
  compartmental solver we need the point below.

\item Before the model can be compiled into the solver's internal
  representation, it needs a reference to what model must be
  simulated.  This is defined in a separate section of the SSP file:

\begin{verbatim}
models:
   granular_parameters:
     - component_name: '/Purk2M9s/segments/b1s20[14]/synchan/synapse'
       description: 'single synaptic event, variable dendritic location'
       field: EVENT_FILENAME
       value: event_data/events.yml
   modelname: /Purk2M9s
   solverclass: compartmental_solver
\end{verbatim}

  Two separate things happen here:

  \begin{itemize}
  \item the event file described above, is linked to a synaptic
    channel, just before the solver gets to see the model.  The
    creation of this synaptic channel is part of the initialization
    sequence of the model container (not shown, see the attached SSP
    file for the details).

  \item the model with name {\bf '/Purk2M9s'} will be feed to the SSP
    object {\bf 'compartmental\_solver'}.  This object was by its
    definition linked to {\bf 'model\_container'} in the point above,
    such that the name {\bf '/Purk2M9s'} effectively identifies the
    model in the neurospaces model container.
  \end{itemize}

\item For this simulation, I am interested in the somatic response
  after the unitary synaptic stimulation.  I first define an
  outputclass that converts floating point numbers to ascii.  The
  low-level code of this function is shipped as a commodity with
  \href{../experiment/experiment.tex}{the {\bf Experiment} package},
  and from perl it is available in the module Experiment (as the perl
  package {\bf Experiment::Output}).  Here is the required SSP snippet
  that makes this function available as an internal object of SSP
  under the name {\bf 'double\_2\_ascii'}:

\begin{verbatim}
outputclasses:
 double_2_ascii:
   module_name: Experiment
   options:
     filename: 'output_generate_single_stimulation_fixed_conductance/Purk2M9s/b1s20[14].output'
   package: Experiment::Output
\end{verbatim}

  Then, I link this output class to the membrane potential of the
  soma:

\begin{verbatim}
outputs:
 - component_name: /Purk2M9s/segments/soma
   field: Vm
   outputclass: double_2_ascii
\end{verbatim}

\item Now, everything has been defined to run the simulation.  The
  result of the previous steps is a single SSP object that connects
  all the parts for a deterministic run of a simulation (except for
  arithmetic rounding, it is a bug if the simulation is not
  deterministic -- numerical rounding is included by means of
  specifying the compartmental solver and its options).  Running the
  simulation is done as follows:

\begin{verbatim}
apply:
 simulation:
   - arguments:
       - 10000
     method: steps
\end{verbatim}

  This calls the internal {\bf 'steps'} method on the global SSP
  object that results after steps 1 - 5.  An more intuitive way to run
  the simulation is to call the {\bf 'advance'} method with a
  simulation time.

  Note again that SSP is independent of the model container, the
  compartmental solver, the input files and the output classes.  For
  instance, plugging in a reference to the neurospaces studio for a
  'solver' will allow you to explore the model that you want to
  simulate, using the GUI and SDL.  This exploration would run on a
  model, after the necessary modifications for this particular
  simulation run (eg. putting the model in a in-vivo mode, or after
  adding a synaptic channel for stimulating a known compartment).
  Another option is to plug in a JSON serializer for communication
  with a web browser.

  Running the simulation will produce a voltage trace with the somatic
  response after synaptic stimulation of compartment b1s20[14].  The
  exact format of the output file depends on the options given to the
  {\bf double\_2\_ascii} output class.  In this case the format would
  be backward compatible with
  \href{http://www.genesis-sim.org/GENESIS/Hyperdoc/Manual-26.html#ss26.7}{the
    GENESIS 2 {\bf ascii\_file} object} (no JSON / YAML), and looks
  something like this:

\begin{verbatim}
0 -0.08
1 -0.08
2 -0.08

...

19 -0.08
20 -0.08
21 -0.08
22 -0.0799999
23 -0.0799999

...


9999 -0.0799987
\end{verbatim}

\item Remember that these values are obtained after stimulation of
  compartment b1s20[14].  If we run a simulation like this, for every
  compartment in the model, we get 1600 raw output files, as there are
  1600 compartments in this model.  Now we would like to process these
  files, to extract the somatic time-to-peak and peak amplitudes.  The
  processing of these files is done using a research-project specific
  script written in perl: this script reads all the raw output files,
  extracts ttp and peak amplitude, and puts them in an aggregate
  output file that has the following format (only amplitude shown):

\begin{verbatim}
---
'/Purk2M9s/segments/b0s01[0]/synchan': -0.0798088
'/Purk2M9s/segments/b0s01[10]/synchan': -0.0798239
'/Purk2M9s/segments/b0s01[11]/synchan': -0.0798216
'/Purk2M9s/segments/b0s01[12]/synchan': -0.0798222
'/Purk2M9s/segments/b0s01[13]/synchan': -0.0798228
...
'/Purk2M9s/segments/b1s20[14]/synchan': -0.0799105
...
\end{verbatim}

  In this file, each line maps a model component -- the stimulated
  compartment -- to a value, in this case the amplitude of the somatic
  synaptic response.  Sometimes the model component will be encoded
  using an integer, for sake of brevity and to enhance compression
  ratios when compressing the file (this will reduce client-server
  communication time when using the project browser).

  Then, this file can be used by the neurospaces studio to colorcode
  the morphology for the Purkinje cell dendritic tree, and -- as an
  example only -- this same file can be used to modify the channel
  densities in the dendritic compartments, because it is essentially a
  simple mapping from compartments to numbers.

  I can give more examples for the interested.  A slightly different
  and interesting example is a voltage clamp, because the voltage
  clamp circuitry runs as a separate solver.

  What I really hope to make clear with this to short explanation, is
  how Genesis 3 works internally, I hope to make clear how it can be
  told exactly what to do, and how input and output are connected to
  the model.  I also hope that it is clear that the explanation is
  relatively independent from the numerical solution method, and from
  the structure of the model, input and output.  It is in the end that
  by abstracting these, that we are able to add new and powerful
  software layers to the simulator, including the already mentioned
  project browser.
\end{enumerate}

His explanation sheds some light on how the internals of the GENESIS 3
simulator work.

The attached SSP configuration, as any other JSON file, can be
converted to XML using the following command:

\begin{verbatim}
$ perl -e 'use YAML; use XML::Simple; local $/; print
XMLout(YAML::Load(<>), NoAttr => 1, RootName => SSP)' 'b1s20[14].yml'
\end{verbatim}

\begin{verbatim}
--- !perl/SSP
apply:
  modifiers:
    - arguments:
        - /usr/local/neurospaces/models/library/channels/nmda.ndf
        - nmda_namespace
      method: import_qualified_filename
      object: neurospaces
    - arguments:
        - nmda_namespace::/NMDA_fixed_conductance
        - '/Purk2M9s/segments/b1s20[14]/synchan'
      method: add_component
      object: neurospaces
  simulation:
    - arguments:
        - 10000
      method: steps
models:
  - conceptual_parameters:
      - component_name: a::/Purk_spine
        description: set resting membrane potential to steady state
        field: Vm_init
        value: -0.0800
      - component_name: b::/maind
        description: set resting membrane potential to steady state
        field: Vm_init
        value: -0.0800
      - component_name: c::/soma
        description: set resting membrane potential to steady state
        field: Vm_init
        value: -0.0800
      - component_name: d::/spinyd
        description: set resting membrane potential to steady state
        field: Vm_init
        value: -0.0800
      - component_name: e::/thickd
        description: set resting membrane potential to steady state
        field: Vm_init
        value: -0.0800
    granular_parameters:
      - component_name: '/Purk2M9s/segments/b1s20[14]/synchan/synapse'
        description: 'single synaptic event, variable dendritic location'
        field: EVENT_FILENAME
        value: event_data/events.yml
    modelname: /Purk2M9s
    solverclass: compartmental_solver
name: 'all models, all dendrites, single synaptic stimulation, fixed synaptic conductance'
outputclasses:
  double_2_ascii:
    module_name: Experiment
    options:
      filename: 'output_generate_single_stimulation_fixed_conductance/Purk2M9s/b1s20[14].output'
    package: Experiment::Output
outputs:
  - component_name: /Purk2M9s/segments/soma
    field: Vm
    outputclass: double_2_ascii
services:
  model_container:
    initializers:
      - arguments:
          - filename: /var/neurospaces/simulation_projects//purkinje-comparison2/morphologies/lab/gp/Purk2M9s.p
          -
            - ./generate_single_stimulation_fixed_conductance
            - -R
            - -A
        method: load
    module_name: Neurospaces
solverclasses:
  compartmental_solver:
    constructor_settings:
      configuration:
        reporting:
          granularity: 1
          tested_things: 6225920
      options:
        iOptions: 4
    module_name: Heccer
    service_name: model_container
\end{verbatim}

\end{document}

%%% Local Variables: 
%%% mode: latex
%%% TeX-master: t
%%% End: 
