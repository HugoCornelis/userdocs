\documentclass[12pt]{article}
\usepackage{verbatim}
\usepackage[dvips]{epsfig}
\usepackage{color}
\usepackage{url}
\usepackage[colorlinks=true]{hyperref}

\begin{document}

\section*{GENESIS: Documentation}

{\bf Related Documentation:}
% start: userdocs-tag-replace-items related-do-nothing
% end: userdocs-tag-replace-items related-do-nothing

\section*{De Schutter: Purkinje Cell Model}

\subsection*{CALCIUM CONCENTRATION}

The distribution of Ca$^{2+}$ during
Purkinje cell activity gives clues to the underlying ionic
mechanisms. This is most true for the dendrites where large
Ca$^{2+}$ fluctuations were seen. The Ca$^{2+}$ concentration in the
soma remained very low (Figs. 10, E and F, and 11 A),
because only CaT channels were present in the soma. We
do not claim that our simulation ofthe Ca$^{2+}$ concentration
in the soma was realistic. However, because there were no
Ca$^{2+}$-activated K$^+$ channels in the soma, it did not affect
the behavior of the model and we have not tried to optimize it
further.

During dendritic spikes, the submembrane Ca$^{2+}$ concentration 
attained a sharper peak than the membrane voltage
(Fig.\,11, {\it C} and {\it D}). It started to increase with a delay 
compared with the depolarization and the CaP channel activation, because Ca$^{2+}$ concentration under these conditions was mainly determined by the integral of Ca$^{2+}$ inflow. It
also decayed rapidly, so that it returned to baseline together
with the membrane potential. The Ca$^{2+}$ images in Fig. 10,
{\it E} and {\it F}, show the same pattern, i.e., the Ca$^{2+}$ rise was
delayed compared with the depolarization and showed
steeper distributions over the dendrite. This specific temporal
and spatial pattern of Ca$^{2+}$ elevation restricted the
activation of Ca$^{2+}$-activated K$^+$ currents (at these elevated
concentrations, mainly KC).

\subsection*{CALCIUM SPIKES}

As the current amplitude injected in a 
Purkinje cell increases, dendritic Ca$^{2+}$ spikes become apparent
as depolarizing spike bursts in the soma. Again, the
model reproduced this behavior reasonably well (Figs.\,3, {\it B}
and {\it C}, and 4, {\it C} and {\it D}). In experimental somatic recordings
the transition between simple spikes and depolarizing
spike bursts is abrupt, whereas the model often showed one
or two small bursts before a full dendritic spike became
apparent.

The transitions between the firing of simple somatic
spikes and the firing of dendritic spikes causes a break in the
linearity of the frequency current ($f-1$) curve, with a second,
shallower slope above this transition. Our $f-I$ curves (Fig.\,6{\it A}) 
were remarkably similar to those published by Llinas
and Sugimori (\cite{R:1980ly}, their Fig.\,5). In the model, dendritic
spike firing frequency remained relatively constant (16--19\,Hz 
in both models) once the threshold was crossed and was
rather insensitive to the current amplitude.

\subsection*{DENDRITIC CURRENTS AND THE CALCIUM SPIKE}

The channels
contributing to the dendritic spike operated in the
same manner in the smooth (Fig.\,11{\it C}) and spiny (Fig.\,11{\it D}) 
dendrites, but the spike itself was much bigger in the
spiny dendrites because of the higher input impedance of
the smaller dendritic branches. The dendritic spikes were
generated by the CaP channel; the CaT channel did not
play a role in spike generation (except during a rebound
spike from hyperpolarization). The dendritic spikes were
repolarized by the KC channel, which activated toward the
end of the spike and deactivated rapidly afterwards. Although
the K2 current amplitude increased a bit during the
spike, it was already quite active before the spike, when it
was a stronger outward current than the KC channel (Fig.\,11{\it C}, {\it inset}).

\subsection*{CONTROL OF DENDRITIC SPIKE GENERATION}

Control of the alternation of somatic and dendritic spiking that determined
the overall behavior of the cell in the simulation was
a complex combination of all the events discussed so far.
However, the important point is that the model clearly demonstrated
that the Ca$^{2+}$-activated K$^+$ channels interacted
with the steady, plateau-generating CaP activation as a negative
feedback loop. The K2 channel dominated this effect
because it had a much lower threshold of Ca$^{2+}$ activation
than the KC channel, resulting in K2 currents larger than
the KC currents between dendritic spikes (Fig.\,10{\it C}, {\it inset}).
One of the clear characteristics of Purkinje cell responses
to large somatic current injections is the delay seen in activation
of full-blown calcium spikes (Fig.\,3, {\it B} and {\it C}) and
then the alternation of dendritic calcium spikes and periods
of somatic action potential firing. In the model this phenomenon
was directly related to the interaction between
the CaP and K2 channels. Initially, depolarization resulted
in a slow buildup of CaP current, which resulted in an increase
of the internal dendritic Ca$^{2+}$ concentration (e.g.,
0.17\,$\mu$M at 200\,ms). This activated a large number of K2
channels, effectively countering the depolarization caused
by the CaP channel. As a consequence after the onset of
current injection there was a fluctuation of the baseline
membrane potential in the dendrites (Fig.\,3{\it B}), but no true
dendritic spikes.

During the current injection, the persistent CaP current
was progressively reduced by $\sim$\,50\,\% because of the slow
inactivation of the CaP channel until a steady state was
reached. This caused a corresponding reduction in the Ca2+
concentration (e.g., 0.11\,$\mu$M between dendritic spikes at
900\,ms) and less activation of the K2 channel, so that the
CaP channel conductance could overcome the counteracting
influence of the K2 channels, resulting in a full dendritic
spike. Once initiated the large influx of Ca$^{2+}$ then
caused the sequence described above for the Ca$^{2+}$ spike,
with the dendrite being repolarized by the KC channel.
Thus it appears that the CaP channel effectively controlled
dynamically the threshold of activation of dendritic
spike firing through a negative feedback from the K2 channel.
Because the K2 channel was sensitive to small changes
in Ca$^{2+}$ concentration caused by the beginning activation
of CaP channels and because the opening of K2 channels
counteracted CaP channel activation, K2 channel activation
could effectively increase the threshold for Ca$^{2+}$ spike
generation. In other words, when the K2 current was high
the CaP channel could only cause plateaus. To generate
dendritic spikes the K2 current had to be low and the CaP
channel had to activate fast. Vice versa, high CaP plateau
currents suppressed dendritic spiking because they activated
most of the K2 channels. This also explained the lack
of full-blown dendritic spikes during the prolonged episodes
of somatic depolarization with high-amplitude
current injections (Fig.\,3{\it D}).

\subsection*{MODELING CALCIUM CONCENTRATIONS}

For a neuron with
Ca$^{2+}$ activity like the Purkinje cell, the detailed modeling of
Ca$^{2+}$ represents one of the biggest technical challenges.
Modeling the diffusion, pumping, internal uptake and release,
cytosolic buffering, etc, of Ca$^{2+}$ is a formidable computational
challenge\,\cite{Sala:1990ys, Yamada-W:1989bs}. 
In the current version of the model we
used a very simple one-shell scheme to model Ca$^{2+}$ concentration.
Calcium concentration in this shell decreased with
a fast exponential decay despite the fact that Ca*+ concentration
is regulated by a complex interaction of, among
others, metabotropic receptors\,\cite{Llano:1991kx, Staub:1992zr}, 
cytoplasmic stores with IP, and ryanodine receptors\,\cite{Takei:1992ac}, 
Ca$^{2+}$ binding proteins like calbindin
and parvalbumin\,\cite{Kadowaki:1993ly}, and Ca$^{2+}$
inflow through the Ca$^{2+}$ channels\,\cite{Hockberger:1989ve, Lev-Ram:1992vn, Ross:339qf}. 
Clearly there is tremendous room for improvement of the model in this
regard.

Our principle reason for modeling internal Ca$^{2+}$ concentration
involved the regulation of conductance in the Ca$^{2+}$-
activated K$^+$ channels. In this case the most relevant region
of the cell is the area immediately adjacent to the internal
membrane, which is presumably where the mechanism for
activating the K$^+$ channels operates. As presented in the
RESULTS, the Ca$^{2+}$ concentration we modeled provided
both the amplitude and fast changes necessary to activate
the BK channel properly. However, at 6\,$\mu$M peak concentrations
[Ca$^{2+}$] was much higher, and the change in concentration
much faster, than those reported for the Purkinje
cell in the literature\,\cite{Hockberger:1989ve, Ross:339qf, Sugimori:1990kx, Tank:1988bh}. One
likely explanation for this discrepancy is the time constants
involved in experimentally measuring Ca$^{2+}$ concentration.
Each of these measurements were made with the calcium
indicator fura-2, whose slow kinetics filter out fast transients
of internal calcium\,\cite{Vranesic:1991dq}.
However, from the Ca$^{2+}$ sensitivity of the BK channel\,\cite{Franciolini:1988fu, Lancaster:1991ye, Reinhart1989:xe, Smart:1987mi} we know that the Ca$^{2+}$ concentrations
must reach the micromolar range. In other neuronal systems
Ca$^{2+}$ concentrations in the micromolar range\,\cite{Muller:1991cr} 
and even in the 200 to 300\,$\mu$M range\,\cite{Llinas:1992nx} 
have been reported. For these reasons
we suspect that the rapid, high-amplitude changes in Ca$^{2+}$
concentration that are necessary in the model to activate
the BK channel properly may very well more accurately
resemble the real submembrane peaks during Ca$^{2+}$ spikes.
However, there is ample evidence for a slower buildup of
Ca$^{2+}$ concentration during repetitive Ca$^{2+}$ spiking\,\cite{Lev-Ram:1992vn, Ross:1990oq}. 
These slower transients
are not reproduced by our model (Fig.\,11, {\it C} and {\it D}) because
of the fast time constant of Ca$^{2+}$ removal in the shell.
A more sophisticated model of Ca$^{2+}$ removal\,\cite{Sala:1990ys, Yamada-W:1989bs} will be necessary
to investigate the consequences of this slow increase
in Ca$^{2+}$.

\bibliographystyle{plain}
\bibliography{../tex/bib/g3-refs.bib}

\end{document}
