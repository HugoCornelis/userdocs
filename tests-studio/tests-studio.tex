\documentclass[12pt]{article}
\usepackage{verbatim}
\usepackage[dvips]{epsfig}
\usepackage{color}
\usepackage{url}
\usepackage[colorlinks=true]{hyperref}

\begin{document}

\section*{GENESIS: Documentation}

{\bf Related Documentation:}
% start: userdocs-tag-replace-items related-do-nothing
% end: userdocs-tag-replace-items related-do-nothing

\section*{Studio Regression Tests}

The {\bf Studio} provides a GUI front-end to the \href{../model-container/model-container.tex}{\bf Model Container} that supports browsing and visualization of a model loaded into the \href{../gshell/gshell.tex}{\bf G-Shell}. Note that the {\bf Studio} is neither a graphical editor nor model construction kit.

For more details see the \href{../studio/studio.tex}{\bf Studio} documentation.

\subsection*{Obtaining Regression Test Specifications}

This \href{http://www.neurospaces.org/neurospaces_project/studio/tests/html/index.html}{\bf link} accesses all regression test specifications for the {\bf Studio}.

\subsection*{Regression Test Documentation}

{\bf Studio} regression test documentation includes:
\begin{itemize}
\item The startup command.
\item The input to the test.
\item The expected output of the test.
\end{itemize}
The regression tests give a good overview of {\bf Studio} functionality. They consist mainly of queries to the {\bf Model Container} using the {\bf Studio} scripting interface.

\subsection*{Scope of Regression Tests}

Regression tests cover the following major functionalities of the {\bf Studio}:
\begin{itemize}

\item[]\href{http://www.neurospaces.org/neurospaces_project/studio/tests/html/specifications/main.html}{\bf Specifications}

Application of operators to reported fields; information reporting on loaded models.

\end{itemize}

\end{document}
