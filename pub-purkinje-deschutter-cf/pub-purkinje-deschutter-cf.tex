\documentclass[12pt]{article}
\usepackage{verbatim}
\usepackage[dvips]{epsfig}
\usepackage{color}
\usepackage{url}
\usepackage[colorlinks=true]{hyperref}

\begin{document}

\section*{GENESIS: Documentation}

{\bf Related Documentation:}
% start: userdocs-tag-replace-items related-do-nothing
% end: userdocs-tag-replace-items related-do-nothing

\section*{De Schutter: Purkinje Cell Model}

\subsection*{Source}

De Schutter E \& Bower JM (1994) An active membrane model of the cerebellar Purkinje cell II. Simulation of synaptic responses. {\it Journal of Nerurophysiology}. {\bf 71}: 401--419.

\subsection*{Climbing Fiber Synapses}
\subsubsection*{SYNAPTIC LOCATIONS}
It is well known that in adults the climbing
fiber input to a single Purkinje cell is provided by a single
climbing fiber synapse evokes an all-or-none massive response,
neuron in the inferior olive\,\cite{Ito:1984uq} and that this contact is
made on the smooth branches of the Purkinje cell dendritic tree,
from close to the soma up to almost the top of the tree\,\cite{Palay:1974fk}.
We modeled the climbing fiber input by placing
synapses from a single axon on all compartments of the main
and smooth dendrites of the cell (see Fig. 1 in\,\cite{deschutter94:_purkin_i}).

\subsubsection*{SYNAPTIC PHARMACOLOGY AND KINETICS}
Activation of the climbing fiber synapse evokes an all-or-none massive response
called the complex spike\,\cite{Eccles:1966kx, Llinas:1980vn}, which involves activation of dendritic Ca$^{2+}$ channels\,\cite{Knopfel:1991ys, Konnerth:1992zr, R:1980pi, Miyakawa:1992ly}.
Climbing fiber inputs are known to
be mediated by $\alpha$-amino-3-hydroxy+methyl-4-isoxazolepropionic acid (AMPA) receptors\,\cite{Knopfel:1990ve, Llano:1991qf}.
In the model, activation kinetics were based on data from
spinal cord and hippocampal neurons\,\cite{Forsythe:1988bh, Holmes:1963dq, Nelson:1986cr},
with an opening time constant of 0.5\,ms and a closing time constant of 1.2\,ms
(peak at 0.8\,ms). Ca$^{2+}$ inflow through this receptor was not modeled,
although it might be present\,\cite{Brorson:1992nx}. This synapse had a reversal potential of 0\,mV\,\cite{Cull-Candy:1989oq, Mayer:1987kl} .

\subsubsection*{SYNAPTIC CONDUCTANCE}

The conductance of each climbing
fiber synaptic connection is unknown and was therefore treated as
a free parameter in the model. Model tuning resulted in the selection
of two different values of maximum synaptic conductance
($\bar g$): 7.5\,mS/cm$^2$ for the main dendrite and 15\,mS/cm$^2$ for the
smooth dendrites. Assuming a total of 300 synaptic contacts\,\cite{Ito:1984uq}, each with the same synaptic conductance, our values
correspond to a conductance of 6.2\,nS at each synaptic contact.

\subsubsection*{SYNAPTIC SPECIALIZATIONS}

Climbing fiber synapses are
known to occur on a specialized ``stubby protuberance'' with a
short, smooth stem\,\cite{Palay:1974fk}. We did not,
however, explicitly model these structures because we do not expect
their absence in the model to affect the voltage transients
caused by climbing fiber activation. Such smooth spines are unlikely
to result in much signal attenuation\,\cite{Rall:1990tg}
and all spines on any particular compartment receive synchronously 
the same synaptic input from a single climbing fiber\,\cite{Ito:1984uq}.

\subsubsection*{SYNAPTIC ACTIVATION}

The climbing fiber synapses were fired as
a volley, with the main dendrite being activated before the more
distal smooth dendrites. The delay between first and last activated
synaptic conductance was 0.9\,ms\,\cite{Llinas:1980vn}.

\bibliographystyle{plain}
\bibliography{../tex/bib/g3-refs}

\end{document}
