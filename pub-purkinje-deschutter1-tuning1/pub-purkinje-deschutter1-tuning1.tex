\documentclass[12pt]{article}
\usepackage{verbatim}
\usepackage[dvips]{epsfig}
\usepackage{color}
\usepackage{url}
\usepackage[colorlinks=true]{hyperref}

\begin{document}

\section*{GENESIS: Documentation}

{\bf Related Documentation:}
% start: userdocs-tag-replace-items related-do-nothing
% end: userdocs-tag-replace-items related-do-nothing

\section*{De Schutter: Purkinje Cell Model}

\subsection*{Tuning the model to replicate in vitro current injections}

The final stage of constructing a complex realistic model involves tuning model parameters to replicate well-described properties of the neuron in question\,\cite{S:1993dz}. Once the model is tuned, additional questions can be explored by holding model parameters fixed and applying different types of inputs (see\,\cite{deschutter94:_purkin_ii}).

Here we  describe the results of optimizing the Purkinje cell model to replicate the characteristic firing patterns of Purkinje cells in slices during current injection in the soma or dendrites. This optimization, or tuning, primarily involved changing the densities and distributions of the different voltage sensitive channels and the decay time constant of the Ca$^{2+}$ concentrations until the model generated biologically realistic output. Due primarily to the work of\,\cite{R:1980ly, R:1980pi}, considerable information is available concerning the intracellular response properties of cerebellar Purkinje cells to current injection in vitro in the slice preparation in the presence and absence of different channel blockers. Accordingly, all the following statements about normal Purkinje cell behavior refer to their work unless a different study is cited.

\bibliographystyle{plain}
\bibliography{../tex/bib/g3-refs.bib}

\end{document}
