\documentclass[12pt]{article}
\usepackage{verbatim}
\usepackage[dvips]{epsfig}
\usepackage{color}
\usepackage{url}
\usepackage[colorlinks=true]{hyperref}

\begin{document}

%one cannot achieve spiritual enlightenment without engaging the world.

\section*{GENESIS: Documentation}

{\bf Related Documentation:}
% start: userdocs-tag-replace-items related-do-nothing
% end: userdocs-tag-replace-items related-do-nothing

\section*{Question raised during initial development of Chemesis-3}

\subsection{2011/06/25: Writing custom G-3 extensions}

It appears that the separation of a model into a biological component
in the model container, and mathematical components in solvers can
make the process of constructing new objects more difficult in G-3
than in G2. Can we minimize the complexity of this, while maintaining
the structure of G-3?

\paragraph{response} This may be the way it appears, but it is
certainly not true.  Because of the strict separation between the
biological aspects of a model and their numerical representation,
extension of G-3 is very different from extension of G-2 and requires
many more steps.  However, many of these steps can be automated and
with the exercise of integrating Chemesis-3, we actually already
automated some of them.  We are currently looking for other solvers to
be candidates for integration.


\subsection{2011/06/25: Selective NDF\_SAVE}

Is there an option to ndf\_save that will allow me to save only /cell?
If not, can I give a model-container command from the gshell that is
equivalent to the one used in conversions.t to generate
simplecell-all.ndf?

What is the easiest way to extract the channel models to get .ndf
files for the channels that appear in converted cell models?  Of
course, I could hand-edit the cell .ndf file to create ones for the
each of the channels, and make a new NDF file for the cell that
references the channels in the library, instead of having all the gate
representations be in the cell file.  But, there has to be a better
way.

\paragraph{response} This needs to be implemented.


\subsection{2011/06/22: Browsing user documentation}

I find it easier moving linearly through a users guide/tutorial, with
occasionally being referred to another document for more in depth
explanations.  I don't want to be sent to different places frequently.

\paragraph{response} The documentation system has no layout algorithm
yet that shows the narrative flows in which the displayed document
participates.


\subsection{2011/06/22: Types of user documentation}

I would be tempted to create 2 (3) tutorial tracks:
\begin{itemize}
\item one for novices, which would basically translate the G-2 tutorials
  into G-3
\item one for G-2 users, which would focus on the new G-3 commands,
  and model compatibility.
\item Possibly a third tutorial track for just running existing
  models, perhaps similar to BoG part 1.
\end{itemize}

\paragraph{response} This has a relationship with the layers of the
old G-2 users community.  See the
\href{../genesis-overview/genesis-overview.tex}{\bf GENESIS overview}.


\subsection{2011/06/14: Morphology}

\begin{verbatim}
morphology_summarize /Purkinje
\end{verbatim}

and also this one:
\begin{verbatim}
morphology_list_spine_heads
\end{verbatim}
but I'm not sure how that command works.  I.e., why are spine\_heads
special?  There is no morphology\_list\_branches.  I would expect that
spines are just additional compartments (at least I treat them that
way in terms of calcium, etc.)


\subsection{2011/06/14: Simulation Constants}

\begin{verbatim}
runtime_parameter_show /Purkinje/segments/soma/cat G_MAX
\end{verbatim}
As I suspected, you consider G\_MAX a simulation constant and not a
model constant.  I am strongly opposed to this.  G\_MAX is no more a
simulation constant than is tau, or V\_half for a channel.

\paragraph{response} When a parameter in the model-container is fixed
to a value, it becomes a model parameter.  But it is important to
understand that the model-container does not make a rigid distinction
between parameter and variables.  For instance binding the same model
parameter to the value of a variable, such as the membrane potential
in a compartment, will turn it into a variable too.

The solvers are responsible to recognize the variables in the model,
or to emit an appropriate error message when the model is
parameterized in a way beyond the solver's capabilities.


\subsection{2011/06/14: Event time tables}

\begin{verbatim}
runtime_parameter_add /Purkinje/segments/branchlets/b1s06/b1s06[182]/Purkinje_spine_0/head/par/synapse EVENT_FILENAME event_data/events.yml
\end{verbatim}

gives me no error message, even though I didn't replace
"EVENT\_FILENAME" with an existing filename.  But I like this
capability.  This is what we use the time\_table for in genesis 2.  Is
there more information on this capability?

\paragraph{response} Currently not, but there are test cases for it in
the gshell, see the test specification single\_synapse.t and
\href{http://neurospaces.sourceforge.net/neurospaces_project/gshell/tests/html/specifications/single_synapse.html}{its
  html version}.  In the model library, the file
'event\_data/events.yml' is an example of an event file.


\subsection{2011/06/14: Renaming a cell model}

How to a load the cell model but give it a different name, or make an
identical copy of a cell, and give it a different name?

\paragraph{response} currently this can only be done by modifying the
NDF file.


\subsection{2011/06/14: Combining existing models}

I would like to combine some of the gates in the NDF library to create
additional channels (e.g., there are no recognizable calcium channels
in the channels library), or modify some of the parameters to tailor
the channels, but I don't know how to do either of these.  Which
documentation would be most helpful for this?


\subsection{2011/06/14: NDF for CHEMESIS-3}

Note that the purpose of NDF is to maybe be readable, but not
necessarily to verify or write.  For most models I have worked with,
it was actually human readable.  For models that resulted after a
conversion from G-2 it was unreadable to a human in most cases.



1. I see you have parameter thick, and then it's commented out with
//? so I'm not sure if you're using it.  I prefer having an outer and
inner diameter or radius, to allow for shells, with the inner radius
allowed to be zero for the inside shell.  Then, the object itself can
calculate thickness as well as all the needed surface areas.  I
suppose with thickness the inner diameter and all surface areas can be
calculated, but I don't see as they are existing fields in the pool
object.

\paragraph{response} This is an interesting question.  The POOL
component is already used for concentrations with exponential decay.
Depending on the model it can use THICK and DIA (inferred from the
parent segment) to compute BETA, and for other models it is convenient
to specify BETA directly.  Generally speaking the implementation of a
component type can define convenience relationships between the
different parameters.  An trivial example is the relationship between
a segment's TAU, RM and CM.

In the example I have used the same POOL model type that is used by
the other models such as the Purkinje cell, the Traub model ao.  I
don't know yet if this is a good idea.


2. I wouldn't call the various pools substrate\_somaCaBuf or
product\_somabuf.  Though certainly that is how they act in the
present simulation, all pools can simultaneously be substrates or
products, so those names will be confusing as soon as the model is
extended at all.

\paragraph{response} these have been renamed.


3. I don't understand the concept of groups, but I think I understand
the concept of bindings - similar to messages?  Are bindables just the
inputs and outputs?

\paragraph{response} Bindings are similar to messages in the sense
that they define the mathematical connections (and also the
topological connections) between model components.  Bindings are
available in the model-container.  But during a simulation the solvers
can decide to do something clever and not compute some of the
variables.

For instance, a compartmental solver does not have to compute an ion
channel current, say soma/kdr->Ik, because it is not required to
compute the membrane potential.  However, when the user lists this
current as a desired output, the compartmental solver can reconfigure
itself to compute ion channel currents.  This is what Heccer does (the
current G-3 compartmental solver).  This is of course dependent on how
smart the implementation is.

A solver can also choose to compute variables that are not directly
available from a model's perspective.  As an example the average
membrane potential or calcium concentration of selected dendrites in a
single time step can be computed.  In G-3, you can tag model
components, then tell Heccer to compute an average value of the tagged
components.

Note that all these features work and are part of the automated test
cases.  The tagging is always available for new model components such
as the reaction type.



4. On the other hand, I definitely do not understand the code
following "An extension file with the definition:" which means I
wouldn't be able to do additional development (though I suppose that
comment is unfair given I'm still learning).

\paragraph{response} Correct: you don't have to write code to create a
new type.  In this case the configuration file mapped the reaction
type to an already existing implementation.  But of course it is
possible to map to a new implementation and in this implementation
code the logic specific to handle the parameters of that component.
For example, the implementation of a segment defines the relationship
between TAU, RM and CM and the implementation of a concentration pool
defines the relationships between volume, diameter, thickness, valency
and beta.

If the reaction.yml would map a reaction type to the segment
implementation, the new reaction type would know about TAU, RM and CM.
Of course this is not what you want, but I am just illustrating the
idea behind.


5. Related to groups, one of the nice things about NeuroRD is the
ability to define a reaction independent of the morphology, and then
the reaction occurs wherever there are the appropriate substrates.  An
good alternative for G-3 is to define a reaction independent of
groups, and then specify which parts of the morphology it occurs in.
In a large model, you don't want to have to define 100 identical
reactions, HOWEVER, that is what I did with G-2 - e.g. I looped
through my 2D array of calcium pools, and created a 2D array of
reactions (pseudo 2D since genesis doesn't allow 2D arrays of
elements).

calcium - the compartments must be much smaller than voltage
compartments.  So, it is necessary to subdivide each voltage segment
or whatever you call them, into multiple calcium compartments.  The
calcium channels bridge these two compartments.  I.e., each calcium
compartment might have its own calcium channel, but each of these
channels receives an identical voltage message.  As an approximation,
you could just take the calcium channel in the voltage compartment,
and subdivide the current by the number of calcium compartments.
However, using GHK to calculate driving potential, each of these
calcium comp could have a different conc, e.g. if it had different
synapses, and thus a different driving potential.


\paragraph{response} The model-container allows to instantiate
repetitive model components in different ways.

It is possible to group arbitrary components and then to reference the
group from many places.  This does not necessarily mean that all
(virtual) instances identified by these references must be
mathematically identical.

As an example, the implementation of a segment defines the parameter
SOMATOPETAL\_DISTANCE (I might be wrong on the name of this
parameter).  The value of this parameter is dynamically inferred from
the path to the soma.  Using the notation '..->DISTANCE' in the
definition of a reaction, pool or any other component allows to set
the relationship of a parameter as dependent of its location in a
morphology.

As I explained in my previous email, the model-container works with
types (data) and algorithms.  A second way of defining repetitive
model components is to first declare 10000 identical model components
then to randomize them (or do something else) using the algorithm
specific for this purpose (it is called Randomize I believe, but have
to look up again).

A third way of defining repetitive model components that is specific
to reading morphologies, is to look at a parameter of a dendritic
section in the morphology, for instance its diameter, and depending on
the value of this parameter map the dendritic section to a predefined
compartmental prototype.  The compartmental prototype may have
arbitrary complexity.  The tool morphology2ndf implements this type of
functionality, internally in the model-container it works with
references to model components.  Section 5 of
http://www.genesis-sim.org/userdocs/technical-guide-1/technical-guide-1.html
explains this.


6. how to create a passive morphology - either specifying a few
"compartments" by hand or reading in a cell from neuromorpho.org, and
how to create different ionic channels.  Second step would be creating
(or using if it exists) either GHK (for Ca channels) or Mg block (for
NMDA channels).


7. The POOL token sounds like the old genesis ca\_concen object.  This
is entirely not sufficient for doing general concentration pools.  In
particular, you absolutely cannot have any exponential decay.  All
decay is provided by various mechanisms.  And you can't have 2-D (or
even 1-D) diffusion without additional geometry factors specified.  By
the time you add and remove stuff, you no longer have the old simple
exp decay.

\paragraph{response} The model-container stores all the different
HH-channel types with the same CHANNEL token.  This token is used for
channels with activation, both activation and inactivation, and for
channels with a concentration dependent gate.

The CHANNEL implementation defines a derived parameter, called
CHANNEL\_TYPE that contains a value that changes dynamically with the
number and type of HH-gates internal to the channel.  This is
convenient for a solver when it needs to allocate its internal
structures, and it is convenient for a user when exploring a model.

The CHANNEL\_TYPE parameter is dynamic in the sense that adding or
removing a gate from a channel will change its value immediately.  The
value is never stored in memory.

The current implementation of the POOL token is exactly that: a pool
of ions, whether exponential decay or with other dynamics.  A modeler
is assumed to think of a pool of ions, independent of whether its with
simple decay, 1D or 2D diffusion.  The parameterization of the pool
token and its internal structure will define the behavior at
simulation time.

Currently three different parameterizations of the POOL token will map
it to one of the G-2 ca\_concen object, the Chemesis-3 pool, or the
Chemesis-3 conservepool objects.

For a user who is exploring a model, this means that querying the
model for all concentration pools and then inspecing the parameters of
the pools becomes trivial.


\subsection{2011/06/14: Creation of the most simple simulations}

If I wanted to create a single cell model with various channels. I
discovered only a few synaptic channels and hodgkin-huxley channels.
Then I looked at the introduction to ndf file format, which wasn't too
helpful, but
http://www.genesis-sim.org/userdocs/model-variables/model-variables.html
was probably the best document, though still not sufficient.  This
shows me how to create a synaptic channel, but I don't need another
synaptic channel.  Presumaly I can just take the existing ones and
modify tau.  It would be nice to see the ndf for a vdep channel, then
I would better understand how to modify it to create other channels.

I must confess another command I really like is the listobjects
commands in G2. I could see what existed, look at the fields and brief
explanation of what it did.  Also, I could look at the source code to
figure out how it worked (I know that this latter step is done by
almost noone.)

\paragraph{response} The equivalent in the gshell of the G-2
listobjects command is 'help command create'.  There are also commands
with similar functionality: 'list structure' list all tokens known to
the model container to structure a model.  As another example 'list
physical' lists all tokens knowns to the model-container that
instantiate model components.  There are a couple of other 'list <this
and that>' commands, but they are of less interest here.


After the tutorials, I still don't know how to get started creating my
own model, i.e.

1. how to define a simple morphology or read in a realistic one

2. how to add channels to various compartments


\subsection{2011/06/14: Tests don't pass on the regression tester.
  What now?}

Then it is highly unlikely it will work on any other developer
machine.


\subsection{2011/06/08: Empty output when compiling the developer package???}

That may wel be.  'make \&\& sudo make install' is the standard way of
installing a Unix software package.  Because the developer package is
a minimal set of utilities, the 'make' phase may be a no-operation
depending on the revision of this package.  The 'sudo make install' is
always required.  Just continue with the installation of the other
packages as explained in the documentation.


\subsection{2011/06/03: Five ways to create simulations in GENESIS-3}

\begin{itemize}
\item Using the g-shell to issue commands (or to read them from a
  file): This is covered in Tutorials 1 and 2 below.

\item Using the "ns-sli" compatibility mode of the g-shell to load or
  run GENESIS 2 SLI scripts.  This is covered in Tutorial 3 below.
  SLI compatibility requires additional bindings than the ones used to
  make objects available to the g-shell.

\item Using Python as a scripting language.  This will become the
  prefered method of scripting in G-3.

\item Using ssp directly from the (unix) command line to load and
  control a model: This is usually used for testing new objects that
  don't have bindings to higher levels of scripting.  Typical commands
  are:

\begin{verbatim}
   $ ssp --help
   $ ssp '--time' '100.0' '--time-step' '5e-5' '--cell' 'traub91_100s.ndf'
   $ ssp --cell cells/purkinje/edsjb1994.ndf --time 0.5 --optimize
   $
\end{verbatim}

\item Using ssp configuration files.  This is used to run many similar
  simulations, for examples to compare the responses of different
  models to similar stimuli, or for a parameter searches.  A number of
  scripts are available that create these ssp configuration files.
\end{itemize}


\subsection{2011/06/14: Tutorial 1}

1. Instead of creating a cell called /n, I created one called /cell1 -
so far so good.  But, then later I forgot, and copied the tutorial
commands "output\_add /n/soma Vm" and there were no complaints/error
messages, even though I never created a /n object. It does give me an
error when I try to run it, but not before then.

2. I can't figure out how to examine the objects that exist for my
use, or the fields/parameters in those objects.
runtime\_parameters\_show shows me the parameters I created using that
command (including my "non\_existent" parameter). model\_parameter\_show
shows me parameters I've added.  list\_elements also helps, but I can't
examine the fields/parameters that already exist in the elements. Does
that mean that I can add any parameter?  That wouldn't make sense,
since parameters such as Rm act differently than parameters such as
INJECT.

3. set\_verbose warning doesn't seem to be doing what it says, though
set\_verbose information does.

4. Are SI units required, or can I use any set of consistent units?
FYI, I really hate being forced to use SI units, since my experiments
and publications all report stuff using physiological units.

\paragraph{response} Representation of parameter values in a
user-chosen unit system is a User Interface issue.  The gshell and its
python based alternative will do the conversion from SI to other unit
systems.  The model-container and solvers always use SI units.  This
is to prevent specification of one model in different unit systems.

Conversion between unit systems is a matter of multiplying and
offsetting the values (when they are numeric).  Dave has recently
written a document about this and we should incorporate this type of
functionality into the gshell such that a user can script a model
using any unit system he likes, and the model-container will always
see, store and provide SI based values to other components, including
(G)UIs and solvers.


5. In the discussion of simulation constants (tutorial1), it seems you
are claiming that conductance of the CaT channels is a sim constant
and not a model constant, but I would strongly disagree.  I think of
the model as analogous to the tissue an experimentalist records from.
conductances are definitely model parameters, where as dt, runtime,
mesh element size, current injection are simulation parameters or
simulation inputs.

6.  Continuing with tutorial1, I tried
\begin{verbatim}
genesis > ndf_load /n myneuron.ndf
\end{verbatim}
but it didn't work:
\begin{verbatim}
Could not find file (number 1, 1), path name (/n)
Set one of the environment variables NEUROSPACES_NMC_USER_MODELS,
NEUROSPACES_NMC_PROJECT_MODELS, NEUROSPACES_NMC_SYSTEM_MODELS or NEUROSPACES_NMC_MODELS
to point to a library where the required model is located,
or use the -m switch to configure where neurospaces looks for models.
*** Error: ndf_load /n
\end{verbatim}

when I used the -m switch, nothing happened, and there was no help for this command.

7. In the tutorial, you suggest storing output in /tmp :
\begin{verbatim}
genesis > sh cat /tmp/output
\end{verbatim}
but I would never want to write my output to that directory.  also,
how is that the default since you specify full path?  I would think
the default would be the directory from which you invoked genesis.
Strangely, this command is not working today.  I keep getting "No such
file or directory" no matter what filename or directory I specify.
OK, I figured it out.  In the tutorial discuss where simulation output
is, and then give the sh cat command \_before\_ you present "run
simulation".  So, of course an error is returned.  I recommend
changing the tutorial so that the sh cat command is explained after
the run simulation to prevent errors, or explain not to invoke the
command till after running the model.  Also, I would change the
default location of the files, and explain how to specify a file
location.

9. model\_state\_load doesn't appear to work.  I tried calling it two
different ways:
\begin{verbatim}
genesis > model_state_load tempmodel
*** Error: <modelname> and <filename> are required
\end{verbatim}
\begin{verbatim}
genesis > model_state_load /n tempmodel
HeccerConstruct: cannot find model /n
/usr/local/bin/genesis-g3: While running GENESIS3 SSP schedule initiated for /n, 0: compile failed (error: HeccerConstruct() from the model_container failed) at /usr/local/glue/swig/perl/SSP.pm line 1756.
 at /usr/local/bin/genesis-g3 line 54
       main::__ANON__('/usr/local/bin/genesis-g3: While running GENESIS3 SSP schedul...') called at /usr/local/glue/swig/perl/SSP.pm line 1756
       SSP::run('SSP=HASH(0x2270958)') called at /usr/local/glue/swig/perl/GENESIS3.pm line 1956
       GENESIS3::Commands::run('/n', 0) called at /usr/local/glue/swig/perl/GENESIS3.pm line 1182
       GENESIS3::Commands::model_state_load('/n', 'tempmodel') called at (eval 50) line 1
       eval 'GENESIS3::Commands::model_state_load( \'/n\', \'tempmodel\', )
;' called at /usr/local/bin/genesis-g3 line 171
       main::interprete('model_state_load /n tempmodel') called at /usr/local/bin/genesis-g3 line 242
       main::loop() called at /usr/local/bin/genesis-g3 line 307
       main::main() called at /usr/local/bin/genesis-g3 line 367
$Result = undef;
$
\end{verbatim}



\subsection{2011/05/28: Customizing GENESIS-3 (1)}

General question \#1: Is this currently possible, or are there parts of
the G-3 infrastructure that need to be implmented in order to make
this task straightforward?  At present, it seems to be a lot harder
than extending G2, although in principle it should be easier. We also
talked about a demo using Python scripting to show how G-3 can
interoperate with an external Python script.  In principle, this
script could invoke PyMOOSE or PyNEURON and incorporate part of a
model in one of these simulators, but I had something much simpler in
mind.  The simplest trivial case (which I would try first) would be a
simple application that makes a graph like G3Plot, but gets the data
directly from /cell/soma Vm as it runs, rather than from a file.  The
demo that I would like to write is an "efield application" (not an
added object) that works as I described in my "Interoperability
example" email, with a Python object that recieves compartment or
channel currents as the simulation runs, calculates the extracellular
potential, and sends it to a plot object.


\subsection{2011/05/28: Customizing GENESIS-3 (2)}

General question \#2: What needs to be done to make this demo
possible?  Is it already possible to have a Python script access
something like asc\_file and get the output directly as it is
generated?  Are their any problems associated with getting the Ik
values?  I was not able to send Gk to an asc\_file.  The
simplecell\_spike\_pulse.g in testchannels-0.9/simplecells was designed
to be another example to convert to Python, but failed to run in
ns-sli for many reasons.


\subsection{2011/05/28: Customizing GENESIS-3 (3)}

General question \#3: I am still trying to get a clear idea of whether
I should choose Python scripting, SLI backwards compatibility
scripting, or basic G-shell scripting for this release.

\subsection{2011/05/28: Customizing GENESIS-3 (4)}

General question \#4: Will scripting in G-3 Python (or anything else)
preserve the GENESIS paradigm of elements coupled with "messages"?  In
GENESIS 2 SLI, a "message" is not the usual one-time event of other
languages, but is more of a persistent data connection.


\subsection{2011/05/21: an efield object}

Suppose that I want to implement an efield object to measure the
extracellular potential at some point outside a Purkinje or traub94 cell.
It needs to know Ik in every compartment at specified intervals (probably
much less than the simulation time step), and the coordinates of each
compartment, in order to calculate the inverse square distance.  This can
be done by either (1) creating a G-3 object in C and making it available to
gshell, ns-sli, or the Python interface; or (2) creating an external
"efield application" in Python that loads and runs the cell model and gets
the needed information for its own Python efield object.

It is the second that I'm most interested in now, and that has the most
relevance to the chapter I'm writing, but I'd like to think ahead to
the first.  When you can find the time, could you try to give me some
general answers in terms of "what things do I need to know about?"
for these questions?  Later, I may pin you down for names of example
code files or specific documentation that exists.

1.  If i write an efield object in C, using the vclamp or PulseGen
code as an example

   a. How do I make it availble to the gshell?  I suspect that the docs
   genesis-extend-model-container will answer a lot of this.  What else do
   I need to know about in order to be sure that the efield object gets
   the Ik information at specified intervals?

   b. How would I make the new efield object accessible to a Python
   script, and use Python to set up the communication with the cell?
   This could be called interoperability of a sort, because Python
   is external to G-3. But it isn't really what I mean by
   interoperability because it requires changes to G-3 to add a new
   object.

   c.  What is involved in doing this for ns-sli?  Is it more or
   less work than doing it for Python?

2.  As a demo of "run-time interoperability" and a useful tool, I'd
like to write a Python script that, as in (1b), would import nmc and
heccer, then issue commands to load a cell model from a NDF file.
But, it would implement its own efield object in Python.

   a.  Do we yet have a way to load a cell from a NDF file in Python?
   The cbi-scripting/scripting-example files just create and intialize
   a soma compartment.  This would seem to be the next step.

   b.  How do I get the Ik information to the Python efield object in an
   efficient manner through the Python interface?

You can ignore question (1) for now.  I posed it just to show the contrast.
At some point I'll need to create documentation on doing this, after trying
it myself, but (2) is more important to me now, and I hope easier.



\bibliographystyle{plain}
\bibliography{../tex/bib/g3-refs.bib}

\end{document}
