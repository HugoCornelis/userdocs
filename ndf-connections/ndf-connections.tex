\documentclass[12pt]{article}
\usepackage[dvips]{epsfig}
\usepackage{lineno}
\usepackage{color}
\usepackage{url}
\usepackage[colorlinks=true]{hyperref}

\begin{document}

\section*{GENESIS: Documentation}

{\bf Related Documentation:}

\href{../ndf-variables/ndf-variables.tex}{\bf Physical Processes and Model Variables}

% start: userdocs-tag-replace-items related-do-nothing
% end: userdocs-tag-replace-items related-do-nothing

\section*{The Model of a Connection}

Connections are made between event-generators and event-receivers.
The Neurospaces / GENESIS 3 model-container defines a template for
event-receiving entities.  Projections then run
\href{../ndf-procedural-description/ndf-procedural-description.tex}{algorithms}
that instantiate these entities and attach them pre-synaptically to
event-generators and post-synaptically to continuous-time state
variables.  The template attributes of 'delay and 'weight' are
inherited by the template instantiations created by the projection
algorithms.

%A change to the template changes the synapse model without changing the
%projection structure nor the projection algorithm.


\section*{Computing Connections}

For most network models connections are algorithmically calculated,
for example based on the distance between neurons and the synaptic
contact points in the dendrite.  That is why the model-container
contains special {algorithms} to compute the connections between
neurons.

\section*{Storing Connections}

The connections in a network model can be stored in two different ways:

\begin{itemize}
\item Connections can be enumerated in the NDF file.  For example:
\begin{verbatim}
  CONNECTION_GROUP NMDA
    CONNECTION n1
      PARAMETERS
        PARAMETER ( PRE = Fiber[0]/spikegen ),
        PARAMETER ( POST = Granule[0]/Granule_soma/mf_NMDA/synapse ),
        PARAMETER ( WEIGHT = 1.0 ),
        PARAMETER ( DELAY = 0.0 ),
      END PARAMETERS
    END CONNECTION
    CONNECTION n2
      PARAMETERS
        PARAMETER ( PRE = Fiber[1]/spikegen ),
        PARAMETER ( POST = Granule[0]/Granule_soma/mf_NMDA/synapse ),
        PARAMETER ( WEIGHT = 1.0 ),
        PARAMETER ( DELAY = 0.0 ),
      END PARAMETERS
    END CONNECTION
  END CONNECTION_GROUP
\end{verbatim}
\item Because connections in a large network model easily outnumber
  the other components of a model, the model-container can store them
  in a memory efficient way.  Most algorithms that compute the
  connections in a network will store them in memory in a compact
  manner.  Some algorithms may compute the connections on the fly when
  they are needed.  This includes algorithms that compress the
  connection matrix.
\end{itemize}



\subsection*{Summary}

\end{document}


%%% Local Variables: 
%%% mode: latex
%%% TeX-master: t
%%% End: 
