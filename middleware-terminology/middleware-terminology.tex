\documentclass[12pt]{article}
\usepackage[dvips]{epsfig}
\usepackage{color}
%e.g.  \textcolor{red,green,blue}{text}
\usepackage{url}
\usepackage[colorlinks=true]{hyperref}
\usepackage{scrtime}
 
\begin{document}

{\bf Related Documentation:}
% start: userdocs-tag-replace-items related-do-nothing
% end: userdocs-tag-replace-items related-do-nothing

\section*{GENESIS: Documentation}

\section{Middleware Terminology}

The following software engineering terms are used in the GENESIS
documentation:

\begin{itemize}
\item Application server software: Software that provides services
  such as security or transaction handling to an application, so that
  the application can contain primarily business logic. An application
  server typically includes a web server, which handles HTTP requests
  from a browser.

\item Enterprise portal: Software that provides a framework and the
  tooling to build dynamic web solutions that deliver a unified,
  interactive interface to personalized information, applications,
  processes and people. It includes service oriented architecture
  (SOA) capabilities, role-based access and rule-based administration,
  collaboration functionality, content management and access to
  enterprise data sources, and integration with heterogeneous back-end
  business systems and processes.

\item Enterprise Service Bus (ESB): Software that evolved from and is
  closely related to message-oriented middleware, message brokering,
  and enterprise application integration (EAI). An ESB handles the
  matching and routing of communications between processes and
  services, converts between different transport protocols, transforms
  different data formats, and identifies and distributes business
  events.

\item Business Process Management (BPM) software: Software that
  includes integration, modeling, monitoring, forms, rules engine and
  workflow. This software is used to design, enact, control and
  analyze operational business processes involving humans,
  organizations, applications, content and other sources of
  information.

\item Service-Oriented Architecture (SOA): Software that provides a
  single platform to find, integrate and orchestrate SOA business
  services, enterprise applications, and other IT assets into
  automated business processes.
\end{itemize}

\end{document}


