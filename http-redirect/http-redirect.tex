\documentclass[12pt]{article}
\usepackage{verbatim}
\usepackage[dvips]{epsfig}
\usepackage{color}
\usepackage{url}
\usepackage[colorlinks=true]{hyperref}

\begin{document}

\section*{GENESIS: Documentation}

{\bf Related Documentation:}
% start: userdocs-tag-replace-items related-do-nothing
% end: userdocs-tag-replace-items related-do-nothing

\section*{HTTP Redirect}

At times it is necessary to direct a user to documentation that is off site since reproducing the content can be time consuming, or require special host configurations (e.g \href{http://www.youtube.com/user/genesissim}{\bf YouTube\,videos}). By using the {\tt redirect:} attribute in your documentation {\it descriptor.yml} file, you can link to URLs on the internet, effectively integrating the linked content into your own documentation.

The only file needed to add a http redirect to your documentation is the {\it descriptor.yml} file containing a user supplied URL and a tag to indicate the level of documentation. The following example shows how to do this:
\begin{verbatim}
---
comment: Optional comment.
description: A redirect to a website
document name: Off Site Documentation
redirect: http://www.directmehere.edu
tags:
  - contents-level5
  - published
\end{verbatim}
The structure is identical to previous {\it descriptor.yml} files with the exception of the redirect attribute. If present, this example would place a link to the URL content in the Level 5 contents page (indicated by the {\tt -\,contents-level-5} tag).

\end{document}
