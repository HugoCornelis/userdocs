% Template for PLoS
% Version 1.0 January 2009
%
% To compile to pdf, run:
% latex plos.template
% bibtex plos.template
% latex plos.template
% latex plos.template
% dvipdf plos.template

\documentclass[10pt]{article}

% amsmath package, useful for mathematical formulas
\usepackage{amsmath}
% amssymb package, useful for mathematical symbols
\usepackage{amssymb}

% graphicx package, useful for including eps and pdf graphics
% include graphics with the command \includegraphics
\usepackage{graphicx}

% cite package, to clean up citations in the main text. Do not remove.
\usepackage{cite}

\usepackage{color} 

% Use doublespacing - comment out for single spacing
\usepackage{setspace} 
\doublespacing

\usepackage{lineno}

% Text layout
\topmargin 0.0cm
\oddsidemargin 0.5cm
\evensidemargin 0.5cm
\textwidth 16cm 
\textheight 21cm

\begin{document}

\section*{A Federated Design for a Neurobiological Simulation Engine: 
The CBI federated software architecture.}
\subsection*{Cornelis H, Coop AD, and Bower JM}
\subsection*{Supplementary Materials 2:} 

RSnet.g is a customizeable GENESIS 2 script for creating a network of simplified
Regular Spiking neocortical neurons with local excitatory connections.
The simulation script is analyzed and explained in the GENESIS Neural
Modeling Tutorials (http://www.genesis-sim.org/GENESIS/Tutorials/).

This version RSnet2.g has been reorganized and modularized for
eventual use under GENESIS 3.

The following snippet (source code lines 405--422) mix up the solver method used and coordinates of the neurons to be simulated.

\linenumbers
\begin{verbatim}
// If hsolve is used, make a solver for cell[0] and duplicate for others
if(hflag)
    pushe /network/cell[0]
    create hsolve solver
    setmethod solver 11 
    setfield solver chanmode {hsolve_chanmode} path "../[][TYPE=compartment]"
    call solver SETUP
    int i
    for (i = 1 ; i < {NX*NY} ; i = i  1)
        call solver DUPLICATE \
            /network/cell[{i}]/solver  ../##[][TYPE=compartment]
             setfield /network/cell[{i}]/solver \
               x {getfield /network/cell[{i}]/soma x} \
               y {getfield /network/cell[{i}]/soma y} \
               z {getfield /network/cell[{i}]/soma z}
    end
    pope
end
\end{verbatim}

\end{document}

