% Template for PLoS
% Version 1.0 January 2009
%
% To compile to pdf, run:
% latex plos.template
% bibtex plos.template
% latex plos.template
% latex plos.template
% dvipdf plos.template

\documentclass[10pt]{article}

% amsmath package, useful for mathematical formulas
\usepackage{amsmath}
% amssymb package, useful for mathematical symbols
\usepackage{amssymb}

% graphicx package, useful for including eps and pdf graphics
% include graphics with the command \includegraphics
\usepackage{graphicx}

% cite package, to clean up citations in the main text. Do not remove.
\usepackage{cite}

\usepackage{color} 

% Use doublespacing - comment out for single spacing
\usepackage{setspace} 
\doublespacing

\usepackage{lineno}

% Text layout
\topmargin 0.0cm
\oddsidemargin 0.5cm
\evensidemargin 0.5cm
\textwidth 16cm 
\textheight 21cm

\begin{document}

\section*{A Federated Design for a Neurobiological Simulation Engine: 
The CBI federated software architecture.}
\subsection*{Cornelis H, Coop AD, and Bower JM}
\subsection*{Supplementary Materials 1:} 

The following example shows how a simulator script file typically
contains a mixture of experiment protocol definitions (e.g. fire
climbing fiber input as an ascending volley, lines 56--63 and then
again lines 103--108, set stimulation frequency, lines 19--20), and
simulation control (e.g.setup the compartmental solver, line 54 and
lines 75--82)
%, define outputs, lines 86--97).

\begin{linenumbers}
\begin{verbatim}

//genesis - Purkinje cell version M9 genesis2.1 master script
/* Copyright E. De Schutter (Caltech and BBF-UIA) */

/* This script simulates a Purkinje cell in vitro, receiving a climbing
**  fiber input 
** It also demonstrates how to output [Ca] and Ik values  */

/* Reference:
** E. De Schutter and J.M. Bower: An active membrane model of the
** cerebellar Purkinje cell: II. Simulation of synaptic responses.
** Journal of Neurophysiology  71: 401-419 (1994).
** http://www.bbf.uia.ac.be/TNB/TNB_pub7.html
** We reconstruct parts of Fig. 4 of this paper.
** See http://www.bbf.uia.ac.be/models/PM9.shtml for general model info.
*/

randseed 
float phertz = 25
float ihertz = 1
float delay = 0.00020	/* sets speed of climbing fiber volley */

str filename = "results/PurkM9_CS"
echo file {filename}
int i
include defaults
cellpath="/Purkinje"
str hstr

/*********************************************************************
** Active membrane lumped Purkinje cell model script  (#2M9)
** E. De Schutter, Caltech 1991-1993 
** Uses the scripts:	Purk_chan, Purk_cicomp, Purk_const, Purk_syn
*********************************************************************/
/* Purkinje cell constants */
include Purk_const.g 

/* special scripts  to create the prototypes */
include Purk_chan 
include Purk_cicomp
include Purk_syn 

/* To ensure that all subsequent elements are made in the library */
ce /library

/* These make the prototypes of channels and compartments that can be
**  invoked in .p files */
make_Purkinje_chans
make_Purkinje_syns
make_Purkinje_comps

/* create the model and set up the run cell mode */
// read cell date from .p file and make hsolve element
readcell tests/scripts/PurkM9_model/Purk2M9.p {cellpath} -hsolve

/* make climbing fiber presynaptic elements */
create neutral {cellpath}/climb_presyn1
disable {cellpath}/climb_presyn1
setfield {cellpath}/climb_presyn1 z 0
//addmsg {cellpath}/climb_presyn1 {cellpath}/main[0-2]/climb ACTIVATION z
addmsg {cellpath}/climb_presyn1 {cellpath}/main[0]/climb ACTIVATION z
. . .
addmsg {cellpath}/climb_presyn5 {cellpath}/br3[16]/climb ACTIVATION z

/* Set the clocks */
for (i = 0; i <= 8; i = i  1)
    setclock {i} {dt}
end
setclock 9 1.0e-4

/* Create the output element */
create asc_file /output/plot_out
useclock /output/plot_out 9

// setup the hines solver
ce {cellpath}
/* we need chanmode 4 for output of Ik and calcmode 0 for backward 
** compatibility (this version M9 of the Purkinje cell model only!)
*/
setfield comptmode 1 chanmode 4  calcmode 0
call . SETUP
setmethod 11

/* Initialize output */
/* Output voltage as in Fig. 4 */
hstr={findsolvefield {cellpath} {cellpath}/soma Vm}
addmsg {cellpath} /output/plot_out SAVE {hstr}
. . .
/* Output [Ca] as in Fig. 4D */
hstr={findsolvefield {cellpath} {cellpath}/b3s44[20]/Ca_pool Ca}
addmsg {cellpath} /output/plot_out SAVE {hstr}
/* Output currents as in Fig. 4D: requires chanmode 4 */
// hstr={findsolvefield {cellpath} {cellpath}/b3s44[20]/CaP Ik}
// addmsg {cellpath} /output/plot_out SAVE {hstr}
. . .
// setfield /output/plot_out filename {filename} initialize 1 append 1 leave_open 1
setfield /output/plot_out filename {filename} leave_open 1

reset

step 0.2 -time

/* fire climbing fiber input as an ascending volley */
setfield {cellpath}/climb_presyn1 z  1
step 1
. . .
setfield {cellpath}/climb_presyn5 z  0
step {delay} -time

step 0.1 -time

quit

\end{verbatim}
\end{linenumbers}

\end{document}

