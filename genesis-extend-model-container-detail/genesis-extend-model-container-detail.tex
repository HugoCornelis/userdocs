\documentclass[12pt]{article}
\usepackage{verbatim}
\usepackage[dvips]{epsfig}
\usepackage{color}
\usepackage{url}
\usepackage[colorlinks=true]{hyperref}

\begin{document}

\section*{GENESIS: Documentation}

{\bf Related Documentation:}
% start: userdocs-tag-replace-items related-do-nothing
% end: userdocs-tag-replace-items related-do-nothing

\section*{Details of Extending the Model\,Container}

For the example of adding a {\it PulseGen} to the \href{../model-container/model-container.tex}{\bf Model\,Container}, with the exception of adding a symbol for the pulse object (see below), the following steps are {\bf automated} by the function {\it pulse\_gen} given in the documentation for \href{../genesis-extend-model-container/genesis-extend-model-container.tex}{\bf Extending\,the\,Model\,Container}.

The following subheadings are in order of execution.

\subsection*{Add the new tokenizer}

First add a new tokenizer to the {\it analyzer.l} file:
\begin{verbatim}
   pulse { return(TOKEN_PULSE); }
\end{verbatim}

\subsection*{Add the new token}

The new token is added to the {\it description.tokens} file located in {\it hierarchy/output/symbols/}:
\begin{verbatim}
   %token TOKEN_PULSE
\end{verbatim}

\subsection*{Add a parser rule}

A series of grammar rules must now be added:
\begin{itemize}
   \item[]{\bf Update {\it hierarchy/parser.start}:} In this file there is a union where we add a symbol for our pulse object:
\begin{verbatim}
   struct symtab_Pulse * ppulse;
\end{verbatim}
The prefix {\tt p} for {\tt pulse} indicates that this is a pointer. \\
{\bf Note:} This is the only step in extending the functionality of the {\bf Model\,Container} that is not automated by the function {\it pulse\_gen}.

   \item[]{\bf Add new declarations for the lexer:} Copying the format of the current declarations, a block like the following will need to be added for the pulse object:
\begin{verbatim}
   %type <phsle> PulseSymbol
   %type <ppulse> PulseSectionFront
   %type <ppulse> PulseSectionFront1
   %type <pidin> PulseSectionFront2
   %type <ppulse> PulseDescription
   %type <phsle> PulseComponent
\end{verbatim}
\end{itemize}

\subsection*{Add rules}

The next step is to add a rule for each of the declarations just made so that the parser knows what to do with them.

\subsubsection*{Rule for {\it PulseSymbol}}

\begin{verbatim}
   PulseSymbol
      :
         PulseSection
         {
#line
            //- put symbol table element on stack
            $$ = &$1->bio.ioh.iol.hsle;
         }
\end{verbatim}
The {\it PulseSymbol} uses a Bio\,Component symbol.

\subsubsection*{Rule for {\it PulseDescription}}

\begin{verbatim}
   PulseDescription        /* <ppulse> */
      :
         {
   #line
            $$ = ParserContextGetActual((PARSERCONTEXT *)pacParserContext);
         }
      |
         PulseDescription
         ChildSectionOptionalInputOptionalParameters
         {
   #line
            //- link children
            if ($2)
            {
               SymbolAddChild(&$1->bio.ioh.iol.hsle, $2);
            }
            
            //- reset actual symbol
            ParserContextSetActual
               ((PARSERCONTEXT *)pacParserContext,
               &$1->bio.ioh.iol.hsle);

            //- put symbol description on stack
            $$ = $1;
         }
      |
         PulseDescription
         Parameters
            {
   #line
            //- link parameters
            SymbolParameterLinkAtEnd(&$1->bio.ioh.iol.hsle, $2);

            //- reset actual symbol
            ParserContextSetActual
               ((PARSERCONTEXT *)pacParserContext,
               &$1->bio.ioh.iol.hsle);

            //- put symbol on stack
               $$ = $1;
         }
      ;
\end{verbatim}

\subsubsection*{Rule for {\it PulseSectionEnd}}

\begin{verbatim}
   PulseSectionEnd
      :
         EndPushedPidin
         TOKEN_PULSE
         {
   #line
         }
      ;
\end{verbatim}

\subsubsection*{Rule for {\it PulseSectionFront}}

\begin{verbatim}
   PulseSectionFront       /* <ppulse> */
      :
         PulseSectionFront1
         PulseSectionFront2
         {
   #line
            //- prepare struct for symbol table
            $$ = $1;

            //- set actual symbol
            ParserContextSetActual
               ((PARSERCONTEXT *)pacParserContext,
               &$$->bio.ioh.iol.hsle);
               
            //- assign name to symbol
            SymbolSetName(&$$->bio.ioh.iol.hsle, $2);
         }
      ;
\end{verbatim}

\subsubsection*{Rule for {\it PulseSectionFront1}}

\begin{verbatim}
   PulseSectionFront1      /* <ppulse> */
      :
         TOKEN_PULSE
         {
   #line
            //- prepare struct for symbol table
            $$ = PulseCalloc();
                    
            //- set actual symbol
            ParserContextSetActual
               ((PARSERCONTEXT *)pacParserContext,
               &$$->bio.ioh.iol.hsle);
         }
      ;
\end{verbatim}
{\bf Note:} This is called by {\it PulseCalloc}, which must be defined for this to work.

\subsubsection*{Rule for {\it PulseSectionFront2}}

\begin{verbatim}
   PulseSectionFront2      /* <ppulse> */
      :
         IdentifierOptionIndexPushedPidin
         {
   #line
            //- put identifier on stack
            $$ = $1;
         }
      ;
\end{verbatim}

\subsubsection*{Rule for {\it PulseSection}}

\begin{verbatim}
   PulseSection    /* <ppulse> */
      :
         PulseSectionFront
            InputOutputRelations
            OptionalItemInputRelations
            PulseDescription
         PulseSectionEnd
         {
   #line
            //- link input/output relations
            SymbolAssignBindableIO(&$4->bio.ioh.iol.hsle, $2);
                    
            //- bind I/O relations
            SymbolAssignInputs(&$4->bio.ioh.iol.hsle, $3);
                    
            //- put finished section info on stack
            $$ = $4;
         }
      ;
\end{verbatim}

\end{document}
