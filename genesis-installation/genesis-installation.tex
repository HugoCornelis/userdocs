\documentclass[12pt]{article}
\usepackage{verbatim}
\usepackage[dvips]{epsfig}
\usepackage{color}
\usepackage{url}
\usepackage[colorlinks=true]{hyperref}

\begin{document}

\section*{GENESIS: Documentation}

{\bf Related Documentation:} \\
\href{../genesis-system/genesis-system.tex}{\bf GENESIS\,System}
% start: userdocs-tag-replace-items related-build-debian
% end: userdocs-tag-replace-items related-build-debian

\section*{GENESIS  Installation}

Currently, there are three types of GENESIS installation:
\begin{itemize}
   \item[(A)]\href{../installation-debian/installation-debian.tex}{\bf User}
   \item[(B)]\href{../installation-developer/installation-developer.tex}{\bf Developer}
   \item[(C)]\href{../installation-independent/installation-independent.tex}{\bf Independent}
\end{itemize}
Typically, for each type of installation, once your machine is correctly configured and any missing \href{../genesis-dependencies/genesis-dependencies.tex}{\bf GENESIS software dependencies} are installed, the GENESIS installation process is highly automated.

\subsection*{A. User Installation}

{\bf For individual students and researchers who want to either create or extend a pre-existing realistic single cell, circuit, or systems level neural model.} \\
Installation is a simple automated process that employs an installer package. The following installer packages are available:
\begin{itemize}
   \item[] \href{../installation-debian/installation-debian.tex}{\bf Debian Installer}
\end{itemize}
   
\subsection*{B. Developer Installation}

{\bf For collaborative research projects and GENESIS software development that provides automated synchronization with the \href{http://www.genesis-sim.org/}{\bf GENESIS web site}}.

\subsubsection*{1. Introduction and Overview}

\begin{itemize}
   \item[]\href{../developer-installation/developer-installation.tex}{\bf Overview of Developer Installation}
\end{itemize}

\subsubsection*{2. Source Code Server Installation}

Installation guides are available for the following operating systems:
\begin{itemize}
   \item[]\href{../installation-debian-server/installation-debian-server.tex}{\bf Debian\,Server\,5.0.3}
   \item[]\href{../installation-fedora10/installation-fedora10.tex}{\bf Fedora\,10}
   \item[]\href{../installation-fedora12/installation-fedora12.tex}{\bf Fedora\,12 and Higher}
   \item[]\href{../installation-ubuntu-lennysid/installation-ubuntu-lennysid.tex}{\bf Ubuntu (Lenny/Sid)}
   \item[]\href{../installation-ubuntu-karmic/installation-ubuntu-karmic.tex}{\bf Ubuntu (Karmic)}
   \item[]\href{../installation-ubuntu-lucid/installation-ubuntu-lucid.tex}{\bf Ubuntu (Lucid)}
   \item[]\href{../installation-ubuntu-maverick/installation-ubuntu-maverick.tex}{\bf Ubuntu (Maverick)}
   \item[]\href{../installation-ubuntu-precise/installation-ubuntu-precise.tex}{\bf Ubuntu (Precise)}
   \item[]\href{../installation-osx/installation-osx.tex}{\bf Mac\,OSX}
\end{itemize}

\subsubsection*{3. Documentation Server Installation}

\subsubsection*{4. Regression Tester Installation}

\begin{itemize}
   \item[]\href{../neurospaces-cron/neurospaces-cron.tex}{\it neurospaces\_cron}
\end{itemize}

\subsubsection*{5. Create a Backup Repository}

Provides a guide to using the \href{../backup-repository/backup-repository.tex}{\it backup\_server} option for repository access.

\subsection*{C. Independent Installation}

For collaborative research projects and software development that is independent from and does not automatically synchronize with the \href{http://www.genesis-sim.org/}{\bf GENESIS\,web\,site}.

\subsubsection*{UNDER CONSTRUCTION}

\subsubsection*{1. Introduction to Independent Development}

The following link provides an introduction to \href{../developer-intro/developer-intro.tex}{\bf Independent\,Development}.

\subsubsection*{2. Source Code Server Installation}

See above for Developer Installation

\subsubsection*{3. Documentation Server Installation}

See above for Developer Installation

\subsubsection*{4. Regression Tester Installation}

See above for Developer Installation

\end{document}
