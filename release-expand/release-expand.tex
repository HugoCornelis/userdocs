\documentclass[12pt]{article}
\usepackage[dvips]{epsfig}
\usepackage{color}
\usepackage{url}
\usepackage[colorlinks=true]{hyperref}

\begin{document}

\section*{GENESIS: Documentation}

{\bf Related Documentation:}
% start: userdocs-tag-replace-items related-do-nothing
% end: userdocs-tag-replace-items related-do-nothing

\section*{\it ReleaseExpand}

The {\it release-expand} utility does keyword expansion in text files. Keywords and replacement texts are both customizable via a configuration file.

Below is the man page of release-expand.

\section*{Details}

\begin{enumerate}
\item[] RELEASE-EXPAND(1) User Contributed Perl Documentation
\end{enumerate}

\subsection*{NAME}

\begin{enumerate}
   \item[] {\it release-expand}--expand keywords before building a release distribution.
\end{enumerate}

\subsection*{SYNOPSIS}

\begin{enumerate}

   \item[]{\it release-expand} options:

   \item[]{\bf $<$package$>$}

   \item[]{\bf $<$major$>$}

   \item[]{\bf $<$minor$>$}

   \item[]{\bf $<$micro$>$}

   \item[]{\bf $<$label$>$}

   \item[]{\bf $<$email$>$}
    
\end{enumerate}

\subsection*{DESCRIPTION}

\begin{enumerate}

    \item[]{\it release-expand} is a keyword expander for arbitrary source code. 

    \item[]{It is intended to be used before building a distribution, to expand keywords, and optionally before a {\it commit} to a repository under version control, to remove keywords. The advantage is that it is programming language independent, there is
    one single clear definition of each keyword, and it allows for user extension. The idea is taken from the user extensible keyword
    expansion mechanism that was implemented in the \href{http://prcs.sourceforge.net/}{\bf prcs} version control system.}

\end{enumerate}

\subsection*{USAGE}

\begin{enumerate}

\item[]{\bf Source file keywords:}

\begin{enumerate}
\item[] In your source files, insert text of the following form, possibly in comments:
\begin{verbatim}
$Format:
<text-with-keywords>
$ 
\end{verbatim}
\item[] The keywords have the format {\tt \${keyword-name}}, the rest of the text is arbitrary, but starts and ends with a dollar sign. 
\end{enumerate}

\item[]{\bf Specifying your keywords:}

\begin{enumerate}
\item[]By default the following keywords are known: 

{\tt package major minor micro label email}

\item[]There is an implicit assumption that the {\tt label} keyword is a concatenation of {\tt major}, {\tt minor}, and {\tt micro}, with dashes in between. 

\item[]Additional keywords can be added in the config file if needed. Beware that at the time of writing, these additional keywords can not be overriden from the command line.

\end{enumerate}

\item[]{\bf Specifying your source files}

\begin{enumerate}
\item[] Not all files are checked for keyword expansion. The files that need keyword expansion must be specified in the config file. 
\end{enumerate}

\item[]{\bf Config file:}

\begin{enumerate}

\item[] The config file is named {\it release-expand.config} and is found in the current directory. It has the following format:

\begin{verbatim}        
#!/usr/bin/perl -w 
my $config
   = {
      files => [
         '/configure.ac', `./heccer.c', `./glue/configure.ac', \
            `./glue/swig/perl/configure.ac', `./test-dist', \
            `./tests/code/configure.ac', `./tests/library/version.t', 
      ],
      labels => {
         package => ", major => ", minor => ", micro => ", \
            label => ", email => ",
      },
   }; 
return $config; 
\end{verbatim}

The {\tt files} section enumerates the files for which keyword expansion is needed. The {\tt labels} section enumerates the keywords and their values. 

You can add custom keywords, e.g. 
\begin{verbatim}
   labels => {
      website => �http://www.neurospaces.org�, optimization => �-O2�, 
   }, 
\end{verbatim}

\item[] The values of the default keywords are replaced by the command line arguments before the expansion algorithm starts.

\end{enumerate}

\item[]{\bf How keyword expansion works:}

\begin{enumerate}

\item[] All the files enumarated in the config file, are checked for occurence of the sequence {\tt \$Format: $\ldots$ \$}. This sequence must occur on one single line in the source file. 

The expansion algorithm will replace the next full line in the file with the text after the colon, with keywords expanded as appropriate. 

An example:

In a C source file, you have the comment: 

\begin{verbatim}
   // $Format: " static char pcVersion=\"${label}\";"$ 
\end{verbatim}

and you use the following command line:
 
\begin{verbatim}
   ./release-expand neurospaces 1 10 pre-release \
      1-10-pre-release `hugo\.cornelis\@gmail\.com'
\end{verbatim}

On expansion the line following this comment line will expand to: 
\begin{verbatim}
   static char pcVersion="1-10-pre-release"; 
\end{verbatim}

In your {\it configure.ac} file, you have the line: 

\begin{verbatim}
   $Format: "AM_INIT_AUTOMAKE(${package}, ${label})"$ 
\end{verbatim}

Again using the command line from above, this would give: 
\begin{verbatim}
   AM_INIT_AUTOMAKE(neurospaces, 1-10-pre-release) 
\end{verbatim}
\end{enumerate}
\end{enumerate}

\subsection*{OPTIONS}

\begin{enumerate}
\item[] The following options are supported: 

\begin{enumerate}
\item[] -v, --verbose
\begin{enumerate}
\item[] Verbose. 
\end{enumerate}

\item[] -h, --help
\begin{enumerate}
\item[] Give help page. 
\end{enumerate}

\item[] -r, --remove
\begin{enumerate}
\item[] Remove keyword expansions, and replace them with an empty line. Use this before {\it checkin} to a version control repository, to avoid generating artificial differences between different versions of your code. This avoids merge conflicts. 
\end{enumerate}
\end{enumerate}
\end{enumerate}

\subsection*{BUGS}
\begin{enumerate}
\item[] Mixing different quoting styles may give problems. 
\end{enumerate}

\subsection*{SEE ALSO}

\begin{enumerate}
\item[] prcs(1) 
\end{enumerate}

\subsection*{AUTHOR}

\begin{enumerate}
\item[] Hugo Cornelis $<$hugo.cornelis@gmail.com$>$ 
\end{enumerate}

\subsection*{COPYRIGHT}

\begin{enumerate}
\item[] Copyright (c) 2007--2008. Hugo Cornelis. All rights reserved. 

\item[] This program is free software; you can redistribute it and/or modify it under the same terms as Perl itself. 

\item[] See $<$http://www.perl.com/perl/misc/Artistic.html$>$

\end{enumerate}

\noindent perl v5.8.7 2009-04-30 RELEASE-EXPAND(1) 

\end{document}
