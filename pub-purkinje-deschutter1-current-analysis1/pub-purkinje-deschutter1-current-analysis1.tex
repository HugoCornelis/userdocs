\documentclass[12pt]{article}
\usepackage{verbatim}
\usepackage[dvips]{epsfig}
\usepackage{color}
\usepackage{url}
\usepackage[colorlinks=true]{hyperref}

\begin{document}

\section*{GENESIS: Documentation}

{\bf Related Documentation:}
% start: userdocs-tag-replace-items related-do-nothing
% end: userdocs-tag-replace-items related-do-nothing

\section*{De Schutter: Purkinje Cell Model}

\subsection*{Analysis of the ionic currents responsible for somatic and dendritic spikes}

Results from parameter variations, of the sort just mentioned,
led to the question of which ionic currents were
most responsible for each of the response properties of the
Purkinje cell. In fact, the parameter sensitivity of model
output to ion channel densities followed directly from an
analysis of the role of the different currents in the Purkinje
cell firing behavior. In some cases the relationship between
ion currents and response properties was straightforward;
in other cases the response of the- cell was a result of the
interaction between different currents. Sometimes this was
further complicated by the fact that the types of channels
and their densities varied between the three different regions
of the model. The following sections first describe the
spatial distribution of the patterns of activity and then consider
in detail the currents that cause each component of
Purkinje cell responses to current injection in the model.
\href{../pub-purkinje-deschutter1-fig-10/pub-purkinje-deschutter1-fig-10.tex}{\bf Figure\,10} shows images representing the membrane potential
distribution and calcium concentration in all parts
of the model during a somatic action potential and a dcndritic
spike.

\href{../pub-purkinje-deschutter1-fig-11/pub-purkinje-deschutter1-fig-11.tex}{\bf Figure\,11} shows the change in membrane potential,
submembrane Ca$^{2+}$ concentration, and amplitude
of all the ionic currents at four representative locations in
the model during a current injection in the soma. The respective
contribution of different channels to the somatic
spikes, dendritic spikes, and the corresponding depolarizing
spike bursts in the soma can be determined from these figures.
In each case the data shown were obtained long after
the initiation of the current injection so that the repetitive
firing properties of the model had settled into a steady state.

\bibliographystyle{plain}
\bibliography{../tex/bib/g3-refs.bib}

\end{document}
