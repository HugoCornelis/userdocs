\documentclass[12pt]{article}
\usepackage{verbatim}
\usepackage[dvips]{epsfig}
\usepackage{color}
\usepackage{url}
\usepackage[colorlinks=true]{hyperref}

\begin{document}

\section*{GENESIS: Documentation}

{\bf Related Documentation:}
% start: userdocs-tag-replace-items related-do-nothing
% end: userdocs-tag-replace-items related-do-nothing

\section*{De Schutter: Purkinje Cell Model}

\subsection*{Source}

De Schutter E \& Bower JM (1994) An active membrane model of the cerebellar Purkinje cell I. Simulation of current clamp in slice. {\it Journal of Nerurophysiology}. {\bf 71}: 375--400.

\noindent De Schutter E \& Bower JM (1994) An active membrane model of the cerebellar Purkinje cell II. Simulation of synaptic responses. {\it Journal of Nerurophysiology}. {\bf 71}: 375--400.

\subsection*{Abstract}

\begin{enumerate}
% 1
\item A detailed compartmental model of a cerebellar Purkinje
cell with active dendritic membrane was constructed. The model
was based on anatomic reconstructions of single Purkinje cells and
included 10 different types of voltage-dependent channels described
by Hodgkin-Huxley equations, derived from Purkinje cell specific
voltage-clamp data where available. These channels included
a fast and persistent Na$^+$ channel, three voltage-dependent
K$^+$ channels, T-type and P-type Ca$^{2+}$ channels, and two types of
Ca$^{2+}$ -activated K+ channels.

% 2
\item The ionic channels were distributed differentially over three
zones of the model, with Na$^+$ channels in the soma, fast K$^+$ channels
in the soma and main dendrite, and Ca$^{2+}$ channels and Ca$^{2+}$-
activated K$^+$ channels in the entire dendrite. Channel densities in
the model were varied until it could reproduce Purkinje cell responses
to current injections in the soma or dendrite, as observed
in slice recordings.

% 3
\item As in real Purkinje cells, the model generated two types of
spiking behavior. In response to small current injections the
model fired exclusively fast somatic spikes. These somatic spikes
were caused by Na$^+$ channels and repolarized by the delayed rectifier.
When higher-amplitude current injections were given, sodium
spiking increased in frequency until the model generated
large dendritic Ca$^{2+}$ spikes. Analysis of membrane currents underlying this 
behavior showed that these Ca$^{2+}$ spikes were caused by
the P-type Ca$^{2+}$ channel and repolarized by the BK-type Ca$^{2+}$-activated
K$^+$ channel. As in pharmacological blocking experiments,
removal of Na$^+$ channels abolished the fast spikes and removal of
Ca$^{2+}$ channels removed Ca$^{2+}$ spiking.

% 4
\item In addition to spiking behavior, the model also produced
slow plateau potentials in both the dendrite and soma. These
longer-duration potentials occurred in response to both short and
prolonged current steps. Analysis of the model demonstrated that
the plateau potentials in the soma were caused by the window
current component of the fast Na$^+$ current, which was much
larger than the current through the persistent Na$^+$ channels.
Plateau potentials in the dendrite were carried by the same P-type
Ca$^{2+}$ channel that was also responsible for Ca$^{2+}$ spike generation.
The P channel could participate in both model functions because
of the low-threshold K2-type Ca$^{2+}$-activated K$^+$ channel, which
dynamically changed the threshold for dendritic spike generation
through a negative feedback loop with the activation kinetics of
the P-type Ca$^{2+}$ channel.

% 5
\item These model responses were robust to changes in the densities
of all of the ionic channels. For most of the channels, modifying
their densities by factors of $\geq$2 resulted only in left or right
shifts of the frequency-current curve. However, changes of $>$20\,\%
to the amount of P-type Ca$^{2+}$ channels or of one of the Ca$^{2+}$-activated
K$^+$ channels in the model either suppressed dendritic spikes
or caused the model to always fire Ca$^{2+}$ spikes. Modeling results
were also robust to variations in Purkinje cell morphology. We
simulated models of two other anatomically reconstructed Purkinje
cells with the same channel distributions and got similar
responses to current injections.

% 6
\item The model was used to compare the electrotonic length of
the Purkinje cell in the presence and absence of active dendritic
conductances. The electrotonic distance from soma to the tip of
the most distal dendrite increased from 0.57\,$\lambda$ in a passive model
to 0.95\,$\lambda$ in a quiet model with active membrane. During a dendritic
spike generated by current injection the distance increased
even more, to 1.57\,$\lambda$.

% 7
\item Finally, the model was used to study the probable accuracy
of experimental voltage-clamp data. Whole-cell patch-clamp conditions
were simulated by blocking most of the K$^+$ currents in the
model. The increased electrotonic length due to the active dendritic
membrane caused space clamp to fail, resulting in membrane
potentials in proximal and distal dendrites that differed critically
from the holding potential in the soma.

% 1
\item Both excitatory and inhibitory postsynaptic channels were
added to the model to examine model responses to synaptic
inputs. Maximum synaptic conductance was the only parameter
that was tuned in these studies. Under these
conditions the model was capable of reproducing physiological
recorded responses to each of the major types of synaptic input.

% 2
\item When excitatory synapses were activated on the smooth
dendrites of the model, the model generated a complex dendritic
Ca$^{2+}$ spike similar to that generated by climbing fiber inputs. Examination
of the model showed that activation of P-type Ca$^{2+}$
channels in both the smooth and spiny dendrites augmented the
depolarization during the complex spike and that Ca$^{2+}$-activated
K$^+$ channels in the same dendritic regions determined the duration
of the spike. When these synapses were activated under simulated
current-clamp conditions the model also generated the characteristic
dual reversal potential of the complex spike. The shape
of the dendritic complex spike could be altered by changing the
maximum conductance of the climbing fiber synapse and thus the
amount of Ca$^{2+}$ entering the cell.

% 3
\item To explore the background simple spike firing properties of
Purkinje cells in vivo we added excitatory ``parallel fiber'' synapses
to the spiny dendritic branches of the model. Continuous
asynchronous activation of these granule cell synapses resulted in
the generation of spontaneous sodium spikes. However, very low
asynchronous input frequencies produced a highly regular, very
fast rhythm (80--120\,Hz), whereas slightly higher input frequencies
resulted in Purkinje cell bursting. Both types of activity are
uncharacteristic of in vivo Purkinje cell recordings.

% 4
\item Inhibitory synapses of the sort presumably generated by stellate
cells were also added to the dendritic tree. When asynchronous
activation of these inhibitory synapses was combined with
continuous asynchronous excitatory input the model generated
somatic action potentials in a much more stochastic pattern typical
of real Purkinje cells. Under these conditions simulated interspike
interval distributions resembled those found in experimental
recordings. Also, as with in vivo recordings, the model did not
generate dendritic bursts. This was mainly due to inhibition that
suppressed the generation of dendritic Ca$^{2+}$ spikes.

% 5
\item In the presence of asynchronous inhibition, changes in the
average frequency of excitatory inputs modulated background simple
spike firing frequencies in the natural range of Purkinje cell
firing frequencies (30--100\,Hz). This modulation was very sensitive
to small changes in the average frequency of excitatory inputs.
In addition, changes in inhibitory frequency caused a parallel shift
of the relationship between excitatory input and spiking frequency.
Because of the specific cerebellar circuitry, inhibitory inputs
may allow Purkinje cells to detect small fluctuations in excitatory
input at any mean frequency of input.

% 6
\item When climbing fiber input was given in the presence of
background asynchronous excitatory and inhibitory inputs the
shape of the complex spike in the soma was significantly affected.
However, the shape of the spike in the dendrites was almost constant.
This difference reflected the more variable excitability of the
soma compared with the dendrites.

% 7
\item Synchronous activation of basket cell inhibitory synapses
during asynchronous activation of granule and stellate cell synapses
interrupted somatic spiking. However, the hyperpolarization
caused by the basket cell synapse did not penetrate far into the
dendrite but stayed localized to the soma and main dendrite.

% 8
\item This simulation work demonstrates that a model based on
voltage-clamp data and tuned entirely on the response of Purkinje
cells to current injection is capable of reproducing a wide range of
synaptically activated responses. In the presence of continuous
granule cell excitation the model showed a stable in vivo state
different from the silent, resting in vitro state. Further, the model
suggests that there may be important functional interactions between
different types of synaptic inputs. In particular, it makes
several specific predictions about the role of stellate cell inhibition.

\end{enumerate}

\end{document}
