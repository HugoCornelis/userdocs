\documentclass[12pt]{article}
\usepackage{verbatim}
\usepackage[dvips]{epsfig}
\usepackage{color}
\usepackage{url}
\usepackage[colorlinks=true]{hyperref}

\begin{document}

\section*{GENESIS: Documentation}

{\bf Related Documentation:}
% start: userdocs-tag-replace-items related-do-nothing
% end: userdocs-tag-replace-items related-do-nothing

\section*{De Schutter: Purkinje Cell Model}

\subsection*{Estimates of Model Parameters}

The \href{../pub-purkinje-deschutter1-equations1/pub-purkinje-deschutter1-equations1.tex}{\bf Hodgkin-Huxley equations employed by this model} require detailed information on channel kinetics and peak conductances. In principle,
all of these data should be obtained from the cell being simulated.
However, all the necessary data are rarely available for any
particular neuron. For this reason it is usually necessary to obtain
kinetic information from a variety of sources.

\subsection*{Estimating channel conductances}

Constructing an accurate
representation of an ionic channel also requires an estimate of
the channel $\bar g$. However, channel conductance values measured in
voltage-clamp experiments are often quite variable\,\cite{McCormick:1992fk} and are only indicative. For this reason,
channel $\bar g$s in the model were not based on experimental measurements\,\cite{S:1993dz, De-Schutter-E:1993fu}.

\subsection*{Estimating channel densities}

Having established and parameterized the equations governing the
voltage-dependent behavior of the ionic conductances in this
cell, it was necessary to determine the channel densities in the
model. Because precise information on channel densities is not
technically obtainable, channel densities, usually expressed as $\bar g$,
are largely a free parameter in detailed single-cell models\,\cite{S:1993dz, L:1992kl, W:1991qa, Traub-R-D:1991mi}.

\subsection*{Estimates of other parameters for ionic channels}

Constructing an accurate
representation of an ionic channel also requires an estimate of
the channel $\bar g$. However, channel conductance values measured in
voltage-clamp experiments are often quite variable\,\cite{McCormick:1992fk} and are only indicative. For this reason,
channel $\bar g$s in the model were not based on experimental measurements\,\cite{S:1993dz, De-Schutter-E:1993fu}.

\subsection*{Temperature dependence of channel kinetics}

Temperature is a well-known critical parameter for channel kinetics\,\cite{Hille:1991zr}.
In the current simulations the in vivo and in vitro data
used to tune the model were collected at 37\,$^\circ$\,C whereas all of the
published voltage-clamp data were obtained at room temperature.
Assuming a Q\,10 factor of 3\,\cite{L:1952fv} all rate constants were multiplied by a factor of 5.

\subsection*{Modeling calcium concentrations}

In the present model realistic simulation of Ca$^{2+}$ concentrations was not attempted. However, computation of Ca$^{2+}$ concentrations was required to activate the
KC and K2 channels. Because these channels are assumed to be
sensitive to the quickly changing, high Ca$^{2+}$ concentrations just
below the membrane surface\,\cite{L:1989ff}, Ca$^{2+}$ concentrations were 
computed in a thin submembrane shell\,\cite{D:1982lh}. These shells integrated the full Ca$^{2+}$ inflow through
the CaP and CaT channels. Their volume and decay time constants
were initially free parameters in the model. The model tuning
resulted in shells 0.2\,$\mu$m deep with a decay time of 0.1\,ms.
The basal internal Ca$^{2+}$ concentration was 0.040\,$\mu$M; the outside
concentration was constant at 2.4\,mM. These concentrations
were also used to compute Nernst potentials\,\cite{Hille:1991zr} for the
CaP and CaT currents.

\bibliographystyle{plain}
\bibliography{../tex/bib/g3-refs}

\end{document}
