\documentclass[12pt]{article}
\usepackage{verbatim}
\usepackage[dvips]{epsfig}
\usepackage{color}
\usepackage{url}
\usepackage[colorlinks=true]{hyperref}

\begin{document}

\section*{GENESIS: Documentation}

{\bf Related Documentation:}
% start: userdocs-tag-replace-items related-do-nothing
% end: userdocs-tag-replace-items related-do-nothing

\section*{De Schutter: Purkinje Cell Model}

\subsection*{Abstract}

In addition to spiking behavior, the model also produced
slow plateau potentials in both the dendrite and soma. These
longer-duration potentials occurred in response to both short and
prolonged current steps. Analysis of the model demonstrated that
the plateau potentials in the soma were caused by the window
current component of the fast Na$^+$ current, which was much
larger than the current through the persistent Na$^+$ channels.
Plateau potentials in the dendrite were carried by the same P-type
Ca$^{2+}$ channel that was also responsible for Ca$^{2+}$ spike generation.
The P channel could participate in both model functions because
of the low-threshold K2-type Ca$^{2+}$-activated K$^+$ channel, which
dynamically changed the threshold for dendritic spike generation
through a negative feedback loop with the activation kinetics of
the P-type Ca$^{2+}$ channel.

\end{document}
