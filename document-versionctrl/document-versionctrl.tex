\documentclass[12pt]{article}
\usepackage[dvips]{epsfig}
\usepackage{url}
\usepackage[colorlinks=true]{hyperref}

\begin{document}

\section*{GENESIS: Documentation}

\section*{Version Control}

Version control of the GENESIS documentation system is currently handled by the monotone software package. Monotone is a distributed version control tool that can help automate many tedious and error-prone tasks in group software development, e.g.
\begin{itemize}
\item Store multiple versions of files you are working on efficiently.
\item Transmit changes to files between you and your colleagues.
\item Merge changes you make with those your colleagues make.
\item Make notes about your opinion of the quality of versions of files.
\item Make decisions about using or ignoring versions, depending on the notes you receive from others. 
\end{itemize}
Please be aware that monotone is a slightly unorthodox version control tool, and many of its concepts are similar--but subtly or significantly different--from concepts with similar names in other version control tools. 

Monotone provides a simple, single-file transactional version store, with fully disconnected operation and an efficient peer-to-peer synchronization protocol. it understands history-sensitive merging, lightweight branches, integrated code review and 3rd party testing. It uses cryptographic version naming and client-side RSA certificates. It has good internationalization support, has no external dependencies and runs on Linux, Solaris, OSX, Windows, and other Unixes.

\subsection*{Obtaining Monotone}
Monotone can be obtained by downloading the package appropriate for your computing platform from the links on the \href{http://monotone.ca/}{monotone home page}.

%\subsection*{Installing Monotone on OS X}
%On the monotone home page select the Mac OS X packaged link and download the DMG file.

\subsection*{Initializing Monotone}
Once the appropriate monotone package has been downloaded and installed on your computer, monotone must  initialized by performing the following steps (commands in the monospaced {\tt typewriter} font should be entered from a Terminal command line prompt):

\begin{enumerate}

\item {\bf Generate a monotone key:} 
\begin{verbatim}
    mtn genkey <identifier>
\end{verbatim}
where the {\tt $<$identifier$>$} is unique to you (typically an email account, e.g. yourname@provider). The identifier is located in a {\it .monotone} directory in your home folder located in the {\it /Users} folder.

\item {\bf Initialize the data base}
\begin{verbatim}
    mtn --db=userdocs.mtn db init
\end{verbatim}
This creates the monotone repository file {\it userdocs.mtn} which is referenced with
\begin{verbatim}
    --db=userdocs.mtn
\end{verbatim}

\item {\bf Create a workspace}
\begin{verbatim}
    mtn --db=userdocs.mtn --branch="0" setup userdocs
\end{verbatim}
This creates a workspace directory on your local machine where you put the source files for any documentation that you create. Importantly, there is a specific required structure and content for documents in the GENESIS documentation system that is described in \href{../create-document/create-document.pdf}{Create a Document}. The path to the workspace directory should be
\begin{verbatim}
    ~/neurospaces_project/userdocs/source/snapshots/0
\end{verbatim}
By default, the workspace directory {\tt 0} contains a folder named {\tt \_MTN}. This is part of the monotone version control system and contains information concerning monotone's version control management.

\end{enumerate}

\subsection*{Monotone Workflow Examples}

The following sections give example workflows for monotone use.

\subsection*{Updating the Local Workspace}

\subsection*{Synchronizing the Local Workspace with the Monotone Repository}

\end{document}