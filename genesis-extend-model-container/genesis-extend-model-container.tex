\documentclass[12pt]{article}
\usepackage{verbatim}
\usepackage[dvips]{epsfig}
\usepackage{color}
\usepackage{url}
\usepackage[colorlinks=true]{hyperref}

\begin{document}

\section*{GENESIS: Documentation}

{\bf Related Documentation:}
% start: userdocs-tag-replace-items related-do-nothing
% end: userdocs-tag-replace-items related-do-nothing

\section*{Extending the Model Container}

This document details the steps a developer must go through in order to add an object called {\bf PulseGen} to the {\bf \href{../model-container/model-container.tex}{\bf Model\,Container}. The following steps are performed in order.

\subsection*{Adding the PulseGen object}

The following subheadings are in order of execution.
Adding a new tokenizer

First I added a new tokenizer to the analyzer.l file:

pulse           {       return(TOKEN_PULSE);    }

This step is automated.
Add a new token

Added a new token to the description.tokens file located in hierarchy/output/symbols/ like so:

%token TOKEN_PULSE

This step is automated.
Adding a parser rule

A series of grammar rules must now be added:

First open the file hierarchy/parser.start file there is a union. In this file there is a union, here we add a symbol for our pulse object:

struct symtab_Pulse * ppulse;

The prefix p for pulse indicates that this is a pointer.

Now some more declarations must be added for the lexer. Copying the format of the current declarations you will need to add a block like this for our pulse object:

%type <phsle> PulseSymbol
%type <ppulse> PulseSectionFront
%type <ppulse> PulseSectionFront1
%type <pidin> PulseSectionFront2
%type <ppulse> PulseDescription
%type <phsle> PulseComponent

This part of this step is automated.

The next step is to add a rule for each of the declarations we just declared so that the parser knows what to do with them.

We need to add a rule for the PulseSymbol?:

PulseSymbol
        :
                PulseSection
                {
#line
                    //- put symbol table element on stack

                    $$ = &$1->bio.ioh.iol.hsle;
                }

The pulse symbol uses a Bio Component symbol.

This step is automated.

We add a rule for PulseDescription?:

PulseDescription        /* <ppulse> */
        :
                {
#line

                    $$ = ParserContextGetActual((PARSERCONTEXT *)pacParserContext);
                }
        |
                PulseDescription
                ChildSectionOptionalInputOptionalParameters
                {
#line
                    //- link children

                    if ($2)
                    {
                        SymbolAddChild(&$1->bio.ioh.iol.hsle, $2);
                    }

                    //- reset actual symbol

                    ParserContextSetActual
                        ((PARSERCONTEXT *)pacParserContext,
                         &$1->bio.ioh.iol.hsle);

                    //- put symbol description on stack

                    $$ = $1;
                }
        |
                PulseDescription
                Parameters
                {
#line
                    //- link parameters

                    SymbolParameterLinkAtEnd(&$1->bio.ioh.iol.hsle, $2);

                    //- reset actual symbol

                    ParserContextSetActual
                        ((PARSERCONTEXT *)pacParserContext,
                         &$1->bio.ioh.iol.hsle);

                    //- put symbol on stack

                    $$ = $1;
                }
        ;

This step is automated.

A rule for PulseSectionEnd?:

PulseSectionEnd
        :
                EndPushedPidin
                TOKEN_PULSE
                {
#line
                }
        ;

This step is automated.

A rule for PulseSectionFront?:

PulseSectionFront       /* <ppulse> */
        :
                PulseSectionFront1
                PulseSectionFront2
                {
#line

                    //- prepare struct for symbol table

                    $$ = $1;

                    //- set actual symbol

                    ParserContextSetActual
                        ((PARSERCONTEXT *)pacParserContext,
                         &$$->bio.ioh.iol.hsle);

                    //- assign name to symbol

                    SymbolSetName(&$$->bio.ioh.iol.hsle, $2);
                }
        ;

This step is automated.

Rule for PulseSectionFront1?:

PulseSectionFront1      /* <ppulse> */
        :
                TOKEN_PULSE
                {
#line

                    //- prepare struct for symbol table

                    $$ = PulseCalloc();

                    //- set actual symbol

                    ParserContextSetActual
                        ((PARSERCONTEXT *)pacParserContext,
                         &$$->bio.ioh.iol.hsle);
                }
        ;

Note that this called PulseCalloc?, which must be defined for this to work.

This step is automated.

Rule for PulseSectionFront2?:

PulseSectionFront2      /* <ppulse> */
        :
                IdentifierOptionIndexPushedPidin
                {
#line

                    //- put identifier on stack

                    $$ = $1;
                }
        ;

This step is automated.

Rule for PulseSection?:

PulseSection    /* <ppulse> */
        :
                PulseSectionFront
                        InputOutputRelations
                        OptionalItemInputRelations
                        PulseDescription
                PulseSectionEnd
                {
#line
                    //- link input/output relations

                    SymbolAssignBindableIO(&$4->bio.ioh.iol.hsle, $2);

                    //- bind I/O relations

                    SymbolAssignInputs(&$4->bio.ioh.iol.hsle, $3);

                    //- put finished section info on stack

                    $$ = $4;
                }
        ;

This step is automated. 

\end{document}
