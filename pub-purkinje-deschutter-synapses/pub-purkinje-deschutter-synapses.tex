\documentclass[12pt]{article}
\usepackage{verbatim}
\usepackage[dvips]{epsfig}
\usepackage{color}
\usepackage{url}
\usepackage[colorlinks=true]{hyperref}

\begin{document}

\section*{GENESIS: Documentation}

{\bf Related Documentation:}
% start: userdocs-tag-replace-items related-do-nothing
% end: userdocs-tag-replace-items related-do-nothing

\section*{De Schutter: Purkinje Cell Model}

\subsection*{Source}

De Schutter E \& Bower JM (1994) An active membrane model of the cerebellar Purkinje cell II. Simulation of synaptic responses. {\it Journal of Nerurophysiology}. {\bf 71}: 401--419.

\subsection*{Climbing Fiber Synapses}

\subsubsection*{SYNAPTIC LOCATIONS}
It is well known that in adults the climbing
fiber input to a single Purkinje cell is provided by a single
climbing fiber synapse evokes an all-or-none massive response,
neuron in the inferior olive\,\cite{Ito:1984uq} and that this contact is
made on the smooth branches of the Purkinje cell dendritic tree,
from close to the soma up to almost the top of the tree\,\cite{Palay:1974fk}.
We modeled the climbing fiber input by placing
synapses from a single axon on all compartments of the main
and smooth dendrites of the cell (see Fig. 1 in\,\cite{deschutter94:_purkin_i}).

\subsubsection*{SYNAPTIC PHARMACOLOGY AND KINETICS}
Activation of the climbing fiber synapse evokes an all-or-none massive response
called the complex spike\,\cite{Eccles:1966kx, Llinas:1980vn}, which involves activation of dendritic Ca$^{2+}$ channels\,\cite{Knopfel:1991ys, Konnerth:1992zr, R:1980pi, Miyakawa:1992ly}.
Climbing fiber inputs are known to
be mediated by $\alpha$-amino-3-hydroxy+methyl-4-isoxazolepropionic acid (AMPA) receptors\,\cite{Knopfel:1990ve, Llano:1991qf}.
In the model, activation kinetics were based on data from
spinal cord and hippocampal neurons\,\cite{Forsythe:1988bh, Holmes:1963dq, Nelson:1986cr},
with an opening time constant of 0.5\,ms and a closing time constant of 1.2\,ms
(peak at 0.8\,ms). Ca$^{2+}$ inflow through this receptor was not modeled,
although it might be present\,\cite{Brorson:1992nx}. This synapse had a reversal potential of 0\,mV\,\cite{Cull-Candy:1989oq, Mayer:1987kl} .

\subsubsection*{SYNAPTIC CONDUCTANCE}

The conductance of each climbing
fiber synaptic connection is unknown and was therefore treated as
a free parameter in the model. Model tuning resulted in the selection
of two different values of maximum synaptic conductance
($\bar g$): 7.5\,mS/cm$^2$ for the main dendrite and 15\,mS/cm$^2$ for the
smooth dendrites. Assuming a total of 300 synaptic contacts\,\cite{Ito:1984uq}, each with the same synaptic conductance, our values
correspond to a conductance of 6.2\,nS at each synaptic contact.

\subsubsection*{SYNAPTIC SPECIALIZATIONS}

Climbing fiber synapses are
known to occur on a specialized ``stubby protuberance'' with a
short, smooth stem\,\cite{Palay:1974fk}. We did not,
however, explicitly model these structures because we do not expect
their absence in the model to affect the voltage transients
caused by climbing fiber activation. Such smooth spines are unlikely
to result in much signal attenuation\,\cite{Rall:1990tg}
and all spines on any particular compartment receive synchronously 
the same synaptic input from a single climbing fiber\,\cite{Ito:1984uq}.

\subsubsection*{SYNAPTIC ACTIVATION}

The climbing fiber synapses were fired as
a volley, with the main dendrite being activated before the more
distal smooth dendrites. The delay between first and last activated
synaptic conductance was 0.9\,ms\,\cite{Llinas:1980vn}.

\subsection*{Granule Cell Synapses}

\subsubsection*{SYNAPTIC LOCATIONS}

In the model we placed granule cell synapses
on all dendrites with diameters $\sim$3.17\,$\mu$m\,\cite{Palay:1974fk}.
These synapses are made on dendritic spines\,\cite{M:1988bh} 
that have been explicitly modeled for each
synaptic contact (see below).

\subsubsection*{SYNAPTIC PHARMACOLOGY AND KINETICS}

These excitatory inputs to the Purkinje cell are also mediated by AMPA receptors
\,\cite{Farrant:1991hc, Garthwaite:1989ij, Lambolez:1992bs}
as well as metabotropic receptors\,\cite{Blackstone:1989fu, Glaum:1992kl, Vranesic:2qa}. Accordingly,
we modeled this excitatory input with the same kinetics as
for the climbing fiber, again ignoring possible Ca$^{2+}$- associated effects.

\subsubsection*{SYNAPTIC CONDUCTANCE}

The maximum conductance of a
single parallel fiber synapseis also unknown but seems to be variable
\,\cite{Hirano:1986fv, Ito:1989dz}. We used a value of 0.7\,
nS for $\bar g$ that falls in the range of values (0.2--50\,nS) that have been
used in other models using glutamate synaptic transmission\,\cite{R:1989cr, W:1991qa, Miller:1985mi, Rapp-M:1992kx, Wehmeier:1989pi, Wilson:1989ff, Zador:1990lh}.
 This synapse also had a reversal
potential of 0\,mV\,\cite{Cull-Candy:1989oq, Mayer:1987kl}.

\subsubsection*{NUMBERS OF SYNAPSES}

It is known that each Purkinje cell receives
$\sim$150,000 granule cell synapses\,\cite{Harvey:1991xz}.
Given computing resources it was not possible in the
current simulation to model all these inputs. Accordingly, in most
of the simulationsp resentedin this paper, granule cell inputs were
delivered on 1,474 spines ( 1\,\% of real), i.e., one on each modeled
spiny dendritic compartment. This simplification has two consequences.
First, there are a large number of dendritic spines missing
from the model. This was compensated for by adding dendritic
membrane (see\,\cite{R:1989cr, Rapp-M:1992kx}).
Second, many synaptic inputs are missing. Under the conditions
of random, asynchronous inputs simulated here, we compensated
for this missing  input by increasing the firing rate of each synapse.
A similar approach has been taken by other Purkinje cell modelers\,\cite{Rapp-M:1992kx}.
Assuming a linear scaling and simulating 1\,\%
of the inputs, an asynchronous firing rate of 1\,Hz in the model
would thus correspond to an average firing rate of $\sim$-0.01\,Hz for
real parallel fibers. We will show in the RESULTS section that this
scaling was appropriate for the current model. All synaptic input
firing rates mentioned are unscaled unless explicitly stated.

\subsubsection*{SYNAPTIC SPECIALIZATIONS}

Granule cell synaptic inputs are
known to terminate on dendritic spines\,\cite{Harvey:1991xz, Palay:1974fk}.
On the basis of electron microscopic
(EM) reconstructions of rat Purkinje cell spines\,\cite{M:1988bh}
we assumed a density of 13 spines per 1\,$\mu$m
dendritic length, resulting in a total of 144,456 spines. Most of
these spines have narrow-diameter spine necks and it is likely that
they have a significant effect on the electrical effects of granule cell
synaptic inputs\,\cite{Rall:1990tg}. For this reason, when
granule cell synaptic effects were studied in the model, spines were
explicitly simulated using two compartments each. One compartment
represented a spherical spine head with a diameter of 0.54\,$\mu$m,
whereas the second compartment represented the cylindrical
spine neck with a diameter of 0.20\,$\mu$m and a length of 0.66\,$\mu$m\,\cite{M:1988bh}. These dimensions resulted in a membrane
surface per spine of 1.33\,$\mu$m$^2$.

Reducing the number of granule cell inputs in the model also
had the effect of reducing the total number of spines. Given the
enormous number of theses pines on a normal Purkinje cell, simply
leaving their membrane area out of the model would significantly
affect simulation results. In our previous efforts to model
the electrical (not synaptic) properties of the Purkinje cell\,\cite{deschutter94:_purkin_i}
the influence of these dendritic spines
was approximated by increasing the membrane surface of compartments
of the region of the dendrite on which spines occur\,\cite{R:1989cr, Jaslove:1992fu, Jaslove:1992fu}.
The same approach was used here when granule cell synaptic effects
were not being simulated. However, for those simulations that did
relate to granule cell input, the total membrane surface was kept
constant by subtracting the membrane surface of modeled spines
(i.e., 1.33\,$\mu$m$^2$ for each spine) from the membrane surface of the
dendritic compartment to which they were connected. As already mentioned,
when no granule cell inputs were modeled,
the simulation was run without spines (i.e., $N_s = 0$), as was
done in the preceding paper\,\cite{deschutter94:_purkin_i}.

\subsubsection*{SYNAPTIC ACTIVATION}

Parallel fiber synaptic input was modeled
as an asynchronous process\,\cite{Bernard:1991ye}.
Activation of each modeled synapse was triggered independently
by a random number generator, which generated a Poisson distribution
around a mean frequency of input.

\subsection*{Stellate Cell Synapses}

\subsubsection*{SYNAPTIC LOCATIONS}

Morphological data about stellate cell
synapse distributions on Purkinje cell dendrites are incomplete.
On the basis of EM data they seem to synapse on both small spiny
and thicker dendritic branches\,\cite{Palay:1974fk} and
in general they do not contact the innervation sites of basket cell
synapses (ibid.), but neither the densities of these synaptic contacts
nor their exact sites of termination are known. Because there
are more stellate cells than Purkinje cells\,\cite{Ito:1984uq} we assumed
that stellate cell synapses were present on all the smooth and spiny
dendritic compartments. We placed two stellate cell synaptic contacts
on the shaft of every smooth dendritic compartment and one
contact on every spiny dendritic compartment, resulting in a total
of 1,695 stellate cell synapses in the model.

\subsubsection*{SYNAPTIC PHARMACOLOGY AND KINETICS}
Stellate cell inhibition
is mediated by $\gamma$-aminobutyric acid-A (GABA$_{\mbox{A}}$) receptors\,
\cite{Gabbot:1986fk, Ito:1984uq, Llano:1991uq}. The kinetics for
this inhibitory synaptic conductance were based on recordings of
miniature inhibitory synaptic currents in pyramidal neurons of
the hippocampus\,\cite{Ropert:1990kx} with an opening time constant
of 0.9\,ms, a closing time constant of 26.5\,ms (peak at 3.2\,
ms), and a reversal potential of -80\,mV.

\subsubsection*{SYNAPTIC CONDUCTANCES}
Postsynaptic conductances on
smooth dendritic compartments had a $\bar g$ of 1.4\,mS/cm$^2$; conductances
on a spiny dendritic compartment had a $\bar g$ of 7\,mS/cm$^2$.

\subsubsection*{SYNAPTIC ACTIVATION}
Stellate cell synapses were fired asynchronously,
following a Poisson distribution around a mean frequency
of input. There was no relation between the timing of
excitatory and inhibitory inputs.

\subsection*{Basket Cell Synapses}

\subsubsection*{SYNAPTIC LOCATIONS}
About 30--50 basket cells terminate on each Purkinje cell on 
the soma and dendritic trun\,\cite{Palay:1974fk, Palkovits:1971vn}. 
Basket cell synaptic receptor
channels were placed on the soma and on all compartments
of the main dendrite (see Fig. 1 in\,\cite{De-Schutter-E:1994vn}).

\subsubsection*{SYNAPTIC PHARMACOLOGY AND KINETICS} 
Basket cell inhibition
is also mediated by GABA, receptors\,\cite{Gabbot:1986fk, Ito:1984uq}. 
We used the same kinetics as for the stellate cell synapses.

\subsubsection*{SYNAPTIC CONDUCTANCES}
As with the excitatory synaptic inputs,
the amplitudes of the conductances of inhibitory synapses
are not known, but the strength of these inputs is assumed to 
vary\,\cite{Bishop:1992ys}. Basket cell synapses on the soma had a $\bar g$ of
100\,pS/cm$^2$ (total conductance 139\,$\mu$S) and on the dendrite a $\bar g$ of
50\,S/cm$^2$ (total conductance for main dendrite 47\,pS) for a
single input.

\subsubsection*{SYNAPTIC ACTIVATION}
In simulations of basket cell activity 20
such basket cell synapses were fired synchronously to provide a
strong inhibitory input.

\bibliographystyle{plain}
\bibliography{../tex/bib/g3-refs}

\end{document}
