\documentclass[12pt]{article}
\usepackage[dvips]{epsfig}
\usepackage{color}
\usepackage{url}
\usepackage[colorlinks=true]{hyperref}

\begin{document}

\section*{GENESIS: Documentation}

\section*{DES: the discrete event system}

The {\it D}iscrete {\it E}vent {\it S}ystem is used for abstract
modeling of action potentials. This functionality is needed for
running network simulations. Action potentials generated at the soma
of one neuron, are translated to a discrete event, that is delivered
at a post-synaptic target of a connected neuron. The {\it DES}
component borrows ideas from previous research in discrete event
simulation.

From a software architecture viewpoint, {\it DES} is a separate
solver: it simulates action potential propagation in an efficient way.
Splitting this functionality from the rest of the simulator,
facilitates customization for modeling of sophisticated learning
rules, especially related to STDP, diffusion and spillover.

Internally, the discrete event system contains two subcomponents, one
component for event distribution that contains a connectivity matrix,
a second component for event queuing.

\subsection*{DES Features}

\begin{itemize}
\item a single event to many targets.
\item queues the event whenever applicable.
\item allows for easy customization, e.g. to associate 'secondary'
  data with a connection, like weight, delay and other data specific
  for the model.
\end{itemize}

Note: at the time of writing this component is distributed as part of the Heccer package. This is likely to change in the near future. 


\end{document}

%%% Local Variables: 
%%% mode: latex
%%% TeX-master: t
%%% End: 
