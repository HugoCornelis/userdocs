\documentclass[12pt]{article}
\usepackage{verbatim}
\usepackage[dvips]{epsfig}
\usepackage{color}
\usepackage{url}
\usepackage[colorlinks=true]{hyperref}

\begin{document}

\section*{GENESIS: Documentation}

{\bf Related Documentation:}
% start: userdocs-tag-replace-items related-do-nothing
% end: userdocs-tag-replace-items related-do-nothing

\section*{De Schutter: Purkinje Cell Model}

\subsection*{Source}

De Schutter E \& Bower JM (1994) An active membrane model of the cerebellar Purkinje cell I. Simulation of current clamp in slice. {\it Journal of Nerurophysiology}. {\bf 71}: 375--400.

\subsection*{Discretization}

\subsubsection*{Parameter values}

\begin{tabular}{ l l l }
                                                                  & {\bf Original}                                   & {\bf Applied}  \\
  Compartment number                        & --                                                       & 1,600              \\
  Electrotonic length ($\lambda$)       & $<$\,0.05\,\cite{W:1966ve}          & 0.009--0.05    \\
                                                                  &                                                          &                          \\
%\multicolumn{3}{l} {*\,Obtained value}%, $^\dag$\,Adjusted value.}                                                                           \\
\end{tabular}

\subsubsection*{Column}

\begin{enumerate}

   \item {\bf Original:} Rapp M, Yarom Y \& Segev I (1994) Physiology, morphology and detailed passive models of guinea-pig cerebellar Purkinje cells. {\it Journal of Physiology} (Lond.). {\bf 474}: 101--118.
   \begin{itemize}
      \item As is standard for compartmental modeling\,\cite{W:1962nx, W:1964oq},
the number of compartments is determined by the morphology of the cell and by
simulation requirements for numerical accuracy. It has been previously shown that when active channels are used
the electrotonic length of compartments should be $<$\,0.05\,$\lambda$\,\cite{W:1966ve}.
   \end{itemize}

   \item {\bf Applied:} All simulations were performed with a model using the morphology of {\it cell I} of Rapp et
al. (1994) unless otherwise noted.
   
   Because electrotonic
lengths were short, more computationally
efficient asymmetric compartments (this GENESIS object corresponds
to a three-element segment, as described by\,\cite{Segev-I:1985kl}) could be used. Simulations done with and without asymmetric compartments
confirmed that the difference in input resistance ($R_N$) and
system time constant ($\tau_0$) for a passive membrane model was
$<$\,1\,\%.

In an active membrane compartment the total $R_m$ is variable,
resulting in a continuously changing space constant and electrotonic 
length\,\cite{Bernander-O:1991tg, Rapp-M:1992kx}.
The total $R_m$ ($RTM_n$) of a compartment $n$ at a specific time was
computed as the sum of all conductances $[ G(V, [ Ca^{2+}] ,t)$ from Eqn.\,(\ref{eqn:el1}) and the passive component, $RM_n$, determined by $R_m$

\begin{equation}
\label{eqn:el1}
   RTM_n(t) = \frac{1}{\frac{1}{RM_n}+\Sigma G_n(V,[ Ca^{2+}] ,t)}
\end{equation}

The compartment's space constant $\lambda_n$ could be derived from $RTM_n$, its axial resistance $RIn$, and its length $L_n$

\begin{equation}
   \lambda_n(t) = \sqrt{\frac{L^2_nRTM_n(t)}{RI_n}}
\end{equation}

The compartment's electrotonic length was the ratio of its length, 
$L_n$, over the space constant, $\lambda_n$. The electronic distance
between two compartments was computed as the sum of the electronic
lengths of all intervening compartments.

For those simulations that
related to granule cell input, the total membrane surface was kept
constant by subtracting the membrane surface of modeled spines
from the membrane surface of the
dendritic compartment to which they were connected.
For a spiny dendritic compartment with length $L$ and diameter $D$, the
membrane surface ($S$ in $\mu$m$^2$), used for the computation of compartment
membrane capacitance and resistance\,\cite{R:1989cr, Rapp-M:1992kx},
depended thus also on the number
of simulated spines ($N_s$) connected to that compartment

\begin{equation}
   S = L\cdot D\cdot \pi + 1.33(13\cdot L - N_s)
\end{equation}

\end{enumerate}

\bibliographystyle{plain}
\bibliography{../tex/bib/g3-refs}

\end{document}
