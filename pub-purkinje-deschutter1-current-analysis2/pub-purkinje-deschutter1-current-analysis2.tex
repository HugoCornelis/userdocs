\documentclass[12pt]{article}
\usepackage{verbatim}
\usepackage[dvips]{epsfig}
\usepackage{color}
\usepackage{url}
\usepackage[colorlinks=true]{hyperref}

\begin{document}

\section*{GENESIS: Documentation}

{\bf Related Documentation:}
% start: userdocs-tag-replace-items related-do-nothing
% end: userdocs-tag-replace-items related-do-nothing

\section*{De Schutter: Purkinje Cell Model}

\subsection*{INTERACTION BETWEEN IONIC CURRENTS}

One of the principle
benefits of modeling neurons at this level of detail is that
the interactions between different currents can be explored.
All too frequently experimentalists assign very specific roles
to particular channel types without taking into account the
often complex interactions between different conductances
that are actually responsible.

In the current model there are numerous examples of this
type of interaction. One of the most striking involves the
interaction of the noninactivating K$^+$ channels and the
Na$^+$ and Ca$^{2+}$ channels responsible for the plateaus. At the
somatic level these K$^+$ conductances served to counteract
the depolarizing effect of the Na$^+$ channels, with their balance
determining the voltage of the plateau potential
(\href{../pub-purkinje-deschutter1-fig-11/pub-purkinje-deschutter1-fig-11.tex}{\bf Fig.\,11{\it A}}).

In the dendrite the Ca$^{2+}$ -activated K$^+$ channels played a
critical role in the dual function of the CaP channel in generating
Ca$^{2+}$ spikes as well as dendritic plateaus. Because
the K2 channel is sensitive to small changes in Ca$^{2+}$ concentration
it could effectively increase the threshold for
Ca$^{2+}$ spike generation by counteracting the activation of
small numbers of CaP channels. The involvement of the
K2 channel in these two very different forms of dendritic
response, and the presence of other mechanisms in the Purkinje
cell dendrite that through changes in the Ca$^{2+}$ concentration\,\cite{Llano:1991kx, Takei:1992ac}
could activate
this channel, suggest interesting possibilities for regulation
of more global dendritic response properties by the K2
channel and other Ca$^{2+}$-inactivated K$^+$ channels that were
not included in the model (like the SK or afterhyperpolarization
(AHP) current,\,\cite{Lancaster:1991ye}.

The model supports the suggestion that the channels responsible
for generating plateau potentials are different in
the soma and dendrite\,cite{R:1980ly}. However,
examination of the model also makes clear that these
different plateaus are not physiologically isolated. A somatic
plateau potential was always accompanied by a dendritic
plateau and vice versa, because any depolarization
will spread throughout the cell.

Other interactions between channels, like the activation
of the anomalous rectifier during prolonged hyperpolarizations,
followed by an activation of CaT channels during the
rebound spike, were not explored in detail. The anomalous
rectifier did not affect the repetitive firing properties of the
model because it was completely deactivated beyond spiking threshold.

\bibliographystyle{plain}
\bibliography{../tex/bib/g3-refs.bib}

\end{document}
