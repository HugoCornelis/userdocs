\documentclass[12pt]{article}
\usepackage{verbatim}
\usepackage[dvips]{epsfig}
\usepackage{color}
\usepackage{url}
\usepackage[colorlinks=true]{hyperref}

\begin{document}

\section*{GENESIS: Documentation}

{\bf Related Documentation:}
% start: userdocs-tag-replace-items related-do-nothing
% end: userdocs-tag-replace-items related-do-nothing

\section*{De Schutter: Purkinje Cell Model}

\subsection*{Effect of active membrane on the electrical properties of Purkinje cells}

\subsection*{CHANNEL ACTIVATION AND ELECTROTONIC LENGTH}

We have used the model to contrast the electrical properties of a
passive Purkinje cell dendrite with those of a dendrite with
the active properties just described.
\href{../pub-purkinje-deschutter1-fig-12/pub-purkinje-deschutter1-fig-12.tex}{\bf Figure 12} shows the effect of including voltage-dependent
conductance on the electrotonic structure of the
model. In this figure, Sholl diagrams\,\cite{Sholl:1953kl} are presented
in units of electrotonic length of the Purkinje cell
under different conditions. At the right side, enlargements
show the electrotonic lengths of a few distal branchlets in a
passive membrane model (no active conductances in the
model; \href{../pub-purkinje-deschutter1-fig-12/pub-purkinje-deschutter1-fig-12.tex}{\bf Fig.\,12{\it B}}) and in an active membrane model during
and after a dendritic spike. One can see that adding active
membrane to the cell roughly doubled its electrotonic size
during the resting state shown in \href{../pub-purkinje-deschutter1-fig-12/pub-purkinje-deschutter1-fig-12.tex}{\bf\,Fig. 12{\it C}} (the dendritic
membrane had a potential of about -60\,mV) and that during
a dendritic spike the electrotonic size was roughly doubled
again (\href{../pub-purkinje-deschutter1-fig-12/pub-purkinje-deschutter1-fig-12.tex}{\bf Fig.\,12{\it D}}). The electrotonic distance from the
soma to the marked dendritic tip (morphological distance
392\,$\mu$m) was 0.57\,$\lambda$ in the passive model; it was 0.95\,$\lambda$ in the
active model in the resting state and 1.57\,$\lambda$ in the active
model during the dendritic spike.

\bibliographystyle{plain}
\bibliography{../tex/bib/g3-refs.bib}

\end{document}
