\documentclass[12pt]{article}
\usepackage[dvips]{epsfig}
\usepackage{color}
\usepackage{url}
\usepackage[colorlinks=true]{hyperref}

\begin{document}

\section*{GENESIS: Documentation}

{\bf Related Documentation:}
% start: userdocs-tag-replace-items related-do-nothing
% end: userdocs-tag-replace-items related-do-nothing

\section*{The {\it Developer} Package}

The {\it Developer} package provides developer utilities that comply
with GENESIS development standards. The most important one is the {\it
  neurospaces\_build} script used for automated software installation
and maintenance on a configurable set of software packages.  Because
this script has many options, most common operations are provided
using frontends.  Other scripts are related to version identification
of the software, source code documentation comments, testing, and
release management.

Some of the utilities currently depend on \href{http://monotone.ca}{monotone} for source code version control due to their configuration. It is possible to work with other version control systems.

\subsection*{Utilities}

\subsubsection*{Main driver script}

\begin{itemize}
	\item {\it neurospaces\_build}\,\,\,Perform various build operations, do 
	\begin{verbatim}
	   neurospaces_build --help
	\end{verbatim}
	   to see how it works. 
\end{itemize} 

%Also look at \href{../release-procedure/release-procedure.tex}{ReleaseProcedure}.

\subsubsection*{Other utilities}

These utilities take optional arguments of ``{\tt --regex}'' to select what packages they operate on, and ``{\tt --verbose}'' to run the command in a more verbose mode. 
\begin{itemize}
	\item {\it neurospaces\_packages}\,\,\,Show what packages are enabled on your local machine.
	\item {\it neurospaces\_versions}\,\,\,Shows what versions of the GENESIS packages are installed.
	\item {\it neurospaces\_create\_directories}\,\,\,Create the correct directory layout required for GENESIS development. 
\end{itemize}
\begin{itemize}
	\item {\it neurospaces\_serve}\,\,\,Starts serving the source code repositories such that other people can {\it pull} and {\it sync} to your machine (note that this locks all your databases).
	\item {\it neurospaces\_pull}\,\,\,Download the source code from a repository.
	\item {\it neurospaces\_status}\,\,\,Check for local source code modification (no network required).
	\item {\it neurospaces\_sync}\,\,\,Synchronize local source code modification with a repository.
	\item {\it neurospaces\_update}\,\,\,Makes the local source code up to date using the repositories locally stored on your PC (so this is a local operation). 
\end{itemize}
\begin{itemize}
	\item {\it neurospaces\_configure}\,\,\,(Re)configures the simulator software (requires packages to be installed already).
	\item {\it neurospaces\_install}\,\,\,Install the simulator software.
	\item {\it neurospaces\_uninstall}\,\,\,Uninstalls the simulator software (including the installer scripts, to reinstall you will have to go to the installer source code directory and run ``{\tt make \&\& sudo make install}'').
	\item {\it neurospaces\_check}\,\,\,Check for correctness of the installed software.
	\item {\it neurospaces\_clean}\,\,\,Clean source code directories. 
\end{itemize}
\begin{itemize}
	\item {\it neurospaces\_docs}\,\,\,Builds documentation on your local machine.
	\item {\it neurospaces\_website\_prepare}\,\,\,Prepare a version of the website on your local PC, and optionally upload it. 
\end{itemize}
\begin{itemize}
	\item {\it neurospaces\_cron}\,\,\,A {\it cron} job based tester script. 
\end{itemize}

\subsubsection*{Soon also}

\begin{itemize}
	\item {\it neurospaces\_download}\,\,\,Download the simulator from a central archive.
\end{itemize}
For updating of version keywords the following are invoked automatically by the {\it neurospaces\_build} script:
\begin{itemize}
	\item {\it release\_extract}\,\,\,Extract release information from a monotone repository. When a Tag is set, that will be the result, otherwise the SHA of the current base revision will be the result.
	\item {\it release\_expand}\,\,\,Do keyword expansion, see the {\it manpage} in the source for more details. 
\end{itemize}

\subsubsection*{Other}

\begin{itemize}
\item {\it mcad2doxy}\,\,\,Convert obsoleted multicad documentation to
  {\it doxygen} format, this has been used to convert the {\bf Heccer}
  developer documentation to {\it doxygen} format. Other packages will
  follow.
\item {\it nstest\_query}\,\,\,Queries test specifications of a package.
\item {\it nspkg-deb}\,\,\,Automated debian package builder.
\item {\it td-labels}\,\,\,Shows all release labels (major and version numbers).
\item {\it td-majors}\,\,\,Shows all major release labels.
\end{itemize}

\subsubsection*{Simple Examples}

The following examples can be used by developers who do not want to
know about the intricate details of the developer tools.  Most
developers fall into this category.

Software developers and laboratories who have added their own private
software components should be ok after reading this section and can
skip reading the more detailed sections.

\begin{enumerate}
\item {\bf What packages does the neurospaces installer recognize:}
\begin{verbatim}
$ neurospaces_build --help-packages
\end{verbatim}

\item {\bf What operations on known packages exist:}
\begin{verbatim}
$ neurospaces_build --help-operations
\end{verbatim}

\item {\bf What actions are taken by default:}
\begin{verbatim}
$ neurospaces_build --help
\end{verbatim}
and inspect the flags at the end of the output. At this time, {\tt --configure}, {\tt --compile}, and {\tt --install} are the default actions. 

\item {\bf How are operations mapped to packages:}
\begin{verbatim}
$ neurospaces_build --help-all
\end{verbatim}

\item {\bf What actions are taken after options have been given:} For example,
\begin{verbatim}
$ neurospaces_build --distcheck --help
\end{verbatim}
enables the {\it distcheck} target of the makefiles, in addition to the default actions. 

\item {\bf What shell commands will be run after options have been given, without really running the commands:}
\begin{verbatim}
$ neurospaces_build --verbose --distcheck --dry-run --developer
\end{verbatim}

\item {\bf Uninstall the currently installed software:}
\begin{verbatim}
$ neurospaces_uninstall
\end{verbatim}

\item {\bf Pull the default repository to obtain the newest version of the
  source code in your local repository:}
\begin{verbatim}
$ neurospaces_pull
\end{verbatim}

\item {\bf Update local source code from the local repository:}
\begin{verbatim}
$ neurospaces_update
\end{verbatim}
This command also automatically merges parallel development branches
  if necessary.

\item {\bf Clean up the local workspace:}
\begin{verbatim}
$ neurospaces_clean
\end{verbatim}

\item {\bf Install the new source code:}
\begin{verbatim}
$ neurospaces_install
\end{verbatim}

\end{enumerate}

\subsubsection*{More Detailed Examples}

See also
\href{../installation-developer/installation-developer.tex}{Developer
  Installation}.

Steps in building a release for the example {\it your-software}
package (see also the
  sections on creating new software components in the \href{../developers-intro/developers-intro.tex}{Developers
  Introduction}).

\begin{itemize}

\item {\bf Tag the code:} The GENESIS convention for creating these tags, which must be unique, is to concatenate an identifier from the name of the lab of origin, the purpose and/or type of software (e.g. build, passive, active, python, userdocs, des, network, i64, purkinje, integration, pools, etc), with an appended numerical identifier. Here, for example we use the generic tag ``{\tt mylab-mysoftware-vnum}''.
\begin{verbatim}
$ neurospaces_build --tag mylab-mysoftware-vnum --verbose \
   --verbose --verbose --no-compile --no-configure --no-install \
   --regex my-software --developer
\end{verbatim}

\item {\bf Build the release for the tagged code:}
\begin{verbatim}
$ neurospaces_build --dist --verbose --verbose --verbose \
   --no-compile --no-configure --no-install --regex my-software \
   --developer
\end{verbatim}

\item {\bf Or with one command line:}
\begin{verbatim}
$ neurospaces_build --tag mylab-mysoftware-vnum --dist \
   --verbose --verbose --verbose --no-compile --no-configure \
   --no-install --regex my-software --developer
\end{verbatim}

\item {\bf Upload the tarballs:}
\begin{verbatim}
$ neurospaces_build --src-tag mylab-mysoftware-vnum \
   --upload-server ftp://upload.sourceforge.net/incoming \
   --verbose --verbose --verbose --no-compile --no-configure \
   --no-install --regex my-software --developer
\end{verbatim}
Do not forget to edit release notes, tag the files as ``Any'', ``Source.gz'', etc. Due to the crappy Sourceforge interfaces, this has to be done manually (anyone have any ideas?).

\end{itemize}


\subsection*{Some Detailed Examples of the Release Procedure}

After putting a tag on the code, you first must check if the package
can be built from the tarball. The autotools {\it distcheck} target
does this (see below). An additional check is to build the tarball,
put it somewhere on your filesystem, and then build the {\it check}
target from the tarball.

For example, first build the tarball and put it somewhere on your filesystem (e.g. {\it /tmp/uploads}):
\begin{verbatim}
   neurospaces_build --developer --verbose --upload-server 
      file://tmp/uploads --src-tag python-5 --no-configure --no-compile \
      --no-install --regex studio
\end{verbatim}
Then build the {\it check} target from the tarball:
\begin{verbatim}
   neurospaces_build --verbose --check --regex studio --src-tag python-5 \
      --src-dir /tmp/uploads --verbose --verbose --unpack --no-install
\end{verbatim}

In a one line summary you need to insert a {\tt --tag}, and {\tt --distcheck} the current code, then make them available on \href{http://sourceforge.net/projects/neurospaces/}{Sourceforge}. In one line of code:
\begin{verbatim}
   neurospaces_build --developer --distcheck --verbose --tag network-5 \
      --src-tag network-5 --upload-server https://frs.sourceforge.net/uploads \
      --verbose
\end{verbatim}
If you don't want to do the upload, for a single package:
\begin{verbatim}
   neurospaces_build --tag network-5 --distcheck --regex 'model-container' \
      --developer --verbose
\end{verbatim}
If that works, you then want to do the upload:
\begin{verbatim}
   neurospaces_build --src-tag network-5 --upload-server \
      https://frs.sourceforge.net/uploads --regex 'model-container' \
      --developer --verbose
\end{verbatim}
The {\tt --verbose} option is there to let you know what is going on during this lengthy procedure.

The same procedure is used to build a release of either individual or all packages. An official public release is made available for download on \href{http://sourceforge.net/projects/neurospaces/}{Sourceforge}. Intermediate releases are for internal use only.

Because official releases are built using automake's {\it distcheck} target, they must pass the tests of the package (on the machine used for the build). So they are considered to be alpha releases (ie. internally well tested).

The release procedure normally checks for package correctness using the automake {\it distcheck} target, meaning that both install and uninstall targets work and are each other's complement. Official releases must always succeed on this target. Intermediate releases can fail.


\subsubsection*{Other Examples}

\begin{itemize}
\item {\bf Uninstall 4 packages on your developer machine:}
\begin{verbatim}
$ neurospaces_build --verbose --verbose --verbose \
   --no-compile --no-configure --uninstall \
   --regex '(my-software|heccer|ssp|studio)' --developer
\end{verbatim}

\item {\bf Build checked releases for four packages:}
\begin{verbatim}
$ neurospaces_build --verbose --verbose --verbose \
   --no-compile --no-configure --no-install --distcheck 
   --regex '(my-software|heccer|ssp|studio)' --developer
\end{verbatim}

\item {\bf After a modification of the developer package, reinstall it:}
\begin{verbatim}
$ neurospaces_build --enable developer --regex developer \
   --developer --verbose --verbose --verbose
\end{verbatim}

\item {\bf Releasing the developer package on Sourceforge:}
\begin{verbatim}
$ neurospaces_build --tag build-25 --dist --src-tag build-25 \
   --upload-server ftp://upload.sourceforge.net/incoming \
   --enable developer --regex '(developer)' --developer --verbose
\end{verbatim}
\end{itemize}

\end{document}
