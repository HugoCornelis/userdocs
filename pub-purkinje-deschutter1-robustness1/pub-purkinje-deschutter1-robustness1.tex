\documentclass[12pt]{article}
\usepackage{verbatim}
\usepackage[dvips]{epsfig}
\usepackage{color}
\usepackage{url}
\usepackage[colorlinks=true]{hyperref}

\begin{document}

\section*{GENESIS: Documentation}

{\bf Related Documentation:}
% start: userdocs-tag-replace-items related-do-nothing
% end: userdocs-tag-replace-items related-do-nothing

\section*{De Schutter: Purkinje Cell Model}

\subsection*{MODEL ROBUSTNESS}

The primary parameters searched in 
this model involved the locations and densities of the different
channels. Although the complexity of the current
model has not allowed us to undertake a full parameter
search, as has been done in some other recent modeling
studies\,\cite{S:1993dz}, we have explored in detail
variations on key parameters to examine the overall
robustness of the results. As can be expected from a model
that shows so much response variability, it was found to be
quite robust to changes in channel densities.

Results were most sensitive to changes in the densities of
the CaP, KC, and K2 channels. However, these three channels
are actually tied together, i.e., modifying the density of
one of them will change the activation of the other two
during the dendritic spiking cycle. They are also linked
through our simple representation of Ca$^{2+}$ concentration,
which drives activation of the KC and K2 channels. For this
reason, robustness of the model could be maintained over a
wider range of channel densities for even these conductances,
provided more than one channel density was
changed at once. In fact, the initial process of model tuning
primarily involved manually changing the densities of these
three channels in an effort to obtain correct levels of dendritic
excitability.

\section*{Robustness of modeling results}

Having tuned the model to replicate in vitro responses to current injection, we ran several simulations to test the robustness of the basic model. These tests helped to build confidence that the model has biological validity.

\subsection*{VARIATIONS IN PURKINJE CELL MORPHOLOGY}

Initial modeling experiments were based on the anatomic reconstruction of one particular Purkinje cell. Accordingly, it was important to determine the sensitivity of the results to this particular morphology. To do this, identical channel equations and densities (PM\,9) were placed into two additional Purkinje cells, reconstructed
by\,\cite{Rapp-P:1994qf}; the results are shown in \href{../pub-purkinje-deschutter1-fig-8/pub-purkinje-deschutter1-fig-8.tex}{\bf Fig.\,8}.

In both cells the response properties of the model fell well within the normal variation seen in Purkinje cell recordings. Simulations of current injections in the soma produced the same typical pattern as in our model of Purkinje {\it cell\,1} (\href{../pub-purkinje-deschutter1-fig-3/pub-purkinje-deschutter1-fig-3.tex}{\bf Fig.\,3}), i.e., at low intensities steady firing of fast somatic spikes superimposed on an increasing plateau potential and at higher intensities the presence of dendritic Ca$^{2+}$ spikes. The details of these firing patterns were, however, quite different for the three cells. {\it Cell\,2} is smaller than {\it cell\,1} (\href{../pub-purkinje-deschutter1-fig-1/pub-purkinje-deschutter1-fig-1.tex}{\bf Fig.\,1}) and thus had a larger $R_N$\,\cite{Rapp-P:1994qf}. As a consequence, the Na$^+$ currents caused a more pronounced plateau potential and the somatic spikes were less well repolarized. This resulted in a progressive attenuation of spike amplitude because of incomplete removal of inactivation of NaF current. Note, however, that despite this buildup of inactivation, spiking did not saturate at a depolarized level during the 1.5\,nA current injection, whereas it did in {\it cell\,3} (not shown). The firing pattern of {\it cell\,3} is more similar to that of {\it cell\,1}, but with a shift in the $f-I$ curve. This can be explained by the small soma and short, thin main dendrite, which caused a smaller total Kdr and KM conductances in the model of this cell (\href{../pub-purkinje-deschutter1-table2/pub-purkinje-deschutter1-table2.tex}{\bf Table\,2}).

\bibliographystyle{plain}
\bibliography{../tex/bib/g3-refs}

\end{document}
