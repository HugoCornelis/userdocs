\documentclass[12pt]{article}
\usepackage[dvips]{epsfig}
\usepackage{color}
\usepackage{url}
\usepackage[colorlinks=true]{hyperref}

\begin{document}

\section*{GENESIS: Documentation}

{\bf Related Documentation:}
% start: userdocs-tag-replace-items related-do-nothing
% end: userdocs-tag-replace-items related-do-nothing

\section*{Physical Processes and Model Variables}

Physical processes are a mathematical model of the functional activity
of a biological entity.  This modeling aspect associates physical
processes with parameters (fixed during a simulation) and variables
(computed by a simulation).

As an example, a synaptic current is commonly modeled using a
second-order differential equation:
\begin{equation}
  \label{eq:second-order-synchan}
  G'' + \alpha G' + \beta G = S
\end{equation}
with $S = \delta(t - t_1) + \cdots + \delta(t - t_N)$.

In computational neuroscience the right-hand side of this equation
corresponds to the level of synaptic activation delivered by spikes
arriving at times $t_i$.

Assuming the initial values $G(0) = 0$ and $G'(0) = 0$, the solution
to this equation is a dual-exponential equation:
\begin{equation}
  \label{eq:dual-exponential}
  G(t) = \frac{\tau_1\tau_2}{\tau_1 - \tau_2}
  \cdot (\mathrm{e}^{\frac{-t}{\tau_1}} - \mathrm{e}^{\frac{-t}{\tau_2}})
\end{equation}
with $\alpha = \frac{\tau_1 + \tau_2}{\tau_1 \tau_2}$ and $\beta =
\frac{1}{\tau_1 \tau_2}$.
%\begin{equation}
%  \label{eq:simple-spatial-cable}
%  \frac{a}{2R_a}\frac{\partial^2V}{\partial x^2} = C_m \frac{\partial V}{\partial t} + \frac{V}{R_m}
%  + I_{\mathrm{HH}} + I_{\mathrm{syn}}
%\end{equation}
This equation has two parameters {\tt TAU1} and {\tt TAU2}, sometimes
called the raise and decay time constants respectively.  Assigning
values to the two parameters, this can be expressed in the NDF format
as:
%EQUATION_EXPONENTIAL exp2

%  ...
\begin{verbatim}
  PARAMETERS
    PARAMETER ( TAU1 = 0.50e-3 ),
    PARAMETER ( TAU2 = 1.20e-3 )
  END PARAMETERS
\end{verbatim}
%END EQUATION_EXPONENTIAL

When a biological entity modulates the behavior of its neighbour,
their physical processes share one or more variables.  In the case of
a synaptic channel the incoming spikes raise the neurotransmitter
concentration level in the synaptic cleft, which then modulates total
ionic conductance through the channel.  The NDF format expresses the
neurotransmitter concentration level as {\tt activation} and
conductance as {\tt G}.  These shared variables must be declared using
the {\tt BINDINGS} keyword:
\begin{verbatim}
  BINDABLES
    INPUT activation, OUTPUT G
  END BINDABLES
\end{verbatim}
Assuming a synapse is present under the label {\tt ../synapse}, its
activation level can be bound to the channel activation level using a
{\tt BINDINGS} clause:
\begin{verbatim}
  BINDINGS
    INPUT ../synapse->activation
  END BINDINGS
\end{verbatim}

Putting everything together using the {\tt EQUATION\_EXPONENTIAL}
token to represent the dual-exponential equation, we get the NDF
snippet:
\begin{verbatim}
EQUATION_EXPONENTIAL exp2
  BINDABLES
    INPUT activation, OUTPUT G
  END BINDABLES
  BINDINGS
    INPUT ../synapse->activation
  END BINDINGS
  PARAMETERS
    PARAMETER ( TAU1 = 0.50e-3 ),
    PARAMETER ( TAU2 = 1.20e-3 )
  END PARAMETERS
END EQUATION_EXPONENTIAL
\end{verbatim}

In this example, the order of the clauses of {\tt BINDABLES}, {\tt
  BINDINGS} and {\tt PARAMETERS} was chosen for readibility.

We finally note that neither the
\href{../ndf-file-format/ndf-file-format.tex}{\bf NDF file format} nor
the \href{../model-container/model-container.tex}{model-container}
make a syntactical distinction between variables and parameters of a
physical process.

\end{document}

%%% Local Variables: 
%%% mode: latex
%%% TeX-master: t
%%% End: 
