\documentclass[12pt]{article}
\usepackage{verbatim}
\usepackage[dvips]{epsfig}
\usepackage{color}
\usepackage{url}
\usepackage[colorlinks=true]{hyperref}

\begin{document}

\section*{GENESIS: Documentation}

{\bf Related Documentation:}
% start: userdocs-tag-replace-items related-do-nothing
% end: userdocs-tag-replace-items related-do-nothing

\section*{New Document}

Here is a
complete simple example with arbitrary values:

#!/usr/local/bin/genesis-g3
#!

create cell /n
create segment /n/soma

set_model_parameter /n/soma Vm_init -0.068
set_model_parameter /n/soma CM 0.01
set_model_parameter /n/soma RM 0.01
set_model_parameter /n/soma RA 0.01
set_model_parameter /n/soma ELEAK 0.01

set_model_parameter /n/soma LENGTH 1
set_model_parameter /n/soma DIA 1

set_runtime_parameter /n/soma INJECT 2e-9

run /n 0.001

echo output follows:\n
sh cat /tmp/output

Typing in all these parameters is a lot of work, and the gshell is not
intended to be used as a model editor.  Therefore you do:

ndf_save /** STDOUT

Or better to a file

ndf_save /** my-first-model.ndf

Then, to check if the syntax is correct, from a unix shell command line, you do

neurospacesparse my-first-model.ndf

Now quit the gshell, and restart it.

genesis >

You can now load your model using:

ndf_load my-first-model.ndf

Check what you have:

list_elements
list_elements /n
show_model_parameters /n/soma

show_parameter /n/soma CM
show_parameter_scaled /n/soma CM

Note that because I used a 1 for the values for diameter and length,
PI comes out as scaled value for CM.

Also note that the system does not care much how one create his
models, manually, using the backward compatibility, or using NDF
files, or both, or even in other ways.  The ndf_save command is
supposed to work always, and allows to convert models.

Third, I renamed 'neuron' to just 'n' because it is a reserved token
in a NDF file.
\end{document}
