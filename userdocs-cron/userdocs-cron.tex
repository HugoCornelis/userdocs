\documentclass[12pt]{article}
\usepackage{verbatim}
\usepackage[dvips]{epsfig}
\usepackage{color}
\usepackage{url}
\usepackage[colorlinks=true]{hyperref}

\begin{document}

\section*{GENESIS: Documentation}

{\bf Related Documentation:}
% start: userdocs-tag-replace-items related-do-nothing
% end: userdocs-tag-replace-items related-do-nothing

\section*{\it userdocs\_cron}

The GENESIS documentation system consumes many resources, particularly memory and cpu time, when being rebuilt. Rebuilding can take a considerable time, particularly on a single processor. For example, rebuilding Levels 1 and 2 takes at least ten minutes, whereas, rebuilding all seven levels takes at least 40 minutes. For these reasons the documentation build is an automated operation. The {\it userdocs} package comes with a {\it cron} script that can be set up to run at periodic intervals to keep your documentation up to date.

\subsection*{Configuration}

To configure the {\it userdocs\_cron} script you must construct a YAML file with the following options:
\begin{itemize}
   \item {\bf MAILTO:} An email recipient to send an email to on completion of the job.
   \item {\bf MAILFROM:} An email address from which to send a completion email.
   \item {\bf HTMLDIR:} A directory which will hold your {\it userdocs} html data directory to be served from the internet.
   \item {\bf OUTPUTDIR:} A directory to write the output log data.
   \item {\bf URL:} An http link which serves the test harness output (for an example see \href{http://neurospaces.sourceforge.net/neurospaces_project/gshell/tests/html/specifications/pclamp.html}{\bf PClamp}).
\end{itemize}
\subsubsection*{Example configuration file}

\begin{verbatim}
---
MAILTO: my@email.com
MAILFROM: userdocs@buildmachine.com
HTMLDIR: /var/www/html
OUTPUTDIR: /tmp
URL: http://mywebsite/userdocs
\end{verbatim}

\end{document}
