\documentclass[12pt]{article}
\usepackage{verbatim}
\usepackage[dvips]{epsfig}
\usepackage{color}
\usepackage{url}
\usepackage[colorlinks=true]{hyperref}

\begin{document}

\section*{GENESIS: Documentation}

{\bf Related Documentation:}\\
\href{../pub-purkinje-deschutter1-passive/pub-purkinje-deschutter1-passive.tex}{\bf Passive\,Membrane\,Properties},
\href{../pub-purkinje-deschutter1-table1/pub-purkinje-deschutter1-table1.tex}{\bf Channel\,Kinetics},
\href{../pub-purkinje-deschutter1-discretization/pub-purkinje-deschutter1-discretization.tex}{\bf Morphology\,Discretization}
% start: userdocs-tag-replace-items related-do-nothing
% end: userdocs-tag-replace-items related-do-nothing

\section*{De Schutter: Purkinje Cell Model}

\subsection*{Methods}

Simulations were performed with the neuronal simulation program
GENESIS\,\cite{Wilson:1989ff} on eight Sun Sparc2 workstations.
We used the Hines algorithm\,\cite{hines84:_effic} available in
GENESIS versions 1.3 and 1.4. The GENESIS implementation of
this algorithm is fast and accurate\,\cite{bhalla92:_rallp}. The simulations
were run with a time step of 20\,$\mu$s. Initial control simulations
at 10\,$\mu$s showed that the 20\,$\mu$s step produced numerically accurate
results. For the model presented here, 550\,ms of Purkinje cell
activity could be simulated in 1\,h. Several thousand simulations
were run, mostly to explore parameter space.

\bibliographystyle{plain}
\bibliography{../tex/bib/g3-refs}

\end{document}
